	\documentclass[12 pt]{article}

	\usepackage[frenchb]{babel}
	\usepackage[utf8]{inputenc}  
	\usepackage[T1]{fontenc}
	\usepackage{amssymb}
	\usepackage[mathscr]{euscript}
	\usepackage{stmaryrd}
	\usepackage{amsmath}
	\usepackage{tikz}
	\usepackage[all,cmtip]{xy}
	\usepackage{amsthm}
	\usepackage{varioref}
	\usepackage{geometry}
	\geometry{a4paper}
	\usepackage{lmodern}
	\usepackage{hyperref}
	\usepackage{array}
	 \usepackage{fancyhdr}
	 \usepackage{float}

	\pagestyle{fancy}
	\theoremstyle{plain}
	\fancyfoot[C]{\thepage} 
	\fancyhead[L]{Angles}
	\fancyhead[R]{2021-2022}
	\newcounter{n}
	\numberwithin{n}{section}
	\newtheorem{theo}{Théorème}
	\labelformat{theo}{théorème}
	\newtheorem*{prop}{Propriété}
	\labelformat{prop}{propriété}
	\newtheorem*{cor}{Corollaire}
	\newtheorem{lm}{Lemme}
	\labelformat{lm}{lemme~#1}
	\newtheorem{hyp}[n]{Hypothèse}
	
	{\theoremstyle{definition}
	\newtheorem*{df}{Définition}
    \newtheorem*{rmq}{Remarque}
	\newtheorem*{nt}{Notation}
	\newtheorem*{voc}{Vocabulaire}
	\newtheorem*{ex}{Exemple}
	\newtheorem*{exs}{Exemples}
	\newtheorem*{formule}{Formule}
	\newtheorem*{formules}{Formules}
	}

	\renewcommand\epsilon{\varepsilon}
	\renewcommand\phi{\varphi}
	\newcommand\R{\mathbb{R}}
	\newcommand\s{\mathbb{S}}
	
	
	
	\title{Cours Chapitre 6}
	\date{}
	\begin{document}

\begin{center}{\Large Chapitre 6 - Angles}\\ 
 \end{center}
\section{Rappel sur les angles et les symétries}
\subsection{Définition des angles}
\begin{df}
On appelle \emph{angle} la zone du plan délimitée par deux demi-droites de même origine. 
Cette origine est appelé \emph{sommet} de l'angle. 
On donne un nom à un tel angle en utilisant celui des demi-droites qui le délimitent : \begin{itemize}
\item L'angle entre les demi-droites $[Ox)$ et $[Oy)$ est noté $\widehat{xOy}$.
\item L'angle entre les demi-droites $[Ox)$ et $[OB)$ est noté $\widehat{xOB}$.
\item L'angle entre les demi-droites $[OA)$ et $[OB)$ est noté $\widehat{AOB}$.
\end{itemize}
\end{df}

\begin{df}
La \emph{mesure} d'un angle de sommet $O$ est définie ainsi : on trace le cercle de centre $O$ 
et de rayon $1$. On rappelle que ce cercle a pour périmètre $2\times \pi\approx 6,28\ldots$. L'angle 
délimite un certain arc de ce cercle, dont on note la longueur $\ell$. Alors la mesure de l'angle est 
définie commme le nombre $\frac{\ell}{2\times \pi} \times 360^o$.
\end{df}

\begin{rmq}
La définition ci-dessus correspond exactement à ce que l'on obtient en mesurant les angles avec le 
rapporteur. Elle revient en fait à considérer comme vraies les quelques propriétés intuitives suivantes. 
\end{rmq}
\begin{prop}
\begin{enumerate}
\item La mesure d'un angle faisant un tour complet est de $360^o$.
\item La mesure d'un angle plat est de $180^o$.
\item La mesure d'un angle droit est de $90^o$.
\item La longueur $\ell$ d'un arc de cercle d'angle au centre $a$ et de rayon $R$ est proportionnelle à l'angle $a$. Autrement dit, on a \[\ell = 2 \times \pi \times R
\times \frac{a}{360^o}. \]
\end{enumerate}
\end{prop}
\begin{figure}[H]\center
\begin{tikzpicture}
\fill[red!50] (0.3, 0) arc (0: 360:0.3);
\fill (0,0) circle (0.05);
\draw (0,1) node {$360^o$};
\draw (0,0)-- (1,0); 
\end{tikzpicture}\ \ \ \ \ \ \ \ \ \ \ \ 
\begin{tikzpicture}
\fill[red!50] (0.3, 0) arc (0: 180:0.3);
\fill (0,0) circle (0.05);
\draw (-1,0)-- (1,0); 
\draw (0,1) node {$180^o$};
\end{tikzpicture}\ \ \ \ \ \ \ \ \ \ \ \ \ 
\begin{tikzpicture}
\fill[red!50] (0.3, 0) arc (0: 90:0.3)--(0,0);
\fill (0,0) circle (0.05);
\draw (0,1)-- (0,0)--(1,0); 
\draw (0,1.4) node {$90^o$};
\end{tikzpicture}   
\caption{Un angle plein, un angle plat, et un angle droit.}
\end{figure}
\begin{figure}[H]\center
\begin{tikzpicture}
\draw (2,0) arc (0:360:2) ; 
\fill[red!60] (0.5,0) arc (0:120:0.5) -- (0,0);
\draw[red, very thick] (2,0) arc (0:120:2);
\draw[red] (3,0) --(0,0) -- (120:3) ;
\draw[red] (0,.8) node{$a$};
\draw[red] (1, -.3) node {$R$};
\draw[red] (1.6,1.6) node {$\ell$};
\draw (6,0) node{
$\ell = \underbrace{2\times \pi \times R}_{\text{circonférence du cercle entier}} \times \frac{a}{360}$};
\end{tikzpicture}
\caption{Longueur d'un arc de cercle}
\end{figure}

\begin{rmq}
On admet aussi que la surface d'un secteur angulaire est proportionnelle à la mesure de l'angle, comme 
résumé sur la figure ci-dessous. \begin{figure}[H]\center
\begin{tikzpicture}
\draw (2,0) arc (0:360:2) ; 
\fill[black!30] (2,0) arc (0:120:2) -- (0,0);
\fill[red!70] (0.5,0) arc (0:120:0.5) -- (0,0);
\draw[red] (3,0) --(0,0) -- (120:3) ;
\draw[red] (0,.8) node{$a$};
\draw[red] (1, -.3) node {$R$};
\draw (6,0) node{
$\mathcal A = \underbrace{\pi \times R}_{\text{aire du disque entier}} \times \frac{a}{360}$};
\end{tikzpicture}
\caption{Aire d'un secteur circulaire (en gris sur la figure)}
\end{figure}
\end{rmq}

\subsection{Vocabulaire}

\begin{df}
On dit que deux angles sont \emph{complémentaires} si la somme de leurs mesures est de $90^o$. 

On dit que deux angles sont \emph{supplémentaires} si la somme de leurs mesures est de $180^o$. 

On dit que deux angles sont \emph{adjacents} s'ils ont le même sommet, et sont de part et d'autre d'un côté commun.
L'angle obtenu en mettant deux angles adjacents côte à côte et en oubliant leur côté commun s'appelle leur réunion.
\end{df}
\begin{figure}[H]\center
\begin{tikzpicture}
\draw[red] (0,0) -- (50: 3);
\fill[red!50] (50:.75) arc (50:90:.75)--(0,0);
\fill[blue!50] (0:.75) arc (0:50:.75)--(0,0);
\draw (0,0) -- (0: 3);
\draw (0,0) -- (90: 2.5);
\end{tikzpicture}\ \ \ \ \ 
\begin{tikzpicture}
\draw[red] (0,0) -- (70: 2.5);
\fill[red!50] (70:.75) arc (70:180:.75)--(0,0);
\fill[blue!50] (0:.75) arc (0:70:.75)--(0,0);
\draw (0,0) -- (0: 3);
\draw (0,0) -- (180: 3);
\end{tikzpicture}
\caption{À gauche : deux angles complémentaires; à droite : deux angles supplémentaires. Dans les deux exemples, les angles bleu et rouge sont adjacents.}
\end{figure}

Étant donné deux angles adjacents, on a la propriété immédiate suivante. 
\begin{prop}
La mesure de la réunion de deux angles adjacents est la somme de leurs mesures respectives. 

\begin{figure}[H]\center
\begin{tikzpicture}
\draw (0,0) -- (10 : 2.5);
\draw[dashed] (0,0) -- (50 : 2);
\draw (0,0) -- (120 : 2.5);
\draw[red] (50:.7) arc (50: 10:.7);
\draw[blue] (50:.6) arc (50: 120:.6);
\draw (30:1) node{$40^o$} (85 : 1) node {$70^o$};
\draw[purple] (10:1.5) arc (10: 120:1.5);
\draw (65:1.9) node{$40+70=110^o$};
\end{tikzpicture}\caption{Deux angles adjacents de $40^o$ et $70^o$ et leur réunion, qui mesure $110^o$.}
\end{figure}
\end{prop}

\subsection{Rappel sur les symétries}
La propriété fondamentale des angles est celle énoncée au chapitre 2 sur les symétries, et que l'on rappelle ici sommairement. 

\begin{theo}[Axiome de conservation des angles]
Les symétries axiales et centrales conservent les mesures des angles. 
\end{theo}

De manière plus détaillée, celle-ci s'utilise ainsi : 

\begin{prop}
Si $A'$, $B'$, et $C'$ sont les symétriques respectifs des points $A$, $B$, et $C$ par rapport à 
une même droite/à un même point, alors les angles $\widehat{ABC}$ et $\widehat{A'B'C'}$ ont la même mesure. 
\end{prop}

\begin{figure}[H]\center
\begin{tikzpicture}
\fill[red!50] (20:.6) arc (20: 80:0.6)--(0,0);
\draw (20:2)-- (0,0)--(80:2); 
\draw (-1, 2) -- (-1, -1);
\draw (-0.6, -1) node {$(\Delta)$};
\draw (-2,0) -- ++(160:2);
\draw (-2,0) -- ++(100:2);
\fill[red!50] (-2,0) -- ++(160:.6) arc (160: 100: 0.6);
\end{tikzpicture}  \ \ \ \ \ \ \ \ \ \ \
\begin{tikzpicture}
\fill[red!50] (-10:.6) arc (-10: -60:0.6)--(0,0);
\draw (-10:2)-- (0,0)--(-60:2); 
\draw (-1,0) ++(.1, .1)--++(-.2, -.2) ++(.2, 0) -- ++(-.2,.2);
\draw (-1, 0.4) node {$O$};
\draw (-2,0) -- ++(170:2);
\draw (-2,0) -- ++(120:2);
\fill[red!50] (-2,0) -- ++(120:.6) arc (120: 170: 0.6);
\end{tikzpicture} 
\caption{Deux angles symétriques par rapport à une droite (à gauche) ou à un point (à droite) sont donc de même mesure.}
\end{figure}
\section{Angles alternes-internes}
Dans cette partie, on étudie une première application des propriétés de la symétrie centrale pour identifier 
des angles égaux ou caractériser des droites parallèles. 
\subsection{Définition}

On se donne trois droites $(d)$,$(d')$ et $(\Delta)$, et on suppose que $(\Delta)$ rencontre $(d)$ en un point $A$,
et $(d')$ en un point $B$, comme sur la figure suivante : 
\begin{figure}[H]\center
\begin{tikzpicture}
\draw (-1, 1) -- ++(10: 3)-- ++(10: -6); 
\draw (1, -1) -- ++(40: 3)-- ++(40:-6);
\draw (-2,2) -- (2,-2); 
\draw (1,1.8) node {$(d)$}; 
\draw (3.5,1.3) node {$(d')$};
\draw (1.5,-2.3) node {$(\Delta)$};
\fill[red!50] (-1,1) ++ (10:.5) arc (10:-45:.5)-- (-1,1);
\fill[blue!50] (1,-1) ++ (135:.5) arc (135:220:.5)-- (1,-1);
\draw (1,-1) ++(.1, .1)--++(-.2, -.2) ++(.2, 0) -- ++(-.2,.2); \draw (-1,1) ++(.1, .1)--++(-.2, -.2) ++(.2, 0) -- ++(-.2,.2);
\draw (1,-1) ++ (0,.5) node {$B$};
\draw (-1,1) ++ (0,.5) node {$A$};
\end{tikzpicture}

 \caption{Figure pour cette partie. Un exemple de paire d'angles alternes-internes est indiquée en rouge et bleu.}
\end{figure}
\begin{df}Sur une telle figure, il y a quatre angles de sommet $A$, et quatre angles de sommet $B$. 
On dit qu'un angle de sommet $A$ et un angle de sommet $B$ de cette figure sont \emph{alternes-internes} s'ils 
vérifient les deux propriétés suivantes : 
\begin{itemize}
\item Les deux angles sont de part et d'autre de $(\Delta)$.
\item Les deux angles sont tous les deux à l'intérieur de la zone située entre $(d)$ et $(d')$. 
\end{itemize}
\end{df}
\begin{rmq}
On aurait pu aussi s'intéresser aux angles \emph{alternes-externes} qui sont définis comme ci-dessus mais en 
remplaçant la seconde propriété par « Les deux angles sont tous les deux à l'extérieur de la zone située entre $(d)$ et $(d')$. ».



\begin{figure}[H]\center
\begin{tikzpicture}
\draw (-1, 1) -- ++(10: 3)-- ++(10: -6); 
\draw (1, -1) -- ++(40: 3)-- ++(40:-6);
\draw (-2,2) -- (2,-2); 
\draw (1,1.8) node {$(d)$}; 
\draw (3.5,1.3) node {$(d')$};
\draw (1.5,-2.3) node {$(\Delta)$};
\fill[red!50] (-1,1) ++ (10:-.5) arc (10:-45:-.5)-- (-1,1);
\fill[blue!50] (1,-1) ++ (135:-.5) arc (135:220:-.5)-- (1,-1);
\draw (1,-1) ++(.1, .1)--++(-.2, -.2) ++(.2, 0) -- ++(-.2,.2); \draw (-1,1) ++(.1, .1)--++(-.2, -.2) ++(.2, 0) -- ++(-.2,.2);
\draw (1,-1) ++ (0,.5) node {$B$};
\draw (-1,1) ++ (0,.5) node {$A$};
\end{tikzpicture}
\caption{ Un exemple de paire d'angles alternes-externes est indiquée en rouge et bleu.}
\end{figure}

\end{rmq}

\subsection{Propriété des angles alternes-internes}

Dans la sous-partie précédente, nous avons défini les angles alternes-internes, ceux-ci sont reliés 
fortement à la notion de parallélisme. 
On a le 
théorème suivant. 

\begin{theo}
On reprend les notations de la sous-partie précédente. Alors : 
\begin{enumerate}
\item Si les deux droites $(d)$ et $(d')$ sont parallèles, alors deux angles alternes-internes ont la même mesure.
\item Si deux angles alternes-internes ont la même mesure, alors les droites $(d)$ et $(d')$ sont parallèles.  
\end{enumerate}
\end{theo}
\begin{rmq}
La propriété ci-dessus est également vraie avec les angles alternes-externes à la place des angles alternes-internes. 
\end{rmq}


\begin{figure}[H]\center
\begin{tikzpicture}
\draw (-1, 1) -- ++(10: 3)-- ++(10: -6); 
\draw (1, -1) -- ++(10: 3)-- ++(10:-6);
\draw (-2,2) -- (2,-2); 
\draw (1,1.8) node {$(d)$}; 
\draw (3.5,-1) node {$(d')$};
\draw (1.5,-2.3) node {$(\Delta)$};
\fill[red!50] (-1,1) ++ (10:.5) arc (10:-45:.5)-- (-1,1);
\fill[red!50] (1,-1) ++ (135:.5) arc (135:190:.5)-- (1,-1);
\draw (1,-1) ++(.1, .1)--++(-.2, -.2) ++(.2, 0) -- ++(-.2,.2);
\draw (-1,1)++(.1, .1)--++(-.2, -.2) ++(.2, 0) -- ++(-.2,.2);
\draw (1,-1) ++ (0,.5) node {$B$};
\draw (-1,1) ++ (0,.5) node {$A$};
\end{tikzpicture}

 \caption{Quand les droites $(d)$ et $(d')$ sont parallèles, les deux angles alternes-internes indiqués en rouge ont la même mesure. Réciproquement, si les deux angles rouges ont la même mesure, les droites sont parallèles.}
\end{figure}
\subsection{Démonstration}
Dans cette sous-partie, on démontre le résultat précédent à partir de propriétés déjà connues du cours (c'est 
le cœur de la géométrie). On va utiliser les propriétés suivantes : \begin{enumerate}
\item Si $I$ est le milieu de $[AB]$, $A$ et $B$ sont symétriques par rapport à $I$. 
\item Le symétrique d'une droite $(d)$ par rapport à un point est une droite parallèle à $(d)$ . 
\item Par un point, il ne passe qu'une seule parallèle à une droite donnée. 
\end{enumerate}
\begin{figure}[H]\center
\begin{tikzpicture}
\draw (-1, 1) -- ++(10: 3)-- ++(10: -6); 
\draw (1, -1) -- ++(10: 3)-- ++(10:-6);
\draw (-2,2) -- (2,-2); 
\draw (1,1.8) node {$(d)$}; 
\draw (3.5,-1) node {$(d')$};
\draw (1.5,-2.3) node {$(\Delta)$};
\fill[red!50] (-1,1) ++ (10:.5) arc (10:-45:.5)-- (-1,1);
\fill[red!50] (1,-1) ++ (135:.5) arc (135:190:.5)-- (1,-1);
\draw (1,-1) ++(.1, .1)--++(-.2, -.2) ++(.2, 0) -- ++(-.2,.2); \draw (-1,1) ++(.1, .1)--++(-.2, -.2) ++(.2, 0) -- ++(-.2,.2); 
\draw (0,0) ++(.1, .1)--++(-.2, -.2) ++(.2, 0) -- ++(-.2,.2);
\draw (1,-1) ++ (0,.5) node {$B$};
\draw (-1,1) ++ (0,.5) node {$A$};
\draw (0,0) ++ (0,.5) node {$I$};

\end{tikzpicture}

 \caption{Figure pour la démonstration. $I$ est le milieu de $[AB]$, on montre que c'est un centre
 de symétrie de la figure.}
\end{figure}
Commençons par démontrer le premier point du théorème. On suppose donc les droites $(d)$ et $(d')$ parallèles. 
On note $I$ le milieu de $[AB]$, ce qui, d'après les rappels ci-dessus, entraîne que $B$ est le symétrique de $A$
par rapport à $I$. En utilisant maintenant le second rappel, on sait que le symétrique de $(d)$ par rapport à 
$I$ est une droite parallèle à $(d)$. Comme $(d)$ passe par $A$, son symétrique passe par $B$ (qui est le symétrique
de $A$). Le symétrique de $(d)$ par rapport à $A$ est donc une droite parallèle à $(d)$ qui passe par $B$. D'après le 
troisième rappel, il n'y a qu'une seule droite vérifiant cela, c'est la droite $(d')$. 
De plus, $I$ appartient à $(\Delta)$, donc $(\Delta)$ est son propre symétrique par rapport à $I$. 
À ce stade, on a démontré : 
\begin{center}
\emph{Le point $I$ est un centre de symétrie de la figure formée par $(d)$, $(d')$ et $(\Delta)$.}
\end{center}
On peut alors conclure que les deux angles alternes-internes en rouge sur la figure sont symétriques l'un de
l'autre par rapport à $I$. Ils ont donc la même mesure. C'est le premier point du théorème. 

\textbf{Plus difficile à comprendre : } Une fois le premier point du théorème établi, le second point en est une conséquence directe car la valeur 
de l'angle rouge en $B$ détermine une unique droite $(d')$ (et d'après le premier point du théorème, cette droite est
la parallèle à $(d)$ passant par $B$.)


\section{Angles d'un triangle}
\subsection{Somme des angles d'un triangle}
Dans cette partie, on énonce une propriété majeure des triangles, que l'on avait déjà constatée en 6e avec la plupart des triangles particuliers. 
\begin{theo}
La somme des trois angles d'un triangle est égale à un angle plat (c'est-à-dire à $180^o$). 

\begin{figure}[H]\center
\begin{tikzpicture}
\draw (0,0) -- (4,3) -- (1, 5)--(0,0); 
\draw (0, -.25) node {$A$} (4.25, 3) node {$B$} (1, 5.25) node {$C$};
\end{tikzpicture} \caption{On a $\widehat{CAB}+\widehat{ABC}+\widehat{BCA}= 180^o$.}
\end{figure}
\end{theo}

Énonçons un corollaire (conséquence immédiate) de ce résultat. 
\begin{cor}
Un triangle ne peut pas avoir plus d'un angle droit ou obtus.\footnote{
O cara piota mia che sì t'insusi\\ 
che, come veggion le terrene menti \\
non capere in trïangol due ottusi, \\ 
così vedi (...)\\

« Chère souche de mon lignage, qui vas si haut \\ 
que, de même que voient les esprits terrestres\\
que deux 
obtus ne tiennent pas dans un triangle, \\ 
de même tu vois (...) »\flushright Dante, La Commedia, LXXXIV, v.13-15
}
\end{cor}
\begin{proof}
La somme de deux angles droits ou obtus dépasse déjà $90^o+90^o=180^o.$ Donc la somme des trois angles 
du triangle serait alors plus grande que $180^o$, ce qui contredirait le résultat précédent.  
\end{proof}

\subsection{Démonstration}

On obtient la propriété de la somme des angles du triangle à partir de celle des angles alternes-internes. 
\begin{figure}[H]\center
\begin{tikzpicture}
\draw[dashed] (-4, -2) -- (4, -2); 
\draw[dashed] (-4, 2) -- (4, 2); 
\draw (-3, -2) -- (2, -2); 
\draw (0,2) -- (-3,-2);
\draw (0,2) -- (2,-2);
\draw (-3, -2.3) node {$B$} (2, -2.3) node {$C$} (0, 2.3) node {$A$};
\fill[red!60] (0,2) -- (-.7,2) arc (180: 232:.7);
\fill[red!60] (-3,-2) -- (-2.3,-2) arc (0: 52:.7);
\fill[blue!60] (0,2) -- (.7,2) arc (0: -63:.7);
\fill[green!60] (0,2) -- ++(-63:.7) arc (-63: -128:.7);
\fill[blue!60] (2,-2) -- (1.3,-2) arc (180:117:.7);
\draw (-4, 2.3) node {$x$} (4, 2.3) node {$y$};
\end{tikzpicture}\caption{Figure pour la démonstration : on montre d'abord l'égalité des deux angles rouges, puis celle des deux angles bleus. Les trois angles bleu, rouge, et vert font alors un angle plat.}
\end{figure}

On trace la droite $(d)$ parallèle à $(BC)$ passant par $A$, comme sur la figure. 
Les angles $\widehat{xAB}$ et $\widehat{ABC}$ sont alternes-internes et les droites $(d)$ et $(BC)$ sont 
parallèles, donc, d'après la propriété des angles alternes-internes, $\widehat{xAB}$ et $\widehat{ABC}$
ont la même mesure. 

De la même manière, les angles $\widehat{yAC}$ et $\widehat{ACB}$ sont alternes-internes (entre les droites parallèles 
$(d)$ et $(BC)$) donc ont la même 
mesure. 

Enfin, l'angle $\widehat{xAB}$, l'angle $\widehat{BAC}$ et l'angle $\widehat{yAC}$ forment un angle plat, donc on a

\[180^o = \widehat{xAB} + \widehat{BAC}+ \widehat{yAC}, \]
et, comme $\widehat{xAB}= \widehat{ABC}$ et $\widehat{yAC}=\widehat{ACB}$, on a bien 
 \[180^o=\widehat{ABC} + \widehat{BAC}+ \widehat{ACB}.\]

\subsection{Triangles particuliers}

\subsubsection{Triangle isocèle}
\begin{prop}
On considère un triangle $ABC$ isocèle en $B$. Alors, les angles de la base ont la même mesure : 
\[ \widehat{BAC}= \widehat{BCA}.\]
\end{prop}
\begin{figure}[H]\center
\begin{tikzpicture}
\fill[red!50] (-2,0) -- (-1.2,0) arc (0: 69:.8);
\fill[red!50] (2,0) -- (1.2,0) arc (180:111:.8);
\draw (-2,0) -- (2,0) -- (0,5) -- (-2,0);
\draw (-2.25, -.25) node {$A$} (2.25, -.25) node {$C$} (0.25, 5.25) node {$B$};
\draw[dashed] (0,-1) --(0,6);
\end{tikzpicture}\caption{Les deux angles à la base d'un triangle isocèle ont la même mesure. }
\end{figure}
\begin{proof}
On considère la droite $(d)$ médiatrice du segment $[AC]$. Comme le point $B$ est à la même distance de $A$ et 
de $C$, il appartient à la médiatrice, donc $(d)$ passe par le point $B$. Les points $A$ et $C$ sont symétriques l'un
de l'autre par rapport à $(d)$, et le point $B$ appartient à $(d)$ donc est son propre symétrique. Le triangle $ABC$ 
admet donc $(d)$ comme axe de symétrie. Les deux angles de la base sont alors symétriques par rapport à $(d)$, 
donc ont la même mesure. 
\end{proof}
\subsubsection{Triangle équilatéral}

\begin{prop}
Les trois angles d'un triangle équilatéral mesurent $60^o$. 
\end{prop}
\begin{figure}[H]\center
\begin{tikzpicture}
\fill[red!50] (90:1.5) -- ++(-60:0.7) arc (-60: -120:.7);
\fill[red!50] (210:1.5) -- ++(0:0.7) arc (0: 60:.7);
\fill[red!50] (-30:1.5) -- ++(180:0.7) arc (180: 120:.7);
\draw (90:1.5) -- (210:1.5) -- (-30:1.5) -- (90:1.5);
\end{tikzpicture}\caption{Les trois angles de $60^o$ d'un triangle équilatéral}
\end{figure}
\begin{proof}
Comme un triangle équilatéral est isocèle en chacun de ses sommets, 
on peut utiliser la propriété précédente, et les trois angles sont donc égaux. 
Comme la somme des trois angles d'un triangle est de $180^o$, et que ces trois angles sont égaux, 
chaque angle mesure $\frac{180^o}3 = 60^o$. 
\end{proof}

\subsubsection{Triangle rectangle}

\begin{prop}
Dans un triangle $ABC$ rectangle en $B$, les deux angles non-droits sont complémentaires (leur somme fait un angle droit). 
\end{prop}\begin{figure}[H]\center
\begin{tikzpicture}
\draw (0,0) -- (3,0) -- (0,4)--(0,0);
\fill[blue!60] (0,4) -- ++(270:.7) arc (270 : 307:.7);
\fill[red!60] (3,0) -- ++(180:.7) arc (180 : 127:.7);
\end{tikzpicture}
\caption{Les angles bleu et rouge sont complémentaires.}
\end{figure}


\begin{proof}
La somme des trois angles d'un triangle fait $180^o$, et il y a un angle de $90^o$. Les deux angles restants 
ont donc pour somme $180^o - 90^o= 90^o$.
\end{proof}

\subsubsection{Triangle rectangle isocèle}
\begin{prop}
Dans un triangle $ABC$ rectangle en $B$, les deux angles non-droits mesurent $45^o$. 
\end{prop}\begin{figure}[H]\center
\begin{tikzpicture}
\draw (0,0) -- (3,0) -- (0,3)--(0,0);
\fill[red!60] (0,3) -- ++(270:.7) arc (270 : 315:.7);
\fill[red!60] (3,0) -- ++(180:.7) arc (180 : 135:.7);
\end{tikzpicture}
\caption{Les deux angles rouges mesurent $45^o$.}
\end{figure}
\begin{proof} 
Comme le triangle est rectangle, ces deux angles sont complémentaires, donc la somme de leurs mesures est $90^o$. 
Comme le triangle est isocèle, ces deux angles ont la 
même mesure. Ils mesurent donc chacun $90^o\div 2 = 45^o$. 
\end{proof}



\section{Angles opposés par le sommet}
On considère deux droites sécantes en un point $O$ comme sur la figure suivante. 
On appelle angle opposés par le sommet, deux angles (parmi les quatre formés par les deux droites), qui sont symétriques l'un de l'autre par rapport au point $O$. 
\begin{figure}[H]\center
\begin{tikzpicture}
\fill[color=red!50](30:.7) arc (30:80:.7)--(0,0);
\fill[color=red!50](30:-.7) arc (30:80:-.7)--(0,0);
\draw (30:3) -- (30:-3);
\draw (80:3) -- (80:-3);
\draw (.25, -.25) node {$O$};
\draw (30:3.25) node{$x$};
\draw (30:-3.25) node{$x'$};
\draw (80:3.25) node{$y$};
\draw (80:-3.25) node{$y'$};
\end{tikzpicture}
\caption{Deux angles opposés par le sommet, donc de même mesure}
\end{figure}
Comme deux angles opposés par le sommet sont symétriques l'un de l'autre par rapport au sommet, on a la 
propriété suivante. 
\begin{prop}
Deux angles opposés par le sommet ont la même mesure.
\end{prop}

\section{Angle au centre d'un cercle}
\subsection{Énoncé}
On donne ici une application intéressante de la somme des angles d'un triangle. 

\begin{figure}[H]
\center
\begin{tikzpicture}
\draw (0,0) circle(2); 
\fill[red!50] (140:.5) arc (140:270:.5) -- (0,0);
\fill [blue!50] (30:2) -- ++ (175:.5) arc(175:240:.5);
\draw (30: 2) -- (140:2) -- (0,0) -- (270:2)-- (30:2);
\draw (0,0) ++ (0.25,.25) node {$O$};
\draw (30:2) ++ (.25,0.25) node {$A$};
\draw (140:2) ++ (-0.25,.25) node {$B$};
\draw (270:2) ++ (0,-.25) node {$C$};
\end{tikzpicture}
\caption{L'angle rouge est égal au double de l'angle bleu}
\end{figure}

\begin{theo}
Soient $A$,$B$ et $C$ trois points d'un même cercle de centre $O$. Alors l'angle au centre\footnote{Attention : il faut choisir de mesurer l'angle $\widehat{BOC}$ de telle sorte qu'il intercepte le même arc de cercle de $B$ à $C$ que l'angle $\widehat{BAC}$.} $\widehat{BOC}$
est deux fois plus grand que l'angle $\widehat{BAC}$ : \[ \widehat{BOC} = 2\times \widehat{BAC}.\]
\end{theo}
\subsection{Démonstration}

On reprend les notations de la figure précédente. Comme $A$, $B$ et $C$ sont sur le même cercle de centre $O$, 
ils sont à la même distance de $O$, donc les triangles $AOB$ et $AOC$ sont isocèles en $O$. 


\begin{figure}[H]
\center
\begin{tikzpicture}
\draw (0,0) circle(2); 
\fill[green!50] (140:2)-- ++(-6:.5) arc (-6:-40:.5) -- (140:2);
\fill[green!50] (30:2)-- ++(176:.5) arc (176:210:.5) -- (30:2);
\fill[purple!50] (30:2)-- ++(210:.5) arc (210:240:.5) -- (30:2);
\fill[purple!50] (270:2)-- ++(60:.5) arc (60:90:.5) -- (270:2);
\draw (30: 2) -- (140:2) -- (0,0) -- (270:2)-- (30:2);
\draw (0,0) -- (30:2);
\draw (0,0) ++ (-0.25,-.25) node {$O$};
\draw (30:2) ++ (.25,0.25) node {$A$};
\draw (140:2) ++ (-0.25,.25) node {$B$};
\draw (270:2) ++ (0,-.25) node {$C$};
\end{tikzpicture}
\caption{Les deux triangles isocèles considérés dans la démonstration. Les angles de mesure $b$ sont en vert, ceux de mesure $c$ sont en violet.}
\end{figure}

Notons $b$ la mesure de l'angle $\widehat{OBA}$ et $c$ la mesure de l'angle $\widehat{OCA}$. \begin{itemize}
\item Comme $AOB$ est isocèle en $O$, les deux angles à sa base sont égaux, donc $\widehat{OAB}=b$. 
Comme la somme des angles du triangle $AOB$ est égale à $180^o$, on a $b + b + \widehat{AOB} =180^o $, donc 
\[\widehat{AOB} =180^o - 2\times b.\]
\item De même, comme $AOC$ est isocèle en $O$, les deux angles à sa base sont égaux, donc $\widehat{OAC}=c$.
Comme la somme des angles du triangle $AOC$ est égale  à $180^o$, on a $c + c + \widehat{AOC} =180^o $, donc 
\[\widehat{AOC} =180^o - 2\times c.\]
\item L'angle $\widehat{BAC}$ est la somme des angles $\widehat{BAO}$ et $\widehat{OAC}$, donc, d'après les deux points précédents, $\widehat{BAC}= b+c$.
\item Enfin, la somme des angles $\widehat{AOB}$, $\widehat{AOC}$ et $\widehat{BOC}$ correspond à un tour complet, donc à $360^o$. On a donc \[360^o = \underbrace{(180^o - 2\times b)}_{\widehat{AOB}} + 
\underbrace{(180^o - 2\times c)}_{\widehat{AOC}} + \widehat{BOC},\] 
c'est-à-dire \[ 360^o = 360^o + \widehat{BOC} - 2 \times b - 2 \times c.\]
On en déduit enfin \[\widehat{BOC} = 2 \times b + 2 \times c = 2\times \underbrace{(b + c)}_{\widehat{BAC}}.\]
\end{itemize}



\subsection{Corollaires importants : lien entre cercles et triangles rectangles}
On énonce deux corollaires (conséquences directes) du théorème précédent. \footnote{« (ni) si du demi-cercle on peut faire un triangle qui n'a pas d'angle droit »/ \\
"o se del mezzo cerchio far si puote\\ trïangol sì ch'un retto non avesse."\\ 
(Dante, Commedia, LXXX, v101-102) }

\begin{cor}
Étant donné un cercle de centre $O$ et de diamètre $[BC]$, n'importe quel point $A$ du cercle 
forme avec $B$ et $C$ un triangle rectangle en $A$. 
\end{cor}

\begin{figure}[H]\center
\begin{tikzpicture}
\draw (0,0) circle(2);
\draw (30: 2)--(210:2) -- (100:2)--(30:2);  
\draw (30:2.25) node{$B$} (210:2.25) node {$C$} (100:2.25) node {$A$};
\fill[red!50] (30:.5) arc (30 : -150:.5);
\fill[blue!50] (100:2) -- ++(-26:.5)arc (-26:-116:.5);
\draw[red] (-70:.8) node {$180^o$};
\end{tikzpicture}\caption{L'angle rouge est plat, donc le bleu mesure la moitié d'un angle plat, donc est droit.}
\end{figure}
\begin{proof}
En effet, on est dans les mêmes conditions que dans le théorème précédent, mais avec $B$, $O$ et $C$ alignés, donc 
$\widehat{BOC}= 180^o$. On a alors $\widehat{BAC}= 180^o\div2 = 90^o$, ce qui montre bien que le triangle 
$ABC$ est rectangle en $A$. 
\end{proof}
On peut même dire un tout petit peu plus.
\begin{cor}
Le cercle circonscrit à un triangle rectangle a son centre au milieu de l'hypoténuse. 

Par conséquent, le segment qui relie le milieu de l'hypoténuse à l'angle droit sépare un triangle rectangle en deux triangles isocèles.
\end{cor}\begin{figure}[H]\center
\begin{tikzpicture}
\draw (0,0) -- (4,0) -- (0,2)--(0,0);
\draw[red](0,0)-- (2,1);
\end{tikzpicture}\caption{Un triangle rectangle, et le segment qui le sépare en deux triangles isocèles.}
\end{figure}


\begin{proof}
Notons $O$ le centre du cercle circonscrit d'un triangle rectangle $ABC$ en $A$.
Toujours d'après le théorème sur l'angle au centre, l'angle $\widehat{BOC}$ est plat, donc $O$ appartient à $[BC]$. 
De plus, comme $B$ et $C$ sont sur un cercle de centre $O$, $OB=OC$, donc $O$ est le milieu de l'hypoténuse $[BC]$. 
\end{proof}
\subsection{Un dernier résultat annexe : droite d'Euler et points particuliers du triangle rectangle}

\begin{figure}[H]\center
\begin{tikzpicture}
\draw (0,0) -- (3,0) -- (0,4)--(0,0);
\draw[red](-0.225,-0.3)-- (2.25,3);
\draw (1.5,2) ++(.1, .1)--++(-.2, -.2) ++(.2, 0) -- ++(-.2,.2); \draw (0,0) ++(.1, .1)--++(-.2, -.2) ++(.2, 0) -- ++(-.2,.2) ;
\draw(1, 1.333333333) ++(.1, .1)--++(-.2, -.2) ++(.2, 0) -- ++(-.2,.2);
\draw (1.75, 2) node {$O$} (0,4.25) node {$B$} (-0.25, .25) node {$A$} (1.25, 1.1) node {$G$} (3.25, 0) node {$C$};
\end{tikzpicture}
\caption{L'orthocentre du triangle rectangle est en $A$, et le centre de son cercle circonscrit en $O$, milieu de l'hypoténuse. Le centre de gravité est donc bien sur la droite $(AO)$, médiane issue de $A$. }
\end{figure}
Étant donné un triangle $ABC$  rectangle en $A$, le centre du cercle circonscrit est donc le milieu 
$O$ de $[BC]$. 

Les hauteurs issues de $B$ et de $C$ sont respectivement les droites $(BA)$ et $(CA)$, donc l'orthocentre est en fait 
le sommet de l'angle droit $A$. 

La droite $(AO)$ passe donc par l'orthocentre du triangle et par le centre de son cercle circonscrit, et 
c'est une médiane du triangle (elle joint le sommet $A$ au milieu de son côté opposé). 
Le centre de gravité est situé sur cette droite (il est à l'intersection des trois médianes). On retrouve ici 
le résultat évoqué (mais non démontré) en chapitre $3$ : le centre de gravité, le centre du cercle circonscrit 
et l'orthocentre sont sur la même droite. Plus précisément, on a démontré le résultat suivant : 
\begin{prop}
La droite d'Euler d'un triangle rectangle est la médiane issue de son angle droit.
\end{prop}







	\end{document}
	


