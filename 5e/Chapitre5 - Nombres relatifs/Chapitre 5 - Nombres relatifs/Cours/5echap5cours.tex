  	\documentclass[12 pt]{article}

	\usepackage[frenchb]{babel}
	\usepackage[utf8]{inputenc}  
	\usepackage[T1]{fontenc}
	\usepackage{amssymb}
	\usepackage[mathscr]{euscript}
	\usepackage{stmaryrd}
	\usepackage{amsmath}
	\usepackage{tikz}
	\usepackage[all,cmtip]{xy}
	\usepackage{amsthm}
	\usepackage{varioref}
	\usepackage{geometry}
	\geometry{a4paper}
	\usepackage{lmodern}
	\usepackage{float}
	\usepackage{hyperref}
	\usepackage{array}
	 \usepackage{fancyhdr}

	\pagestyle{fancy}
	\theoremstyle{plain}
	\fancyfoot[C]{\thepage} 
	\fancyhead[L]{Nombres relatifs}
	\fancyhead[R]{2021-2022}
	\newcounter{n}
	\numberwithin{n}{section}
	\newtheorem{theo}{Théorème}
	\labelformat{theo}{théorème}
	\newtheorem*{cor}{Corollaire}
	\newtheorem*{prop}{Propriété}
	\newtheorem{lm}{Lemme}
	\labelformat{lm}{lemme~#1}
	\newtheorem{hyp}[n]{Hypothèse}
	
	{\theoremstyle{definition}
	\newtheorem*{df}{Définition}
    \newtheorem*{rmq}{Remarque}
	\newtheorem*{nt}{Notation}
	\newtheorem*{voc}{Vocabulaire}
	\newtheorem*{ex}{Exemple}
	\newtheorem*{exs}{Exemples}
	\newtheorem*{formule}{Formule}
	\newtheorem*{formules}{Formules}
	}

	\renewcommand\epsilon{\varepsilon}
	\renewcommand\phi{\varphi}
	\newcommand\R{\mathbb{R}}
	\newcommand\s{\mathbb{S}}
	
	
	
	\title{Cours Chapitre 5}
	\date{}
	\begin{document}

\begin{center}{\Large Chapitre 5 - Nombres relatifs}\\ 
 \end{center}

\section{Définition}
\begin{df}
Les \emph{entiers positifs} sont les entiers obtenus en comptant à partir de $1$ : $1$, $2$, $3$, $\ldots$. 
On peut les faire précéder d'un signe $+$ ou non sans que cela ne change rien. 

Les \emph{entiers négatifs} sont les entiers obtenus en faisant précéder les entiers positifs (non signés) d'un signe $-$. Ce sont donc $-1$, $-2$, $-3$, $\ldots$. 

Les \emph{entiers relatifs} sont $0$, les entiers positifs, et les entiers négatifs. 

Plus généralement, un nombre relatif est obtenu en plaçant un signe $+$ ou $-$ devant un nombre. Un tel nombre est 
positif si le signe est un $+$ (auquel cas on peut choisir de ne pas le mettre) et négatif si le signe est un $-$.

Attention, $0 = -0 = + 0$. 
\end{df}
\begin{exs}
Les nombres suivants sont des entiers relatifs : $2,34$; $\frac{3}4$; $-4$; $+4,3$; $-\frac45$; $- 2,3$. Dans la liste, $-4$, $-\frac45$ et $-2,3$ sont négatifs, et les autres nombres sont positifs. $-4$ est un entier relatif, mais 
$-2,3$ n'est pas un entier relatif (car $2,3$ n'est pas un entier). 
\end{exs}

\begin{df}
L'\emph{opposé} d'un nombre relatif est le nombre relatif obtenu en changeant son signe ($+$ ou $-$) en le signe 
contraire. 
\end{df}
\begin{ex}
L'opposé de $-4,5$ est $+4,5$. L'opposé de $3$ est $-3$. L'opposé de $O$ est $O$.
\end{ex}
\begin{prop}
\begin{enumerate}
\item L'opposé d'un nombre positif est un nombre négatif. 
\item L'opposé d'un nombre négatif est un nombre positif. 
\item Le nombre nul $0$ est le seul nombre à être son propre opposé. 
\item L'opposé de l'opposé d'un nombre est lui-même. (En changeant deux fois le signe, on revient au nombre 
d'origine).
\end{enumerate}
\end{prop}



\section{Opérations}

\subsection{Soustraction de nombres positifs}

Pour soustraire deux nombres positifs, on peut se retrouver face à deux cas : 
\begin{itemize}
\item Si le premier est plus grand, on sait déjà faire la soustraction, et le résultat est un nombre positif. 
\item Si le second est plus grand, on ne pouvait pas faire la soustraction jusqu'ici. On pourra désormais la calculer
en calculant la différence du second terme par le premier, et en mettant un signe $-$ devant. Le résultat est alors négatif. 
\end{itemize}

\begin{exs}
$4 - 5 = - (5-4) = -1$

$\frac14 - \frac34 = - (\frac34 - \frac14) = - \frac24 = -\frac12$. 


\end{exs}

\begin{formule}
On peut résumer ce qui précède ainsi, en notant $a$ et $b$ deux nombres positifs : 

\[  a - b = \begin{cases}
+ (a-b) & \text{si $a >b$}\\
- (b-a) & \text{si $a <b$}
\end{cases}\]
\end{formule}

\subsection{Addition de nombres relatifs}

On cherche maintenant à additionner des nombres relatifs entre eux, donc à donner un sens à des calculs comme 
$(+2) +  (-3)$, $-1+ (-\frac12)$ ou $(-4) + 5,3$. 
\subsubsection{Addition de deux nombres relatifs}

On veut additionner deux nombres relatifs $a$ et $b$. Il y a trois cas possibles : 

\begin{itemize}
\item Si les deux nombres sont positifs, on se contente de les additionner sans les signes, et le résultat est positif. 

\item Si les deux nombres sont négatifs, on les additionne sans les signes, puis on fait précéder le résultat du signe $-$. On obtient un nombre négatif. 

\item Si l'un des nombres est positif et l'autre négatif, on fait la différence entre le nombre positif et le nombre
négatif sans son signe. On peut alors obtenir un nombre positif ou négatif suivant le signe de la différence. 

\end{itemize}
\begin{exs}
\[(+ 2) + (+3, 5) = +(2+3,5) = + 5,5 = 5,5\]
\[ (- 4 )+ (-3) = - (4+3) = -7\]
\[ (+ 5) + (-2, 4) = 5 - 2,4 = 2,6\]
\[ (-2,4) + 5 =  5 - 2,4 = 2,6\]
\[ (-4,5) + 1 = 1 - 4,5 = - 3,5\]

\end{exs}

On peut résumer ce qui précède avec les formules suivantes, où $a$ et $b$ sont des nombres positifs : 
\begin{formules}
\[ (+ a) + (+b) = + (a+b)\]
\[ (- a) + (-b) = - (a+b) \]
\[ (+a) + (-b) = a-b\]
\[ (-a) + (+b) = b-a\]

\end{formules}

\subsubsection{Addition de plusieurs nombres relatifs}
L'addition ne dépend pas de l'ordre des termes. 
\begin{formule}
\[a+b = b+a\]
\end{formule}

Pour additionner plusieurs nombres relatifs, il suffit de faire la somme de toutes les valeurs 
positives, et de retrancher la somme de toutes les valeurs négatives.
\begin{ex} \[ (+2) +  (-4) + (+5) + (-7) + 3 = (2 + 5 + 3) - (4+ 7) = 10 - 11 = -1.\]
\end{ex}

\subsection{Soustraction de nombres relatifs}



\subsection{Soustraction de deux nombres relatifs}



\begin{df}Si $a$ et $b$ sont deux nombres relatifs,
la \emph{différence} entre $a$ et $b$ est la somme de $a$ et de l'opposé de $b$. 
On la note $a - b$. L'opération qui tranforme les deux nombres $a$ et $b$ en
leur différence $a - b$ est la \emph{soustraction}.
\end{df}

\begin{ex}
La différence entre $3$ et $5$ est la somme de $3$ et $-5$, donc $3- 5 = 3 + (-5)$, ce qui est rassurant, 
puisque c'est ce que l'on avait vu dans la partie sur la soustraction des nombres positifs. 

La différence entre $3$ et $-4$ est la somme de $3$ et de $+4$ donc $ 3 - (- 4) = 3 + 4 = 7$. 

La différence entre $-5$ et $2$ est la somme de $-5$ et de $-2$ donc $ - 5 - 2 = -5 + (-2) = - 7$. 

La différence entre $-2$ et $-6$ est la somme de $-2$ et de $+6$ donc $-2 - (-6) = -2+6 = 4$. 
\end{ex}

\begin{formule}
On peut résumer les règles ci-dessus en un certain nombre de formules : 
\[ (+a) - (+b) = a - b \]
\[ (+a) - (-b) = a + b \]
\[ (-a) - (+b) = - a - b = - (a + b) \]
\[ (-a) - (-b) =  - a + b = b - a \]
Celles-ci se résument aux deux principes suivants : 
\[ - (+b) = + (-b) = - b\]
\[ - ( - b) = + b\]
\end{formule}
\begin{rmq}
Lorsque deux signes ($-$ ou $+$) se touchent immédiatement, on peut les simplifier en un seul signe, positif si les deux signes étaient égaux, négatif s'il y avait un $+$ et un $-$. 
\end{rmq}

\subsection{Soustractions (et additions) multiples}

\begin{df}
Une \emph{somme algébrique} est une suite d'additions et de soustractions de nombres relatifs, avec 
éventuellement des parenthèses. 
\end{df}

\begin{exs}Voici deux exemples de sommes algébriques : 
\[ 2 + (-3 ) - (4 + (-5) + 3) + (4 - (-6) ) \]

\[ 3 + 4 - (-5) + (3 + (-4) ) - (4 - ((-3) + 5))\]
\end{exs}

Pour calculer une telle somme, on peut se débarrasser des parenthèses en utilisant les deux règles suivantes : 
\begin{itemize}
\item si une parenthèse est précédée d'un signe $+$ on peut l'oublier.
\item si une parenthèse est précédée d'un signe $-$, on peut la retirer, mais après avoir changé le signe de tous 
les termes à l'intérieur de la parenthèse. 
\end{itemize}
\begin{exs}
\begin{eqnarray*}
 2 + (-3 ) - (4 + (-5) + 3) + (4 - (-6) )  &=& 2 - 3 - (4 - 5 + 3) + (4 + 6) \\
 &=& 2 - 3 - 4 + 5 + 3 + 4 + 6
\end{eqnarray*}

\begin{eqnarray*}
  3 + 4 - (-5) + (3 + (-4) ) - (4 - ((-3) + 5))  &=& 3 + 4 + 5 + (3 - 4) - ( 4 - (-3+5) ) \\
 &=& 3 + 4 + 5 + 3 - 4 - ( 4 + 3  - 5)   \\
  &=& 3 + 4 + 5 + 3 - 4 -  4 - 3 + 5 
\end{eqnarray*}
\end{exs}


Une fois arrivé à cette expression, on a simplement des additions et soustractions de nombres positifs, 
et on peut regrouper tous les termes qui sont additionnés et leur retirer la somme de tous les nombres qui sont 
soustraits. Pour reprendre les mêmes exemples:

\begin{eqnarray*}
 2 + (-3 ) - (4 + (-5) + 3) + (4 - (-6) ) 
 &=& 2 - 3 - 4 + 5 + 3 + 4 + 6\\
 &=& \underbrace{(2+ 5 + 3 + 4  +6)}_{=20} - \underbrace{(3+4)}_{=7} = 13.
\end{eqnarray*}
\begin{eqnarray*}
  3 + 4 - (-5) + (3 + (-4) ) - (4 - ((-3) + 5)) &= & 3 + 4 + 5 + 3 - 4 -  4 - 3 + 5 \\ 
  &=&\underbrace{(3+4+5+3+5)}_{=20}-\underbrace{(4+4+3)}_{=11}= 9
\end{eqnarray*}


\section{Comparaison et repérage}

\subsection{Comparaison de nombres relatifs}

\begin{df}
Pour comparer deux nombres relatifs, on utilise les règles suivantes : 
\begin{itemize}
\item pour deux nombres positifs, on sait déjà les comparer (cf. comparaison de nombres décimaux en $6^e$)
\item pour deux nombres négatifs, le plus grand est celui qui serait le plus petit sans les signes ($-2$ est plus grand 
que $-4$)
\item pour deux nombres de signes contraires, le nombre positif est toujours plus grand que le nombre négatif.
\end{itemize}
\end{df}

Par conséquent, pour ranger une série de nombres relatifs dans l'ordre croissant, on classe les 
nombres négatifs dans l'ordre décroissant des nombres non signés, puis on classe les 
nombres positifs
dans l'ordre croissant. Si on doit aussi ranger $0$, il est entre les nombres négatifs et les nombres positifs. 

\begin{ex}
On veut les nombres suivants dans l'ordre croissant :  $3$ ; $-2,1$ ; $4,2$  ; $-1$ ; $0$ ; $2,41$.

\begin{itemize}
\item on range les nombres positifs ($3$ ; $4,2$ ; $2,41$) dans l'ordre croissant : $2,41  < 3 < 4,2$. 
\item on range les nombres négatifs ($-2,1$; $-1$) dans l'ordre décroissant de leurs valeurs non signées :$ -2,1 < 
-1$.
\item on écrit les nombres négatifs ainsi triés, puis $0$ puisqu'il était à classer, puis les 
nombres positifs : \[ -2,1 < -1  < 0 < 2,41 < 3 < 4,2.\]
\end{itemize}
\end{ex}

\subsection{Repérage sur une droite graduée orientée}

Une des manières les plus naturelles de se représenter les nombres relatifs est d'utiliser une droite graduée. 
\begin{df}
Une droite graduée est une droite munie d'une origine (un point $O$), d'une orientation (soit de la droite vers la gauche, soit de la gauche vers la droite) et d'une graduation (une longueur 
de référence $\ell$). 

\begin{figure}[H]\center
\begin{tikzpicture}
\draw [->, >= latex] (-6, 0) -- (6, 0);
\draw (-5, -.2) -- (-5, .2);
\draw (-4, -.2) -- (-4, .2);
\draw(-3, -.2) -- (-3, .2);
\draw(-2, -.2) -- (-2, .2);
\draw (-1, -.2) -- (-1, .2);
\draw (5, -.2) -- (5, .2);
\draw (4, -.2) -- (4, .2);
\draw(3, -.2) -- (3, .2);
\draw(2, -.2) -- (2, .2);
\draw (1, -.2) -- (1, .2);
\draw (0, -.2) -- (0, .2);
\draw (0, .5) node {$O$};
\end{tikzpicture}
\caption{Une droite graduée orientée vers la droite.}
\end{figure}

Sur une telle droite, on associe à chaque point $M$ un nombre relatif appelé son \emph{abscisse} de la manière suivante :
\begin{itemize}
\item on mesure la longueur $OM$ entre l'origine et le point,
\item on regarde à combien de graduations correspond cette longueur, ce qui revient à calculer la division 
$OM\div \ell$. (Par exemple, $OM=1,4$ cm correspond à $10$ graduations si une graduation mesure $1,4$ mm, mais à $7$
graduations si une graduation mesure $2$ mm.)
\item on détermine le signe de l'abscisse en fonction de l'orientation de la droite : si $M$ est après l'origine (à droite si la droite va de gauche à droite comme c'est le plus courant, à gauche si elle va de droite à gauche), on 
met un signe $+$, si $M$ est avant l'origine, on met un signe $-$. 
\end{itemize}
\end{df}

\begin{figure}[H]\center
\begin{tikzpicture}
\draw [->, >= latex] (-6, 0) -- (6, 0);
\draw (-5, -.2) -- (-5, .2);
\draw (-4, -.2) -- (-4, .2);
\draw(-3, -.2) -- (-3, .2);
\draw(-2, -.2) -- (-2, .2);
\draw (-1, -.2) -- (-1, .2);
\draw (5, -.2) -- (5, .2);
\draw (4, -.2) -- (4, .2);
\draw(3, -.2) -- (3, .2);
\draw(2, -.2) -- (2, .2);
\draw (1, -.2) -- (1, .2);
\draw (0, -.2) -- (0, .2);
\draw (0, .5) node {$O$};
\fill (2.4, 0) circle (0.05); 
\draw (2.4, .5) node {$A$}; 
\draw (2.4, -.5) node {$+2,4$}; 
\fill (-4.6, 0) circle (0.05); 
\draw (-4.6, .5) node {$B$}; 
\draw (-4.6, -.5) node {$-4,6$};
\fill (5.5, 0) circle (0.05);  
\draw (5.5, .5) node {$C$}; 
\draw (5.5, -.5) node {$+5,5$};
\fill (4, 0) circle (0.05);  
\draw (4, .5) node {$D$}; 
\draw (4, -.5) node {$+4$}; 
\fill (-1.3333, 0) circle (0.05);  
\draw (-1.33333, .5) node {$E$}; 
\draw (-1.3333333, -.5) node {$-\frac43$}; 
\end{tikzpicture}
\caption{Cinq points d'une droite graduée orientée, dont les abscisses sont indiquées en-dessous.}
\end{figure}
Notons la propriété immédiate suivante : 
\begin{prop}
Étant donnés des points d'une droite graduée orientée, leur ordre sur la droite est l'ordre 
de leurs abscisses. 
\end{prop}
\begin{ex}
Sur la figure précédente, on voit que l'ordre des points $A$ à $E$ est $B$, $E$, $A$, $D$, $C$. 
Cela nous dit que les abscisses sont dans le même ordre, donc 
\[ - 4, 6 < -\frac43 < 2,4 < 4 < 5,5\]
\end{ex}


\subsection{Repérage dans le plan muni d'un repère}

\begin{df}
Un \emph{repère} du plan est donné par deux droites orientées de même origine. 
En pratique, on travaillera au collège dans des repères \emph{orthogonaux}, 
c'est-à-dire où les deux droites sont perpendiculaires. 

Habituellement\footnote{Comprendre : sans doute toujours dans votre expérience et vos exercices à venir.}, la première 
droite est horizontale et orientée de gauche à droite, et la seconde droite est verticale et orientée de bas en haut.
Ce qui changera plus couramment en revanche est la longueur des graduations. 
\end{df}
\begin{figure}[H]
\center
\begin{tikzpicture}
\draw[->] (-2, 0) -- (2, 0); 
\draw[->] (0,-2) -- (0, 2); 
\draw (-1.75, -.08) -- (-1.75, .08);
\draw (-1.5, -.08) -- (-1.5, .08);
\draw(-1.25, -.08) -- (-1.25, .08);
\draw(-1, -.08) -- (-1, .08);
\draw (-.75, -.08) -- (-.75, .08);
\draw (-.5, -.08) -- (-.5, .08);
\draw (-.25, -.08) -- (-.25, .08);
\draw (1.75, -.08) -- (1.75, .08);
\draw (1.5, -.08) -- (1.5, .08);
\draw(1.25, -.08) -- (1.25, .08);
\draw(1, -.08) -- (1, .08);
\draw (.75, -.08) -- (.75, .08);
\draw (.5, -.08) -- (.5, .08);
\draw (.25,-.08) -- (.25 , .08);

\draw (-.08, -1.75) -- (.08, -1.75);
\draw (-.08, -1.5) -- (.08, -1.5);
\draw(-.08, -1.25) -- (.08, -1.25);
\draw(-.08, -1) -- (.08, -1);
\draw (-.08, -.75) -- (.08, -.75);
\draw (-.08, -.5) -- (.08, -.5);
\draw (-.08, -.25) -- (.08, -.25);
\draw (-.08, 1.75) -- (.08, 1.75);
\draw (-.08, 1.5) -- (.08, 1.5);
\draw(-.08, 1.25) -- (.08, 1.25);
\draw(-.08, 1) -- (.08, 1);
\draw (-.08, .75) -- (.08, .75);
\draw (-.08, .5) -- (.08, .5);
\draw (-.08, .25) -- (.08, .25);
\end{tikzpicture}
\caption{Un repère orthogonal du plan}
\end{figure}  

\begin{df}
Étant donné un repère formé de deux droites $(d)$ et $(d')$ et un point $M$ du plan, on peut 
associer au point deux nombres relatifs appelés \emph{abscisse} de $M$ et \emph{ordonnée} de $M$ de la manière suivante : 
\begin{itemize}
\item on trace la parallèle à $(d')$ passant par $M$, et l'on note $A$ son intersection avec $(d)$, 
\item on trace la parallèle à $(d)$ passant par $M$ , et l'on note $B$ son intersection avec $(d')$, 
\item l'abscisse de $M$ est l'abscisse $x$ de $A$ sur la droite orientée $(d)$,
\item l'ordonnée de $M$ est l'abscisse $y$ de $B$ sur la droite orientée $(d')$.
\end{itemize}
On note alors $(x;y)$ les coordonnées du point $M$ dans le repère.
\end{df}

\begin{figure}[H]
\center
\begin{tikzpicture}
\draw[->,>= latex] (-4, 0) -- (4, 0); 
\draw[->, >= latex] (0,-4) -- (0, 4);
\draw(-3, -.2) -- (-3, .2);
\draw(-2, -.2) -- (-2, .2);
\draw (-1, -.2) -- (-1, .2);
\draw(3, -.2) -- (3, .2);
\draw(2, -.2) -- (2, .2);
\draw (1, -.2) -- (1, .2);
\draw( -.2, -3) -- ( .2, -3);
\draw( -.2, -2) -- ( .2,-2);
\draw ( -.2, -1) -- ( .2, -1);
\draw( -.2, 3) -- ( .2,3);
\draw( -.2, 2) -- ( .2,2);
\draw ( -.2,1) -- ( .2,1);
\fill (3.5, -1.4) circle (0.06);  
\draw[dashed] (3.5, -1.4) -- (3.5, 0) (3.5, -1.4) -- (0,-1.4);
\fill (3.5, 0) circle (0.06) (0, -1.4) circle (0.06); 
\draw (3.5, 0.5) node {$A$} (-.5, -1.4) node {$B$} (3.75, -1.8) node {$M$};
\draw (3.85,- 0.35) node {$3,5$} (.5, -1.65) node {$-1,4$} ;
\fill (-2, 3) circle (0.06); \draw (-2.2, 3.2) node {$N$};
\fill (-4, -1.5) circle (0.06); \draw (-4.2, -1.7) node {$P$};
\end{tikzpicture}
\caption{Exemple de repérage d'un point du plan. Le point $M$ a pour abscisse $3,5$ et pour ordonnée $-1,4$.
On note donc ses coordonnées $(3,5; -1,4)$. On a placé aussi pour l'exemple un point $N$ de coordonnées $(-2; 3)$ et un point $P$ de coordonnées $(-4; -1,5)$}
\end{figure}  
\subsection{Applications ultérieures}

Le repérage dans le plan aura dans la suite un certain nombre d'applications pour vous : \begin{itemize}
\item Tracer la courbe d'une fonction (fin de collège). 
\item Faire une correspondance entre les constructions géométriques et les calculs sur les coordonnées. Par exemple : 
\begin{itemize}
\item  si on prend deux points $M$ et $N$ du plan de coordonnées respectives $(x; y)$ et $(x'; y')$, leur milieu est le point $I$ de coordonnées $((x+x')\div 2; 
(y+y')\div 2)$. 
\item si on prend trois points $A$, $B$ et $C$ de coordonnées respectives $(x;y)$, $(x'; y')$ et $(x''; y'')$, on peut
démontrer que le centre de gravité du triangle est le point de coordonnées $(\frac{x+x'+x''}3; \frac{y+y'+y''}3)$. 
\item En $4^e$, le théorème de Pythagore permet de calculer la distance entre deux points en fonction de leurs 
coordonnées dans un repère orthogonal. 
\end{itemize}
\item À terme, les calculs sur les coordonnées permettent de remplacer un long raisonnement géométrique à base de 
propriétés élémentaires en une suite de calculs sur des nombres relatifs (les coordonnées des points). Cela permet de 
faire des démonstrations plus facilement (une fois que la technique est maîtrisée), mais on perd parfois un peu en 
clarté et visibilité. 
\end{itemize}



	\end{document}
	
