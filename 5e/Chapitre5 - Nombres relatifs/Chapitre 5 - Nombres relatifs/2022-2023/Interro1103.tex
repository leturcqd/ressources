	\documentclass[14pt]{extreport}
\usepackage{extsizes}
	\usepackage[frenchb]{babel}
	\usepackage[utf8]{inputenc}  
	\usepackage[T1]{fontenc}
	\usepackage{amssymb}
	\usepackage[mathscr]{euscript}
	\usepackage{stmaryrd}
	\usepackage{amsmath}
	\usepackage{tikz}
	\usepackage[all,cmtip]{xy}
	\usepackage{amsthm}
	\usepackage{varioref}
	\usepackage[ margin=1in]{geometry}
	\geometry{a4paper}
	\usepackage{lmodern}
	\usepackage{hyperref}
	\usepackage{array}
	\usepackage{easytable}
	 \usepackage{fancyhdr}\usepackage{longtable}

	\pagestyle{fancy}
	\theoremstyle{plain}
	\fancyfoot[C]{\empty} 
	\fancyhead[L]{Interrogation}
	\fancyhead[R]{11 mars 2022}
	
	
	\title{Interrogation chapitre 5}
	\date{}
	\begin{document}

\begin{center}{\Large Interrogation chapitre 5}\\ \textbf{Soignez votre présentation et votre rédaction.
Exercices 3 et 4 sur l'énoncé, tout le reste sur votre copie double.}\end{center}


\textbf{Exercice 1} % 4.5 points

Pour les propriétés suivantes, donnez un exemple de nombres relatifs pour lesquelles elles sont vraies, ou dire qu'elles sont impossibles. 

1. Un nombre qui est égal à son propre opposé.

2. Deux nombres positifs dont la différence est $-3$. 

3. Un nombre positif et un nombre négatif dont le produit est $-4$. 

%4. Un nombre positif et un nombre négatif dont le produit est égal à $-6$. 

\textbf{Exercice 2}  % 4 points

Recopier les opérations suivantes en remplaçant les $\ldots$ par des nombres relatifs. 

\[ 8 + (-3) =\ldots \]
\[ 9 - (-5) = \ldots \]
\[ -1 + \ldots = 4 \]
\[ -1 - \ldots = 7 \]

\textbf{Exercice 3} % 1 / 1.5 / 1.5  -> 4 points

Calculer les sommes suivantes en détaillant vos étapes. 

\[ -1  + 3 + 2 + (-4) + (-5) + 6 \]
\[ 1,03  + 3,4 + 6 + (-4,13)   + (-5,4) + 2,1  \]
\[ \frac23 +  \left(- \frac43\right) + \frac13\]
%\[ \frac23 +  \left(- \frac13\right) + \frac43\]
%\[ \frac23 +  \left(- \frac16\right) + \frac12\]




\textbf{Exercice 4} % 6 points
Complèter le carré magique suivant avec des entiers relatifs pour que les quatre lignes, les quatre colonnes, 
et les deux diagonales aient une somme égale à $6$. 
\[
\begin{TAB}(e,1cm){|c|c|c|c|}{|c|c|c|c|}
    1 & & & 2 \\
    1 &  & -3 &  \\
      & 2 & 0 & \\
      & -7 &  & -2
\end{TAB}
\]

%
%\[
%\begin{TAB}(e,1cm){|c|c|c|c|}{|c|c|c|c|}
%    -2 & & & 5 \\
%    5 &  & 2 &  \\
%      & -3 & 7 & \\
%      & -1 &  & 1
%\end{TAB}
%\]

\newpage
\textbf{Exercice 5}

%\begin{longtable}{|c|c|c|c|c|c|c|c|}
%\hline 
% Jour & L & Ma & Mer & J & V & S & D \\
%\hline
%Température maximale & 11 & 12,3 & 12,7 & 10,2 & 10,3 & 9,4 & 10 \\
%Température minimale & 2 & 3,1 & -1,2 & 0,4 & -0,3 & -2 & -0,7 \\
%Amplitude thermique & 9 & & & & & & \\
%\hline
%\end{longtable} 
\begin{longtable}{|c|c|c|c|c|c|c|c|}
\hline 
 Jour & L & Ma & Mer & J & V & S & D \\
\hline
Température maximale & 11 & 12,3 & 12,1 & 10,7 & 10,3 & 9,4 & 10 \\
Température minimale & 2 & 3,1 & -1,4 & 0,4 & -0,8 & -1,5 & -1,7 \\
Amplitude thermique & 9 & & & & & & \\
\hline
\end{longtable} 

a. Remplir le tableau suivant avec les amplitudes thermiques, 
qui sont les différences entre la température maximale et la température minimale d'une journée.

b. Quel jour a-t-il fait le plus chaud ? .......................

c. Quel jour a-t-il fait le plus froid ? ........................

d. Quel jour l'amplitude thermique a-t-elle été la plus forte ? .................


\textbf{Exercice 6} % 4 points

Compléter les opérations suivantes. 

\[ 8 \times (- 3) =  \ldots \]
\[ -1 \times \ldots = 5 \]
\[ (-3) \times (-5) = \ldots  \]
\[ -5 + 2 \times \ldots = 9 \]


%
%\textbf{Exercice 5}
%
%Voici un tableau des températures moyennes pour chaque mois de l'année dans plusieurs villes : Brest, Montréal, Samarcande, Pékin.
%
%
%
%\begin{longtable}{|c|c|c|c|c|c|c|c|c|c|c|c|c|}
%\hline 
% & Jan. & Fév. & Mars & Avril & Mai & Juin & Juil. & Août & Sept. & Oct. & Nov. & Déc.\\
%\hline
%B & 6,9 	& 6,8  & 8,4 & 9,6 & 12,6 & 15 & 16,9 & 17  & 15,4 & 12,7 & 9,5 & 7,3\\
%M & - 9,7 & - 7,7 & -2 & 6,4 & 13,4 & 18,6 & 21,2 & 20,1 & 15,5 & 8,5 & 2,1 & -5,4\\
%S & 2,3 & 4 & 9,3 & 15,2 & 20,4 & 25,4 & 27,2 & 25,6 & 20,6 & 14,1 & 8 & 3,7\\
%P &-2,7&0,6&6,8&14,8&20,8&25,1&27&25,9&21&14&5,3&-0,7\\
%\hline
%\end{longtable} 
%
%
%1. Pour chacune des villes, quelle est le mois le plus chaud ? le mois le plus froid ? 
%
%2. Quelle est l'amplitude thermique de chacune des quatre villes (la différence entre la température maximale et la 
%température minimale) ?
%
%3. Quelle est la ville ayant la plus grande amplitude thermique ? La plus petite ? 

\end{document}