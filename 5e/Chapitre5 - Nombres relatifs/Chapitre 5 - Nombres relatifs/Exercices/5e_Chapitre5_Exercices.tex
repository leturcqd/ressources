	\documentclass[12 pt]{article}

	\usepackage[frenchb]{babel}
	\usepackage[utf8]{inputenc}  
	\usepackage[T1]{fontenc}
	\usepackage{amssymb}
	\usepackage[mathscr]{euscript}
	\usepackage{stmaryrd}
	\usepackage{amsmath}
	\usepackage{tikz}
	\usepackage[all,cmtip]{xy}
	\usepackage{amsthm}
	\usepackage{varioref}
	\usepackage{geometry}
	\geometry{a4paper}
	\usepackage{lmodern}
	\usepackage{hyperref}
	\usepackage{array}
	 \usepackage{fancyhdr}
\usepackage{float}
	\pagestyle{fancy}
	\theoremstyle{plain}
	\fancyfoot[C]{\empty} 
	\fancyhead[L]{Exercices}
	\fancyhead[R]{2021-2022}
	\newcounter{n}
	\numberwithin{n}{section}
	\newtheorem{df}[n]{Définition}
	\newtheorem{theo}{Théorème}
	\labelformat{theo}{théorème}
    \newtheorem{rmq}[n]{Remarque}
	\labelformat{rmq}{remarque~#1}
	\newtheorem{cor}[n]{Corollaire}
	\newtheorem{lm}{Lemme}
	\labelformat{lm}{lemme~#1}
	\newtheorem{hyp}[n]{Hypothèse}
	\newtheorem{nt}[n]{Notation}

	\renewcommand\epsilon{\varepsilon}
	\renewcommand\phi{\varphi}
	\newcommand\R{\mathbb{R}}
	\newcommand\s{\mathbb{S}}
	
	
	
	\title{Exercices Chapitre 5}
	\date{}
	\begin{document}

\begin{center}{\Large Chapitre 5 - Nombres relatifs}\\ 
 \end{center}


\section{Définitions}



1) Dans la liste suivante, dire quels nombres sont positifs, et quels nombres sont négatifs : 
\[ 2 ; \ -3 ; \ -4,5 ; \ \frac23 ; + 5 ; 4 + 3. \]

2) Donner l'opposé des nombres suivants : 

\[ 2 ; -3,4 ; 4,5 ; +5 \]

3) Dans les nombres des deux questions précédentes, lesquels sont des entiers relatifs ?

\section{Opérations}


\subsection{Additions}

Calculer les additions suivantes de nombres relatifs : 

\[ 2 + (-3)\]
\[ (-2) + 4 + (-5) + 3\]
\[ \frac13 + (-\frac 56) + \frac12 \]
\[ 2,1 + (-3,4) + 1,3 + (+4,7) + (- 2,7)\]

\subsection{Soustractions}

Calculer les soustractions suivantes, et dire si elles sont positives ou négatives. 

\[ 5 - 6 \]
\[ 7 - 9 \]
\[ 4 - 3 - 2\] 
\[ 4 - ( 3 - 2 ) \]
\[ 7 - (-1) \]
\[ 5,1 - 2,7\]
\[ (-3,4) - (-2,1) \]
\[ (-3,1) - (5,7)\]

%
%\subsection{Multiplications}
%
%Calculer les opérations suivantes. 
%\[ 2 \times (-4) - (+3) \times (-5) + (4) \times (-1) \]
%\[ (3 - 2) \times (4 - 1) \] 
%\[ (2\times 4 + (- 13))  \times (6 - (-1) ) \]


\subsection{Opérations multiples}

Calculer les opérations suivantes. 
\[ 2 + (-4) - (+3) - (-5) + (4) + (-1) \]
\[ (3 - 2) - (4 - 1) \] 
\[ (2\times 4 - 13) + (6 - (-1) ) \]

%\[ (2,5 \times 4,1 - 1,3) + (6 - (-1)\times(-7) ) \]
%
%\[ \left(\frac23\times 4 - \left(-\frac13\right)\right) + ( (-1) - ((-3) \times (1-4) ) \]


%
\section{Repérage et comparaison}
\subsection{Comparaison}

Ranger dans l'ordre décroissant les nombres suivants : 

\[ -1 ; 3,4 ; -0,5 ; 1,17 ; 2 - (-4)\]

\subsection{Comparaison (II)}

Ranger dans l'ordre décroissant les nombres suivants : 

\[ -1 ; \frac34 ; +0,5 ; -\frac17 ; 2 - (-4)\]
\subsection{Repérage}

Tracer une droite graduée de $20$ carreaux, avec l'origine $O$ au dixième carreau. On place $1$ au douzième carreau (le deuxième après $O$). Placer sur la règle les nombres suivants : $ -2,5$; $\frac32$; $-\frac12$; $1,5$; $+3$. 
En déduire la liste de ces nombres dans l'ordre croissant. 


\subsection{Repérage (II)}

Tracer une droite graduée de $18$ carreaux, avec l'origine $O$ au neuvième carreau. On place $1$ au douzième carreau (le troisième après $O$). Placer sur la règle les nombres suivants : $+ 3$;$\frac23$; $-\frac46$; $\frac{15}{9} $; $-\frac{49}{21}$. 
En déduire la liste de ces nombres dans l'ordre croissant. 

\subsection{Repérage dans le plan}

Tracer un repère du plan, en y choisissant $1$ carreau comme graduation pour chaque droite, et y placer les points 
suffisants : $ A(3 ; 4)$, $B(4; -1)$, $C(-3; 2)$, $D(-2; -4)$. 

\subsection{Repérage dans le plan (II)}

Tracer un repère du plan, en y choisissant $1$ carreau comme graduation pour chaque droite, et y placer les points 
suffisants : $ A(1 ; 2)$, $B(-1; 2)$, $C(-1; -2)$, $D(1; -2)$. 

Quelle est la nature du quadrilatère $ABCD$ ? 
\newpage

\subsection{Repérage dans le plan (III)}
Donnez les coordonnées des points $A$, $B$, $C$, $D$, et $E$. 

Quelle est la nature du triangle $ABC$ ?

Que dire des points $A$, $B$ et $D$ ?

Placer le point $F$ symétrique de $D$ par rapport à $B$. Donner les coordonnées du point $F$. 

Que dire du triangle $ACF$ ?
\begin{figure}[H]
\begin{tikzpicture}
\draw[dotted](-5.5,-5.5) grid (5.5,5.5);
\draw[very thick, ->] (-5.5, 0)--(5.5,0);
\draw[very thick, ->] (0,-5.5)--(0,5.5);
\fill (3,2) circle(.1);
\draw (3,2) ++ (.25, .25) node {$A$};
\fill (-1,-2) circle(.1);
\draw (-1,-2) ++ (.25, .25) node {$B$};
\fill (5,0) circle(.1);
\draw (5,0) ++ (.25, .25) node {$C$};
\fill (-3,4) circle(.1);
\draw (-3,4) ++ (.25, .25) node {$E$};
\fill (-3,-4) circle(.1);
\draw (-3,-4) ++ (.25, .25) node {$D$};
\draw (1,.3) node {$1$};
\draw (.3,1) node {$1$};
\end{tikzpicture}
\end{figure}
\newpage
\subsection{Repérage dans le plan (IV)}
Donnez les coordonnées des points $A$, $B$, $C$, et $D$. 

\begin{figure}[H]
\begin{tikzpicture}
\draw[dotted](-5.5,-5.5) grid (5.5,5.5);
\draw[very thick, ->] (-5.5, 0)--(5.5,0);
\draw[very thick, ->] (0,-5.5)--(0,5.5);
\fill (3,2) circle(.1);
\draw (3,2) ++ (.25, .25) node {$A$};
\fill (-1,-2) circle(.1);
\draw (-1,-2) ++ (.25, .25) node {$B$};
\fill (-3,4) circle(.1);
\draw (-3,4) ++ (.25, .25) node {$C$};
\fill (1,1) circle(.1);
\draw (1,1) ++ (.25, .25) node {$D$};
\draw (1,.3) node {$1$};
\draw (.3,1) node {$1$};
\end{tikzpicture}
\end{figure}
On construit un point $A'$ de la manière suivante : 
\begin{itemize}
\item l'abscisse du point $A'$ est deux fois l'abscisse du point $D$ moins l'abscisse du point $A$. 
\item l'ordonnée du point $A'$ est deux fois l'ordonnée du point $D$ moins l'ordonnée du point $A$. 
\end{itemize}
On procède de même pour construire un point $B'$ et un point $C'$ (en remplaçant les coordonnées de $A$ par celles
de $B$ ou de $C$ dans la construction précédente).

Que dire de la figure $A'B'C'$ ainsi obtenue ?
	\end{document}
