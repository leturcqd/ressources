	\documentclass[14pt]{extreport}
\usepackage{extsizes}
	\usepackage[frenchb]{babel}
	\usepackage[utf8]{inputenc}  
	\usepackage[T1]{fontenc}
	\usepackage{amssymb}
	\usepackage[mathscr]{euscript}
	\usepackage{stmaryrd}
	\usepackage{amsmath}
	\usepackage{tikz}
	\usepackage[all,cmtip]{xy}
	\usepackage{amsthm}
	\usepackage{varioref}
	\usepackage[ margin=1in]{geometry}
	\geometry{a4paper}
	\usepackage{lmodern}
	\usepackage{hyperref}
	\usepackage{array}
	\usepackage{float}
	\usepackage{easytable}
	 \usepackage{fancyhdr}\usepackage{longtable}
	 \usetikzlibrary{shapes.misc}
\newlength{\taillecellule}
\setlength{\taillecellule}{2cm}
\newcolumntype{C}{@{}>{\centering\arraybackslash}p{\taillecellule}@{}}

\usepackage{pstricks,multido}
\usepackage{arrayjob}
\usepackage{calc,xlop}
\tikzset{cross/.style={cross out, draw=black, minimum size=2*(#1-\pgflinewidth), inner sep=0pt, outer sep=0pt},
%default radius will be 1pt. 
cross/.default={1pt}}

	\pagestyle{fancy}
	\theoremstyle{plain}
	\fancyfoot[C]{\empty} 
	\fancyhead[L]{Interrogation chapitre 5}
	\fancyhead[R]{12 mars 2025}
	
	
	\title{Interrogation chapitre 5}
	\date{}
	\begin{document}



\subsection*{Exercice 1 (8 points)}  % 4 points
 
Recopiez et remplissez les pointillés :

\[ a)\ 18 + (-3) =\ldots \ \ \ \ \ \ \ \ \ \ \ \ \ \ 
 b)\ 9 - (-15) = \ldots \]
\[ c)\ -8 + 34 =\ldots \ \ \ \ \ \ \ \ \ \ \ \ \ \ 
 d)\ -39 - 7 = \ldots \]
\[ e)\ -5 + (-24) =\ldots \ \ \ \ \ \ \ \ \ \ \ \ \ \ 
 f)\ -15 - (-15) = \ldots \]
\[ g)\ -12 + \ldots = 4  \ \ \ \ \ \ \ \ \ \ \ \ \ \  
h)\ -14 - \ldots = 7\]

\subsection*{Exercice 2 (6 points)} % 1 / 1.5 / 1.5  -> 4 points

Calculez les sommes suivantes en détaillant vos étapes. 

\[ a) -21  + 33 + 29 + (-54) + (-53) + 6  = \]
\[ b) -41  - (-33) - 12 + (-24) + (-95) + 16  = \]
\[ c) -21  + 43 + 41 - (-14) + (-17) + 12  = \]
\[ d) -16  - ( 13 + 2 + (-4) ) - ((-5) + 16)= \]
\[ e) 1,3  - 2,1 + (-3,4) - (- 5,6) - (+1,4)  = \]
\[ f) 1,03  + 3,4 + 6 + (-4,13)   + (-5,4) + 2,1 = \]

\subsection*{Exercice 3 (6 points) }

Remplissez les pyramides additives suivantes (chaque nombre est la somme des deux nombres en-dessous de lui.)

\begin{figure}[H]
\center
\begin{tikzpicture}[scale=.9,every node/.style={draw,minimum width=1.8cm,minimum height=.9cm}]
\draw (0,0)  node {\phantom{9}};
\draw(-1,-1) node {\phantom {8}} ++(2,0) node {\phantom {1}};
\draw(-2,-2) node {\phantom{6,5}} ++(2,0) node {\phantom{1,5}} ++(2,0) node {\phantom{-0,5}};
\draw(-3,-3) node {\phantom{}{7,5}} ++(2,0) node {\phantom{}{-1}} ++(2,0) node {\phantom{}{2,5}} ++(2,0) node {\phantom{}{-3}};
\end{tikzpicture}
\
\begin{tikzpicture}[scale=.9,every node/.style={draw,minimum width=1.8cm,minimum height=.9cm}]
\draw (0,0)  node {\phantom{2,7}};
\draw(-1,-1) node {\phantom {1,2}} ++(2,0) node {\phantom {}{1,5}};
\draw(-2,-2) node {\phantom{1,8}} ++(2,0) node {\phantom{}{-0,6}} ++(2,0) node {\phantom{2,1}};
\draw(-3,-3) node {\phantom{}{4,1}} ++(2,0) node {\phantom{}{-2,3}} ++(2,0) node {\phantom{1,7}} ++(2,0) node {\phantom{0,4}};
\end{tikzpicture}
\end{figure}
 
 
 \newpage 
 \subsection*{Exercice 1 (8 points)}  % 4 points
 
Recopiez et remplissez les pointillés :

\[ a)\ 13 + (-3) =\ldots \ \ \ \ \ \ \ \ \ \ \ \ \ \ 
 b)\ 9 - (-13) = \ldots \]
\[ c)\ -8 + 31 =\ldots \ \ \ \ \ \ \ \ \ \ \ \ \ \ 
 d)\ -35 - 7 = \ldots \]
\[ e)\ -5 + (-26) =\ldots \ \ \ \ \ \ \ \ \ \ \ \ \ \ 
 f)\ -15 - (-15) = \ldots \]
\[ g)\ -11 + \ldots = 4  \ \ \ \ \ \ \ \ \ \ \ \ \ \  
h)\ -11 - \ldots = 7\]

\subsection*{Exercice 2 (6 points)} % 1 / 1.5 / 1.5  -> 4 points

Calculez les sommes suivantes en détaillant vos étapes. 

\[ a) -21  + 34 + 29 + (-54) + (-53) + 6  = \]
\[ b) -41  - (-39) - 12 + (-24) + (-95) + 16  = \]
\[ c) -21  + 45 + 41 - (-14) + (-17) + 12  = \]
\[ d) -16  - ( 11 + 2 + (-4) ) - ((-5) + 16)= \]
\[ e) 1,3  - 2,1 + (-3,4) - (- 5,6) - (+1,4)  = \]
\[ f) 1,03  + 3,4 + 6 + (-4,13)   + (-5,4) + 2,1 = \]

\subsection*{Exercice 3 (6 points) }

Remplissez les pyramides additives suivantes (chaque nombre est la somme des deux nombres en-dessous de lui.)

\begin{figure}[H]
\center
\begin{tikzpicture}[scale=.9,every node/.style={draw,minimum width=1.8cm,minimum height=.9cm}]
\draw (0,0)  node {\phantom{9}};
\draw(-1,-1) node {\phantom {8}} ++(2,0) node {\phantom {1}};
\draw(-2,-2) node {\phantom{6,5}} ++(2,0) node {\phantom{1,5}} ++(2,0) node {\phantom{-0,5}};
\draw(-3,-3) node {\phantom{}{7,5}} ++(2,0) node {\phantom{}{-1}} ++(2,0) node {\phantom{}{1,5}} ++(2,0) node {\phantom{}{-3}};
\end{tikzpicture}
\
\begin{tikzpicture}[scale=.9,every node/.style={draw,minimum width=1.8cm,minimum height=.9cm}]
\draw (0,0)  node {\phantom{2,7}};
\draw(-1,-1) node {\phantom {1,2}} ++(2,0) node {\phantom {}{1,5}};
\draw(-2,-2) node {\phantom{1,8}} ++(2,0) node {\phantom{}{-0,6}} ++(2,0) node {\phantom{2,1}};
\draw(-3,-3) node {\phantom{}{4,1}} ++(2,0) node {\phantom{}{-5,3}} ++(2,0) node {\phantom{1,7}} ++(2,0) node {\phantom{0,4}};
\end{tikzpicture}
\end{figure}
 
 
\end{document}