\documentclass[14 pt]{extarticle}

	\usepackage[frenchb]{babel}
	\usepackage[utf8]{inputenc}  
	\usepackage[T1]{fontenc}
	\usepackage{amssymb}
	\usepackage[mathscr]{euscript}
	\usepackage{stmaryrd}
	\usepackage{amsmath}
	\usepackage{tikz}
	\usepackage[all,cmtip]{xy}
	\usepackage{amsthm}
	\usepackage{varioref}
	\usepackage{geometry}
	\geometry{a4paper}
	\usepackage{lmodern}
	\usepackage{hyperref}
	\usepackage{array}
	 \usepackage{fancyhdr}
	 \usepackage{float}
\renewcommand{\theenumi}{\alph{enumi})}
	\pagestyle{fancy}
	\theoremstyle{plain}
	\fancyfoot[C]{} 
	\fancyhead[L]{Contrôle}
	\fancyhead[R]{13 février 2022}\geometry{
 a4paper,
 total={170mm,257mm},
 left=20mm,
 top=20mm,
 }
	
	
	\title{Interrogation chapitre 4}
	\date{}
	\begin{document}

\begin{center}{\Large Interrogation chapitre 5}\\ 
 \end{center}
 
 Nom : \ldots\ldots\ldots\\
 Prénom : \ldots\ldots\ldots
 
 
\subsection*{Exercice 1}
 Pour chacune des grandeurs suivantes, dire si elle peut être représentée par des nombres relatifs : 
 
 \begin{enumerate}
 \item la durée d'un événement 
 \item la masse d'un objet
 \item la température ambiante
 \item l'altitude d'un lieu
 \end{enumerate}
\subsection*{Exercice 2}
 
 Ranger dans l'ordre décroissant les nombres suivants. 
 

\[ 2,12 ; \ \ 2,1 ; \ \ -2,4 ; \ \ -2,3 ; \ \ 2 ; \ \ 0 ; \ \ -1,23 \]
 
\subsection*{Exercice 3}
Tracer une droite graduée de $18$ carreaux, avec l'origine $O$ au neuvième carreau. 

On place $1$ au douzième carreau (le troisième après $O$). 

Placer sur la règle les nombres suivants :
\[ 3 ; \ \ -2 ; \ \ \frac43 ; \ \ -\frac{14}{21}\]


\newpage 
\subsection*{Exercice 4}
\begin{enumerate}


\item Donnez les coordonnées des points $A$, $B$, $C$, $D$, et $E$. 

\item Quelle est la nature du triangle $ABC$ ?


\item Placer le point $F$ symétrique de $D$ par rapport à $B$. Donner les coordonnées du point $F$. 
\end{enumerate}
\begin{figure}[H]
\begin{tikzpicture}
\draw[dotted](-5.5,-5.5) grid (5.5,5.5);
\draw[very thick, ->] (-5.5, 0)--(5.5,0);
\draw[very thick, ->] (0,-5.5)--(0,5.5);
\draw (3,2) -- ++(45: .2) (3,2) -- ++(135: .2) (3,2) -- ++(-45: .2) (3,2) --++(-135: .2) ;
\draw (3,2) ++ (.25, .25) node {$A$};
\draw (-1,-2) -- ++(45: .2) (-1,-2) -- ++(135: .2) (-1,-2) -- ++(-45: .2) (-1,-2) --++(-135: .2) ;
\draw (-1,-2) ++ (.25, .25) node {$B$};
\draw (5,0) -- ++(45: .2) (5,0) -- ++(135: .2) (5,0) -- ++(-45: .2) (5,0) --++(-135: .2) ;
\draw (5,0) ++ (.25, .25) node {$C$};
\draw (-3,4) -- ++(45: .2) (-3,4) -- ++(135: .2) (-3,4) -- ++(-45: .2) (-3,4) --++(-135: .2) ;
\draw (-3,4) ++ (.25, .25) node {$E$};
\draw (-3,-4) -- ++(45: .2) (-3,-4) -- ++(135: .2) (-3,-4) -- ++(-45: .2) (-3,-4) --++(-135: .2) ;
\draw (-3,-4) ++ (.25, .25) node {$D$};
\draw (1,.3) node {$1$};
\draw (.3,1) node {$1$};
\end{tikzpicture}
\end{figure}

\newpage 


\begin{center}{\Large Interrogation chapitre 5}\\ 
 \end{center}
 
 Nom : \ldots\ldots\ldots\\
 Prénom : \ldots\ldots\ldots

\subsection*{Exercice 1}
 Pour chacune des grandeurs suivantes, dire si elle peut être représentée par des nombres relatifs : 
 
 \begin{enumerate}
 \item l'âge d'une personne
 \item les années du calendrier
 \item la longueur d'un trajet
 \item l'altitude d'un lieu
 \end{enumerate}
\subsection*{Exercice 2}
 
 Ranger dans l'ordre décroissant les nombres suivants. 
 

\[ 2,31 ; \ \ 2,1 ; \ \ -2,2 ; \ \ -2,3 ; \ \ 1 ; \ \ 0 ; \ \ -1,23 \]
 
\subsection*{Exercice 3}
Tracer une droite graduée de $18$ carreaux, avec l'origine $O$ au neuvième carreau. 

On place $1$ au douzième carreau (le troisième après $O$). 

Placer sur la règle les nombres suivants :
\[ 2 ; \ \ -1 ; \ \ -\frac73 ; \ \ \frac{14}{21}\]

\newpage 

\subsection*{Exercice 4}
\begin{enumerate}


\item Donnez les coordonnées des points $A$, $B$, $C$, $D$, et $E$. 

\item Quelle est la nature du triangle $CDE$ ?


\item Placer le point $F$ symétrique de $D$ par rapport à $B$. Donner les coordonnées du point $F$. 
\end{enumerate}
\begin{figure}[H]
\begin{tikzpicture}
\draw[dotted](-5.5,-5.5) grid (5.5,5.5);
\draw[very thick, ->] (-5.5, 0)--(5.5,0);
\draw[very thick, ->] (0,-5.5)--(0,5.5);
\draw (3,2) -- ++(45: .2) (3,2) -- ++(135: .2) (3,2) -- ++(-45: .2) (3,2) --++(-135: .2) ;
\draw (3,2) ++ (.25, .25) node {$A$};
\draw (-1,-2) -- ++(45: .2) (-1,-2) -- ++(135: .2) (-1,-2) -- ++(-45: .2) (-1,-2) --++(-135: .2) ;
\draw (-1,-2) ++ (.25, .25) node {$B$};
\draw (5,0) -- ++(45: .2) (5,0) -- ++(135: .2) (5,0) -- ++(-45: .2) (5,0) --++(-135: .2) ;
\draw (5,0) ++ (.25, .25) node {$C$};
\draw (-3,3) -- ++(45: .2) (-3,3) -- ++(135: .2) (-3,3) -- ++(-45: .2) (-3,3) --++(-135: .2) ;
\draw (-3,3) ++ (.25, .25) node {$E$};
\draw (-3,-3) -- ++(45: .2) (-3,-3) -- ++(135: .2) (-3,-3) -- ++(-45: .2) (-3,-3) --++(-135: .2) ;
\draw (-3,-3) ++ (.25, .25) node {$D$};
\draw (1,.3) node {$1$};
\draw (.3,1) node {$1$};
\end{tikzpicture}
\end{figure}


 	\end{document}
