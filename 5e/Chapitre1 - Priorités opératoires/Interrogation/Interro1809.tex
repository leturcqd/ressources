\documentclass[14 pt]{extarticle}

	\usepackage[frenchb]{babel}
	\usepackage[utf8]{inputenc}  
	\usepackage[T1]{fontenc}
	\usepackage{amssymb}
	\usepackage[mathscr]{euscript}
	\usepackage{stmaryrd}
	\usepackage{amsmath}
	\usepackage{tikz}
	\usepackage[all,cmtip]{xy}
	\usepackage{amsthm}
	\usepackage{varioref}
	\usepackage{geometry}
	\geometry{a4paper}
	\usepackage{lmodern}
	\usepackage{hyperref}
	\usepackage{array}
	 \usepackage{fancyhdr}
\renewcommand{\theenumi}{\alph{enumi})}
	\pagestyle{fancy}
	\theoremstyle{plain}
	\fancyfoot[C]{} 
	\fancyhead[L]{Interrogation}
	\fancyhead[R]{19 septembre 2023}\geometry{
 a4paper,
 total={170mm,257mm},
 left=20mm,
 top=20mm,
 }
	
	
	\title{Interrogation Chapitre 1}
	\date{}
	\begin{document}

\begin{center}{\Large Interrogation Chapitre 1}\\ 
 \end{center}
 
  

\subsection*{Exercice 1}

Effectuer les calculs suivants en numérotant les opérations et en détaillant les étapes. \begin{enumerate}
\item $10 + 3 \times 4$
\item $ 7 \times 3 + 2$
\item $18\div 3 \times 3$
\item $(2+5) \times (3+2) - 2$
\item $ (26 - 1 + 5)\div 2 \times 5$
\end{enumerate}

\subsection*{Exercice 2}

Parmi les égalités suivantes, certaines sont fausses, et d'autres vraies. Recopier chaque égalité en rajoutant des parenthèses si nécessaire pour les rendre toutes vraies.

\begin{enumerate}
\item $18 \div 2 \div 3 = 3$
\item $27 - 1 + 3 = 23$
\item $(4+ 6 )\times 5-3 = 20$
\item $4 \times 9\times 7 - 13 = 200$. 
\end{enumerate}
 

\subsection*{Exercice 3}

On achète trois stylos à $2$ euros l'unité, et cinq cahiers à $4$ euros l'unité. 

\begin{enumerate}
\item Écrire le prix total sous la forme d'un seul calcul. 
\item A-t-on besoin de parenthèses ?
\item Effectuer le calcul. 
\end{enumerate}
 
\newpage


\begin{center}{\Large Interrogation Chapitre 1}\\ 
 \end{center}
 
  

\subsection*{Exercice 1}

Effectuer les calculs suivants en numérotant les opérations et en détaillant les étapes. \begin{enumerate}
\item $10 \times 2 + 4$
\item $ 7 + 2 \times 3$
\item $27\div 3 \times 3$
\item $(2+4) \times (3+2) - 2$
\item $ (26 - 1 + 5)\div 5 \times 2$
\end{enumerate}

\subsection*{Exercice 2}

Parmi les égalités suivantes, certaines sont fausses, et d'autres vraies. Recopier chaque égalité en rajoutant des parenthèses si nécessaire pour les rendre toutes vraies.

\begin{enumerate}
\item $24 \div 2 \div 3 = 4 $
\item $27 - 1 + 3 = 23$
\item $(4+ 6 )\times 5-3 = 20$
\item $4 \times 9\times 7 - 3 = 240$. 
\end{enumerate}
 

\subsection*{Exercice 3}

On achète trois stylos à $3$ euros l'unité, et cinq cahiers à $4$ euros l'unité. \begin{enumerate}
\item Écrire le prix total sous la forme d'un seul calcul. 
\item A-t-on besoin de parenthèses ?
\item Effectuer le calcul. 
\end{enumerate}
 
 	\end{document}
