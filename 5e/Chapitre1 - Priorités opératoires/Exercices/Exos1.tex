\documentclass[12 pt]{extarticle}

	\usepackage[frenchb]{babel}
	\usepackage[utf8]{inputenc}  
	\usepackage[T1]{fontenc}
	\usepackage{amssymb}
	\usepackage[mathscr]{euscript}
	\usepackage{stmaryrd}
	\usepackage{amsmath}
	\usepackage{tikz}
	\usepackage[all,cmtip]{xy}
	\usepackage{amsthm}
	\usepackage{varioref}
	\usepackage{geometry}
	\geometry{a4paper}
	\usepackage{lmodern}
	\usepackage{hyperref}
	\usepackage{array}
	 \usepackage{fancyhdr}
\renewcommand{\theenumi}{\alph{enumi})}
	\pagestyle{fancy}
	\theoremstyle{plain}
	\fancyfoot[C]{} 
	\fancyhead[L]{Fiche d'exercices}
	\fancyhead[R]{2022-2023}\geometry{
 a4paper,
 total={170mm,257mm},
 left=20mm,
 top=20mm,
 }
	
	
	\title{Exercices Chapitre 1}
	\date{}
	\begin{document}

\begin{center}{\Large Chapitre 1 - Priorités opératoires, distributivité}\\ 
 \end{center}
 
 
 \subsection*{Exercice 1}

 
1) Effectuer les calculs suivants : $(2 + 3) \times (4 - 1)$ ; 
$ 2 + ( 3 \times (4 - 1) ) $ ; 
$( ( 2 + 3 ) \times 4 ) - 1 $ ; 
$ ( 2 + ( 3 \times 4 ) ) - 1$ ; 
$2 + ((3 \times 4) - 1)$. 

2) Combien de résultats différents trouve-t-on ? 

3) Lequel de ces cinq calculs correspond à $ 2 + 3 \times 4 - 1$ ?

\subsection*{Exercice 2}
 
 
 Pour chacune des opérations suivantes, numéroter les opérations 
 pour indiquer dans quel ordre les effectuer, puis calculer le 
 résultat. 
 \begin{enumerate}
 \item $15 + 1 - 4 \times 2$
 \item$4 \div 2 + 3 \times 5$ 
 \item $7 + 9 \div 3 + 1$
 \item $12 - 6 \times 2 \div 4$
 \item $40 \div 8 + 8 \times 8$
 \item $12 \times 6 \div 3 \times 2$
 \item $(1 + 3 \times 5) \div 2$
 \item $(64\div 16) \div 2$
 \item $64\div (16 \div 2)$
 \item $3 \times ( 7 \times 4 -1)$
 \item $35 \div 7\times (47-12)$
 \item $ 7\times 3  - (6 + 63\div 7)$
  \item $ (5 \times 6+ ((9-7)\times 4))\div 2$
 \end{enumerate}
 
 \subsection*{Exercice 3} 
 
1) Écrire une expression parenthésée correspondant aux calculs suivants : \\
a) la somme du produit de $2$ par $3$ et de $4$ ;\\
b) la différence entre $5$ et le produit de $2$ par la différence entre $7$ et  $5$ ; \\
c) le quotient de $4$ par le produit de $2$ par $2$ ; \\
d) le produit de la somme de $3$ et $4$ par la différence entre $6$ et $5$ ;
 
2) Dans les quatre expressions précédentes, peut-on supprimer des parenthèses ? Si oui, lesquelles ? 

3) Effectuer les calculs.
 \newpage
\subsection*{Exercice 4}
Décrire en français (cf. exercice 3) les opérations suivantes : 
\begin{enumerate}
\item $(3 + 5)\times 4 $ :
\item $ 3 - 5 \times 4$ ; 
\item $  (2 - 5) \div 3 $ ;
\item $ 6\div (3+2)$ ;
\item $ 6\div 3+2$.
\end{enumerate}
\subsection*{Exercice 5}

Parmi les égalités suivantes, déterminer lesquelles sont vraies, et lesquelles sont fausses. Pour ces dernières, placer des parenthèses de manière à ce que l'égalité soit vraie. 
\begin{enumerate}
\item $ 3 + 4 \times 5 = 35$ ;
\item $ 5 - 2 + 3 = 0$ ;
\item $ 12 - 4 \times 2 = 4 $ ;
\item $ 3 + 5 \times 4 = 32$ ;
\item $ 7 + 3 \times 5 -4 = 10$ ;
\item $ 12 + 1\times 4 - 5 = 11$ ;
\item $ 20 \div 4 + 3 \times 2 = 16$ ; 
\item $ 16 \div 5 + 3 - 2 = 0$ ;
\item $ 72 \div 6 \div 2 - 3 = 3 $ ;
\item $ 4 + 3 \times 5 - 8 \div 2 + 2 = 33$.
\end{enumerate}

\subsection*{Exercice 6}

Factoriser les expressions suivantes pour les calculer plus rapidement. \begin{enumerate}
\item $ 2 \times 3 + 2 \times 5$ ;
\item $ 4 \times 31 - 4 \times 30 $ ;
\item $ 5 \times 2 + 2 \times 15$ ;
\item $ 4 \times 3 \times 7 + 4 \times 5$ ;
\item $ 4 \times 3 \times 7 + 4 \times 7$.
\end{enumerate}

\subsection*{Exercice 7}

Simplifier les calculs suivants pour les effectuer.

\begin{enumerate}
\item $ 7 + 5 + 4 - 5 - 7 - 4$ ; 
\item $ 3 \times 5 \times 2 \div 3 \div 2$ ;
\item $ (4 + 5) \times 5 + (4+1) \times 91$ ;
\item $ 16 \times 8 - 4 \times 32$.

\end{enumerate}


 	\end{document}
