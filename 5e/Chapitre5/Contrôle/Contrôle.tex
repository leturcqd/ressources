	\documentclass[14pt]{extreport}
\usepackage{extsizes}
	\usepackage[frenchb]{babel}
	\usepackage[utf8]{inputenc}  
	\usepackage[T1]{fontenc}
	\usepackage{amssymb}
	\usepackage[mathscr]{euscript}
	\usepackage{stmaryrd}
	\usepackage{amsmath}
	\usepackage{tikz}
	\usepackage[all,cmtip]{xy}
	\usepackage{amsthm}
	\usepackage{varioref}
	\usepackage[ margin=1in]{geometry}
	\geometry{a4paper}
	\usepackage{lmodern}
	\usepackage{hyperref}
	\usepackage{array}
	\usepackage{float}
	\usepackage{easytable}
\renewcommand{\theenumi}{\alph{enumi})}
	 \usepackage{fancyhdr}\usepackage{longtable}
	 \usetikzlibrary{shapes.misc}

\tikzset{cross/.style={cross out, draw=black, minimum size=2*(#1-\pgflinewidth), inner sep=0pt, outer sep=0pt},
%default radius will be 1pt. 
cross/.default={1pt}}

	\pagestyle{fancy}
	\theoremstyle{plain}
	\fancyfoot[C]{\empty} 
	\fancyhead[L]{Interrogation}
	\fancyhead[R]{18 mars 2024}
	
	
	\title{Interrogation chapitre 5}
	\date{}
	\begin{document}

\begin{center}{\Large Interrogation chapitre 5}\end{center}



\subsection*{Exercice 1 (4 points)}
 
 Rangez dans l'ordre décroissant les nombres suivants. 
 

\[ 2,12 ; \ \ 2,1 ; \ \ -2,4 ; \ \ -2,3 ; \ \ 2 ; \ \ 0 ; \ \ -1,23 \]
 
 
 
\subsection*{Exercice 2 (2 points)}
 
 Calculez : \begin{enumerate}
 \item $12 - 5 - (-4) $
 \item $ 5 - 2 - (14 + 12 + ( -7) - 4) $
 \end{enumerate}
 
\subsection*{Exercice 3 (4 points)}
Tracer une droite graduée de $18$ carreaux, avec l'origine $O$ au neuvième carreau. 

On place $1$ au douzième carreau (le troisième après $O$). 

Placer sur la règle les nombres suivants :
\[ 3 ; \ \ -2 ; \ \ \frac43 ; \ \ -\frac{14}{21}\]



\subsection*{Exercice 4 (4 points)} 
 
1) Indiquer les abscisses des points $B$, $C$, $D$ et $E$.
 
2) Placer sur la droite graduée le point $F$ d'abscisse $-1$. 

3) Quel point a une abscisse opposée de celle de $D$ ? 
 \begin{figure}[H]\center
\begin{tikzpicture}
\draw [->, >= latex] (-8, 0) -- (6, 0);
\draw (-7, -.2) -- (-7, .2);
\draw (-6, -.2) -- (-6, .2);
\draw (-5, -.2) -- (-5, .2);
\draw (-4, -.2) -- (-4, .2);
\draw(-3, -.2) -- (-3, .2);
\draw(-2, -.2) -- (-2, .2);
\draw (-1, -.2) -- (-1, .2);
\draw (5, -.2) -- (5, .2);
\draw (4, -.2) -- (4, .2);
\draw(3, -.2) -- (3, .2);
\draw(2, -.2) -- (2, .2);
\draw (1, -.2) -- (1, .2);
\draw (0, -.2) -- (0, .2);
\draw (0, .5) node {$O$};
\draw (0, -.5) node {$0$};
\draw (2, 0) node[cross=5pt, rotate = 0] {}; 
\draw (2, .5) node {$A$}; 
\draw (2, -.5) node {$1$}; 
\fill (-6, 0) node[cross=5pt, rotate = 0] {}; 
\draw (-6, .5) node {$B$}; 
\fill (5, 0) node[cross=5pt, rotate = 0] {};  
\draw (5, .5) node {$C$}; 
\fill (4, 0) node[cross=5pt, rotate = 0] {};  
\draw (4, .5) node {$D$}; 
\fill (-4, 0) node[cross=5pt, rotate = 0] {};  
\draw (-4, .5) node {$E$}; 
\end{tikzpicture}
\end{figure}

\subsection*{Exercice 5 (6 points)}

\begin{figure}[H]\center
\begin{tikzpicture}
\draw[dotted](-5.5,-4.5) grid (7.5,6.5);
\draw[very thick, ->] (-5.5, 0)--(7.5,0);
\draw[very thick, ->] (0,-4.5)--(0,6.5);
\fill (2,-3) node[cross=5pt, rotate = 0] {};
\draw (2,-3) ++ (.25, .25) node {$A$};
\fill (-5,2) node[cross=5pt, rotate = 0] {};
\draw (-5,2) ++ (.25, .25) node {$B$};
\fill (3,4) node[cross=5pt, rotate = 0] {};
\draw (3,4) ++ (.25, .25) node {$C$};
\draw (1,.3) node {$1$};
\draw (.3,1) node {$1$};
\end{tikzpicture}
\end{figure}
1) Indiquer les coordonnées des points $A$, $B$ et $C$. 

2) Placer le point $D$ de coordonnées $(2; 5)$ et le point $E$ de coordonnées $(-3; 2)$. 

3) Placer le point $F$ qui a la même abscisse que $A$ et la même 
ordonnée que $B$. 



4) Placer le point $G$ dont l'abscisse est le tiers de la somme des 
abscisses de $A$, $B$ et $C$, et dont l'ordonnée est le tiers de la 
somme des ordonnées de $A$, $B$ et $C$. 

5) Tracer les droites $(AG)$, $(BG)$ et $(CG)$, et le triangle $ABC$. 

6) Que représentent les droites ainsi tracées pour le triangle $ABC$ ?
\newpage 

\begin{center}{\Large Interrogation chapitre 5}\end{center}

\subsection*{Exercice 1 ( 4 points)}
 
 Ranger dans l'ordre décroissant les nombres suivants. 
 

\[ 2,31 ; \ \ 2,1 ; \ \ -2,2 ; \ \ -2,3 ; \ \ 1 ; \ \ 0 ; \ \ -1,23 \]
 \subsection*{Exercice 2 (2 points)}
 
 Calculez : \begin{enumerate}
 \item $12 - 5 - (-6) $
 \item $ 5 - 2 - (15 + 12 + ( -7) - 4) $
 \end{enumerate}
 
\subsection*{Exercice 3 (4 points)}
Tracer une droite graduée de $18$ carreaux, avec l'origine $O$ au neuvième carreau. 

On place $1$ au douzième carreau (le troisième après $O$). 

Placer sur la règle les nombres suivants :
\[ 2 ; \ \ -1 ; \ \ -\frac73 ; \ \ \frac{14}{21}\]


\subsection*{Exercice 4 (4 points)} 
 
1) Indiquer les abscisses des points $B$, $C$, $D$ et $E$.
 
2) Placer sur la droite graduée le point $F$ d'abscisse $-2$. 

3) Quel point a une abscisse opposée de celle de $D$ ? 
 \begin{figure}[H]\center
\begin{tikzpicture}
\draw [->, >= latex] (-8, 0) -- (6, 0);
\draw (-7, -.2) -- (-7, .2);
\draw (-6, -.2) -- (-6, .2);
\draw (-5, -.2) -- (-5, .2);
\draw (-4, -.2) -- (-4, .2);
\draw(-3, -.2) -- (-3, .2);
\draw(-2, -.2) -- (-2, .2);
\draw (-1, -.2) -- (-1, .2);
\draw (5, -.2) -- (5, .2);
\draw (4, -.2) -- (4, .2);
\draw(3, -.2) -- (3, .2);
\draw(2, -.2) -- (2, .2);
\draw (1, -.2) -- (1, .2);
\draw (0, -.2) -- (0, .2);
\draw (0, .5) node {$O$};
\draw (0, -.5) node {$0$};
\draw (2, 0) node[cross=5pt, rotate = 0] {}; 
\draw (2, .5) node {$A$}; 
\draw (2, -.5) node {$1$}; 
\fill (-6, 0) node[cross=5pt, rotate = 0] {}; 
\draw (-6, .5) node {$B$}; 
\fill (5, 0) node[cross=5pt, rotate = 0] {};  
\draw (5, .5) node {$C$}; 
\fill (4, 0) node[cross=5pt, rotate = 0] {};  
\draw (4, .5) node {$D$}; 
\fill (-3, 0) node[cross=5pt, rotate = 0] {};  
\draw (-3, .5) node {$E$}; 
\end{tikzpicture}
\end{figure}

\subsection*{Exercice 5 (6 points)} 

\begin{figure}[H]\center
\begin{tikzpicture}
\draw[dotted](-5.5,-4.5) grid (7.5,6.5);
\draw[very thick, ->] (-5.5, 0)--(7.5,0);
\draw[very thick, ->] (0,-4.5)--(0,6.5);
\fill (2,-3) node[cross=5pt, rotate = 0] {};
\draw (2,-3) ++ (.25, .25) node {$A$};
\fill (-2,2) node[cross=5pt, rotate = 0] {};
\draw (-2,2) ++ (.25, .25) node {$B$};
\fill (6,4) node[cross=5pt, rotate = 0] {};
\draw (6,4) ++ (.25, .25) node {$C$};
\draw (1,.3) node {$1$};
\draw (.3,1) node {$1$};
\end{tikzpicture}
\end{figure}
1) Indiquer les coordonnées des points $A$, $B$ et $C$. 

2) Placer le point $D$ de coordonnées $(2; 5)$ et le point $E$ de coordonnées $(-3; 2)$. 

3) Placer le point $F$ qui a la même abscisse que $A$ et la même 
ordonnée que $C$. 



4) Placer le point $G$ dont l'abscisse est le tiers de la somme des 
abscisses de $A$, $B$ et $C$, et dont l'ordonnée est le tiers de la 
somme des ordonnées de $A$, $B$ et $C$. 

5) Tracer les droites $(AG)$, $(BG)$ et $(CG)$, et le triangle $ABC$. 

6) Que représentent les droites ainsi tracées pour le triangle $ABC$ ?
\end{document}