	\documentclass[14pt]{extreport}
\usepackage{extsizes}
	\usepackage[frenchb]{babel}
	\usepackage[utf8]{inputenc}  
	\usepackage[T1]{fontenc}
	\usepackage{amssymb}
	\usepackage[mathscr]{euscript}
	\usepackage{stmaryrd}
	\usepackage{amsmath}
	\usepackage{tikz}
	\usepackage[all,cmtip]{xy}
	\usepackage{amsthm}
	\usepackage{varioref}
	\usepackage[ margin=1in]{geometry}
	\geometry{a4paper}
	\usepackage{lmodern}
	\usepackage{hyperref}
	\usepackage{array}
	\usepackage{float}
	\usepackage{easytable}
	 \usepackage{fancyhdr}\usepackage{longtable}
	 \usetikzlibrary{shapes.misc}

\tikzset{cross/.style={cross out, draw=black, minimum size=2*(#1-\pgflinewidth), inner sep=0pt, outer sep=0pt},
%default radius will be 1pt. 
cross/.default={1pt}}

	\pagestyle{fancy}
	\theoremstyle{plain}
	\fancyfoot[C]{\empty} 
	\fancyhead[L]{Interrogation}
	\fancyhead[R]{16 janvier 2024}
	
	
	\title{Interrogation chapitre 4}
	\date{}
	\begin{document}

\begin{center}{\Large Interrogation chapitre 4}\end{center}

\subsection*{Exercice}

1) Parmi les consignes suivantes, laquelle permet de tracer un triangle ? Justifiez votre réponse. 

\[ AB = 3\text{ cm}, AC=8\text{ cm},BC = 4\text{ cm}\]
\[ AB = 5\text{ cm}, AC=8\text{ cm},BC = 6\text{ cm}\]
\[ AB = 5\text{ cm}, AC=10\text{ cm},BC = 4\text{ cm}\]
Réponse : \dotfill

\dotfill

\dotfill

2) Au dos de la feuille, tracez ledit triangle. 

3) Tracez en vert les médiatrices des trois côtés du triangle et leur point de concours $O$. 

4) Tracez le cercle passant par $A$, $B$ et $C$. \\Comment s'appelle ce cercle ? \dotfill 

5) Tracez en rouge les hauteurs issues de chacun des trois sommets du triangle et leur point de concours $H$. 

6) Tracez en bleu les médianes issues de chacun des trois sommets et leur point de concours $G$. 

7) Que remarque-t-on sur les points $O$, $H$ et $G$ ? \dotfill 

 \  \dotfill
 
 \newpage 
 

\begin{center}{\Large Interrogation chapitre 4}\end{center}

 
 
\subsection*{Exercice}

1) Parmi les consignes suivantes, laquelle permet de tracer un triangle ? Justifiez votre réponse. 

\[ AB = 6\text{ cm}, AC= 11\text{ cm},BC = 4\text{ cm}\]
\[ AB = 3\text{ cm}, AC=3 \text{ cm},BC = 7\text{ cm}\]
\[ AB = 7\text{ cm}, AC=10\text{ cm},BC = 9\text{ cm}\]
Réponse : \dotfill

\dotfill

\dotfill

2) Au dos de la feuille, tracez ledit triangle. 

3) Tracez en vert les médiatrices des trois côtés du triangle et leur point de concours $O$. 

4) Tracez le cercle passant par $A$, $B$ et $C$. \\Comment s'appelle ce cercle ? \dotfill 

5) Tracez en rouge les hauteurs issues de chacun des trois sommets du triangle et leur point de concours $H$. 

6) Tracez en bleu les médianes issues de chacun des trois sommets et leur point de concours $G$. 

7) Que remarque-t-on sur les points $O$, $H$ et $G$ ? \dotfill 

 \  \dotfill
\end{document}