\documentclass[14 pt]{extarticle}

	\usepackage[frenchb]{babel}
	\usepackage[utf8]{inputenc}  
	\usepackage[T1]{fontenc}
	\usepackage{amssymb}
	\usepackage[mathscr]{euscript}
	\usepackage{stmaryrd}
	\usepackage{amsmath}
	\usepackage{tikz}
	\usepackage[all,cmtip]{xy}
	\usepackage{amsthm}
	\usepackage{varioref}
	\usepackage{geometry}
	\geometry{a4paper}
	\usepackage{lmodern}
	\usepackage{hyperref}
	\usepackage{array}
	 \usepackage{fancyhdr}
	 \usepackage{float}
\renewcommand{\theenumi}{\alph{enumi})}
	\pagestyle{fancy}
	\theoremstyle{plain}
	\fancyfoot[C]{} 
	\fancyhead[L]{Contrôle}
	\fancyhead[R]{6 janvier 2025}\geometry{
 a4paper,
 total={170mm,257mm},
 left=20mm,
 top=20mm,
 }
	
	
	\title{Évaluation fractions}
	\date{}
	\begin{document}

\begin{center}{\Large Contrôle chapitre 3}\\ 
 \end{center}
 \subsection*{Exercice 1 (4 points)}
 Réduire au même dénominateur : 
 
 \begin{enumerate}
 \item $\frac13$ et $\frac49$
 \item $\frac1{10}$ et $\frac12$
 \item $\frac16$ et $\frac4{15}$
 \end{enumerate}

\subsection*{Exercice 2 (4,5 points)}
\begin{enumerate}
\item 
Réduire au même dénominateur : 
\( \frac23, \frac15, \frac4{15}\)
\item Classer dans l'ordre croissant les fractions précédentes. 
\end{enumerate}
\subsection*{Exercice 3 (4,5 points)}

Tracer une droite graduée de $14$ carreaux avec $1$ au quatrième carreau et y placer les fractions suivantes : 
\[ \frac34, \frac32, \frac{10}8, \frac5{10}\]

\subsection*{Exercice 4 (4,5 points)}

Pour chacune des fractions suivantes, donner son écriture décimale arrondie au centième, puis l'exprimer en pourcentage.
\begin{enumerate}
\item $\frac23$,
\item $\frac25$,
\item $\frac49$,
\item $\frac{1}{15}$. 
\end{enumerate}

\subsection*{Exercice 5 (2 points)}
Parmi les longueurs suivantes, lesquelles permettent de construire un triangle ? Justifier vos réponses.
\[ 2 \text{cm}, 4 \text{cm}, 7\text{cm}\]
\[ 3 \text{cm}, 5 \text{cm}, 7\text{cm}\]


 	\end{document}
