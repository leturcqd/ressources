	\documentclass[12 pt]{article}

	\usepackage[frenchb]{babel}
	\usepackage[utf8]{inputenc}  
	\usepackage[T1]{fontenc}
	\usepackage{amssymb}
	\usepackage[mathscr]{euscript}
	\usepackage{stmaryrd}
	\usepackage{amsmath}
	\usepackage{tikz}
	\usepackage[all,cmtip]{xy}
	\usepackage{amsthm}
	\usepackage{varioref}
	\usepackage{geometry}
	\geometry{a4paper}
	\usepackage{lmodern}
	\usepackage{hyperref}
	\usepackage{array}
	 \usepackage{fancyhdr}

	\pagestyle{fancy}
	\theoremstyle{plain}
	\fancyfoot[C]{\thepage} 
	\fancyhead[L]{Fiche d'exercices}
	\fancyhead[R]{2021-2022}
	\newcounter{n}
	\numberwithin{n}{section}
	\newtheorem{theo}{Théorème}
	\labelformat{theo}{théorème}
	\newtheorem{cor}[n]{Corollaire}
	\newtheorem{lm}{Lemme}
	\labelformat{lm}{lemme~#1}
	\newtheorem{hyp}[n]{Hypothèse}
	
	{\theoremstyle{definition}
	\newtheorem{df}[n]{Définition}
    \newtheorem{rmq}[n]{Remarque}
	\newtheorem{nt}[n]{Notation}
	}

	\renewcommand\epsilon{\varepsilon}
	\renewcommand\phi{\varphi}
	\newcommand\R{\mathbb{R}}
	\newcommand\s{\mathbb{S}}
	
	
	
	\title{Exercices Chapitre 4}
	\date{}
	\begin{document}

\begin{center}{\Large Chapitre 4 - Géométrie du triangle}\\ 
 \end{center}

\section{Inégalité triangulaire, triangles constructibles}

\begin{theo}[Inégalité triangulaire]
Si $ABC$ est un triangle non aplati\footnote{Un triangle est \emph{plat} ou \emph{aplati} si ses trois sommets sont alignés}, 
on a les trois inégalités suivantes : \[AB < AC + CB \] \[AC < AB + BC \] \[BC < BA + AB \]
\end{theo}

\begin{df}
Un triplet de longueurs $(a,b,c)$ est \emph{constructible} s'il est possible de construire un triangle avec $AB= c$, 
$BC=a$, et $AC=b$. 

Par exemple, $(3,4,5)$ est constructible car il existe un triangle de côtés $3$ cm, $4$ cm, et $5$ cm, mais $(1,1, 100000)$ ne l'est pas (il n'y a aucun triangle avec deux côtés de $1$ cm et un côté de $1$ km. 
\end{df}

\begin{theo}[Critère de constructibilité]
Le triplet de longueurs $(a,b,c)$ est constructible s'il vérifie les trois inégalités $a < b+c$, $b < a+c$, et $c<a+b$. 
En pratique, il suffit pour cela de vérifier que la plus grande des trois longueurs est plus petite que la somme des deux plus petites. 
\end{theo}
Par exemple, il existe un triangle avec des côtés de longueurs respectives $3$ cm, $4$ cm, et $5$ cm car la plus grande longueur ($5$ cm) est plus petite que la somme des deux autres ($4+3=7$ cm).

En revanche, il n'existe pas de triangle avec comme longueurs de côtés $3$ cm, $5$ cm et $9$ cm car $9 > 3+5$. 

\begin{rmq}
Ces inégalités traduisent la propriété suivante : « \emph{le chemin le plus court d'un point à un autre est toujours
le segment de droite} ». En effet, $AB <AC+CB$ signifie que l'on va plus vite de $A$ à $B$ en passant par le segment 
$[AB]$ qu'en faisant un détour par le point $C$. 
\end{rmq} 

\section{Droites particulières du triangle}
Dans cette partie, on étudie quatre familles de droites particulières du triangle, 
conduisant à quatre points particuliers de celui-ci. Tous les triangles considérés ici sont non aplatis, je ne le repréciserai pas à chaque énoncé. 
\subsection{Médiatrices (et cercle circonscrit)}

\begin{df}
La \emph{médiatrice} du segment $[AB]$ est la droite perpendiculaire à $(AB)$ passant par le milieu de $[AB]$.
\end{df}

\begin{theo}[Propriété des points de la médiatrice]\begin{itemize}
\item Si $M$ est un point de la médiatrice de $[AB]$, $AM = BM$. 
\item Si $AM=BM$, $M$ appartient à la médiatrice de $[AB]$.
\end{itemize}Autrement dit : \emph{les points de la médiatrice de'un segment sont exactement les points qui sont 
équidistants de ses deux extrémités}. 
\end{theo}

\begin{theo}
Dans un triangle $ABC$, les médiatrices des trois côtés du triangle sont concourantes\footnote{elles se rencontrent en un seul point}. 
Leur point de concours (d'intersection) $O$ est équidistant de $A$, $B$ et $C$. 
C'est donc le centre du seul cercle contenant les points $A$, $B$
et $C$. Ce cercle est appelé \emph{cercle circonscrit}\footnote{circum + scribere : écrit autour} du triangle $ABC$, et le point $O$ est donc 
\emph{le centre du cercle circonscrit}. 
\end{theo}

\begin{rmq}
Pour trouver le centre du cercle circonscrit, il suffit donc de tracer deux des médiatrices, et de regarder leur 
point d'intersection $O$. On obtient le cercle circonscrit en traçant le cercle de centre $O$ passant par n'importe lequel 
des sommets, qui passera automatiquement par les deux autres sommets. 
\end{rmq}

J'énonce aux passages deux corollaires de cette propriété du cercle circonscrit (un corollaire est un résultat qui se 
déduit d'un résultat plus important). 
\begin{cor}
\begin{enumerate}
\item Par trois points non alignés, il passe toujours un unique cercle.
\item Étant donné un cercle, on peut trouver son centre en traçant deux cordes et en cherchant le point d'intersection de leurs médiatrices.
\end{enumerate}
\end{cor}



\subsection{Hauteurs (et orthocentre)}

\begin{df}
Dans un triangle $ABC$, la \emph{hauteur issue du sommet $A$} (on dit parfois relative à $[BC]$)
est la droite passant par $A$ perpendiculaire à la droite $(BC)$.\footnote{J'insiste ici sur le fait qu'il faut 
bien ici parler de la droite $(BC)$ et non du segment $[BC]$, car si le triangle a un angle obtus, 
il a deux hauteurs qui ne rencontrent pas le côté auquel elles sont perpendiculaires, 
mais uniquement la droite le prolongeant.}

Le point d'intersection de cette hauteur et de la droite $(BC)$\footnote{même remarque} est appelé \emph{pied 
de la hauteur issue de $A$}. (C'est un point de la droite $(BC)$, mais pas nécessairement du côté $[BC]$. C'est le point où la hauteur forme son angle droit avec la droite $(BC)$.)
\end{df}

\begin{theo}
Avec les notations de la définition précédente, si on note $H$ le pied de la hauteur issue de $A$, l'aire du triangle 
$ABC$ est donnée par la formule \[\mathcal A = \frac{AH \times BC}2. \]\end{theo}

\begin{rmq}\begin{enumerate}
\item Souvent, cette formule est résumée par « base fois hauteur divisé par deux ». Évidemment, on obtient deux autres formules en considérant les hauteurs issues des deux autres sommets, qui donnent toutes trois le même résultat.
\item La formule ci-dessus entraîne le principe suivant : \emph{quand on déplace un sommet d'un triangle parallèlement
au côté qui lui est opposé, on ne change pas sa surface}. Ce principe est central pour démontrer la plupart des 
théorèmes de géométrie (Pythagore, Thalès, etc.) \end{enumerate}
\end{rmq}

\begin{theo}
Dans un triangle $ABC$, les hauteurs issues des trois sommets sont concourantes. Leur point de concours est appelé 
l'\emph{orthocentre} du triangle $ABC$. 
\end{theo}

\subsection{Médianes et centre de gravité}


\begin{df}
Dans un triangle $ABC$, la \emph{médiane issue du sommet $A$} (on dit parfois relative à $[BC]$)
est la droite passant par $A$ et le milieu du segment $[BC]$.
\end{df}

\begin{theo}
Dans un triangle $ABC$, les médianes issues des trois sommets sont concourantes. Leur point de concours est appelé 
le \emph{centre de gravité} du triangle $ABC$. 
\end{theo}

\begin{rmq}
Le centre de gravité du triangle est le point sur l'on peut faire reposer en équilibre un triangle solide (dont la 
masse est répartie de manière homogène). On parle pour cette raison parfois aussi de centre de masse/barycentre.
\end{rmq}


\subsection{Bissectrice et cercle inscrit}

\begin{df}
La \emph{bissectrice} d'un angle est la droite passant par son sommet et le séparant en deux angles adjacents égaux. 
\end{df}

\begin{rmq}
Pour tracer une bissectrice, on fixe l'écartement du compas, on le plante au sommet de l'angle, et on obtient ainsi deux points $A$ et $B$, un sur chaque côté de l'angle. Depuis ces deux points, on réutilise le compas, pour obtenir un 
point $C$, tel que $OA=OB=AB=AC$ (autrement dit $OACB$ est un losange). La bissectrice est alors obtenue en traçant la droite $(OC)$. 
\end{rmq}

\begin{theo}

Les bissectrices des trois angles d'un triangle sont concourantes. Leur point de concours est le centre du seul cercle 
tangent\footnote{Un cercle est \emph{tangent} à une droite quand il ne la touche qu'en un point.} aux trois côtés du triangle. Ce cercle est appelé \emph{cercle inscrit} du triangle $ABC$, et le point de concours est donc le \emph{centre du cercle inscrit}. 
\end{theo}

\begin{rmq}Si l'on veut tracer le cercle inscrit, il suffit donc de tracer deux bissectrices pour trouver 
son centre $I$. 
On doit cependant ensuite encore fixer le rayon. Pour ce faire, on trouve le point $K$ sur $[AB]$ tel que $(IK)$ et
$(AB)$ soient perpendiculaires. Le cercle inscrit est alors le cercle de centre $I$ passant par $K$.  
\end{rmq}


\subsection{[Culture générale/clairement hors programme] : Droite et cercle d'Euler}

On termine le chapitre avec deux théorèmes importants de géométrie du triangle mettant en jeu les notions précédentes. 

\begin{theo}[Droite d'Euler]
Dans tout triangle, le centre de gravité, le centre du cercle circonscrit, et l'orthocentre sont alignés. 

Dans un triangle équilatéral, ces trois points sont confondus (ils sont au même endroit).

Dans un triangle non équilatéral, ces points sont distincts, et la droite qui passe par eux est appelée \emph{droite
d'Euler} du triangle. 

En général, elle ne contient pas le centre du cercle inscrit.
\end{theo}


\begin{theo}[Cercle d'Euler]

Dans tout triangle, les milieux des trois côtés, et les pieds des trois hauteurs appartiennent tous les six à un 
même cercle. Ce cercle est appelé \emph{cercle d'Euler} du triangle. 

\end{theo}
	\end{document}
	
