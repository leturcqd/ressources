	\documentclass[12 pt]{article}

	\usepackage[frenchb]{babel}
	\usepackage[utf8]{inputenc}  
	\usepackage[T1]{fontenc}
	\usepackage{amssymb}
	\usepackage[mathscr]{euscript}
	\usepackage{stmaryrd}
	\usepackage{amsmath}
	\usepackage{tikz}
	\usepackage[all,cmtip]{xy}
	\usepackage{amsthm}
	\usepackage{varioref}
	\usepackage{geometry}
	\geometry{a4paper}
	\usepackage{lmodern}
	\usepackage{hyperref}
	\usepackage{array}
	 \usepackage{fancyhdr}

	\pagestyle{fancy}
	\theoremstyle{plain}
	\fancyfoot[C]{\thepage} 
	\fancyhead[L]{Fiche d'exercices}
	\fancyhead[R]{2021-2022}
	\newcounter{n}
	\numberwithin{n}{section}
	\newtheorem{df}[n]{Définition}
	\newtheorem{theo}{Théorème}
	\labelformat{theo}{théorème}
    \newtheorem{rmq}[n]{Remarque}
	\labelformat{rmq}{remarque~#1}
	\newtheorem{cor}[n]{Corollaire}
	\newtheorem{lm}{Lemme}
	\labelformat{lm}{lemme~#1}
	\newtheorem{hyp}[n]{Hypothèse}
	\newtheorem{nt}[n]{Notation}

	\renewcommand\epsilon{\varepsilon}
	\renewcommand\phi{\varphi}
	\newcommand\R{\mathbb{R}}
	\newcommand\s{\mathbb{S}}
	
	
	
	\title{Exercices Chapitre 4}
	\date{}
	\begin{document}

\begin{center}{\Large Chapitre 4 - Géométrie au compas : cercles, triangles, périmètres}\\ 
 \end{center}


\textbf{Exercice 1}

1) Tracer un segment $[OA]$ de longueur $3$ cm, et tracer le cercle $(\mathcal C)$ de centre $O$ passant par $A$.

2) Quel est le rayon de $({\mathcal C})$ ? Justifier\footnote{En géométrie, \emph{justifier} est distinct de \emph{mesurer} et d'\emph{observer} (ou \emph{constater}) : 
on \emph{observe} un résultat sur une figure, on \emph{mesure} une longueur ou un angle, 
mais on \emph{justifie} un résultat en utilisant des hypothèses de l'énoncé et des propriétés du cours. En particulier, une justification peut s'aider d'une figure, mais n'a jamais \textbf{besoin} de celle-ci.} la réponse. 

3) Soit\footnote{« être », au subjonctif : « soit » est équivalent à dire « on se donne ». Ainsi, la phrase « Soit $B$ un point de $(\mathcal C)$ » a deux effets : elle introduit un nouveau point $B$ dans les données de l'exercice 
et ajoute aux hypothèses que celui-ci appartient au cercle $(\mathcal C)$.} $B$ un point de $(\mathcal C)$. 
Quel est la nature du triangle $AOB$ ? Justifier la réponse. 

\textbf{Exercice 2}

1) Tracer un segment $[AB]$ de longueur $4$ cm et tracer le cercle $(\mathcal C)$ de diamètre $[AB]$. 

2) Soit $O$ le centre de $(\mathcal C)$. Justifier que $O$ est le milieu de $[AB]$. 

3) a. Placer un point $C$ sur le cercle $(\mathcal C)$ et tracer le triangle $ABC$.
 
b. Que remarque-t-on sur la figure à propos des droites $(AC)$ et $(BC)$ ? 

c. Quelle est donc la nature du triangle $ABC$ ? 

4) Recommencer la question $3$ deux fois en remplaçant le point $C$ par d'autres points ($D$ puis $E$) du cercle. 

5) Compléter la conjecture\footnote{Une conjecture est une propriété dont l'on prédit qu'elle est vraie par 
l'observation mais que l'on ne démontre (=justifie) pas.} suivante : « Si $M$ est un point du cercle de diamètre $[AB]$,
le triangle $AMB$ est $\ldots$. »

\textbf{Exercice 3}

1) Tracer un segment $[AB]$ de longueur $4$ cm, et les cercles $(\mathcal C_1)$ de centre $A$ passant par $B$ 
et $(\mathcal C_2)$ de centre $B$ passant par $A$. 

2) Placer les points d'intersection $C$ et $D$ de $(\mathcal C_1)$ et $(\mathcal C_2)$. 

3) Que dire des triangles $ABC$ et $ABD$ ? Justifier la réponse. 

4) Que dire du quadrilatère $ABCD$ ? Justifier la réponse. 

\textbf{Exercice 4}

1) Tracer un segment $[AB]$ de longueur $4$ cm, et le cercle $\mathcal C$ de diamètre $[AB]$. 

2) Tracer un autre diamètre $[CD]$ du cercle $\mathcal C$. 

3) Qu'observe-t-on à propos du quadrilatère $ACBD$ ? 

\textbf{Exercice 5}

1) Tracer un segment $[AB]$ de longueur $4$ cm, et le cercle $(\mathcal C)$ de diamètre $[AB]$. 

2) Tracer la droite $(\Delta)$ perpendiculaire à $(AB)$ passant par son milieu. 

3) Placer les points d'intersection $C$ et $D$ de la droite $(\Delta)$ et du cercle $(\mathcal C)$.

4) Qu'observe-t-on à propos du quadrilatère $ACBD$ ? 


\textbf{Exercice 6 - Médiatrice}

1) Tracer un segment $[AB]$ de longueur $4$ cm. 

2) Avec le compas, tracer deux cercles de même rayon respectivement centrés en $A$ et en $B$. 
Placer leurs points d'intersection $C$ et $D$, et tracer la droite $(CD)$. 

3) Que dire des droites $(CD)$ et $(AB)$ ? Que représente leur point d'intersection pour $[AB]$ ?

4) Pour un point quelconque $M$ de cette droite, mesurer les distances $AM$ et $BM$. 

\textbf{Remarque} : On observe donc que tout point équidistant de $A$ et $B$ est sur la droite perpendiculaire à
$[AB]$ passant par son milieu. Réciproquement, tout point de cette droite est équidistant de $A$ et $B$. 
Cette droite est appelée la \emph{médiatrice} du segment $[AB]$. 

\textbf{Exercice 7 - Cercle circonscrit à un triangle}

1) Tracer un triangle $ABC$. 

2) Tracer les médiatrices (voir remarque précédente) $(d_1)$, $(d_2)$ et $(d_3)$. des segments $[AB]$, $[BC]$ et $[AC]$. 

3) On observe que ces trois droites sont concourantes\footnote{Trois droites sont \emph{concourantes} si elles se rencontrent en un unique point commun.}. On va le démontrer. 

4)[Difficile] On montre d'abord que $(d_1)$ et $(d_2)$ sont sécantes. Si elles étaient parallèles, montrer que $(d_1)$ serait perpendiculaire à $(BC)$. En déduire que $(AB)$ et $(BC)$ seraient parallèles. Ceci étant faux, on en déduit bien que les droites $(d_1)$ et $(d_2)$ sont sécantes. 

5) Soit $O$ le point d'intersection de $(d_1)$ et $(d_2)$. Justifier (revoir la remarque précédente) que $O$ est à la même distance de $A$, $B$ et $C$. 

6) En déduire que $O$ appartient à $(d_3)$. On a donc montré que les trois médiatrices d'un triangle sont concourantes. 

7) a. Justifier que $A$, $B$ et $C$ sont sur le même cercle de centre $O$. 

b. Si $O'$ est un point tel que $A$, $B$ et $C$ soient sur un même cercle centré en $O'$, montrer que $O'$ est le point $O$. (On pourra démontrer qu'il appartient à chacune des trois médiatrices précédentes.)

\textbf{Conclusion} : On a démontré que tout triangle est inscrit\footnote{Un polygone est \emph{inscrit} dans un cercle lorsque tous ses sommets sont sur le cercle. On dit alors que le cercle est \emph{circonscrit} au polygone.} dans un unique cercle, dont le centre est le point d'intersection des médiatrices des trois sommets. Ce cercle est appelé le \emph{cercle circonscrit au triangle $ABC$}.
\newpage

\textbf{Exercice 8 - Non-existence d'un cercle circonscrit pour les quadrilatères}

Dans cet exercice, on vérifie que certains quadrilatères ne sont inscrits dans aucun cercle. 

1) Tracer un triangle $ABC$ tel que $AB=BC= 5$ cm et $AC= 3$ cm. 

2) Placer le point $D$ de telle sorte que $ABCD$ soit un losange.  

3) Tracer les diagonales $[AC]$ et $[BD]$ et vérifier que $(AC)$ est la médiatrice de $[BD]$, et $(BD)$ 
celle de $[AC]$. 

4) Supposons qu'il existe un cercle contenant $A$, $B$, $C$, et $D$, et notons $O$ son centre. 

a. Justifier que $O$ appartient à la médiatrice de tout segment dont les extrémités sont choisies parmi les quatre
points $A$, $B$, $C$ et $D$. 

b. En déduire que $O$ est l'intersection des deux diagonales. (Utiliser 3.)

c. Mesurer $OA$ et $OB$, et en déduire que $A$ et $B$ ne peuvent donc pas être sur le même cercle de centre $O$. 

\textbf{Conclusion} : On a donc vérifié que si un cercle contenait les quatre sommets de notre losange, il ne pouvait pas contenir $A$ et $B$ à la fois, ce qui est contradictoire. La seule possibilité est donc qu'aucun tel cercle n'existe : le losange $ABCD$ ne peut être inscrit dans aucun cercle. Généralement, un parallélogramme ne s'inscrit dans un cercle que si c'est un rectangle. Pour un losange, la seule possibilité est donc d'être un carré. 


	\end{document}
