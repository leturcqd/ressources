	\documentclass[12 pt]{article}

	\usepackage[frenchb]{babel}
	\usepackage[utf8]{inputenc}  
	\usepackage[T1]{fontenc}
	\usepackage{amssymb}
	\usepackage[mathscr]{euscript}
	\usepackage{stmaryrd}
	\usepackage{amsmath}
	\usepackage{tikz}
	\usepackage[all,cmtip]{xy}
	\usepackage{amsthm}
	\usepackage{varioref}
	\usepackage{geometry}
	\geometry{a4paper}
	\usepackage{lmodern}
	\usepackage{hyperref}
	\usepackage{array}
	 \usepackage{fancyhdr}

	\pagestyle{fancy}
	\theoremstyle{plain}
	\fancyfoot[C]{\empty} 
	\fancyhead[L]{Contrôle}
	\fancyhead[R]{9 fevrier 2022}
	
	
	\title{Contrôle chapitre 4}
	\date{}
	\begin{document}

\begin{center}{\Large Contrôle chapitre 4}\\ \textbf{Réponse aux questions de tracé sur la feuille de sujet. 
Tout le reste sur votre copie double, en indiquant les numéros de question. Rédigez en détail vos justifications. }\end{center}


\textbf{Cours} % 5 points

Dans un triangle $ABC$, rappeler la définition de : \begin{enumerate}
\item la médiane issue de $A$, 
\item la médiatrice de $[AB]$, 
\item la hauteur issue de $B$.
\end{enumerate}



\textbf{Exercice 1} % 4,5 points

Parmi les mesures suivantes, lesquelles permettent de construire un triangle ? (on ne demande pas de les construire)
\textbf{Justifiez la réponse.}

\[ AB = 4 \text{ cm}, AC = 3 \text{ cm}, BC = 2 \text{ cm}\]
\[ AB = 6 \text{ cm}, AC = 3 \text{ cm}, BC = 2 \text{ cm}\]
\[ AB = 4 \text{ cm}, AC = 3 \text{ cm}, BC = 9 \text{ cm}\]


\textbf{Exercice 2}  % 4,5 points

1) Sur le triangle ci-dessous, tracer les médiatrices des trois côtés. 
\[\begin{tikzpicture}
\draw (0,0) -- (4, 1) -- (5,5) -- (0,0);
\draw (-0.25,-.25) node {A} (4.25,.65) node {B} (5.25,5.25) node {C};
\end{tikzpicture}\]
2) Tracer le cercle passant par les points $A$, $B$ et $C$. 


\newpage

\textbf{Exercice 3}  % 4,5 points

1) Sur le triangle ci-dessous, tracer les bissectrices des trois angles. 
\[\begin{tikzpicture}
\draw (0,0) -- (3, .3) -- (5,5) -- (0,0);
\draw (-0.25,-.25) node {A} (3.25,.15) node {B} (5.25,5.25) node {C};
\end{tikzpicture}\]

2) Que peut-on dire des trois droites ainsi tracées ? Comment appelle-t-on leur point de concours ?

3)[Bonus] Tracer le cercle inscrit du triangle. 


\textbf{Exercice 4}

On considère un triangle $ABC$ isocèle en $A$, et on note $(d)$ la médiatrice de $[BC]$ et $I$ le milieu de $[BC]$.

1) Rappelez la propriété de cours concernant les points de la médiatrice d'un segment. 

2) Démontrer que $A$ appartient à la droite $(d)$. 

3) Démontrer que $(d)$ est aussi la hauteur issue de $A$, et la médiane issue de $A$. 

4) Démontrer que le centre de gravité, l'orthocentre, et le centre du cercle circonscrit sont sur la droite $(d)$. 


\end{document}