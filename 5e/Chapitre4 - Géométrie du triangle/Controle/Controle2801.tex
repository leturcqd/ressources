	\documentclass[14pt]{extreport}
\usepackage{extsizes}
	\usepackage[frenchb]{babel}
	\usepackage[utf8]{inputenc}  
	\usepackage[T1]{fontenc}
	\usepackage{amssymb}
	\usepackage[mathscr]{euscript}
	\usepackage{stmaryrd}
	\usepackage{amsmath}
	\usepackage{tikz}
	\usepackage[all,cmtip]{xy}
	\usepackage{amsthm}
	\usepackage{varioref}
	\usepackage[ margin=1in]{geometry}
	\geometry{a4paper}
	\usepackage{lmodern}
	\usepackage{hyperref}
	\usepackage{array}
	\usepackage{float}
	\usepackage{easytable}
\usepackage{stackengine}
	 \usepackage{fancyhdr}\usepackage{longtable}
	 \usetikzlibrary{shapes.misc}


\newcommand\xrowht[2][0]{\addstackgap[.5\dimexpr#2\relax]{\vphantom{#1}}}

	\pagestyle{fancy}
	\theoremstyle{plain}
	\fancyfoot[C]{\empty} 
	\fancyhead[L]{Contrôle}
	\fancyhead[R]{28 janvier 2024}
	
	
	\title{Contrôle chapitre 4}
	\date{}
	\begin{document}

\begin{center}{\Large Contrôle chapitre 4}\end{center}

\subsection*{Exercice 1 (6 points)}

Parmi les consignes suivantes, laquelle/lesquelles permet(tent) de tracer un triangle ? Justifiez votre réponse pour chaque triangle. \begin{enumerate}
\item $AB = 3\text{ cm}, AC=9\text{ cm},BC = 4\text{ cm}$
\item $AB = 13\text{ cm}, AC=8\text{ cm},BC = 4\text{ cm}$
\item $AB = 7\text{ cm}, AC=7\text{ cm},BC = 3\text{ cm}$
\end{enumerate}

\subsection*{Exercice 2 (6 points)}

Remplissez le tableau suivant. 

\begin{tabular}{|c|c|c|c|}
		\hline \xrowht{20pt}
		Nombre & signe & valeur absolue & opposé\\
        \hline\xrowht{20pt}
        $2$ &    &   &   \\ \hline \xrowht{20pt}
         $-4,1$ &   &   &   \\ \hline\xrowht{20pt}
        &   -& $3,3$  &   \\ \hline\xrowht{20pt}
          &   &   & $-2$  \\ \hline
\end{tabular} 


\subsection*{Exercice 3 (5 points)}

Soit ABC un triangle isocèle en C. 
\begin{enumerate}
\item Recopiez et complétez : « Un point appartient à la médiatrice du segment $[AB]$ exactement si ... »
\item  Montrez que le point $C$ appartient à la médiatrice de $[AB]$. 
\item Déduisez-en que la médiatrice de $[AB]$ est la hauteur issue de $C$. 
\end{enumerate}



\subsection*{Exercice 4 (3 points +1)}

Soit ABCD un quadrilatère avec des angles droits en $A$ et $B$, des longueurs $AD< BC$, et notons $E$ le point de $[BC]$ tel que $\widehat{DEC}$ soit droit. 
\begin{enumerate}
\item  Faites une figure.
\item Calculez l'aire de $ABED$ en fonction de $AB$ et $BE$.
\item Calculez l'aire de $EDC$ en fonction de $EC$ et $AB$. 
\item Donnez une formule de l'aire totale en fonction de $AB$, $AD$ et $BC$.
\end{enumerate}


\newpage

\begin{center}{\Large Contrôle chapitre 4}\end{center}

\subsection*{Exercice 1 (6 points)}

Parmi les consignes suivantes, laquelle/lesquelles permet(tent) de tracer un triangle ? Justifiez votre réponse pour chaque triangle. \begin{enumerate}
\item $AB = 6\text{ cm}, AC=9\text{ cm},BC = 4\text{ cm}$
\item $AB = 11\text{ cm}, AC=8\text{ cm},BC = 4\text{ cm}$
\item $AB = 5\text{ cm}, AC=3\text{ cm},BC = 5\text{ cm}$
\end{enumerate}

\subsection*{Exercice 2 (6 points)}

Remplissez le tableau suivant. 

\begin{tabular}{|c|c|c|c|}
		\hline \xrowht{20pt}
		Nombre & signe & valeur absolue & opposé\\
        \hline\xrowht{20pt}
        $-5$ &    &   &   \\ \hline \xrowht{20pt}
         $2,4$ &   &   &   \\ \hline\xrowht{20pt}
        & -  & $6$  &   \\ \hline\xrowht{20pt}
          &   &   & $+3,1$  \\ \hline
\end{tabular} 


\subsection*{Exercice 3 (5 points)}

Soit ABC un triangle isocèle en B. 
\begin{enumerate}
\item Recopiez et complétez : « Un point appartient à la médiatrice du segment $[AC]$ exactement si ... »
\item  Montrez que le point $B$ appartient à la médiatrice de $[AC]$. 
\item Déduisez-en que la médiatrice de $[AC]$ est la hauteur issue de $B$. 
\end{enumerate}



\subsection*{Exercice 4 (3 points +1)}

Soit ABCD un quadrilatère avec des angles droits en $A$ et $B$, des longueurs $AD> BC$, et notons $E$ le point de $[AD]$ tel que $\widehat{AEC}$ soit droit. 
\begin{enumerate}
\item  Faites une figure.
\item Calculez l'aire de $ABCE$ en fonction de $AB$ et $AE$.
\item Calculez l'aire de $EDC$ en fonction de $ED$ et $AB$. 
\item Donnez une formule de l'aire totale en fonction de $AB$, $AD$ et $BC$.
\end{enumerate}
\end{document}