\documentclass[14 pt]{extarticle}

	\usepackage[frenchb]{babel}
	\usepackage[utf8]{inputenc}  
	\usepackage[T1]{fontenc}
	\usepackage{amssymb}
	\usepackage[mathscr]{euscript}
	\usepackage{stmaryrd}
	\usepackage{amsmath}
	\usepackage{tikz}
	\usepackage[all,cmtip]{xy}
	\usepackage{amsthm}
	\usepackage{varioref}
	\usepackage{geometry}
	\geometry{a4paper}
	\usepackage{lmodern}
	\usepackage{hyperref}
	\usepackage{array}
	 \usepackage{fancyhdr}
	 \usepackage{float}
\renewcommand{\theenumi}{\alph{enumi})}
	\pagestyle{fancy}
	\theoremstyle{plain}
	\fancyfoot[C]{} 
	\fancyhead[L]{Contrôle}
	\fancyhead[R]{16 janvier 2022}\geometry{
 a4paper,
 total={170mm,257mm},
 left=20mm,
 top=20mm,
 }
	
	
	\title{Interrogation chapitre 4}
	\date{}
	\begin{document}

\begin{center}{\Large Interrogation chapitre 4 - corrigé}\\ 
 \end{center}
 
 
 \subsection*{Exercice 1 (4 points)}
 Recopiez et complétez. 
 \begin{enumerate}
 \item La médiatrice du segment $[AB]$ est {\color{red}{la droite qui coupe le segment en son milieu perpendiculairement.}}
 \item Les points de la médiatrice de $[AB]$ sont exactement les points qui {\color{red}{sont à la même distance de $A$ et de $B$.}}
 \end{enumerate}
 
 \subsection*{Exercice 2 (6 points)}
 
Parmi les longueurs suivantes, lesquelles permettent de construire un triangle ? Justifiez vos réponses. Quand cela est possible, tracez le triangle.
\begin{enumerate}
\item $AB = 4$ cm, $AC = 3$ cm, $BC = 5$ cm.
{\color{red}Le triangle est constructible car $AC + AB = 3 + 4 = 7 \text{ cm} > 5 \text{ cm} = BC$.}


\item $AB = 4$ cm, $AC = 3$ cm, $BC = 8$ cm. {\color{red}Le triangle n'est pas constructible car $AC + AB = 3 + 4 = 7 \text{ cm} < 8 \text{ cm} = BC$.}
\item $AB = 1$ cm, $AC = 3$ cm, $BC = 5$ cm. {\color{red}Le triangle n'est pas constructible car $AC + AB = 1 + 3 = 4 \text{ cm} < 5 \text{ cm} = BC$.}
\end{enumerate}


\subsection*{Exercice 3 (4 points)}
\begin{enumerate}
\item Calculez l'écriture décimale de $\frac2{11}$. {\color{red} On trouve $2\div 11 = 0,18181818\ldots$}.
\item Réduisez $\frac25$ et $\frac32$ au même dénominateur. 
{\color{red} On trouve $\frac25 = \frac4{10}$ et $\frac32 = \frac{15}{10}$.}
\item Simplifiez $\frac{210}{735}$ le plus possible. 
{\color{red} En simplifiant par $5$, puis $3$, puis $7$, on trouve $\frac{2}{7}$.}
\end{enumerate}


\subsection*{Exercice 4 (6 points)}
\begin{enumerate}
\item Tracez un triangle $ABC$, isocèle en $A$, avec $AB = 4$ cm et $BC = 6$ cm. 
\item Tracez la médiatrice $(d)$ du segment $[BC]$. 
\item \textbf{Démontrez à l'aide d'une propriété du cours} que le point $A$ appartient à la droite $(d)$. 

{\color{red} Le triangle $ABC$ est isocèle en $A$ donc $AB=AC$. 

Un point qui est situé à la même distance de $B$ et de $C$ est sur la médiatrice de $[BC]$, donc $A\in (d)$. }
\item Que représente la droite $(d)$ pour le triangle $ABC$ ?
{\color{red} Comme $A\in(d)$, $A$ est son propre symétrique par rapport à $(d)$. 

Comme $(d)$ est la médiatrice de $[BC]$, par définition de la symétrie axiale (revoir chap.2), $B$ et $C$ sont symétriques par rapport à $(d)$. 

Donc la symétrie d'axe $(d)$ envoie $A$, $B$ et $C$ sur $A$, $C$ et $B$. Le triangle $ABC$ admet donc $(d)$ comme axe de symétrie. }
\end{enumerate}

\newpage 


\begin{center}{\Large Interrogation chapitre 4}\\ 
 \end{center}
 
 \subsection*{Exercice 1 (4 points)}
 Recopiez et complétez. 
 \begin{enumerate}
 \item La médiatrice du segment $[AB]$ est {\color{red}{la droite qui coupe le segment en son milieu perpendiculairement.}}
 \item Les points de la médiatrice de $[AB]$ sont exactement les points qui {\color{red}{sont à la même distance de $A$ et de $B$.}}
 \end{enumerate}
 
 
 \subsection*{Exercice 2 (6 points)}
 
Parmi les longueurs suivantes, lesquelles permettent de construire un triangle ? Justifiez vos réponses. Quand cela est possible, tracez le triangle.
\begin{enumerate}
\item $AB = 4$ cm, $AC = 2$ cm, $BC = 7$ cm. {\color{red}Le triangle n'est pas constructible car $AC + AB = 2 + 4 = 6 \text{ cm} < 7 \text{ cm} = BC$.}
\item $AB = 2$ cm, $AC = 3$ cm, $BC = 6$ cm. {\color{red}Le triangle n'est pas constructible car $AB + AC = 2 + 3 = 5 \text{ cm} < 6 \text{ cm} = BC$.}
\item $AB = 4$ cm, $AC = 3$ cm, $BC = 5$ cm. {\color{red}Le triangle est constructible car $AB + AC = 3 + 4 = 7 \text{ cm} > 5 \text{ cm} = BC$.}
\end{enumerate}


\subsection*{Exercice 3 (4 points)}
\begin{enumerate}
\item Calculez l'écriture décimale de $\frac7{11}$. {\color{red} On trouve $7\div 11 = 0,63636363\ldots$}.
\item Réduisez $\frac45$ et $\frac32$ au même dénominateur. 
{\color{red} On trouve $\frac45 = \frac8{10}$ et $\frac32 = \frac{15}{10}$.}
\item Simplifiez $\frac{420}{735}$ le plus possible. 
{\color{red} En simplifiant par $5$, puis $3$, puis $7$, on trouve $\frac{4}{7}$.}
\end{enumerate}


\subsection*{Exercice 4 (6 points)}
\begin{enumerate}
\item Tracez un triangle $ABC$, isocèle en $B$, avec $AC = 4$ cm et $BC = 5$ cm. 
\item Tracez la médiatrice $(d)$ du segment $[AC]$. 
\item \textbf{Démontrez à l'aide d'une propriété du cours} que le point $B$ appartient à la droite $(d)$. {\color{red} Le triangle $ABC$ est isocèle en $B$ donc $AB=BC$. 

Un point qui est situé à la même distance de $A$ et de $C$ est sur la médiatrice de $[AC]$, donc $B\in (d)$. }
\item Que représente la droite $(d)$ pour le triangle $ABC$ ?
{\color{red} Comme $B\in(d)$, $B$ est son propre symétrique par rapport à $(d)$. 

Comme $(d)$ est la médiatrice de $[AC]$, par définition de la symétrie axiale (revoir chap.2), $A$ et $C$ sont symétriques par rapport à $(d)$. 

Donc la symétrie d'axe $(d)$ envoie $A$, $B$ et $C$ sur $C$, $B$ et $A$. Le triangle $ABC$ admet donc $(d)$ comme axe de symétrie. }
\end{enumerate}


 	\end{document}
