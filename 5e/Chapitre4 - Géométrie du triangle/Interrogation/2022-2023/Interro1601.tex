\documentclass[14 pt]{extarticle}

	\usepackage[frenchb]{babel}
	\usepackage[utf8]{inputenc}  
	\usepackage[T1]{fontenc}
	\usepackage{amssymb}
	\usepackage[mathscr]{euscript}
	\usepackage{stmaryrd}
	\usepackage{amsmath}
	\usepackage{tikz}
	\usepackage[all,cmtip]{xy}
	\usepackage{amsthm}
	\usepackage{varioref}
	\usepackage{geometry}
	\geometry{a4paper}
	\usepackage{lmodern}
	\usepackage{hyperref}
	\usepackage{array}
	 \usepackage{fancyhdr}
	 \usepackage{float}
\renewcommand{\theenumi}{\alph{enumi})}
	\pagestyle{fancy}
	\theoremstyle{plain}
	\fancyfoot[C]{} 
	\fancyhead[L]{Contrôle}
	\fancyhead[R]{16 janvier 2022}\geometry{
 a4paper,
 total={170mm,257mm},
 left=20mm,
 top=20mm,
 }
	
	
	\title{Interrogation chapitre 4}
	\date{}
	\begin{document}

\begin{center}{\Large Interrogation chapitre 4}\\ 
 \end{center}
 
 
 \subsection*{Exercice 1 (4 points)}
 Recopiez et complétez. 
 \begin{enumerate}
 \item La médiatrice du segment $[AB]$ est \ldots
 \item Les points de la médiatrice de $[AB]$ sont exactement les points qui\ldots. 
 \end{enumerate}
 
 \subsection*{Exercice 2 (6 points)}
 
Parmi les longueurs suivantes, lesquelles permettent de construire un triangle ? Justifiez vos réponses. Quand cela est possible, tracez le triangle.
\begin{enumerate}
\item $AB = 4$ cm, $AC = 3$ cm, $BC = 5$ cm ;
\item $AB = 4$ cm, $AC = 3$ cm, $BC = 8$ cm ;
\item $AB = 1$ cm, $AC = 3$ cm, $BC = 5$ cm. 
\end{enumerate}


\subsection*{Exercice 3 (4 points)}
\begin{enumerate}
\item Calculez l'écriture décimale de $\frac2{11}$. 
\item Réduisez $\frac25$ et $\frac32$ au même dénominateur. 
\item Simplifiez $\frac{210}{735}$ le plus possible. 
\end{enumerate}


\subsection*{Exercice 4 (6 points)}
\begin{enumerate}
\item Tracez un triangle $ABC$, isocèle en $A$, avec $AB = 4$ cm et $BC = 6$ cm. 
\item Tracez la médiatrice $(d)$ du segment $[BC]$. 
\item \textbf{Démontrez à l'aide d'une propriété du cours} que le point $A$ appartient à la droite $(d)$. 
\item Que représente la droite $(d)$ pour le triangle $ABC$ ?
\end{enumerate}

\newpage 


\begin{center}{\Large Interrogation chapitre 4}\\ 
 \end{center}
 
 \subsection*{Exercice 1 (4 points)}
 Recopiez et complétez. 
 \begin{enumerate}
 \item La médiatrice du segment $[AB]$ est \ldots
 \item Les points de la médiatrice de $[AB]$ sont exactement les points qui\ldots. 
 \end{enumerate}
 
 \subsection*{Exercice 2 (6 points)}
 
Parmi les longueurs suivantes, lesquelles permettent de construire un triangle ? Justifiez vos réponses. Quand cela est possible, tracez le triangle.
\begin{enumerate}
\item $AB = 4$ cm, $AC = 2$ cm, $BC = 7$ cm ;
\item $AB = 2$ cm, $AC = 3$ cm, $BC = 6$ cm ;
\item $AB = 4$ cm, $AC = 3$ cm, $BC = 5$ cm. 
\end{enumerate}


\subsection*{Exercice 3 (4 points)}
\begin{enumerate}
\item Calculez l'écriture décimale de $\frac7{11}$. 
\item Réduisez $\frac45$ et $\frac32$ au même dénominateur. 
\item Simplifiez $\frac{420}{735}$ le plus possible. 
\end{enumerate}


\subsection*{Exercice 4 (6 points)}
\begin{enumerate}
\item Tracez un triangle $ABC$, isocèle en $B$, avec $AC = 4$ cm et $BC = 5$ cm. 
\item Tracez la médiatrice $(d)$ du segment $[AC]$. 
\item \textbf{Démontrez à l'aide d'une propriété du cours} que le point $B$ appartient à la droite $(d)$. 
\item Que représente la droite $(d)$ pour le triangle $ABC$ ?
\end{enumerate}


 	\end{document}
