\documentclass[14 pt]{extarticle}

	\usepackage[frenchb]{babel}
	\usepackage[utf8]{inputenc}  
	\usepackage[T1]{fontenc}
	\usepackage{amssymb}
	\usepackage[mathscr]{euscript}
	\usepackage{stmaryrd}
	\usepackage{amsmath}
	\usepackage{tikz}
	\usepackage[all,cmtip]{xy}
	\usepackage{amsthm}
	\usepackage{varioref}
	\usepackage{geometry}
	\geometry{a4paper}
	\usepackage{lmodern}
	\usepackage{hyperref}
	\usepackage{array}
	 \usepackage{fancyhdr}
	 \usepackage{float}
\renewcommand{\theenumi}{\alph{enumi})}
	\pagestyle{fancy}
	\theoremstyle{plain}
	\fancyfoot[C]{} 
	\fancyhead[L]{Interrogation}
	\fancyhead[R]{18 novembre 2024}\geometry{
 a4paper,
 total={170mm,257mm},
 left=20mm,
 top=20mm,
 }
	
	
	\title{Interrogation chapitre 3}
	\date{}
	\begin{document}

\begin{center}{\Large Interrogation chapitre 3}\\ 
 \end{center}
 \subsection*{Exercice 1 (3 points)}
 
 \begin{enumerate}
 \item Donner une fraction dont le dénominateur est $3$ et le numérateur est $4$. 
 \item Donner une fraction dont le numérateur est impair. 
 \item Donner une fraction dont le dénominateur est le double du numérateur. 
 \end{enumerate}
 
 \subsection*{Exercice 2 (7 points)}
Trouver l'écriture décimale des fractions suivantes (si elle est infinie, entourez la période se répétant): 
\begin{enumerate}
\item $\frac18$
\item $\frac53$
\item $\frac{9}{40}$
\item $\frac7{11}$
\item $\frac5{12}$
\item $\frac47$
\item $\frac{432}{100}$
\end{enumerate}
 \subsection*{Exercice 3 (5 points)}
 
 Simplifier les fractions suivantes : 
 \begin{enumerate}
 \item $\frac{16}{32}$
 \item $\frac{315}{1050}$
 \item $\frac{25}{75}$
 \item $\frac{42}{49}$
 \item $\frac{176}{495}$
 \end{enumerate}
 
 
 \subsection*{Exercice 4 (5 points)}
 
 Effectuer les calculs suivants en détaillant les étapes : 
 \begin{enumerate}
 \item $2 + 5 \times 4$
 \item $ 16 \div 2 \times 2$
 \item $ 30 \div 2 + 4$
 \item $ (15 - 1 + 4) \div 2 + 7$
 \item $ 5 + 2 \times 3 + 4$
 \end{enumerate} 
 
\newpage 

\begin{center}{\Large Interrogation chapitre 3}\\ 
 \end{center}
 \subsection*{Exercice 1 (3 points)}
 
 \begin{enumerate}
 \item Donner une fraction dont le dénominateur est $5$ et le numérateur est $4$. 
 \item Donner une fraction dont le numérateur est pair. 
 \item Donner une fraction dont le dénominateur est le triple du numérateur. 
 \end{enumerate}
 
 \subsection*{Exercice 2 (7 points)}
Trouver l'écriture décimale des fractions suivantes (si elle est infinie, entourez la période se répétant): 
\begin{enumerate}
\item $\frac38$
\item $\frac53$
\item $\frac{11}{40}$
\item $\frac2{11}$
\item $\frac5{12}$
\item $\frac47$
\item $\frac{142}{100}$
\end{enumerate}
 \subsection*{Exercice 3 (5 points)}
 
 Simplifier les fractions suivantes : 
 \begin{enumerate}
 \item $\frac{18}{32}$
 \item $\frac{315}{735}$
 \item $\frac{50}{75}$
 \item $\frac{42}{49}$
 \item $\frac{176}{495}$
 \end{enumerate}
 
 
 \subsection*{Exercice 4 (5 points)}
 
 Effectuer les calculs suivants en détaillant les étapes : 
 \begin{enumerate}
 \item $2 + 4 \times 4$
 \item $ 16 \div 2 \times 2$
 \item $ 30 \div 2 + 4$
 \item $ (15 - 1 + 4) \div 2 + 7$
 \item $ 5 + 2 \times 3 + 4$
 \end{enumerate} 
 
 	\end{document}
 	
