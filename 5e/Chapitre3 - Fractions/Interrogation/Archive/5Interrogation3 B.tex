\documentclass[14 pt]{extarticle}

	\usepackage[frenchb]{babel}
	\usepackage[utf8]{inputenc}  
	\usepackage[T1]{fontenc}
	\usepackage{amssymb}
	\usepackage[mathscr]{euscript}
	\usepackage{stmaryrd}
	\usepackage{amsmath}
	\usepackage{tikz}
	\usepackage[all,cmtip]{xy}
	\usepackage{amsthm}
	\usepackage{varioref}
	\usepackage{geometry}
	\geometry{a4paper}
	\usepackage{lmodern}
	\usepackage{hyperref}
	\usepackage{array}
	 \usepackage{fancyhdr}
	 \usepackage{float}
\renewcommand{\theenumi}{\alph{enumi})}
	\pagestyle{fancy}
	\theoremstyle{plain}
	\fancyfoot[C]{} 
	\fancyhead[L]{Contrôle}
	\fancyhead[R]{1 novembre 2022}\geometry{
 a4paper,
 total={170mm,257mm},
 left=20mm,
 top=20mm,
 }
	
	
	\title{Interrogation chapitre 3}
	\date{}
	\begin{document}

\begin{center}{\Large Interrogation chapitre 3}\\ 
 \end{center}
 Nom : \\
 Prénom : \\
 \subsection*{Exercice 1 (6 points)}
 

\begin{enumerate}
\item Écrire une fraction ayant $4$ comme dénominateur. 
\item Écrire une fraction ayant $7$ comme numérateur. 
\item Écrire une fraction égale à $5$ ayant $7$ comme dénominateur.
\item Écrire une fraction égale à $4$ ayant $12$ comme numérateur. 
\end{enumerate}

\subsection*{Exercice 2 (6 points)}
Pour chacune des fractions suivantes, 
rechercher son écriture décimale. Si celle-ci est finie, la donner en entier. Sinon, soulignez le motif qui se répète. 

\begin{enumerate}
\item $\frac25$
\item $\frac1{625}$ 
\item $\frac{6}{7}$
\end{enumerate}
\subsection*{Exercice 3 (4 points)}

Compléter les pointillés. 

\begin{enumerate}
\item $\frac37= = \frac{3 \times \ldots}{7 \times \ldots}= 
\frac{\ldots}{42}$
\item $\frac1{9}= \frac{11}{\ldots}$
\item $\frac{56}{98}= \frac\ldots{\ldots}$  (plusieurs choix possibles : choisir celui avec les plus petits numérateur et dénominateur entiers)
\end{enumerate}

\subsection*{Exercice 4 (4 points)}

Regroupez les fractions suivantes par fractions égales. 

\[ \frac25, \ \  
 \frac{12}{15}, \ \    
 \frac{12}{28}, \ \  
 \frac{24}{30}, \ \  
  \frac{24}{60}, \ \  
   \frac37, \ \  
      \frac{12}{30}, \ \  
       \frac{4}{5}\]


 	\end{document}
