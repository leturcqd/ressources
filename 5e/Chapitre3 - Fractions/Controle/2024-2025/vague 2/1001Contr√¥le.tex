\documentclass[14 pt]{extarticle}

	\usepackage[frenchb]{babel}
	\usepackage[utf8]{inputenc}  
	\usepackage[T1]{fontenc}
	\usepackage{amssymb}
	\usepackage[mathscr]{euscript}
	\usepackage{stmaryrd}
	\usepackage{amsmath}
	\usepackage{tikz}
	\usepackage[all,cmtip]{xy}
	\usepackage{amsthm}
	\usepackage{varioref}
	\usepackage{geometry}
	\geometry{a4paper}
	\usepackage{lmodern}
	\usepackage{hyperref}
	\usepackage{array}
	 \usepackage{fancyhdr}
	 \usepackage{float}
\renewcommand{\theenumi}{\alph{enumi})}
	\pagestyle{fancy}
	\theoremstyle{plain}
	\fancyfoot[C]{} 
	\fancyhead[L]{Nom, prénom : }
	\fancyhead[R]{}\geometry{
 a4paper,
 total={170mm,257mm},
 left=10mm,
 top=15mm,
 right = 10 mm,
 bottom = 10 mm
 }
	
	
	\title{Contrôle chapitre 3 étape 2}
	\date{}
	\begin{document}
% Nom : \phantom{Blablablabablablabalb}
% Prénom : 
 \subsection*{Exercice 1 (4 points)}
 
Compléter les pointillés.
\begin{enumerate}
\item $\displaystyle\frac59=\frac{5\times \ldots}{9\times \ldots} = \frac{25}\ldots$
\item $\displaystyle\frac57=\frac\ldots{56}$
\item $\displaystyle\frac34=\frac{\ldots}{12}$
\end{enumerate}
\subsection*{Exercice 2 (4 points)}
Réduire les fractions suivantes au même dénominateur, et les classer dans l'ordre croissant. 
\[ \frac5{12}, \frac7{30}, \frac9{20}, \frac7{60}\]
\subsection*{Exercice 3 (4 points)}

Sur la droite graduée suivante, placer les fractions (détailler le raisonnement sur la copie) : 
\[ \frac56, \frac{26}{12}, \frac{54}{36}, \frac{48}{72}\]
\begin{center}
\begin{tikzpicture}
\draw[->, >= latex] (0,0) -- (19,0);
\draw (0,-.2) -- (0,.2);
\draw (6,-.2) -- (6,.2);
\draw (12,-.2) -- (12,.2);
\draw (18,-.2) -- (18,.2);
\draw (1,-.1) -- (1,.1);
\draw (2,-.1) -- (2,.1);
\draw (3,-.1) -- (3,.1);
\draw (4,-.1) -- (4,.1);
\draw (5,-.1) -- (5,.1);
\draw (10,-.1) -- (10,.1);
\draw (9,-.1) -- (9,.1);
\draw (6,-.1) -- (6,.1);
\draw (7,-.1) -- (7,.1);
\draw (8,-.1) -- (8,.1);
\draw (11,-.1) -- (11,.1);
\draw (12,-.1) -- (12,.1);
\draw (13,-.1) -- (13,.1);
\draw (14,-.1) -- (14,.1);
\draw (15,-.1) -- (15,.1);
\draw (16,-.1) -- (16,.1);
\draw (17,-.1) -- (17,.1);
\draw (0,-.5) node{$0$};
\draw (6,-.5) node{$1$};
\draw (12,-.5) node{$2$};
\draw (18,-.5) node{$3$};
\end{tikzpicture}\end{center}
\subsection*{Exercice 4 (4 points)}

Pour chacune des fractions suivantes, donner son écriture décimale et l'exprimer comme un pourcentage, arrondi à l'unité. 
\begin{enumerate}
\item $\displaystyle\frac23$,
\item $\displaystyle\frac45$,
\item $\displaystyle\frac89$,
\item $\displaystyle\frac{2}{15}$. 
\end{enumerate}

\subsection*{Exercice 5 (4 points)}
\begin{enumerate}
\item Peut-on construire un triangle dont les côtés mesurent $2$ cm, 3 cm et 6 cm ? Justifier votre réponse.
\item Peut-on construire un triangle dont les côtés mesurent 4 cm, 3 cm et 5 cm ? Justifier votre réponse.
\end{enumerate}
 	\end{document}
