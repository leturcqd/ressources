\documentclass[14 pt]{extarticle}

	\usepackage[frenchb]{babel}
	\usepackage[utf8]{inputenc}  
	\usepackage[T1]{fontenc}
	\usepackage{amssymb}
	\usepackage[mathscr]{euscript}
	\usepackage{stmaryrd}
	\usepackage{amsmath}
	\usepackage{tikz}
	\usepackage[all,cmtip]{xy}
	\usepackage{amsthm}
	\usepackage{varioref}
	\usepackage{geometry}
	\geometry{a4paper}
	\usepackage{lmodern}
	\usepackage{hyperref}
	\usepackage{array}
	 \usepackage{fancyhdr}
	 \usepackage{float}
\renewcommand{\theenumi}{\alph{enumi})}
	\pagestyle{fancy}
	\theoremstyle{plain}
	\fancyfoot[C]{} 
	\fancyhead[L]{Contrôle}
	\fancyhead[R]{10 décembre 2024}\geometry{
 a4paper,
 total={170mm,257mm},
 left=20mm,
 top=20mm,
 }
	
	
	\title{Contrôle chapitre 3}
	\date{}
	\begin{document}

\begin{center}{\Large Contrôle chapitre 3}\\ 
 \end{center}
 Nom : \\
 Prénom : 
 \subsection*{Exercice 1 (4 points)}
 
Compléter les pointillés.
\begin{enumerate}
\item $\frac29=\frac{2\times \ldots}{9\times \ldots} = \frac{12}\ldots$
\item $\frac34=\frac\ldots{8}$
\item $\frac23=\frac{\ldots}{12}$
\end{enumerate}

\subsection*{Exercice 2 (4 points)}
Réduire les fractions suivantes au même dénominateur, et les classer dans l'ordre croissant. 
\[ \frac49, \frac23, \frac56, \frac7{18}\]
\subsection*{Exercice 3 (4 points)}

Sur la droite graduée suivante, placer les fractions (détailler le raisonnement sur la copie) : 
\[ \frac25, \frac6{10}, \frac{36}{30}, \frac{225}{125}\]

\begin{tikzpicture}
\draw[->, >= latex] (0,0) -- (16,0);
\draw (0,-.2) -- (0,.2);
\draw (5,-.2) -- (5,.2);
\draw (10,-.2) -- (10,.2);
\draw (15,-.2) -- (15,.2);
\draw (1,-.1) -- (1,.1);
\draw (2,-.1) -- (2,.1);
\draw (3,-.1) -- (3,.1);
\draw (4,-.1) -- (4,.1);
\draw (9,-.1) -- (9,.1);
\draw (6,-.1) -- (6,.1);
\draw (7,-.1) -- (7,.1);
\draw (8,-.1) -- (8,.1);
\draw (11,-.1) -- (11,.1);
\draw (12,-.1) -- (12,.1);
\draw (13,-.1) -- (13,.1);
\draw (14,-.1) -- (14,.1);
\draw (0,-.5) node{$0$};
\draw (5,-.5) node{$1$};
\draw (10,-.5) node{$2$};
\draw (15,-.5) node{$3$};
\end{tikzpicture}

\subsection*{Exercice 4 (4 points)}

Pour chacune des fractions suivantes, donner son écriture décimale et l'exprimer comme un pourcentage, arrondi à l'unité. 
\begin{enumerate}
\item $\frac13$,
\item $\frac35$,
\item $\frac79$,
\item $\frac{4}{15}$. 
\end{enumerate}

\subsection*{Exercice 5 (4 points)}
Regrouper les fractions suivantes par fractions égales. 

\[\frac25, \frac6{10}, \frac{25}{30},\frac{10}{15},
\frac{24}{60},\frac{14}{35}, \frac{18}{27},
\frac{10}{12}.
 \]

 	\end{document}
