\documentclass[14 pt]{extarticle}

	\usepackage[frenchb]{babel}
	\usepackage[utf8]{inputenc}  
	\usepackage[T1]{fontenc}
	\usepackage{amssymb}
	\usepackage[mathscr]{euscript}
	\usepackage{stmaryrd}
	\usepackage{amsmath}
	\usepackage{tikz}
	\usepackage[all,cmtip]{xy}
	\usepackage{amsthm}
	\usepackage{varioref}
	\usepackage{geometry}
	\geometry{a4paper}
	\usepackage{lmodern}
	\usepackage{hyperref}
	\usepackage{array}
	 \usepackage{fancyhdr}
	 \usepackage{float}
\renewcommand{\theenumi}{\alph{enumi})}
	\pagestyle{fancy}
	\theoremstyle{plain}
	\fancyfoot[C]{} 
	\fancyhead[L]{Contrôle}
	\fancyhead[R]{11 décembre 2024}\geometry{
 a4paper,
 total={170mm,257mm},
 left=20mm,
 top=20mm,
 }
	
	
	\title{Contrôle chapitre 3}
	\date{}
	\begin{document}

\begin{center}{\Large Contrôle chapitre 3}\\ 
 \end{center}
 Nom : \\
 Prénom : 
 \subsection*{Exercice 1 (4 points)}
 Réduire au même dénominateur : 
 
 \begin{enumerate}
 \item $\frac12$ et $\frac38$
 \item $\frac13$ et $\frac15$
 \item $\frac56$ et $\frac2{15}$
 \item $\frac7{30}$, $\frac3{20}$ et $\frac{11}{12}$
 \end{enumerate}

\subsection*{Exercice 2 (4 points)}
Réduire les fractions suivantes au même dénominateur, et les classer dans l'ordre croissant. 
\[ \frac49, \frac5{12}, \frac56, \frac7{18}\]
\subsection*{Exercice 3 (4 points)}

Tracer une droite graduée de $14$ carreaux avec $1$ au cinquième carreau et y placer les fractions suivantes : 
\[ \frac25, \frac6{10}, \frac{36}{30}, \frac{72}{24}, \frac{225}{125}\]

\subsection*{Exercice 4 (4 points)}

Pour chacune des fractions suivantes, donner son écriture décimale et l'exprimer comme un pourcentage, arrondi à l'unité. 
\begin{enumerate}
\item $\frac13$,
\item $\frac35$,
\item $\frac79$,
\item $\frac{4}{15}$. 
\end{enumerate}

\subsection*{Exercice 5 (4 points)}
Regrouper les fractions suivantes par fractions égales. 

\[\frac25, \frac6{10}, \frac{25}{30},\frac{10}{15},
\frac{24}{60},\frac{14}{35}, \frac{18}{27},
\frac{10}{12}.
 \]

 	\end{document}
