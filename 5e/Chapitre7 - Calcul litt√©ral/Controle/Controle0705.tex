	\documentclass[14pt]{extreport}
\usepackage{extsizes}
	\usepackage[frenchb]{babel}
	\usepackage[utf8]{inputenc}  
	\usepackage[T1]{fontenc}
	\usepackage{amssymb}
	\usepackage[mathscr]{euscript}
	\usepackage{stmaryrd}
	\usepackage{amsmath}
	\usepackage{tikz}
	\usepackage[all,cmtip]{xy}
	\usepackage{amsthm}
	\usepackage{varioref}
	\usepackage[ margin=0.6in]{geometry}
	\geometry{a4paper}
	\usepackage{lmodern}
	\usepackage{hyperref}
	\usepackage{array}
	\usepackage{float}
	\usepackage{easytable}
	 \usepackage{fancyhdr}\usepackage{longtable}
	 \usetikzlibrary{shapes.misc}
	 \newcommand\ang[1]{$#1{}^o$}
\newlength{\taillecellule}
\setlength{\taillecellule}{2cm}
\newcolumntype{C}{@{}>{\centering\arraybackslash}p{\taillecellule}@{}}
 

	\pagestyle{fancy}
	\theoremstyle{plain}
	\fancyfoot[C]{\empty} 
	\fancyhead[L]{Interrogation chapitre 7}
	\fancyhead[R]{6 mai 2024}
	
	
	\title{Interrogation chapitre 7}
	\date{}
	
	
	\begin{document}
\subsection*{Exercice 1}	
	
Dire si les tableaux suivants sont des tableaux de proportionnalité. Justifier la réponse. 

\begin{enumerate}

\item $\begin{tabular}{|c|c|c|}
\hline 
6 & 8 & 5 \\
\hline 
10 & 12 & 9\\
\hline
\end{tabular}$

\item $\begin{tabular}{|c|c|c|}
\hline 
6 & 8 & 5 \\
\hline 
9 & 12 & 7,5\\
\hline
\end{tabular}$

\item $\begin{tabular}{|c|c|c|}
\hline 
12 & 16 & 44 \\
\hline 
21 & 28 & 77\\
\hline
\end{tabular}$

\item $\begin{tabular}{|c|c|}
\hline 
 6 413 451 &  6 413 457 \\
\hline 
6 413 454 & 6 413 452 \\
\hline
\end{tabular}$

\end{enumerate}	
	
	
	\subsection*{Exercice 2 }
	
	Recopier et remplir les tableaux de proportionnalité suivant en expliquant votre calcul. 
	
	\begin{enumerate}
	
\item $\begin{tabular}{|c|c|c|c|c|}
\hline 
24 & 15 & 39 & \phantom{9} & 78 \\
\hline 
56 & \phantom{35} & \phantom{91} & 21 & \phantom{182}\\
\hline
\end{tabular}$

\item $\begin{tabular}{|c|c|c|c|c|}
\hline 
2,4 & 4,8 & \phantom{7,2} & 36 & \phantom{12} \\
\hline 
\phantom{30} & 60 & 90 & \phantom{450} & 150\\
\hline
\end{tabular}$

\end{enumerate}
    
 
\subsection*{Exercice 3} 

La vitesse du son dans l'air est de $340$ mètres par seconde. 

\begin{enumerate}
\item Convertir cette vitesse en kilomètres par heure. 
\item On entend la foudre treize secondes après que l'éclair ait été visible. À quelle distance a-t-il frappé ? 
\item En douze secondes, le son parcourt sous l'eau une distance de $18$ km. Calculer la vitesse du son dans l'eau en mètres par secondes, et en kilomètres par heure. 


\end{enumerate}


\subsection*{Exercice 4} 

\begin{enumerate}
\item On souhaite participer à un cadeau de $234$ euros à hauteur de $12$ \%. Calculer la participation nécessaire. 
\item Dans une classe de $30$ élèves, $12$ élèves sont des garçons. Exprimer la proportion de garçons sous la forme d'un pourcentage. 
\item La France compte $68,4$ millions d'habitants, et représente $15,3$ \% de la population de l'Union européenne. Quelle est la population de l'Union européenne ? 
\end{enumerate}

 
\end{document}

