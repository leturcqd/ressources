\documentclass[14 pt]{extarticle}

	\usepackage[frenchb]{babel}
	\usepackage[utf8]{inputenc}  
	\usepackage[T1]{fontenc}
	\usepackage{amssymb}
	\usepackage[mathscr]{euscript}
	\usepackage{stmaryrd}
	\usepackage{amsmath}
	\usepackage{tikz}
	\usepackage[all,cmtip]{xy}
	\usepackage{amsthm}
	\usepackage{varioref}
	\usepackage{geometry}
	\geometry{a4paper}
	\usepackage{lmodern}
	\usepackage{hyperref}
	\usepackage{array}
	 \usepackage{fancyhdr}
\renewcommand{\theenumi}{\alph{enumi})}
	\pagestyle{fancy}
	\theoremstyle{plain}
	\fancyfoot[C]{} 
	\fancyhead[L]{Contrôle}
	\fancyhead[R]{25 septembre 2023}\geometry{
 a4paper,
 total={170mm,257mm},
 left=20mm,
 top=20mm,
 }
	
	
	\title{Controle Chapitre 1}
	\date{}
	\begin{document}

\begin{center}{\Large Contrôle Chapitre 1}\\ 
 \end{center}
 
  

\subsection*{Exercice 1} 
 
Effectuer les calculs suivants \emph{en détaillant les étapes} : 
 
 \begin{enumerate}
 \item $1,12 + 7 + 1, 88 + 3$
 \item $ 17 + 19 + 21 + 23 + 25 + 27 + 29 + 31 + 33 $
 \item $5 \times 17\times 4 \times 2 \times 25$
 \item $ 2,5 \times 13 \times 2 $
 \item $ 15 \div 3 + 2 \times 5$
 \item $ (14 +1)\div 3 \times 5$
 \item $23 \div 20 \div 2 + 3 $
 \item $ 1 + 4 \times (20 + 5) \times 8 + 2$
 \item $ (26 - 1 + 5)\div 2 \times 5$
 \item $ ( ( 8 + 4 ) \div 3 ) \times ( 15 - ( 1 + 4 ) )$
 \end{enumerate}
 
 \subsection*{Exercice 2}
 
 Recopier les égalités suivantes en rajoutant des parenthèses si nécessaire pour les rendre vraies. 
 
 \begin{enumerate}
 \item $2 + 8 \times 3 + 1 = 31$
 \item $5 + 6 + 10 \div 2  = 13$
 \item $18 \div 2 + 7 + 4 = 6 $
 \item $ 8 \div 2 + 2 \times 2 = 8 $
 \end{enumerate}
 

 \subsection*{Exercice 3} 
 
 Calculer : 
 \begin{enumerate}
 \item $18 \times 36 + 18\times 64 $
 \item $ 361 \times 110 + 361 \times 90 $
 \item $ 993 \times 4 + 7 \times 4$
 \item $ 101 \times 54 - 91 \times 54$
 \end{enumerate}
 
 	\end{document}
