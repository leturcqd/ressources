\documentclass[12 pt]{extarticle}

	\usepackage[frenchb]{babel}
	\usepackage[utf8]{inputenc}  
	\usepackage[T1]{fontenc}
	\usepackage{amssymb}
	\usepackage[mathscr]{euscript}
	\usepackage{stmaryrd}
	\usepackage{amsmath}
	\usepackage{tikz}
	\usepackage[all,cmtip]{xy}
	\usepackage{amsthm}
	\usepackage{varioref}
	\usepackage{geometry}
	\geometry{a4paper}
	\usepackage{lmodern}
	\usepackage{hyperref}
	\usepackage{array}
	 \usepackage{fancyhdr}
	 \usepackage{float}
	\pagestyle{fancy}
	\theoremstyle{plain}
	\fancyfoot[C]{\thepage} 
	\fancyhead[L]{Fiche d'exercices}
	\fancyhead[R]{2022-2023}
	
	
\renewcommand{\theenumi}{\alph{enumi})}

\newlength{\taillecellule}
\setlength{\taillecellule}{2cm}
\newcolumntype{C}{@{}>{\centering\arraybackslash}p{\taillecellule}@{}}
\usetikzlibrary{calc}
\usepackage{pstricks,multido}
\usepackage{arrayjob}
\usepackage{calc,xlop}

\newcounter{AlphNode}
\renewcommand*{\theAlphNode}{\Alph{AlphNode}}

	\title{Exercices Chapitre 4}
	\date{}
	\begin{document}

\begin{center}{\Large Chapitre 4 - Triangles égaux}\\
 \end{center} 
 
 \subsection*{Exercice 1}
 \begin{figure}[H]
 \center
 \begin{tikzpicture}[scale=1.5]
    \foreach \x in {0,1,2,3,4}
    \foreach \y in {0,1,2,3,4}
    {
    \draw (\x,\y) -- ++(0, .1); 
    \draw (\x,\y) -- ++(0, -.1); 
    \draw (\x,\y) -- ++(.1,0); 
    \draw (\x,\y) -- ++(-.1,0); 
    \def\lettre{(\x+1)+5*(\y)};
    \pgfmathsetcounter{AlphNode}{Mod(\lettre,26)};
  \draw (\x, \y) ++ (.12,.2)node {\scriptsize\theAlphNode};
    }
  \end{tikzpicture}
 \end{figure}
  
  \begin{enumerate}
  \item Citer un triangle égal au triangle $AHL$. 
  \item Citer trois triangles égaux au triangle $AMP$. 
  \item Donner tous les triangles égaux au triangle $PVL$ ayant $A$ comme sommet.
   \item Donner tous les triangles égaux au triangle $PVL$ ayant $M$ comme sommet.
  \item Combien y a-t-il de triangles égaux à $PVI$ ayant $E$ comme sommet ? 
  \end{enumerate}
  
  \subsection*{Exercice 2}
  
 On considère un triangle ABC. \\
 On place deux points $M$ et $N$ sur $[BC]$ tels que $BM=CN$ 
 et que B, M, N, et C soient alignés dans cet ordre. \\
 On trace la droite $(d_1)$ passant par $M$ et parallèle à $(AC)$, et on note $P$ son point d'intersection avec $[AB]$. \\
 On trace ensuite la droite $(d_2)$ passant par $N$ et parallèle à $(AB)$, et on note $Q$ son point d'intersection avec $[AC]$. \\
 
 \begin{enumerate}
 \item Faire une figure illustrant le programme de construction supra. 
 \item En utilisant le cours de $5^e$, démontrez que $\widehat{BMP}=\widehat{BCA}$, et que $\widehat{BPM}= \widehat{BAC}$. 
 \item En raisonnant de même, démontrez que $\widehat{CNQ}=\widehat{CBA}$, et que $\widehat{CQN}= \widehat{CAB}$. 
 \item Démontrer que les triangles $BMP$ et $NCQ$ sont égaux. 
 \end{enumerate}
 
\subsection*{Exercice 3}
 
 \begin{enumerate}
 
 \item Tracez un carré $ABCD$ et ses deux diagonales, qui se coupent en un point $O$. Combien a-t-on de triangles égaux au triangle $ABC$ sur la figure ainsi tracée ? Combien y a t-il de triangles égaux au triangle $AOB$ ? 
  
\item Même question avec un rectangle (non carré). 
 
 \item Même question avec un losange (non carré). 

\item Démontrer proprement les résultats précédents. 

\item Démontrer que si les quatre triangles $AOB$, $BOC$, $COD$ et $DOA$ sont égaux, alors $ABCD$ est un losange (ou un carré). 

\item Démontrer que si les quatre triangles $ABC$, $BCD$, $CDA$ et $DAB$ sont égaux, alors $ABCD$ est un rectangle. 
\end{enumerate}



 
 	\end{document}
