\documentclass[14 pt]{extarticle}

	\usepackage[frenchb]{babel}
	\usepackage[utf8]{inputenc}  
	\usepackage[T1]{fontenc}
	\usepackage{amssymb}
	\usepackage[mathscr]{euscript}
	\usepackage{stmaryrd}
	\usepackage{amsmath}
	\usepackage{tikz}
	\usepackage[all,cmtip]{xy}
	\usepackage{amsthm}
	\usepackage{varioref}
	\usepackage{geometry}
	\geometry{a4paper}
	\usepackage{lmodern}
	\usepackage{hyperref}
	\usepackage{array}
	 \usepackage{fancyhdr}
	 \usepackage{float}
\renewcommand{\theenumi}{\alph{enumi})}
	\pagestyle{fancy}
	\theoremstyle{plain}

	\fancyfoot[C]{} 
	\fancyhead[L]{}
	\fancyhead[R]{\date{}}\geometry{
 a4paper,
 total={170mm,257mm},
 left=20mm,
 top=20mm,
 }
	
	
	\title{Interrogation chapitres 7-8}
	\date{}
	\begin{document}
 
 Nom : \ldots\ldots\ldots\ldots\ldots\ldots\ldots\ldots\ldots Prénom : \ldots\ldots\ldots \\ 
 
 \subsection*{Exercice 1}
 
 Compléter : 
 
 \[ \frac34 + \frac54 = \frac\ldots\ldots   \]
 \[ \frac34 - \frac54 = \frac\ldots\ldots   \]
 \[ \frac34 \times \frac54 = \frac\ldots\ldots   \]
 \[ \frac34 \div \frac54 = \frac\ldots\ldots   \]
 
\subsection*{Exercice 2}

Compléter : 

\[ 3^4 \times 3^7 = \ldots^{\ldots\ldots}
\]

\[ \frac{2^5}{2^7} = 2^{\ldots\ldots} =\frac{\ldots}{\ldots}  \]
\[(10^2)^3 = \ldots^{\ldots\ldots} = \ldots\ldots\]

\[ 2^5 \times 5^5 = ({\ldots\ldots})^{\ldots} = {\ldots}^{\ldots} = \ldots\]

 
 \subsection*{Exercice 3}
 
 Donner le signe des expressions suivantes : \begin{enumerate}
 \item $(-3)^5$ 
 \item $3^{-5}$
 \item $(-3)^{-5}$
 \item $ -2^0$
 \item $(-2)^0$
 \end{enumerate}
 
\subsection*{Exercice 4}

On considère un vase contenant $100$ boules de couleur numérotées. Une boule sur cinq est rouge, Une boule sur quatre est bleue, les autres sont noires. Un cinquième des boules noires portent un numéro pair. 

\begin{enumerate}
\item Quelle est la proportion de boules noires ? (Détailler le calcul). 
\item Combien de boules noires portent un numéro impair ? 
\end{enumerate}


\subsection*{Exercice 5 (Bonus)}
On veut trouver un nombre dont la somme des quotients par $5$, $7$ et $9$ est égale à $429$. \begin{enumerate}
\item Quel est le plus petit nombre positif (non nul) tel que ces trois quotients soient entiers ? 
\item Pour ce nombre, que vaut le quotient par $5$ ? par $7$ ? par $9$ ? et leur somme ? 
\item Conclure. 
\end{enumerate}
\newpage

 
 Nom : \ldots\ldots\ldots\ldots\ldots\ldots\ldots\ldots\ldots Prénom : \ldots\ldots\ldots \\ 
 
 \subsection*{Exercice 1}
 
 Compléter : 
 
 \[ \frac14 + \frac94 = \frac\ldots\ldots   \]
 \[ \frac14 - \frac94 = \frac\ldots\ldots   \]
 \[ \frac14 \times \frac94 = \frac\ldots\ldots   \]
 \[ \frac14 \div \frac94 = \frac\ldots\ldots   \]
 
\subsection*{Exercice 2}

Compléter : 

\[ 2^4 \times 2^6 = \ldots^{\ldots\ldots}
\]

\[ \frac{3^3}{3^7} = 3^{\ldots\ldots} =\frac{\ldots}{\ldots}  \]
\[(10^4)^2 = \ldots^{\ldots\ldots} = \ldots\ldots\]

\[ 5^4 \times 2^4 = ({\ldots\ldots})^{\ldots} = {\ldots}^{\ldots} = \ldots\]

 
 \subsection*{Exercice 3}
 
 Donner le signe des expressions suivantes : \begin{enumerate}
 \item $(-2)^{-5}$ 
 \item $2^{-5}$
 \item $2^{5}$
 \item $ -13^0$
 \item $(-13)^0$
 \end{enumerate}
 
\subsection*{Exercice 4}

On considère un vase contenant $100$ boules de couleur numérotées. Une boule sur cinq est rouge, Une boule sur quatre est bleue, les autres sont noires. Un cinquième des boules noires portent un numéro pair. 

\begin{enumerate}
\item Quelle est la proportion de boules noires ? (Détailler le calcul). 
\item Combien de boules noires portent un numéro impair ? 
\end{enumerate}


\subsection*{Exercice 5 (Bonus)}
On veut trouver un nombre dont la somme des quotients par $5$, $7$ et $9$ est égale à $429$. \begin{enumerate}
\item Quel est le plus petit nombre positif (non nul) tel que ces trois quotients soient entiers ? 
\item Pour ce nombre, que vaut le quotient par $5$ ? par $7$ ? par $9$ ? et leur somme ? 
\item Conclure. 
\end{enumerate}


 	\end{document}
