\documentclass[14 pt]{extarticle}

	\usepackage[frenchb]{babel}
	\usepackage[utf8]{inputenc}  
	\usepackage[T1]{fontenc}
	\usepackage{amssymb}
	\usepackage[mathscr]{euscript}
	\usepackage{stmaryrd}
	\usepackage{amsmath}
	\usepackage{tikz}
	\usepackage[all,cmtip]{xy}
	\usepackage{amsthm}
	\usepackage{varioref}
	\usepackage{geometry}
	\geometry{a4paper}
	\usepackage{lmodern}
	\usepackage{float}
	\usepackage{hyperref}
	\usepackage{array}
	 \usepackage{fancyhdr}
\renewcommand{\theenumi}{\alph{enumi})}
	\pagestyle{fancy}
	\theoremstyle{plain}
	\fancyfoot[C]{} 
	\fancyhead[L]{Contrôle}
	\fancyhead[R]{15 décembre 2022}\geometry{
 a4paper,
 total={170mm,257mm},
 left=20mm,
 top=20mm,
 }
	
	
	\title{Contrôle Chapitre 4}
	\date{}
	\begin{document}

\begin{center}{\large Contrôle Chapitre 4}\\ 
 \end{center}
 
 
 \subsection*{Exercice 1 (5 points)}
 
Les triangles suivants sont-ils égaux ? \textbf{(Le dessin n'est pas à l'échelle)}. Répondre en justifiant si la réponse est oui. 
\begin{enumerate}
\item
\begin{figure}[H]
\center
 \begin{tikzpicture}
\draw (0,0) -- (3,2) -- (5,1) -- (0,0); 
\end{tikzpicture}
 et  \ \ \ \ \ \ \ 
 \begin{tikzpicture}
\draw (0,0) -- (3,2) -- (5,1) -- (0,0); 
\end{tikzpicture}
\end{figure}
 \item
\begin{figure}[H]
\center
 \begin{tikzpicture}
\draw (0,0) -- (3,2) -- (5,1) -- (0,0); 
\end{tikzpicture}
 et  \ \ \ \ \ \ \ 
 \begin{tikzpicture}
\draw (0,0) -- (3,2) -- (5,1) -- (0,0); 
\end{tikzpicture}
\end{figure}
\item
\begin{figure}[H]
\center
 \begin{tikzpicture}
\draw (0,0) -- (3,2) -- (5,1) -- (0,0); 
\end{tikzpicture}
 et  \ \ \ \ \ \ \ 
 \begin{tikzpicture}
\draw (0,0) -- (3,2) -- (5,1) -- (0,0); 
\end{tikzpicture}
\end{figure}
\end{enumerate}

\subsection*{Exercice 2}
 
 Trouver la mesure manquante sachant que les deux triangles de la question sont égaux. Justifier la réponse.

\begin{enumerate}
\item
\begin{figure}[H]
\center
 \begin{tikzpicture}
\draw (0,0) -- (3,2) -- (5,1) -- (0,0); 
\end{tikzpicture}
 et  \ \ \ \ \ \ \ 
 \begin{tikzpicture}
\draw (0,0) -- (3,2) -- (5,1) -- (0,0); 
\end{tikzpicture}
\end{figure}
 \item
\begin{figure}[H]
\center
 \begin{tikzpicture}
\draw (0,0) -- (3,2) -- (5,1) -- (0,0); 
\end{tikzpicture}
 et  \ \ \ \ \ \ \ 
 \begin{tikzpicture}
\draw (0,0) -- (3,2) -- (5,1) -- (0,0); 
\end{tikzpicture}
\end{figure}
\end{enumerate}

\subsection*{Exercice 3}
 
 Tracer un losange $ABCD$, dont les diagonales se croisent en un point $O$. Démontrer que les triangles $ABC$ et $BCD$ sont égaux. 
 
\subsection*{Exercice 4}
Tracé : 
\begin{enumerate}

\item Tracer un triangle $ABC$, isocèle en $A$, avec $\widehat{BAC}= 120^o$, et $AB= 3 cm$.
\item Compléter la figure en plaçant un point $D$ tel que $\widehat{DBC}= 30^o$ et $BD= 30^o$. 
\end{enumerate}
Raisonnement : 
\begin{enumerate}

\item Montrer que les triangles $BCD$ et $ABC$ sont égaux. 

\item En déduire que $ABDC$ est un parallélogramme. 
\end{enumerate}
 	\end{document}
