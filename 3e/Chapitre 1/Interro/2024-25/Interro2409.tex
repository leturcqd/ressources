\documentclass[14 pt]{extarticle}

	\usepackage[frenchb]{babel}
	\usepackage[utf8]{inputenc}  
	\usepackage[T1]{fontenc}
	\usepackage{amssymb}
	\usepackage[mathscr]{euscript}
	\usepackage{stmaryrd}
	\usepackage{amsmath}
	\usepackage{tikz}
	\usepackage[all,cmtip]{xy}
	\usepackage{amsthm}
	\usepackage{varioref}
	\usepackage{geometry}
	\geometry{a4paper}
	\usepackage{lmodern}
	\usepackage{hyperref}
	\usepackage{array}
	 \usepackage{fancyhdr}
\renewcommand{\theenumi}{\alph{enumi})}
	\pagestyle{fancy}
	\theoremstyle{plain}
	\fancyfoot[C]{} 
	\fancyhead[L]{Interrogation chapitre 1}
	\fancyhead[R]{24 septembre 2023}\geometry{
 a4paper,
 total={170mm,257mm},
 left=20mm,
 top=20mm,
 }
	
	
	\title{Chapitre 1-  Arithmétique}
	\date{}
	\begin{document}

\begin{center}{\Large Interrogation : Chapitre 1 - Arithmétique}\\ 
 \end{center}




\subsection*{Exercice 1 (3 pts)}
 Poser les divisions euclidiennes suivantes. 
 
 
 \begin{enumerate}
 \item $5746$ par $5$
 \item $1634$ par $7$ 
 \item $12 264$ par $11$. 
 \end{enumerate}
 \subsection*{Exercice 2 (6 pts)}
 
 Pour chacun des nombres suivants, dire s'il est premier ou non. Justifier la réponse. 
\begin{enumerate}
\item $654$,
\item $41$,
\item $69$,
\item $61$,
\item $187$.
\end{enumerate}

\subsection*{Exercice 3 (6 pts)}
Décomposer en facteurs premiers les nombres suivants. Détailler la méthode. 

\begin{enumerate}
\item $91$,
\item $128$,
\item $420$,
\item $252$,
\item $3960$
\end{enumerate}
\subsection*{Exercice 4 (4 pts)}
Faire la liste des diviseurs des nombres suivants : 
\begin{enumerate}
\item $30$
\item $120$
\item $71$
\item $1~000$
\end{enumerate}

\subsection*{Exercice 5 (1 point et +)}

Trouver un multiple de 35 ayant exactement 8 diviseurs. Combien y a-t-il de solutions ?
\newpage\begin{center}{\Large Interrogation : Chapitre 1 - Arithmétique}\\ 
 \end{center}




\subsection*{Exercice 1 (3 pts)}
 Poser les divisions euclidiennes suivantes. 
 
 
 \begin{enumerate}
 \item $5476$ par $5$
 \item $1364$ par $7$ 
 \item $12 246$ par $11$. 
 \end{enumerate}
 \subsection*{Exercice 2 (6 pts)}
 
 Pour chacun des nombres suivants, dire s'il est premier ou non. Justifier la réponse. 
\begin{enumerate}
\item $544$,
\item $43$,
\item $69$,
\item $61$,
\item $221$.
\end{enumerate}

\subsection*{Exercice 3 (6 pts)}
Décomposer en facteurs premiers les nombres suivants. Détailler la méthode. 

\begin{enumerate}
\item $91$,
\item $64$,
\item $420$,
\item $252$,
\item $2~640$
\end{enumerate}
\subsection*{Exercice 4 (4 pts)}
Faire la liste des diviseurs des nombres suivants : 
\begin{enumerate}
\item $70$
\item $200$
\item $83$
\item $1~500$
\end{enumerate}

\subsection*{Exercice 5 (1 point et +)}

Trouver un multiple de 21 ayant exactement 8 diviseurs. Combien y a-t-il de solutions ?




 	\end{document}
