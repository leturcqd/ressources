\documentclass[12 pt]{extarticle}

	\usepackage[frenchb]{babel}
	\usepackage[utf8]{inputenc}  
	\usepackage[T1]{fontenc}
	\usepackage{amssymb}
	\usepackage[mathscr]{euscript}
	\usepackage{stmaryrd}
	\usepackage{amsmath}
	\usepackage{tikz}
	\usepackage[all,cmtip]{xy}
	\usepackage{amsthm}
	\usepackage{varioref}
	\usepackage{geometry}
	\geometry{a4paper}
	\usepackage{lmodern}
	\usepackage{hyperref}
	\usepackage{array}
	 \usepackage{fancyhdr}
%\renewcommand{\theenumi}{\alph{enumi})}
	\pagestyle{fancy}
	\theoremstyle{plain}
	\fancyfoot[C]{} 
	\fancyhead[L]{Devoir maison}
	\fancyhead[R]{Mai 2024}\geometry{
 a4paper,
 total={170mm,257mm},
 left=20mm,
 top=20mm,
 }
	
	
	\title{Applications de la similitude}
	\date{}
	\begin{document}

\begin{center}{\Large Applications de la similitude}\\ 
 \end{center}
 
 \subsection*{Exercice 1- Puissance d'un point par rapport à un cercle} 
 
 On se donne dans tout l'exercice un cercle de centre $O$ et de rayon $R$ et un point $M$ qui n'est pas situé sur le cercle. On trace deux droites sécantes en
 $M$ croisant le cercle en $A$ et $B$ pour la première, en $C$ et $D$ pour la seconde.
 
 \begin{enumerate}
 \item Faire une figure dans le cas où $M$ est situé à l'intérieur du disque. Tracer également les segments $[AC]$ et $[BD]$. On se place désormais dans ce cas.
 \item Justifier l'égalité $\widehat{AMC}=\widehat{DMB}$.
 \item On admet que $\widehat{DBA}=\widehat{DCA}$. Montrer que les triangles $MDB$ et $MAC$ sont semblables. 
 \item En déduire que $MA.MB=MC.MD$. 
 \end{enumerate}
  
On a montré l'énoncé suivant : « Étant donné une droite passant par $M$ et coupant le cercle en deux points $A$ et $B$, le produit des distances $MA$ et $MB$ ne dépend pas de la droite choisie. » Ce produit s'appelle la \emph{puissance de $M$ par rapport au cercle.}


\subsection*{Exercice 2 - Expression de la puissance d'un point par rapport à un cercle en fonction de sa distance au centre}

D'après l'exercice précédent, on peut donc choisir n'importe quel droite passant par $M$ pour calculer sa puissance par rapport au cercle. Dans cet exercice, on prend donc la droite $(OM)$. 
\begin{enumerate}
\item Faire une figure avec $A$ et $B$ les points d'intersection de $(OM)$ avec le cercle. 
\item On note $R$ le rayon du cercle et $d$ la distance $IM$. Exprimer $MA$ et $MB$ en fonction de $R$ et $d$.
\item En déduire une expression du produit $MA\times MB$, la développer et la réduire.  
\end{enumerate}

On a montré l'énoncé suivant : la puissance d'un point intérieur d'un disque $M$ au cercle est la différence entre le carré du rayon et le carré de la distance de $M$ au centre de ce cercle. 


\subsection*{Exercice 3 - Une dernière formule géométrique}

Traçons enfin en partant du point $M$ la droite $(d)$ perpendiculaire à $(OM)$. Elle coupe le cercle en deux points $P$ et $Q$. 
\begin{enumerate}
\item Montrer que la puissance de $M$ par rapport au cercle est le carré de la distance $MP$. 
\item On note $a$ l'angle $\widehat{MOP}$. Montrer que la puissance de $M$ par rapport au cercle s'écrit $R^2(\sin(a))^2$. 
\end{enumerate}

\subsection*{Exercice 4 - Définition des fonctions trigonométriques}

On se donne deux triangles $ABC$ et $A'B'C'$ rectangles en $B$ et $B'$. 

\begin{enumerate}
\item On suppose que $\widehat{BAC}=\widehat{B'A'C'}$.
\begin{enumerate}
\item Montrer que les triangles sont semblables. 
\item En déduire que $\frac{AB}{AC} = \frac{A'B'}{A'C'}$.
\item De même, en déduire que $\frac{BC}{AC} = \frac{B'C'}{A'C'}$. 
\item De même, en déduire que $\frac{BC}{AB} = \frac{B'C'}{A'B'}$. 
\end{enumerate} 
\item On ne suppose plus $\widehat{BAC}=\widehat{B'A'C'}$. 
\begin{enumerate}
\item Supposons $\frac{AB}{AC} = \frac{A'B'}{A'C'}$. 
Montrer que les triangles sont semblables. On montrerait de même qu'ils le sont lorsqu'un des deux autres rapports des questions 1.b et 1.c est le même.
\item En déduire qu'alors, $\widehat{BAC}=\widehat{B'A'C'}$.
\end{enumerate}
\end{enumerate}

On a montré le résultat suivant : Étant donné un angle $x$, dans tout triangle rectangle en $B$ avec un angle au sommet $A$ de mesure $x$, les rapports $\frac{AB}{AC}$, $\frac{BC}{AC}$ et $\frac{BC}{AB}$ sont déterminés, et chacun de ces rapports déterminent totalement l'angle $x$. On les appelle respectivement \emph{cosinus}, \emph{sinus}, et \emph{tangente} de l'angle $x$. On les note plus brièvement $\cos(x)$, $\sin(x)$ et $\tan(x)$. 

\begin{enumerate}
\item[3.] À partir du théorème de Pythagore, calculer $(\cos(x))^2 + (\sin(x))^2$.
\item[4.] Montrer la relation $\tan(x)= \frac{\sin(x)}{\cos(x)}$. 
\item[5.] Déduire des deux questions précédentes la relation $1 + (\tan(x))^2 = \frac1{(\cos(x))^2}$. 
\end{enumerate}


\subsection*{Exercice 5 - Relations dans un triangle rectangle}


Dans un triangle rectangle $ABC$ en $B$, on trace la hauteur issue de $B$. Elle croise l'hypoténuse $[AC]$ en un point $H$. 
\begin{enumerate}
\item 
Montrer que les triangles $AHB$, $BHC$ et $ABC$ sont semblables.
\item En déduire les relations $AH = \frac{AB^2}{AC}$, $CH= \frac{CB^2}{AC}$ et $BH = \frac{AB.BC}{AC}$. 
\end{enumerate}
 	\end{document}
