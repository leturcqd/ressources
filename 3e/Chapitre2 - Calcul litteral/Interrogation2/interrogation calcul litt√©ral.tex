\documentclass[14 pt]{extarticle}

	\usepackage[frenchb]{babel}
	\usepackage[utf8]{inputenc}  
	\usepackage[T1]{fontenc}
	\usepackage{amssymb}
	\usepackage[mathscr]{euscript}
	\usepackage{stmaryrd}
	\usepackage{amsmath}
	\usepackage{tikz}
	\usepackage[all,cmtip]{xy}
	\usepackage{amsthm}
	\usepackage{varioref}
	\usepackage{geometry}
	\geometry{a4paper}
	\usepackage{lmodern}
	\usepackage{hyperref}
	\usepackage{array}
	 \usepackage{fancyhdr}
	 \usepackage{float}
\renewcommand{\theenumi}{\alph{enumi})}
	\pagestyle{fancy}
	\theoremstyle{plain}
\newcommand{\exo}[8]{
 \ \\ \ \\
 Nom : \ldots\ldots\ldots\ldots\ldots\ldots\ldots\ldots\ldots Prénom : \ldots\ldots\ldots \\ 
 Calculez :\\ \ \\ 
 \[ (- (#5 \times x +   =   \] 
 \[ (-3)^{-#6} = \] 
 \[ #7^2 = \]
 \[ \sqrt{#8} = \]
  }
	\fancyfoot[C]{} 
	\fancyhead[L]{}
	\fancyhead[R]{}\geometry{
 a4paper,
 total={170mm,257mm},
 left=20mm,
 top=20mm,
 }
	
	
	\title{Interrogation chapitre 5}
	\date{}
	\begin{document}
 \ \\ \ \\
 Nom : \ldots\ldots\ldots\ldots\ldots\ldots\ldots\ldots\ldots Prénom : \ldots\ldots\ldots \\ 
Développez et réduire :\\ \ \\ 
 \[ - (2x - 3y +5) =   \] 
 \[ (a + b) \times 5 = \] 
 \[ (x^2 + x - 1) \times x = \]
 \[ (x+1)\times 2 - (2 + x) \times x = \]
 
 \hrule
 \ \\ \ \\
 Nom : \ldots\ldots\ldots\ldots\ldots\ldots\ldots\ldots\ldots Prénom : \ldots\ldots\ldots \\ 
Développez et réduire :\\ \ \\ 
 \[ - (3x - 2y - 5) =   \] 
 \[ (a + d) \times 5 = \] 
 \[ (x^2 + 2x - 1) \times x = \]
 \[ (x+1)\times 3 - (2 + x) \times x = \]
 
 \hrule
 \ \\ \ \\
 Nom : \ldots\ldots\ldots\ldots\ldots\ldots\ldots\ldots\ldots Prénom : \ldots\ldots\ldots \\ 
Développez et réduire :\\ \ \\ 
 \[ - (2x + 3y -5) =   \] 
 \[ (x + z) \times 5 = \] 
 \[ (c^2 + c + 1) \times c = \]
 \[ (x+1)\times 2 - (3 + x) \times x = \]
 
 \hrule
 \ \\ \ \\
 Nom : \ldots\ldots\ldots\ldots\ldots\ldots\ldots\ldots\ldots Prénom : \ldots\ldots\ldots \\ 
Développez et réduire :\\ \ \\ 
 \[ - (2x - 3y -2) =   \] 
 \[ (a - b) \times 5 = \] 
 \[ (x^2 - x - 1) \times x = \]
 \[ (x+1)\times 4 - (2 + x) \times x = \]
 
 \hrule
 
 
 
 	\end{document}
