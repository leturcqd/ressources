\documentclass[14 pt, fleqn, pstricks]{extarticle}

	\usepackage[frenchb]{babel}
	\usepackage[utf8]{inputenc}  
	\usepackage[T1]{fontenc}
	\usepackage{amssymb}
	\usepackage[mathscr]{euscript}
	\usepackage{stmaryrd}
	\usepackage{amsmath}
	\usepackage{tikz}
	\usepackage[all,cmtip]{xy}
	\usepackage{amsthm}
	\usepackage{varioref}
	\usepackage{geometry}
	\usepackage{tabularx}
	\geometry{a4paper}
	\usepackage{lmodern}
	\usepackage{hyperref}
	\usepackage{array}
	 \usepackage{fancyhdr}
	 \usepackage{pstricks,pst-plot,pst-tree,pstricks-add}
\usepackage{pst-eucl}% permet de faire des dessins de géométrie simplement
\usepackage{pst-text}
\usepackage{pst-node,pst-all}
\usepackage{pst-func,pst-math,pst-bspline,pst-3dplot}  %%% POUR LE BAC %%%
	 \usepackage{float}\usepackage{setspace}
\setlength{\mathindent}{1cm}
\renewcommand{\theenumi}{\alph{enumi})}
	\pagestyle{fancy}
	\theoremstyle{plain}
	\fancyfoot[C]{} 
	\fancyhead[L]{}
	\fancyhead[R]{}\geometry{
 a4paper,
 total={170mm,257mm}
 }
	
	
	\title{Exercices de calcul}
	\date{}
	\begin{document}
	 


\subsection*{Exercice 1 (6 points) }	 
	 
On considère le programme de calcul suivant : 
\begin{itemize}
\item Prendre un nombre
\item Retirer 3
\item Multiplier par le nombre de départ 
\item Ajouter 2
\end{itemize}
\begin{enumerate}
\item Calculez le résultat du programme en choisissant $1$ comme nombre de départ.
\item  On note $h(x)$ le résultat du programme lorsqu'on choisit le nombre $x$ au départ. Donnez une expression algébrique de $h(x)$.
\item  Recopiez et remplissez le tableau suivant : 
	 
\begin{figure}[H]
\center
$\begin{array}{|c|c|c|c|c|}
\hline
 x &  3 & -4 & \frac13 &      \\
\hline
h(x)&   & & & 0    \\
\hline
\end{array}$
\end{figure}
\item  Développez et réduisez l'expression de $h(x)$. 
\end{enumerate} 

\subsection*{Exercice 2 (8 points)}

Développez et réduisez les expressions suivantes : 
\begin{enumerate}
\item $(2x-3y)(4x-2)$
\item $(2a+3b)(-4a+6b)$
\item $(2a+5b)^2 + (3a+b)^2$
\item $(2a+5b)(3a-2b)-(2a-1)(3a+2b)-(a-2b)(5b-1)$
\end{enumerate}
    
\subsection*{Exercice 3 (6 points)}
    
Factorisez les expressions suivantes : 
\begin{enumerate}
\item $(2x+1)(3x-4) + (2x+1)(7x+4) $
\item $x(x+1) + (x-2)x$
\item $(x+1)(x+3) - (x+1)(2x+2)$
\item $(x+1)^2 - 4$, en utilisant une identité remarquable.
\end{enumerate}

 	\end{document}







