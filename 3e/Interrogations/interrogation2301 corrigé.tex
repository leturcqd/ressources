\documentclass[14 pt, fleqn]{extarticle}

	\usepackage[frenchb]{babel}
	\usepackage[utf8]{inputenc}  
	\usepackage[T1]{fontenc}
	\usepackage{amssymb}
	\usepackage[mathscr]{euscript}
	\usepackage{stmaryrd}
	\usepackage{amsmath}
	\usepackage{tikz}
	\usepackage[all,cmtip]{xy}
	\usepackage{amsthm}
	\usepackage{varioref}
	\usepackage{geometry}
	\geometry{a4paper}
	\usepackage{lmodern}
	\usepackage{hyperref}
	\usepackage{array}
	 \usepackage{fancyhdr}
	 \usepackage{float}\usepackage{setspace}
\setlength{\mathindent}{1cm}
\renewcommand{\theenumi}{\alph{enumi})}
	\pagestyle{fancy}
	\theoremstyle{plain}
	\fancyfoot[C]{} 
	\fancyhead[L]{}
	\fancyhead[R]{}\geometry{
 a4paper,
 total={170mm,257mm},
 left=5mm,
 top=5mm,
 bottom = 0mm
 }
	
	
	\title{Interrogation chapitre 5}
	\date{}
	\begin{document}
	

\baselineskip=1.2\baselineskip


 Nom : \ldots\ldots\ldots\ldots\ldots\ldots\ldots\ldots\ldots Prénom : \ldots\ldots\ldots \\ 
Développez et réduire :
 \( (y+2)\times 2 - (3 - y) \times 2 ={\color{red}   
 (y \times 2 + 2 \times 2) - ( 3 \times 2 - y \times 2)
 } \)
 \[ = {\color{red}   
 (2y + 4) - (6 - 2y)
 }
 =
 {\color{red}   
 2y + 4 - 6 + 2y = 4y - 2
 }\]
 Convertir : 
 \( 12,54 \text{m}^3 = 
 {\color{red}   12, 54 \times 1 000\ \text{dm}^3 = 12 540 } \text{L}\)
 
Un produit qui coûte 260 euros après une hausse de $30\%$ coûtait auparavant : $ {\color{red} \frac{260}{1+\frac{30}{100}} = \frac{260}{1,3}= 200
 }$ euros.
 \ \\  
 \hrule
 \ \\ \ \\
 Nom : \ldots\ldots\ldots\ldots\ldots\ldots\ldots\ldots\ldots Prénom : \ldots\ldots\ldots \\ 
Développez et réduire :
\((x-2)\times 3 - (3 - x) \times 3 =
 {\color{red} (x \times 3 - 2 \times 3) - ( 3\times 3 - x \times 3)} \)
 \[ =  {\color{red}(3x - 6) - (9 - 3x)  }= 
  {\color{red} 3x - 6 - 9 + 3x = 6x - 15}
 \]
 Convertir : 
 \( 96,12\text{cm}^3 =  {\color{red} 96, 12 \times 0,001\ \text{dm}^3= 0,09612}\ \text{L}\)
 
Un produit qui coûte 480 euros après une baisse de $20\%$ coûtait auparavant  :$ {\color{red} \frac{480}{1-\frac{20}{100}}= \frac{480}{0,8}= 600}$ euros.
 \ \\  
 
 \hrule
 \ \\ \ \\
 Nom : \ldots\ldots\ldots\ldots\ldots\ldots\ldots\ldots\ldots Prénom : \ldots\ldots\ldots \\ 
Développez et réduire :
 \( (a+3)\times 3 - (3- a) \times 2  = {\color{red} (a\times 3 + 3 \times 3) - (3\times 2 - a \times2)} \)
 \[ = 
  {\color{red} (3a + 9)-(6 - 2a)}=  
  {\color{red} 3a + 9 - 6 + 2a} = 
   {\color{red} 5a + 3}\]
 Convertir : 
 \( 56,124 \text{dm}^3 = 56, 124\ \text{L} = 56 124 \  \text{mL}\)
 
Un produit qui coûte 420 euros après une hausse de $40\%$ coûtait auparavant : ${\color{red}\frac{420}{1+\frac{40}{100}} = \frac{420}{1,4} = 300} $ euros.
 \ \\ 
 \hrule
 \ \\ \ \\
 Nom : \ldots\ldots\ldots\ldots\ldots\ldots\ldots\ldots\ldots Prénom : \ldots\ldots\ldots \\ 
Développez et réduire :
 \( (a+3)\times 2 - (3 - a) \times 2  =
  {\color{red} (a\times 2 + 3\times 2) - (3\times 2- a \times 2)} \)
 \[ =  {\color{red} (2a+6) - (6-2a) } =
  {\color{red} 2a + 6 - 6 + 2a = 4a}\]
 Convertir : 
 \( 78,24 \text{m}^3 =  {\color{red} 78, 24\times 1 000\ \text{dm}^3 = 78 240}\ \text{L}\)
 
Un produit qui coûte 420 euros après une baisse de $30\%$ coûtait auparavant  :${\color{red}\frac{420}{1-\frac{30}{100}} = \frac{420}{0,7} = 600} $ euros.
 \hrule
 
 
 
 	\end{document}
