
 \chapter{Application aux problèmes}
 \begin{enumerate}
 \item Trouver un nombre dont la somme des quotients par 
 4 et par 7 soit égale à 55. 
 \item Trouver un nombre dont la somme des quotients par 4, 6 et 8 soit égale à 78.
 \item Les trois quarts d'un nombre surpassent les deux tiers de ce nombre de 17. Trouve ce nombre.
 \item Trouver un nombre dont les $\frac45$ surpassent de 14 les $\frac23$ de ce même nombre. 
 \item En retranchant 18 aux $\frac34$ d'un nombre on trouve le même résultat que si l'on avait ajouté 7 au 
 $\frac13$ de ce nombre. Trouver ce nombre.
 \item Trouver un nombre tel que si on en ajoute les $\frac9{14}$ aux $\frac23$ le résultat surpasse de $50$ les $\frac56$ de ce même nombre.
\item Un marchand a acheté deux tonneaux de vin de même 
 contenance pour 270 F. Lors de la mise en bouteille, il 
 en perd 15 litres du premier tonneau qu'il revend 0,80 F 
 le litre et 25 litres du second qu'il revend 0,90 F le 
 litre. Il fait ainsi un bénéfice de 78 G. Trouver la 
 capacité de chacun des tonneaux. 
\item Une libraire fait éditer trois livres à un même 
 nombre d'exemplaires chacun ; le premier est vendu 3,60 
 F, le deuxième 3 F et le troisième 4,50 F. Il reste 1~300 
 exemplaires invendus du premier, 980 du deuxième, et 640 
 du troisième. Sachant que la vente a rapporté 45~000 F, 
 on demande à combien d'exemplaires chacun de ces livres a 
 été édité.
 \item Un marchand achète deux pièces du même drap, l'une 
 de 64 mètres, l'autre de 50 mètres. Il fait un bénéfice 
 de 96 F par mètre sur la première et de 105 F par mètre 
 sur la seconde et retire ainsi de sa vente 51~750 F. 
 Combien a-t-il payé le mètre de drap ? 
 \item Si un vigneron vend son vin 68 F l'hectolitre, il 
 lui restera 320 F après avoir acheté un champ. Mais s'il 
 ne le vend que 60 F l'hectolitre, il lui manquera 48 F. 
 Combien d'hectolitres de vin a-t-il récoltés ? 
 \item Un ouvrier calculer qu'il dépense par mois les 
 $\frac23$ de son salaire plus 48 F. Sachant qu'il 
 économise ainsi 102 F, trouver son salaire mensuel. 
 \item Un fermier espère payer son propriétaire avec 
 le prix de sa récolte de blé. S'il la vend 26,50 F le 
 quintal, il lui restera 195 F. S'il ne la vend que 25 F, 
 il lui manquera 225 F. Combien de quintaux de blé a-t-il 
 récoltés ? 
 \item Un marchand a acheté des moutons 93 F l'un et les 
 revend 105 F au marché de la Villette à Paris. Le 
 transport lui revient à 97,50 F et l'un des moutons 
 est mort en route ; il gagne malgré tout 241,50 F. 
 Combien de moutons avait-il acheté ? 
 \item Un litre de lait pur pèse 1,033 kg. Une laitière a acheté 50 litres de lait et trouve qu'ils ne pèsent que 51,485 kg. Quelle quantité d'eau contient ce lait ? 
 \item On a acheté une pièce d'étoffe à raison de 20 F les 3 mètres. On la revend à raison de 60 F les 7 mètres. On fait un bénéfice de 100 F. Quelle est la longueur de 
 la pièce ? 
 \item On veut placer des élèves dans une salle de projection. En mettant dix élèves par banc, il y en a 11
 qui ne sont pas placés. En mettant 11 élèves par banc, il reste alors 7 places disponibles. Quel est le nombre 
 d'élèves ? 
 \item Un groupe d'enfants doit faire une excursion qui 
 revient à 150 F chacun. Au moment du départ, trois d'entre eux sont absents et chacun des autres doit payer 
 15 F en plus. Quel était le prix de revient total de 
 l'excursion. 
 \item Un enfant veut placer ses billes en tas égaux contenant chacun 12 billes. Il lui reste alors 16 billes.
 En mettant 3 billes de plus par tas, il lui manque 5 billes. Combien de billes possède-t-il ? 
 \item Un chemisier a acheté des chemises à 9 F la pièce. Il en vend la moitié à 12 F pièce, le tiers à 11,40 F et le reste à 9,60 F. Il fait ainsi un bénéfice de 115,20 F. Combien de chemises a-t-il vendues ? 
 \item Un marchand a acheté une pièce d'étoffe à 7,20 F 
 le mètre. Il en revend 13 mètres à 9 F puis les $\frac35$ du reste à 9,60 F le mètre et le nouveau reste à 7,50 F le mètre. Il fait ainsi un bénéfice de 109,20 F. Trouver la longueur de cette pièce. 
 \item On mélange du café à 7,20 F le kg avec du café à 8,40 F le kg. Quelle quantité du premier doit-on prendre pour 10 kg du second, si l'on veut obtenir un mélange qui revienne à 8 F le kg. 
 \item  Deux trains partent en même temps, le premier de Paris, le second de Tours, se dirigeant l'un vers l'autre. Le premier marche à 90 km à l'heure, et le deuxième à 80 km à l'heure. Sachant que la distance Paris-Tours est de 238 km, on demande à quelle distance de Paris ils vont se croiser. 
 \item Un express part de Paris à 8h30 et se dirige vers le Mans à la vitesse de 75 km à l'heure. À 9h01, un rapide part du Mans vers Paris et marche à 84 km à l'heure. La distance Paris-Le Mans étant de 211 km, trouver l'heure de la rencontre. 
 \item Deux nombres ont pour différence 25 et leur somme est égale à 109. Quels sont ces deux nombres ? 
 \item La différence de deux nombres est 14, et le double
 du plus grand surpasse de 5 le triple du plus petit. Quels sont ces deux nombres ? 
 \item Trouver trois nombres impairs consécutifs, sachant que leur somme est égale à 141. 
 \item Trouver les dimensions d'un rectangle sachant que le périmètre est égal à 272 mètres et que la longueur est les $\frac53$ de la largeur. 
 \item La longueur d'un champ rectangulaire est inférieure de 15 mètres au double de la largeur. Trouver sa surface sachant que le demi-périmètre est égal à 186 mètres. 
 \item Trois enfants ont ensemble 33 ans. L'âge du premier dépasse de 2 ans l'âge du deuxième et est le double de l'âge du troisième. Trouver l'âge de chacun d'eux. 
 \item La somme de deux nombres est 496. En les divisant l'un par l'autre, on trouve 6 pour quotient entier et 48 pour le reste. Trouver ces deux nombres. 
 \item La différence de deux nombres est 516. Le quotient entier de ces deux nombres est 13 et le reste de leur division est 24. Trouver ces deux nombres.
 \item Deux enfants ont ensemble 105 billes. Si le premier en avait 15 de plus, il en aurait trois fois plus que le second. Combien de billes ont-ils chacun ? 
 \item On écrit 3 nombres à la suite l'un de l'autre.
 Chacun d'eux surpasse de 5 le double de celui qui le 
 précède. Sachant que leur somme est égale à 160, trouver ces trois nombres. 
 \item La somme de deux nombres est 162. En ajoutant 13 à chacun d'eux, l'un d'eux devient le triple de l'autre. Trouver ces deux nombres. 
 \item Partager une somme de 1~850 F entre 3 personnes,
 sachant que la première reçoit 250 F de moins que la deuxième et deux fois moins que la troisième.
 \item Un cultivateur vend pour 994 F du blé à 26 F le quintal et de l'avoine à 15 F le quintal. Sachant que le poids de l'avoine est le triple de celui du blé, combien de quintaux de chaque sorte a-t-il vendus ? 
 \item Un cycliste qui fait 30 km à l'heure rejoint au bout de 1 h 20 min un piéton qui marche à 6 km à l'heure. 
 Quelle était l'avance du piéton au moment du départ du 
 cycliste ? 
 \item Une usine emploie 171 ouvriers. Le nombre de femmes est le tiers de celui des hommes et celui des enfants\footnote{Ahem.} la moitié de celui des femmes. Trouver le nombre d'hommes, de femmes, et d'enfants employés à l'usine.
 \item Un mètre de drap coûte 7,20 F de plus qu'un mètre de toile. Sachant que 10 m de drap et 12 m de toile coûtent ensemble 256,80 F, trouver le prix du mètre de chacune des deux étoffes.
 \item Un épicier vend 1 kg de café et 3 kg de sucre pour 13,20 F, puis une autre fois 9 kg de café et 13 kg de sucre pour 102 F. Quel est le prix du kg de café et celui 
 du kg de sucre ? 
 \item Une fermière vend 3 canards et 4 poulets pour 75 F. Sachant qu'un canard et un poulet valent 21 F, trouver
 le prix d'un canard et celui d'un poulet.
 \item Deux ouvriers gagnent à eux deux 30 F par jour. En un mois le premier a travaillé 24 jours et le deuxième 20 
 jours et ils ont reçu à eux deux 656 F. Quel est le salaire journalier de chacun d'eux ? 
 \item Un fabricant a vendu 5 mètres de toile et 10 mètres de drap pour 210 F; puis une autre fois 27 mètres de toile et 23 mètres de drap pour 631,80 F. Trouver le prix d'un mètre de toile et celui d'un mètre de drap.
 \item Un épicier vend 15 litres de liqueurs pour 96 F. L'une de ces liqueurs est vendue 5,40 F le litre, et l'autre 6,90 F le litre. Combien de litres de chaque sorte a-t-il vendus ? 
 \item Pour payer une somme de 890 F, on a donné 46 pièces, les unes de 20 F, les autres de 10 F. Combien de pièces de chaque sorte a-t-on données ? 
 \item Un négociant a vendu 37 quintaux de blé, les uns à 25 F, les autres à 27 F. Il a retiré 975 F de sa vente. Calculer le nombre de 
 quintaux de chaque sorte. 
 Une somme de 329~000 F doit être partagée entre trois personnes. La
 part de la deuxième est les $\frac43$ de celle de la première et celle
 de la troisième est la moitié de celle de la première plus 23~000 F.
 Calculer les trois parts.
 \item Une salle de cinéma comprend des places à 3 F, à 2,40 F, et à 2 F. Il y a deux fois plus de places à 2,40 F que de places à 3 F et le 
 nombre de places à 2 F est les $\frac5{11}$ du nombre total. La 
 salle complète fournit une recette de 1~792 F. Trouver le nombre 
 total de places.
 
 \end{enumerate}