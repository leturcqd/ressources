
 
 \chapter{Caractères de divisibilité}
 
\begin{enumerate}
\item  $a$ désignant un nombre entier, montrer qu'un nombre pair peut s'écrire $2a$, qu'un nombre impair peut s'écrire $2a+1$ ou $2a-1$. Vérifier que la somme ou la différence de deux nombres impairs est un nombre pair.  
\item Montrer que tout nombre qui n'est pas multiple de $3$ peut s'écrire $3a+1$ ou $3a-1$, $a$ désignant un nombre entier.
Vérifier que le produit de trois nombres entiers consécutifs est toujours multiple de $3$. 
\item Montrer que tout nombre qui n'est pas multiple de $5$ peut s'écrire $5a+1$, $5a-1$, $5a+2$ ou $5a-2$, $a$ désignant un nombre entier. Que peut-on dire du produit de cinq nombres entiers consécutifs ? 
\item Vérifier qu'un nombre entier est divisible par $11$ lorsque la différence entre la somme des chiffres
de rang impair et la somme des chiffres de rang pair 
est multiple de $11$. 
\item Déterminer les lettres $x$ et $y$ pour que le nombre qui s'écrit : 
\begin{enumerate}
\item $5x9$ soit divisible par $9$ ; 
\item $7x6$ soit divisible par $3$ ; 
\item $1~3x4$ soit divisible par $9$ ; 
\item $6~x5y$ soit divisible par $2$ et par $9$ ; 
\item $7~5xy$ soit divisible par $5$ et par $9$. 
\end{enumerate}
\item Deux nombres sont composés des mêmes chiffres écrits dans un ordre différent. Montrer que leur différence est un multiple de $9$. Exemple : $2~468$
et $6~482$. 
\item On échange le chiffre des dizaines et celui des unités d'un nombre de deux chiffres. Montrer que la différence des deux nombres est le produit par $9$
de la différence de leur deux chiffres. 

\emph{Application.} Trouver un nombre de deux chiffres sachant que la somme de ses chiffres est $11$ et que l'échange de ses deux chiffres le fait augmenter de $63$. 
\item Montrer que toute puissance de $1~000$ est un 
multiple de $37$ augmenté de $1$. En déduire que le reste de la division de $254~438~906$ par $37$ est le même que celui de la division de $254+438+906$ par $37$. 
\item Déterminer le plus grand nombre de $3$ chiffres, puis le plus grand nombre de $4$ chiffres terminés par 
un $5$ et divisibles par $9$. 
\item Dans les opérations suivantes une erreur a été
commise. Expliquer pourquoi cette erreur n'est pas 
mise en évidence par la preuve par $9$ de ces opérations.

  $ \begin{tabular}{cccc}
    & 2 & 5 & 7 \\ 
    &  & 7 & 8 \\
   \hline 
    2 & 0 & 5 & 6 \\
   1 & 7 & 9 & 9  \\ 
   \hline 
   3 & 8 & 5 & 5 
  \end{tabular}
 $\phantom{meooooow}
  $ \begin{tabular}{cccc}
    & 2 & 1 & 3 \\ 
    & 3 & 0 & 5 \\
   \hline 
    1 & 0 & 6 & 5 \\
   6 & 3 & 9 &   \\ 
   \hline 
   7 & 4 & 5 & 5 
  \end{tabular}
 $\phantom{meooooow}
  $ \begin{tabular}{cccc|cc}
  7 & 2 & 3 & 8 & 2 & 4\\
  & 1 & 2 & 8 & 3 & 5 \\
  & &  & 8  & &  
  
  \end{tabular}
 $
 \item La somme des chiffres d'un nombre inconnu est égale à $23$. La division de ce nombre par $9$ donne $96$ pour quotient. Trouver ce nombre. 
 \item On sait que les seuls nombres divisibles par $25$ sont les nombres terminés par $00$, $25$, $50$ ou 
 $75$. 
 
 En déduire que tous les nombres de $3$ chiffres divisibles à la fois par $9$ et par $25$. Les comparer au plus petit d'entre eux. 
 \item Un commerçant a vendu un certain nombre d'articles à $13$ F et à $18$ F pour une somme totale de 282 F. Montrer que le nombre d'articles vendus à 13 F est obligatoirement un multiple de 2 et un multiple de 3. En déduire le nombre d'articles de chaque sorte
 vendu. 
 \item Une somme de $5,10$ F est formée par des pièces de 50 centimes, et des pièces de 20 centimes : 
 \begin{enumerate}
 \item[a)]Trouver le nombre de pièces de chaque sorte
 sachant qu'il y en a 15 en tout.
 \item[b)] Y a-t-il d'autres façons de former une somme de $5,10$ F avec des pièces de $50$ centimes et de 20 centimes ? Si oui, les déterminer. 
 \end{enumerate}
 \item Une somme de $1~382$ F est formée par des billets de $50$ F, 10 F, et 5 F, et des pièces de 1 F. 
 Le nombre de billets de 5 F est le triple de celui des pièces de 1 F ; le nombre des billets de 10 F 
 dépasse de 5 celui des pièces de 1 F ; le nombre de 
 billets de 50 F dépasse de 2 celui des billets de 5 F. Calculer le nombre de billets ou de pièces de chaque sorte. 
 \item Établir la liste des diviseurs des deux nombres suivants, puis celle de leurs diviseurs communs : 
 
\begin{center}\begin{tabular}{ccc}
63 et 171. & 84 et 180. & 60 et 105.\\
120 et 216. & 126 et 210. & 108 et 252. \\ 
100 et 140. & 140 et 175. & 132 et 198. \\
112 et 231. & 95 et 225. & 1~815 et 2~385. \\
45 ; 108 et 135. &  & 55 ; 121 et 165.\\
252 ; 315 et 441. & & 378 ; 432 et 648. 
\end{tabular} \end{center}

\item Établir la liste des dix premiers multiples des
nombres suivants, puis celle de leurs cinq premiers
multiples communs. 
\begin{center}\begin{tabular}{ccc} 
36 et 54. & 42 et 54. & 66 et 110. \\
40; 45 et 72. & & 91 ; 117 et 273.\end{tabular}
\end{center}
\end{enumerate}