
\chapter{Médiatrice d'un segment. Constructions géométriques}
\begin{enumerate}
\item Partager géométriquement un segment $[AB]$ en huit parties égales. 
\item Partager géométriquement un angle donné $\widehat{xOy}$ en $4$ angles égaux. Même problème pour un arc donné. 
\item Diviser géométriquement un cercle en $16$ arcs égaux. 
\item On se donne quatre points $ABCD$ tels que $(AB)$ soit perpendiculaire à $(CD)$ et $(AC)$ à $(BD)$. Construire les quatre cercles passant par trois de ces points.
Que remarque-t-on au sujet de leurs rayons ?
\item Trois cercles égaux de $3$ cm de rayon passent par un même point $A$ et 
se recoupent deux à deux en $B$, $C$ et $D$. Construire le cercle $BCD$ et mesurer
son rayon. 
\item \begin{enumerate}
\item Construire géométriquement le cercle circonscrit à un triangle $ABC$. 
\item Tracer sur la même figure les trois médianes du triangle. Que remarquez-vous ?
\end{enumerate}
\item Construire géométriquement les trois hauteurs d'un triangle, dans le cas où les trois angles sont aigus, puis où un angle est obtus. Que remarquez-vous ? 
\item Construire les trois bissectrices intérieures et les trois bissectrices extérieures du triangle $ABC$. Y a-t-il des points communs à trois de ces droites ?
\item Trois cercles égaux de centres donnés $A$, $B$, $C$ se coupent deux à deux. 
Démontrer que les cordes communes à deux de ces cercles concourent en un point $O$. 
Que représente ce point pour le triangle $ABC$ ? 
\item 
\begin{enumerate}
\item Construire sur une droite donnée $(\Delta)$ (ou sur un cercle $C$) un point équidistant de deux points donnés $A$ et $B$. 
\item Construire un cercle passant par $A$ et $B$ sachant que son centre est sur $(\Delta)$ (ou sur $C$).
\end{enumerate}
\item On mène d'un point $O$ la perpendiculaire $OH$ et deux obliques $OA$ et $OB$ de part et d'autre de $OH$ à une droite donnée $(xy)$.
\begin{enumerate}
\item Montrer que l'une des égalités $OA=OB$ et $HA=HB$ entraîne l'autre. 
\item Montrer que chacune des égalités précédentes est aussi équivalente à $\widehat{OAH} = \widehat{OBH}$ et à $\widehat{HOA}=\widehat{HOB}$.
\end{enumerate}
\item Soit un triangle isocèle $ABC$ de base $[BC]$ et soit $(xy)$ la bissectrice extérieure de l'angle $\widehat{A}$. On porte sur $(xy)$ deux segments égaux $[AD]$
et $[AE]$ ($D$ du côté de $B$ et $E$ du côté de $C$).\begin{enumerate}
\item Que représente la hauteur $[AH]$ du triangle $ABC$ pour le segment $[DE]$ ?
Comparer $HD$ et $HE$. 
\item Comparer les triangles $ABD$ et $ACE$ puis $ABE$ et $ACD$. Conséquences
pour $BD$ et $CE$ ainsi que pour $BE$ et $CD$ ? 
\end{enumerate}
\item On considère un triangle isocèle $AOB$ de sommet $O$. On prend un point $A'$ sur le côté $[OA]$ et un point $B'$ sur le côté $[OB]$ tels que les angles 
$\widehat{ABA'}$ et $\widehat{BAB'}$ soient égaux. $[AB']$ et $[AB]$ se coupent en $I$.\begin{enumerate}
\item Montrer que le triangle $IAB$ est isocèle. Préciser la position du point $I$
par rapport à la hauteur $[OH]$ du triangle $OAB$. 
\item Réciproquement,montrer que lorsque $I$ appartient au segment $[OH]$, les 
angles $\widehat{ABA'}$ et $\widehat{BAB'}$ sont égaux. Comparer $AA'$ et $BB'$.
\end{enumerate}
\item On prolonge la base $[AB]$ du triangle isocèle $[OAB]$ de deux longueurs égales
$AC$ et $BD$.\begin{enumerate}
\item Montrer que la hauteur $[OH]$ du triangle $OAB$ est la médiatrice de $[CD]$.
\item Démontrer que le triangle $OCD$ est isocèle puis que les triangles $OAC$ 
et $OBD$ sont égaux ainsi que les triangles $OAD$ et $OBC$.
\end{enumerate}
\item Soient $D$, $E$, $F$ les symétriques d'un point $M$ intérieur à un triangle $ABC$ par rapport aux côtés $[BC]$, $[CA]$ et $[AB]$.\begin{enumerate}
\item Que représente $A$ pour le cercle $MEF$ ? Propriété analogue des points $B$
et $C$ ? En déduire une construction au compas des points $D$, $E$, $F$. 
\item Les médiatrices du triangle $DEF$ se coupent en $P$. Montrer que $[AP]$ est
la bissectrice de l'angle $\widehat{EAF}$. Sachant que $\widehat{BAM}=16^o$ et $\widehat{CAM}=45^o$ trouver la valeur de l'angle $\widehat{CAP}$ et la comparer à celle de l'angle $\widehat{BAM}$. 
\end{enumerate}
\item Dans un triangle $ABC$ les médiatrices des côtés $[AB]$ et $[AC]$ se coupent
en un point $O$ de $[BC]$. On désigne par $M$ et $N$ les milieux de $[AB]$ et $[AC]$.
\begin{enumerate}
\item Montrer que $O$ est le milieu de $[BC]$ et que le cercle de diamètre $[BC]$ passe par $A$. 
\item Quelle est la nature du triangle $MON$ ? Montrer que l'angle $\widehat{A}$ du 
triangle $ABC$ est égal à la somme des angles $\widehat{B}$ et $\widehat{C}$ de ce triangle.
\end{enumerate}
\item Les médiatrices des côtés $[AB]$ et $[AC]$ du triangle $ABC$ se coupent en $O$
à l'intérieur de l'angle $\widehat{BAC}$. Comparer la somme des angles $\widehat{B}$
et $\widehat{C}$ du triangle à la somme des angles $\widehat{ABO}$ et $\widehat{ACO}$
puis à l'angle $\widehat{A}$ du triangle suivant que le point $O$ est : \begin{enumerate}
\item intérieur au triangle ; \item sur le côté $[BC]$ ; \item extérieur au triangle.
\end{enumerate}
\item Dans un quadrilatère $ABCD$ les médiatrices des segments $[AB]$, $[AC]$ et 
$[AD]$ se coupent en un même point $O$. \begin{enumerate}
\item Montrer que les points $A$, $B$, $C$, $D$ sont sur un même cercle. 
\item Montrer que les médiatrices du triangle $BDC$ se coupent en $O$.
\end{enumerate}
\item On considère un quadrilatère convexe $ABCD$ dans lequel la médiatrice de 
$[AB]$ est également la médiatrice de $[CD]$. Soit $O$ le point où cette médiatrice coupe la médiatrice de $[BC]$. \begin{enumerate}
\item Montrer que les quatre points $A$, $B$, $C$ et $D$ sont situés sur un même cercle de centre $O$. 
\item Démontrer que les médiatrices de $[AD]$, $[AC]$ et $[BD]$ passent également par $O$. 
\end{enumerate}
\item \begin{enumerate}
\item Construire un triangle isocèle $ABC$ tel que $BC = 5$ cm et $\widehat{B}=\widehat{C}= 53^o$ ; le point $D$ de $[BC]$ tel que $DA=DB$ et enfin le point $E$ 
du segment $[AD]$ tel que $AE=CD$.
\item Comparer les triangles $ABE$ et $CAD$. Nature du triangle $BDE$.
\end{enumerate}
\end{enumerate}