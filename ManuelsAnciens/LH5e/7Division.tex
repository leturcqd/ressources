
 
 \chapter{Division des nombres entiers}
 \begin{enumerate}
 \item Trouver tous les nombres entiers dont le 
 produit par 62 est inférieur à 685. 
 \item Montrer que le nombre des chiffres du quotient
 dans une division est égal au plus petit nombre de zéros qu'il 
 faut écrire à la droite du diviseur pour 
 obtenir un nombre supérieur au dividende.
 \item Montrer que, dans une division, le dividende est
 supérieur au double du reste. 
 \item Dans une division, le diviseur est 9. 
 Quels sont les restes possibles ? 
 \item Trouver les nombres qui, divisés par 13, 
 donnent un quotient et un reste égaux entre eux. 
 \item Quels sont les nombres qui, divisés par 7, 
 donnent un quotient égal à la moitié du reste ? 
 \item Quels sont les nombres qui, divisés par 5, 
 donnent un quotient égal au triple du reste. 
 \item Trouver tous les couples de nombres entiers $x$
 et $y$ qui satisfont à la relation suivante :
 \[ 287 = 17x + y\]
 \item Le quotient d'une division est 5 et le reste 32
 Trouver la plus petite valeur du diviseur et du dividende. Le dividende étant inférieur à 225, quelles 
 sont les valeurs possibles pour le dividende et le
 diviseur ? 
 \item Trouver un nombre terminé par deux zéros qui,
 divisé par 67, donne pour quotient 129. 
 \item Trouver deux nombres connaissant leur somme, 
 958 et sachant qu'en divisant le premier par le second on trouve 3 comme quotient et 98 comme reste.
 \item Trouver deux nombres connaissant leur différence, 291, et en sachant qu'en divisant le 
 premier par le second on trouve 13 pour quotient et 15 pour reste. 
 \item Effectuer la division de 272 par 57. 
 De combien peut-on augmenter le dividende sans 
 changer le quotient ? De combien peut-on diminuer le 
 dividende sans changer le quotient ? Généraliser lorsque le dividende et le diviseur sont deux nombres
 donnés $a$ et $b$, et $q$ et $r$ le quotient et le 
 reste de leur division. 
 \item Le quotient d'une division est 5, le reste 28. 
 En additionnant le dividende, le diviseur, le
 quotient et le reste, on trouve 283. Trouver 
 le dividende et le diviseur.
 \item On considère la division de 272 par 57. Montrer que le quotient ne change pas lorsqu'on multiplie le dividende et le diviseur par un même nombre. Que devient le reste ? 
 \item On considère la division de 236 par 36. Montrer que le quotient ne change pas lorsqu'on divise le dividende et le diviseur par un même nombre. Que devient le reste ? 
 \item Dans une division, le quotient est 21 et le
 reste est 8. Si on ajoute 27 au dividende
 sans changer le diviseur, le quotient est 22 et le
 reste est nul. Trouver le dividende et le diviseur 
 initiaux. 
 \item On augmente le dividende d'une division de 
 35 et le diviseur de 5. Il se trouve que ni le 
 quotient, ni le reste ne change. Quel est le quotient
 ?
 \item On dispose d'un certain nombre de billes. 
 En les rangeant par dizaines, il en reste 8. Mais il manque 5 billes pour pouvoir en ajouter une de plus par groupe. Trouver le nombre de billes. 
 \item On dispose de 225 g d'argent avec lequel on se propose de faire frapper des médailles au titre\footnote{Cela signifie qu'il y a 0,9 gramme d'argent dans chaque gramme de la médaille.} de 
 0,900 et pesant 15 g chacune. Combien pourra-t-on 
 en fabriquer ? 
 \item Une pièce de drap de 36 m de long et coûtant 
 23 F le mètre a été utilisée pour confectionner
 des costumes. On compte pour 3,20 m de tissu par 
 costume et 85 F de frais de main-d'œuvre et de fournitures. Les costumes sont vendus 189 F. Calculer 
 le bénéfice réalisé par le fabricant. 
 \item Une tente qui a pour base un rectangle de 6 m sur 2 m est fermée à ses extrémités par deux triangles isocèles verticaux de 2 m de base et de 1,15 m de 
 hauteur. Latéralement, elle se compose de deux parties inclinées rectangulaires. 
 \begin{enumerate}
 \item Faire un dessin à main levée.
 \item Calculer le volume intérieur de cette tente.  \item Combien d'hommes pourra-t-on
 y abriter si l'on veut que chacun dispose de 0,7 m${}^3$ ?
 \end{enumerate}
 \item Un cultivateur a fait venir en gare un wagon 
 d'engrais. Ce wagon mesure 6 m de long, 2,50 m de large et est chargé sur une hauteur de 80 cm. 
 L'engrais pèse 130 kg à l'hectolitre. Le cultivateur dispose d'un tombereau qui peut supporter 2,4 tonnes. 
 \begin{enumerate}
 \item Combien de voyages seront nécessaires pour enlever
 tout l'engrais ?
 \item  Afin de ménager son attelage, le cultivateur décide de faire un voyage de plus et de répartir la charge également sur les différents voyages. Quel masse charge-t-on à chaque voyage ? 
 \end{enumerate}
 \item Deux caisses contiennent chacune 145 oranges. On retire 25 oranges de la première caisse pour les mettre dans la deuxième. 
 \begin{enumerate}
 \item Combien la deuxième caisse contient-elle alors d'oranges de plus que la première ? 
 \item On répartit les oranges de chacune des caisses 
 dans des caissettes qui en contiennent chacune 25. 
 Combien de caissettes pourra-t-on remplir avec chaque
 caisse ? Pourrait-on, en réunissant les oranges restant dans les deux caisses, remplir une caissette
 de plus ? Y aurait-il encore du reste ? 
 \item Quel est le plus petit nombre d'oranges qu'il eût suffi d'ajouter à chacune des caisses initiales pour que la répartition en caissettes, effectuée après l'opération du (a), se fasse sans reste ? 
 \end{enumerate}
 \item Compléter les divisions suivantes. 
 
 \end{enumerate}