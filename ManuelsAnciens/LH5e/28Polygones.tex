
\chapter{Polygones. Triangles} 
\begin{enumerate}
\item Construire les médianes, les hauteurs, les bissectrices intérieures et extérieures d'un triangle. Que remarquez-vous ? 
\item Construire un triangle rectangle et isocèle dont les côtés de l'angle droit valent $32$ mm. 
\item Construire un triangle rectangle connaissant l'hypoténuse $58$ mm et un côté de 
l'angle droit $42$ mm. 
\item Construire un triangle rectangle connaissant l'hypoténuse $44$ mm et un angle
$32^o$. 
\item Construire un triangle rectangle connaissant un côté de l'angle droit $37$ mm et un angle adjacent à ce côté $53^o$. 
\item Construire un triangle isocèle dont les deux côtés égaux mesurent $54$ mm et le troisième côté $60$ mm. 
\item Construire un triangle équilatéral de $29$ mm de côté. 
\item Construire un triangle $ABC$ connaissant $\widehat{A}=52^o$, $AB=27$ mm, $AC=38$ mm. 
\item Construire un triangle $ABC$ connaissant $BC = 57$ mm, $\widehat{B}= 63^o$, $\widehat{C}= 51^o$. 
\item Construire un triangle $ABC$ connaissant la hauteur $AH=18$ mm, $AB=25$ mm, et $BC=50$ mm (on supposera $H$ entre $B$ et $C$). 
\item Construire un triangle $ABC$ connaissant la hauteur $AH= 18$ mm, $AB=25$ mm, et $AC=30$ mm.
\item Construire un triangle $ABC$ connaissant la hauteur $AH=22$ mm, $AB=29$ mm, et $\widehat{A}=37^o$. 
\item Construire un triangle $ABC$ connaissant $AH=30$ mm, $BH= 12$ mm, et $AC=40$ mm. 
\item Construire un triangle $ABC$ connaissant $AB=38$ mm, la médiane $AM=23$ mm, et $BM=23$ mm. Mesurer les angles de ce triangle. 
\item Construire un triangle $ABC$ connaissant $AB=41$ mm, $\widehat{BAC}= 54^o$
et la bissectrice intérieure $AD=45$ mm. 
\item Construire un triangle $ABC$ connaissant : $\widehat{A}= 120^o$, $AB= 30$ mm, et $BC = 57$ mm. 
\item Construire un triangle isocèle $ABC$ sachant que $AB=AC=32$ mm, et que $\widehat{B}= 65^o$. 
\item Découper dans une feuille de bristol ou de léger carton une fausse équerre triangulaire $ABC$ telle que $BC=10$ cm, $\widehat{B}=60^o$ et $\widehat{C}=72^o$. 
Vérifier que l'angle $\widehat{A}$ mesure $48^o$.
\item Utiliser la fausse équerre de l'exercice précédent pour mener par un point extérieur $A$ une oblique $[AM)$ faisant un angle de $60^o$ avec une droite donnée $(xy)$. Nombre de solutions ?
\item Construire un triangle $ABC$ rectangle en $A$ tel que $AB=40$ mm et $\widehat{C}= 72^o$. (Utiliser la fausse équerre précédente.)
\item Construire un triangle $ABC$ tel que $AB= 32$ mm, $\widehat{B}= 65^o$, et $\widehat{C}= 47^o$. (Utiliser la fausse équerre précédente.)
\item Construire un triangle isocèle $ABC$ tel que $\widehat{B}= \widehat{C}= 72^o$ avec une hauteur $AH=48$ mm. (Utiliser la fausse équerre précédente.)
\item Construire un triangle $ABC$ sachant la hauteur $AH=45$ mm et que les angles $\widehat{B}$ et $\widehat{C}$ sont égaux à $60^o$ et $72^o$. (Utiliser la fausse équerre précédente.)
\item Construire un triangle $ABC$ connaissant la hauteur $AH= 36$ mm, l'angle $\widehat{B}=120^o$ et l'angle $\widehat{C}= 48^o$. 
\item Construire un quadrilatère convexe $ABCD$ sachant que $AB=AC= BC=5\cm$, $AD=3\cm$ et $CD=4\cm$. Mesurer au rapporteur les angles $\widehat{B}$ et $\widehat{D}$ du quadrilatère obtenu.
\item Construire un quadrilatère convexe $ABCD$ connaissant $\widehat{A}=120^o$, $AB=3\cm$, $AD=2\cm$, $BC=3,5\cm$, et $CD=4\cm$. Mesurer au rapporteur les angles
$\widehat{B}$, $\widehat{C}$ et $\widehat{D}$ du quadrilatère obtenu.
\end{enumerate}