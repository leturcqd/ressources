
\chapter*{Simplification des fractions}

\begin{enumerate}
\item Simplifier les fractions 
\[ \frac{77}{121}; \phantom{meow}
\frac{156}{208};\phantom{meow}
\frac{225}{375};\phantom{meow}
\frac{125}{1~000};\phantom{meow}
\frac{130}{273}; \phantom{meow}
\frac{2~352}{5~376}; \phantom{meow}
\frac{30~752}{37~800}.\]
\item Simplifier les fractions : 
\[\frac{18\times 35\times77}{66\times 21\times9}; 
\phantom{meow} \frac{45\times 38\times 34\times 100}{25\times 95\times 17};\phantom{meow}\frac{3\times 5^2\times7^4}{3^2\times5^4\times 7};\phantom{meow}\frac{2^5\times3^2\times11}{2^5\times3^3\times17}.\]
\item Réduire au même dénominateur les fractions : 
\[ \frac{17}{25}\text{  et  }\frac{19}{45};\phantom{meow}
\frac{5}{26}\text{  et  }\frac{7}{39};\phantom{meow}
\frac1{350}\text{  et  }\frac1{420};\phantom{meow}
\frac{13}{77}\text{  et  }\frac{9}{66}.\]
\item Réduire au même dénominateur les fractions : 
\[\frac{18}{36}, \frac{19}{38}\text{  et  }\frac{17}{40}; \phantom{meow} 
\frac{11}{20}, \frac{34}{60} \text{  et  }\frac{128}{120}; \phantom{meow}
\frac{15}{17}, \frac{11}{60} et \frac{15}{66}; \phantom{meow}
\frac58, \frac79, \frac3{24} et \frac5{18}.\]
\item Comparer les fractions : \[\frac{15}{16}\text{  et  }\frac{14}{15}; \phantom{meow}\frac{13}{21}\text{  et  }\frac{15}{28}; \phantom{meow}\frac{84}{115}\text{  et  }\frac{36}{43}\]
\item Ranger par ordre de grandeur les fractions et les nombres : \[\frac74 \phantom{meow}\frac78\phantom{5}{12}\phantom{meow}\frac{23}{45}
\phantom{meow}\frac{27}{50}\phantom{meow}2\phantom{meow}\frac{39}{12}\phantom{meow}3.\]
\item \begin{enumerate}
\item Réduire au même dénominateur les fractions : 
\[\frac12\phantom{meow}\frac23\phantom{meow}\frac34
\phantom{meow}\frac45\phantom{meow}\frac56\phantom{meow}\frac67.\]
\item Les réduire au même numérateur ; 
\item les comparer. 
\end{enumerate}
\item Trouver toutes les fractions égales à $\frac{60}{72}$ et dont les termes soient plus petits que ceux
de cette fraction. 
\item Montrer qu'en ajoutant aux deux termes de la fraction $\frac49$ les produits de $4$ et de $9$ par un même nombre entier, on obtient un fraction égale à $\frac49$. Généraliser. 
\item Démontrer que les fractions $\frac{141}{329}$ 
et $\frac{111}{259}$ sont égales. Simplifier ces deux fractions, puis comparer la fraction irréductible 
trouvée à chacune des deux fractions : 
\[\frac{141-111}{329-259}\text{   et   }\frac{141+111}{329+259}.\]
\item Deux roues font, l'une 17 tours en 5 secondes et l'autre 7 tours en 2 secondes. Quelle est celle qui
tourne le plus vite ? 
\item La capacité d'un flacon est égale aux $\frac3{16}$ de la capacité d'un vase A et aux $\frac4{21}$ de celle d'un vase B. Comparer les capacités des deux vases. Quelle fraction de A représente B ? 
\item Une pièce de toile a une longueur double de celle d'une pièce de soie. On vend le tiers de la pièce de toile et les $\frac47$ de la pièce de soie. Comparer les longueurs des coupons vendus. 
\item Deux réservoirs identiques sont emplis d'eau l'un aux $\frac47$, l'autre aux $\frac59$. Quel est celui qui contient le plus d'eau ? Sachant que l'un d'eux contient $5$ litres d'eau de plus que l'autre, trouver la capacité de l'un de ces réservoirs et le contenu de chacun. 
\item Deux angles A et B représentent, l'un 
les $\frac35$, l'autre les $\frac47$ d'un angle C. 
\begin{enumerate}
\item Quelle fraction de l'angle A représente l'angle B ? 
\item Sachant que la différence entre les angles A et B est égale à $3^o45'$ calculer les valeurs des trois angles A, B et C. 
\end{enumerate}
\item On considère les fractions $\frac{a}{b}$ et $\frac{a+1}{b+1}$. 

Montrer que la seconde est supérieure à la première si $a< b$. 

Montrer que la seconde est supérieure à la première si $a > b$. 

Plus généralement, comparer les fractions $\frac{a}{b}$ et $\frac{a+n}{b+n}$; puis $\frac{a}{b}$ et $\frac{a-n}{b-n}$ (n étant inférieur à a et b). En supposant : a inférieur à b, a supérieur à b. 

\item On a vendu trois coupons d'une même pièce d'étoffe. Le premier représente le $\frac13$ de la pièce, le second les $\frac5{18}$, et les $\frac{7}{24}$. \begin{enumerate}
\item Classer les 3 coupons par ordre de grandeur. 
\item Sachant que le 3${}^e$ coupon mesure 10,50 m, 
calculer la longueur de la pièce et celle de chacun des deux autres coupons. 
\end{enumerate}
\item Un marchand achète deux lots de marchandises qu'il paie 44~100 F chacun. Il revend le premier lot 56~920 F et il fait sur le second un bénéfice de 16\% sur le prix de vente. \begin{enumerate}
\item Quelle fraction d'un prix d'achat représente chacun des bénéfices réalisés ? 
\item Calculer le prix de vente du deuxième lot et 
le bénéfice sur ce lot. 
\end{enumerate} 
\item Un négociant calcule un prix de vente. Il hésite entre un bénéfice de 18\% sur le prix d'achat et un bénéfice de 15\% sur le prix de vente. \begin{enumerate}
\item Quelles fractions du prix d'achat représentent chacun des bénéfices envisagés ? Quel est le plus important ? 
\item Calculer la différence entre ces deux bénéfices 
sachant que le prix d'achat est de 76~000 F. 
\end{enumerate}
\item \begin{enumerate}
\item On donne la fraction $\frac25$. On ajoute 14 au numérateur et 35 au dénominateur. Comparer la fraction obtenue à la fraction donnée. 
\item Quels nombres faut-il ajouter au numérateur et au dénominateur de la fraction $\frac25$ pour obtenir une fraction égale dont le dénominateur soit 150 ?
\end{enumerate}
\end{enumerate}