
\chapter{Le cercle} 
\begin{enumerate}
\item Calculer en degrés la mesures des arcs égaux à $\frac13$, $\frac23$, $\frac15$, $\frac25$, $\frac18$, $\frac38$, $\frac1{10}$ et $\frac3{10}$ d'un cercle et construire ces arcs.
\item Effectuer en grades les calculs précédents.
\item Construire sur un cercle de $3\cm$ de rayon les arcs successifs $AB$, $BC$, $CD$ et $DE$ mesurant respectivement : $27^o$, $53^o$, $75^o$ et $108^o$. Calculer la mesure de l'arc $EA$ qui reste sur le cercle. 
\item Démontrer le théorème suivant : \emph{La bissectrice d'un angle au centre $\widehat{AOB}$ passe par le milieu $M$ de l'arc intercepté par cet angle.}\\ Énoncer et démontrer la réciproque de ce théorème. 
\item Énoncer et démontrer la réciproque de chacune des propositions : 
\begin{enumerate}
\item \emph{Si un point est intérieur à un cercle, sa distance au centre est inférieure au rayon.}
\item \emph{Si un point est extérieur à un cercle, sa distance au centre est supérieure au rayon.}
\end{enumerate}
\item On donne deux points $A$ et $B$ distants de $5\cm$. Construire un point $C$ tel que $AC= 6\cm$ et $BC = 4\cm$. Combien y a-t-il de solutions ?
\item Quelle est la figure formée par l'ensemble des centres des cercles de $5\cm$ de rayon passant par un point donné $A$ ? \\
Construire un cercle de $5\cm$ de rayon passant par deux points donné $A$ et $B$ tels que $AB=4\cm$. 
\item On donne un cercle de centre $O$, de rayon $4\cm$ et un point $A$ de ce cercle. Construire les cordes issues de $A$ ayant $2,5\cm$ de longueur. 
\item Construire les milieux $M$ et $N$ des deux arcs d'extrémités $A$ 
et $B$ d'un cercle donné de centre $O$. Que représentent $OM$ et $ON$ pour les angles saillant et rentrant $\widehat{AOB}$ ? Comment sont disposés les 3 points $M$, $O$ et $N$ ?
\item Quatre points $A$, $B$, $C$, $D$ sont disposés dans cet ordre sur un demi-cercle. \begin{enumerate}
\item Sachant que $\arc{AB}= \arc{CD}$, comparer les arcs $\arc{AC}$ et $\arc{BD}$. 
\item Dans ce cas montrer que les arcs $\arc{AD}$ et $\arc{BC}$ ont même milieu $M$. 
\end{enumerate}
\item Trois points $A$, $B$ et $C$ sont situés dans cet ordre sur un cercle. Les arcs $AB$ et $AC$ mesurent respectivement $88^o$ et $154^o$. Soit $M$ le milieu de l'arc $\arc{BC}$.\begin{enumerate}
\item Montrer que $\arc{AM}=\arc{AB}+\arc{BM}=\arc{AC}-\arc{BM} = \frac12(\arc{AB}+\arc{AC}).$
\item Évaluer l'angle $\widehat{AOM}$. Comparer sa valeur à la somme des angles 
$\widehat{AOB}$ et $\widehat{AOC}$. 
\end{enumerate}
\item On considère sur un cercle deux arcs consécutifs $\arc{BA}=78^o$ et $\arc{AC}=54^o$. 
\begin{enumerate}
\item Construire les milieux $M$ et $N$ de ces deux arcs. 
\item Calculer la mesure de l'arc $\arc{MAN}$ et la comparer à la mesure de l'arc $\arc{BAC}$. Généraliser. 
\end{enumerate}
\item Soient sur un cercle deux arcs de même sens : $\arc{AB}=42^o$ et $\arc{AC}= 108^o$. \begin{enumerate}
\item Construire les milieux $M$ et $N$ de ces deux arcs et calculer la mesure de 
l'arc $\arc{MBN}$. 
\item Comparer cette mesure à celle de l'arc $\arc{BNC}$. Généraliser. 
\end{enumerate}
\end{enumerate}