\chapter{Numération}
 
 \begin{enumerate}
 \item Combien faut-il de mots différents pour nommer tous les nombres jusqu'à un million ? 
 \item On écrit les $237$ premiers nombres. Combien, au total, a-t-on écrit de chiffres ? Même question pour les nombres entre $94$ et $237$. 
 \item On écrit tous les nombres de deux chiffres. Combien en écrit-on ? 
 Combien de chiffres écrit-on au total ? Même question pour les nombres de trois chiffres. 
 \item Pour numéroter les pages d'un livre, on emploie $408$ caractères d'imprimerie. Quel est le nombre de pages de ce livre ?
 \item On écrit les $467$ premiers nombres. Combien de fois écrit-on le chiffre $3$ ? Combien de fois écrit-on le chiffre $5$ ? Combien de fois écrit-on le chiffre $8$ ? 
 \item Former tous les nombres de trois chiffres qui s'écrivent avec les chiffres $3$, $5$, $7$. Classer ces nombres dans l'ordre croissant.\\
 Même question pour les nombres de quatre chiffres qui s'écrivent avec $3$, $5$, $7$, $9$. 
 \item Combien faut-il de dizaines, de centaines, de mille pour former un million, un milliard, $35$ millions, $17$ milliards ?
 \item Dans un nombre de deux chiffres, le chiffre des dizaines est $7$, on place un zéro entre les deux chiffres de ce nombre. De combien augmente-t-on ainsi sa valeur ? \\
 Soit le nombre $672$. On intercale un zéro entre les chiffres $6$ et $7$ et un zéro entre les chiffres $7$ et $2$. De combien augmente-t-il ainsi ? 
 \item Quels sont les plus petit et le plus grand nombre de $4$ chiffres ? Combien y a-t-il de nombres ayant moins de $4$ chiffres ? moins de $5$ chiffres ? En déduire combien il existe de nombres de $4$ chiffres. Généraliser. 
 \item Écrire en chiffres romains les nombres suivants : 
 \[ 349 \ \ \ \ 654 \ \ \ \ 1\ 794 \ \ \ \ 2\ 497. \]
 Écrire en chiffres indo-arabes les nombres suivants : 
 \[ \rm CXLIX \ \ \ \ CDLXVII \ \ \ \ MCCXLIV \ \ \ \ MCDXCIV. \]
 
 \item Dans un nombre de deux chiffres, le chiffre des dizaines est le double du chiffre des unités, et la somme de ces deux chiffres est $12$. Trouver ce nombre. 
 \item Dans un nombre de trois chiffres, le chiffre des unités dépasse de $2$ celui des dizaines et ce dernier est le triple du chiffre des centaines. La somme des trois chiffres est $16$. Trouver ce nombre. 
 \item Une loterie comprend $5\ 000$ billets numérotés de $1$ à $5\ 000$. Les frais d'organisation s'élèvent à $33,50$ F. Tous les billets ont été vendus $1$ F l'un. Les billets se terminant par $27$ gagnent $10$ F. Tous les billets se terminant par $135$ gagnent $200$ F et le numéro $2\ 791$ gagne le gros lot, soit $1\ 000$ F. Quel est le bénéfice réalisé ? 
 \item On organise une loterie comprenant $1\ 000$ billets numérotés de $1$ à $1 000$ et qui sont tous vendus $0,50$ F chacun. Les frais d'organisation se montent à $50$ F. Les billets terminés par $7$ gagnent $1$ F, les billets terminés par $35$ gagnent $10$ F et le gros lot est gagné par le numéro $794$.
 Le bénéfice réalisé est de $150$ F. Quel est le montant du gros lot ? 
 
 \end{enumerate}