
 \chapter{Produits de deux nombres}
 \begin{enumerate}
 \item Dans un nombre entier de deux chiffres, on appelle $a$ le chiffre des dizaines, et $b$ celui des unités. Montrer que la valeur de ce nombre est $10a+b$.
 Même exercice pou un nombre de trois chiffres en désignant par $a$ le chiffre des centaines, $b$ celui des dizaines, et $c$ celui des unités.
 \item Un libraire achète cinq douzaines de livres à 
 24 F la douzaine et les revend 3 F pièce. Trouver le 
 bénéfice réalisé, sachant que l'éditeur donne 13 livres pour 12 au libraire. 
 \item La lumière parcourt 300~000 kilomètres par
 seconde. Évaluer la distance de la Terre au soleil,
 sachant que la lumière met $8$ min $30$ à parcourir
 cette distance.
 \item Trouver un nombre de $2$ chiffres sachant que la somme de ses chiffres est $12$, et qu'en 
 retranchant de ce nombre le nombre écrit dans l'ordre
 inverse on trouve 18. 
 \item Écrire plus simplement les sommes suivantes :
 \[ (a + b + c) + (a + b + c) + (a + b + c).\]
 \[(a - b) + (a - b) + (a - b) + (a - b).\]
 \item Le périmètre d'un rectangle est 386 m ; 
 la longueur a 23 m de plus que la largeur.
 Trouver la surface du rectangle.
 \item Dans la multiplication de 243 par 405, on ne tient pas compte du O au multiplicateur. Trouver, sans faire la multiplication, l'erreur ainsi commise. 
 \item En multipliant un nombre par 207, on oublie de 
 tenir compte du zéro du multiplicateur. On fait ainsi
 une erreur de 64~080. Retrouver le multiplicande\footnote{Dans un produit $a\times b$, $a$ est le \emph{multiplicande} et $b$ le \emph{multiplicateur}.} et le résultat correct de la multiplication.
 \item On considère le produit $56 \times 43$. On augmente le multiplicateur de $8$. Trouver sans effectuer les multiplications l'augmentation du produit. 
 \item Une mercière vend une première fois 52 mètres de drap à 36 francs le mètre, et une seconde fois 65 mètres de drap à 42 francs le mètre. Trouver, sans calculer les deux prix de vente, la différence entre ces deux prix. 
 \item Le produit de deux nombres est 109~450. Trouver ces deux nombres sachant que le multiplicateur a deux chiffres, que le chiffre de ses unités est 5 et que le premier produit partiel\footnote{La première ligne lorsque vous posez le produit.} de l'opération est 21~890. 
 \item On veut clore un jardin rectangulaire de 42 m de longueur et de 30 m de largeur à l'aide d'un grillage de 2 m de haut soutenu par des poteaux en 
 ciment distants de 2 m. Le grillage pèse 4 kg au mètre carré et revient à 72 F le quintal. Calculer la dépense sachant qu'un poteau coûte 4,50 F et qu'il faut ajouter une dépense supplémentaire de 25 F pour 
 le bâti de la porte d'entrée.
 \item Une école de trois classes brûle par jour et par classez deux seaux de charbon contenant 8 kg de 
 combustible. Calculer la dépense en une année sachant
 que l'on a chauffé pendant 25 semaines à raison de 
 5 jours par semaine et que le charbon utilisé revient à 180 F la tonne. 
 \item Une ruche produit en moyenne 10 kg de miel et 15 kg de cire. Le miel vaut 5,20 F le kg et la cire 3,60 F le kg. Calculer le rapport annuel d'un rucher 
 de 18 ruches sachant que les frais d'entretien s'élèvent au quart du produit total. 
 \item La toiture d'un hangar est composée de deux 
 trapèzes isocèles égaux dont les bases mesurent 
 10 m et 4 m et de deux triangles isocèles égaux de 6 m de base. La hauteur des trapèzes et des triangles est de 4,50 m. On recouvre la toiture de plaques de 
 fibrociment qui revient à 8 F le mètre carré. Calculer la dépense.\footnote{La surface d'un trapèze est donnée par la formule $\mathcal A = \text{moyenne des bases}\times \text{hauteur}$.}
 \item La façade d'un magasin a la forme d'un rectangle de 9 m de long et de 3,50 m de hauteur. Elle
 comprend trois baies vitrées. Chacune d'elles se compose d'un rectangle de 2 m de large et de 1,60 m de haut surmonté d'un demi-cercle de 2 m de diamètre. 
 \begin{enumerate}
 \item Faire le croquis de la façade en prenant 1 cm 
 pour 1 m, sachant qu'il y a un intervalle de 50 cm
 entre deux baies vitrées et que celle du milieu occupe le centre de la façade. 
 \item On fait recouvrir cette façade de plaques de marbre qui reviennent à 4,50 F le mètre carré. Calculer la dépense. 
 \end{enumerate}
\item \begin{enumerate}
\item Soit $a$ l'un des nombres entiers de 0 à 10. 
Établir les tableaux de correspondance entre $a$ et 
les nombres $b$, $c$ et $d$ tels que : 
\[ b= 3 a ; \phantom{meowmeowmeow} c = 3a + 2 ;
\phantom{meowmeowmeow} d = 3a + 5\]
\item Construire les graphiques correspondants. 
\end{enumerate}
\item \begin{enumerate}
\item Soit $a$ l'un des nombres entiers de 5 à 15. 
Établir les tableaux de correspondance entre $a$ et 
les nombres $b$, $c$ et $d$ tels que : 
\[ b= 2 a ; \phantom{meowmeowmeow} c = 2a + 4 ;
\phantom{meowmeowmeow} d = 2a - 5\]
\item Construire les graphiques correspondants. 
\end{enumerate}
\item Compléter les multiplications suivantes : 

  $ \begin{tabular}{cccccc}
   & & 4 & . & 5 & 3 \\
   & &   &   & . & 7  \\
   \hline 
   & . & . & 2 & . & . \\
   . & . &. &. & . &\\
   \hline 
   . &. &. &. & 1 &. 
  \end{tabular}
 $\phantom{meowmeow}
   $ \begin{tabular}{cccccc}
   & & & 9 & 7 & .\\
   & & & . & . &  7 \\
   \hline
   & & . & . &. & 2 \\
   & . & . & . & . & \\
   . & . & . & . & & \\
   \hline 
   . & . & 8 & 4 & 3 & . 
  \end{tabular}
 $
   $ \begin{tabular}{ccccc}
   & & . & 9 & 6 \\ 
   & & 2 & . & 8 \\
   \hline 
   & 3 & 1 & . & . \\
   . & . & . & & \\ 
   \hline 
   . & . & . & . &.
  \end{tabular}
 $

 \end{enumerate}
 