
\chapter{Cas d'égalité des triangles rectangles}
\begin{enumerate}
\item \begin{enumerate}
\item Dans un triangle $ABC$ le pied de la hauteur $[AH]$ se trouve entre $B$ et $C$.
Montrer que les angles $\widehat{B}$ et $\widehat{C}$ sont aigus. 
\item Lorsque $H$ est sur le prolongement $[Bx)$ de $[CB]$, montrer que l'angle $\widehat{B}$ est obtus et les angles $\widehat{C}$ et $\widehat{A}$ aigus. Combien
y a-t-il au moins d'angles aigus dans un triangle ? 
\end{enumerate}
\item Deux triangles $ABC$ et $A'B'C'$ sont égaux. Comparer les hauteurs homologues 
$AH$ et $A'H'$.
\item Démontrer que dans un triangle isocèle $ABC$ de sommet $A$ les hauteurs $[BH]$ et $[CK]$ sont égales. Énoncer et démontrer la réciproque. (Utiliser les triangles 
$ABH$ et $ACK$ ou $BCH$ et $CBK$.)
\item Montrer que dans un triangle isocèle les hauteurs issues des sommets de la base se coupent sur l'axe de symétrie du triangle. 
\item Démontrer que les sommets $B$ et $C$ du triangle $ABC$ sont équidistants de la médiane $[AM]$. La propriété subsiste-t-elle pour toute droite $(xy)$ passant par $M$ ?
\item Dans un cercle donné ou dans deux cercles égaux deux cordes égales sont équidistantes du centre. Énoncer et démontrer la réciproque. 
\item On mène d'un point $O$ la perpendiculaire $(OH)$ et les obliques $(OA)$ et $(OB)$ à une droite $(xy)$. Démontrer que l'égalité $OA=OB$ entraîne $HA=HB$ et réciproquement. 
\item Les deux segments concourants $[AB]$ et $[CD]$ ont même milieu $O$. 
\begin{enumerate}
\item Comparer les triangles $OAC$ et $OBD$ ainsi que les hauteurs $OH$ et $OK$ de ces deux triangles. 
\item Montrer que le point $O$ est le milieu de $[HK]$.
\end{enumerate}
\item Dans un quadrilatère $ABCD$ on a $AD=BC$ et $\widehat{DAB}= \widehat{ABC}<90^o$. On mène les perpendiculaires $(DH)$ et $(CK)$ à $(AB)$. \begin{enumerate}
\item Comparer les triangles $ADH$ et $BCK$ puis $DH$ et $CK$. 
\item Montrer que $AB$ et $HK$ ont le même milieu $I$ et que $HC=KD$.
\end{enumerate}
\item Construire un triangle isocèle $ABC$ sachant que $AB=AC=5$ cm et que la hauteur $AH=4$ cm. Mesurer $[BC]$.
\item On connaît en position l'hypoténuse $BC = 40$ mm d'un triangle rectangle $ABC$ et la longueur $AB=24$ mm. \begin{enumerate}
\item Construire le point $D$ du prolongement de $[BA]$ tel que $AD=AB$ puis le point 
$A$. Mesurer $AC$. 
\item Généraliser.
\end{enumerate}
\item Construire un triangle $ABC$ tel que $BC=60$ mm, la hauteur $AH=32$ mm et la médiane $AM=35$ mm. 
\item Dans le triangle $ABC$ la hauteur $AH$ mesure $38$ mm et la bissectrice $AD$ $42$ mm. Effectuer la construction du triangle $AHD$ puis celle du triangle $ABC$ sachant que :\begin{enumerate}
\item $\widehat{BAC}=64^o$ ; 
\item $AB=45$ mm.
\end{enumerate}
\item On considère un triangle $ABC$ rectangle en $A$ et on construit le segment $[BD]$ perpendiculaire à $(BC)$ et égal à $BC$, puis la perpendiculaire $(Bx)$ à
$(AB)$. On mène de $D$ la perpendiculaire $(DE)$ à $(Bx)$.\begin{enumerate}
\item Comparer les angles $\widehat{ABC}$ et $\widehat{EBD}$ puis les triangles
$ABC$ et $EBD$.
\item Démontrer que $AC=DE$ et $AB=BE$.
\end{enumerate}
\item Deux points $A$ et $B$ situés de part et d'autre de $(xy)$ sont tels que leurs distances $AH$ et $BK$ soient égales. Démontrer que $(xy)$ passe par le milieu $O$ 
de $[AB]$.
\item Comparer deux triangles isocèles ayant : \begin{enumerate}
\item Des bases égales et des angles au sommet égaux.
\item Les hauteurs relatives à la base égales et des angles à la base égaux.
\end{enumerate}
\item Deux triangles $ABC$ et $A'B'C'$ ont $AB=A'B'$, $\widehat{B}=\widehat{B'}$ et $\widehat{C}=\widehat{C'}$. On mène les hauteurs $[AH]$ et $[A'H']$. \begin{enumerate}
\item Comparer les triangles $ABH$ et $A'B'H'$, puis les triangles $AHC$ et $A'H'C'$.
\item En déduire l'égalité des triangles $ABC$ et $A'B'C'$ et énoncer le cas 
d'égalité correspondant.
\end{enumerate}
\item La médiatrice du côté $[AB]$ du triangle isocèle $ABC$ coupe la base $[BC]$ en $D$. Le cercle de centre $B$ passant par $D$ recoupe $[AD]$ en $E$. \begin{enumerate}
\item Nature des triangles $DAB$ et $BDE$. Comparer les angles $\widehat{ACD}$ et 
$\widehat{BAE}$. 
\item Démontrer, en utilisant le résultat de l'exercice précédent, l'égalité
des triangles $ACD$ et $BAE$ puis que $CD=AE$.
\end{enumerate}
\item \begin{enumerate}
\item Deux triangles $ABC$ et $A'B'C'$ ont $AB=A'B'$, $\widehat{B}+\widehat{B'}=180^o$ et $\widehat{C}= \widehat{C'}$. Montrer que les hauteurs $AH$ et $A'H'$ sont égales, puis que les côtés $[AC]$ et $[A'C']$ sont égaux.
\item Montrer que réciproquement si $AB=A'B'$, $AC=A'C'$ et $\widehat{B}+\widehat{B'}=180^o$, les angles $\widehat{C}$ et $\widehat{C'}$ sont égaux. \\ (On pourra amener $[A'B']$ sur $[AB]$ et $C'$ sur le prolongement de $[CB]$.)
\item Dans un triangle isocèle $OAB$, on prend un point $C$ du côté $[OA]$ et on
prolonge $[OB]$ d'une longueur $BD=AC$. Le segment $[CD]$ coupe $[AB]$ en $M$. 
\begin{enumerate}
\item Comparer les hauteurs $CH$ et $DK$ des triangles $MAC$ et $MBD$. 
\item Montrer que $M$ est le milieu de $[HK]$ et de $[CD]$.
\end{enumerate}
\end{enumerate}
\end{enumerate}