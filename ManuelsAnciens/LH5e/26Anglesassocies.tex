
\chapter{Angles associés} 
\begin{enumerate}
\item Construire et calculer le supplément des angles suivants : 
\begin{enumerate}
\item $30^o$; $45^o$; $60^o$. 
\item $67^o35'$; $113^o43'23"$. 
\item $25$ gr; $50$ gr; $75$ gr. 
\item $56,327$ gr; $141,943$ gr. 
\end{enumerate}
\item Construire et calculer le complément des angles suivants : 
\begin{enumerate}
\item $24^o$; $36^o$; $63^o$. 
\item $23^o37'$; $67^o26'30"$.
\item $30$ gr; $48$ gr; $65$ gr. 
\item $31,09$ gr; $58,542$ gr. 
\end{enumerate}
\item Construire à l'aide du rapporteur, deux angles adjacents $\widehat{AOB}$ et $\widehat{AOC}$ de $54^o$ et $96^o$. Calculer l'angle $\widehat{MON}$ de leurs bissectrices. Le comparer à l'angle $\widehat{BOC}$.
\item Deux angles adjacents ont pour somme $102^o$. Quel est l'angle de leurs bissectrices ? Calculer et construire ces deux angles sachant que leur différence vaut $36^o$. 
\item Construire deux angles adjacents supplémentaires dont la différence vaut $54^o$. 
\item Deux angles adjacents $\widehat{AOB}$ et $\widehat{BOC}$ sont complémentaires et l'un est les $\frac23$ de l'autre. Calculer et construire ces deux angles. Construire les bissectrices $[OM)$ et $[ON)$. Calculer leur angle $\widehat{MON}$. 
\item Les bissectrices de deux angles adjacents $\widehat{AOB}$ et $\widehat{BOC}$ font un angle droit $\widehat{MON}$. Démontrer que ces deux angles sont supplémentaires. 
\item Les bissectrices $[OM)$ et $[ON)$ de deux angles égaux sont dont le prolongement l'une de l'autre. Démontrer que ces deux angles sont opposés par le sommet. 
\item Deux angles non adjacents $\widehat{AOB}$ et $\widehat{AOC}$ ont pour différence un angle droit. 
\begin{enumerate}
\item Calculer l'angle de leurs bissectrices. 
\item Construire ces deux angles sachant que $\widehat{AOB}= \frac38\widehat{AOC}$.
\end{enumerate}
\item Les bissectrices $[OM)$ et $[ON)$ de deux angles non adjacents $\widehat{AOB}$ et $\widehat{AOC}$ font un angle de $36^o$. \begin{enumerate}
\item Calculer l'angle $\widehat{BOC}$, lorsque $\widehat{AOB}= 84^o$. Comparer la valeur trouvée à celle de l'angle $\widehat{MON}$. 
\item Généraliser pour des valeurs quelconques des angles donnés. 
\end{enumerate}
\item Quatre angles consécutifs de même sommet $\widehat{AOB}$, $\widehat{BOC}$, 
$\widehat{COD}$ et $\widehat{DOA}$ sont tels que : \[ \widehat{AOB}= \widehat{COD}
\phantom{meow}\text{et}\phantom{meow}\widehat{BOC}=\widehat{DOA}.\]
Démontrer que leurs côtés sont deux à deux dans le prolongement de l'autre. 
\item Démontrer que lorsque deux angles $\widehat{AOB}$ et $\widehat{AOC}$ ont même sommet et un côté commun, l'angle $\widehat{MON}$ de leurs bissectrices et la moitié de l'angle de leurs côtés non communs, que ces angles soient adjacents ou non. 
\item On construit un angle $\widehat{AOB}$ de $48^o$ puis les angles droits $\widehat{AOA'}$ et $\widehat{BOB'}$ non adjacents au premier. 
\begin{enumerate}
\item Comparer les angles $\widehat{AOB'}$ et $\widehat{A'OB}$. Montrer que les angles $\widehat{AOB}$ et $\widehat{A'OB'}$ sont supplémentaires. 
\item Montrer que les angles $\widehat{AOB}$ et $\widehat{A'OB'}$ ont même bissectrice $[OP)$. Calculer l'angle $\widehat{MON}$ des bissectrices des angles $\widehat{AOB'}$ et $\widehat{A'OB}$. 
\end{enumerate}
\item Construire l'angle $\widehat{AOB}=108^o$ puis les angles droits $\widehat{AOA'}$ et $\widehat{BOB'}$ adjacents à celui-ci. 
\begin{enumerate}
\item Montrer que les angles $\widehat{AOB}$ et $\widehat{A'OB'}$ sont supplémentaires. Comparer les angles $\widehat{AOB'}$ et $\widehat{A'OB}$. 
\item Démontrer que les bissectrices $[OP)$ et $[OQ)$ des angles $\widehat{AOB}$ et $\widehat{A'OB'}$ sont en ligne droite et calculer l'angle $\widehat{MON}$ des bissectrices des angles $\widehat{AOB'}$ et $\widehat{A'OB}$. 
\end{enumerate}
\item Calculer la valeur et construire deux angles $\widehat{AOB}$ et $\widehat{BOC}$ sachant que leurs bissectrices font un angle de $40^o$ et que le premier est égal aux $\frac35$ de l'autre. \begin{enumerate}
\item Dans le cas où les angles $\widehat{AOB}$ et $\widehat{BOC}$ sont adjacents.
\item Dans le cas où ces angles ne sont pas adjacents. 
\end{enumerate}
\end{enumerate}