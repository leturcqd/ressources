
 \chapter{Résolution littérale de problèmes}
 \begin{enumerate}
 \item Trouver un nombre dont le produit par 11 surpasse de 108 le produit de 
 ce même nombre par 8. 
 \item Le produit d'un nombre par 9 diminué de 23 surpasse de 19 le produit de ce même nombre par 6. Quel est ce nombre ? 
 \item Un père a 30 ans, son fils a 4 ans. Dans combien d'années l'âge du père sera-t-il le triple de l'âge de son fils ?
 \item Un père âgé de 37 ans a trois enfants âgés respectivement de 3, 7 et 11 ans. Dans combien d'années l'âge du père sera-t-il égal à la somme des âges de ses enfants ? 
 \item Deux trains séparés par une distance de 310 km marchent à la rencontre l'un de l'autre. Le premier fait 90 km à l'heure et le deuxième 65 km à l'heure. Dans combien de temps se fera la rencontre ? 
 \item Un cycliste qui roule à la vitesse de 30 km a 60 km d'avance sur un motocycliste qui le suit à la vitesse de 50 km à l'heure. On demande combien de temps il faudra à ce dernier pour rejoindre le cycliste. 
 \item Un épicier achète de l'huile 2,10 F le litre. Il la revend 2,70 F le litre et fait ainsi un bénéfice de 138 F. Quelle quantité d'huile avait-il achetée ? 
 \item Deux pièces de la même étoffe valent l'une 2~340 F, l'autre 5~040 F. Sachant que la première mesure 3 mètres de moins que la seconde, trouver le
 prix du mètre de cette étoffe.
 \item Un marchand a acheté 5 pièces de vin pour 575 F. En revendant ce 
 vin 805 F, il fait un bénéfice de 0,20 F par litre. Trouver la contenance de chaque pièce. 
 \item Un libraire achète des casiers 8 F la douzaine. Il les vend 1 F pièce et gagne ainsi 480 F. Combien de douzaines de cahiers a-t-il vendues ? 
 \item Un chemisier achète des chemises 108 F la douzaine, mais il en 
 reçoit une treizième par douzaine achetée. Il les revend 12 F la pièce et
 fait ainsi un bénéfice de 192 F. Combien de douzaines de chemises a-t-il acheté ?
 \item Une pièce d'étoffe vaut 675 F. On la diminue de 7 mètres. Elle ne vaut plus alors que 486 F. Quelle est la valeur du mètre de cette étoffe ?
 \item La somme de deux nombres impairs consécutifs est égale à 124. Quels sont ces deux nombres ? 
 \item Trouver deux nombres sachant que le premier surpasse de 8 le deuxième et que, si on l'augmente de 14, il est égal au triple du second. 
 \item La somme de trois nombres est égale à 388. Trouver ces trois nombres
 sachant que le premier surpasse de 11 le second et est inférieur de 15 
 au troisième. 
 \end{enumerate}