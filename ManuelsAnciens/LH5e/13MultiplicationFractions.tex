\chapter{Multiplication des fractions}
 
 \begin{enumerate}
 \item Découper une bande de papier de 20 cm de long. 
 En prendre les $\frac35$. Construire les $\frac78$ de la bande obtenue. Vérifier que la longueur finale est les $\frac{21}{40}$ de la bande primitive. 
 \item Effectuer les produits suivants et simplifier les résultats : 
 \[ \frac47\times \frac59\times \frac3{10}; 
 \phantom{meowmeow} \frac74\times5\times\frac8{21};
  \phantom{meowmeow}  \frac34\times \frac45\times \frac56.\] 
   \[ \frac{45}{64}\times\frac{24}{27}; 
 \phantom{meowmeow} \frac{35}{36}\times\frac{48}{49};
  \phantom{meowmeow}  \frac{21}{36}\times\frac{24}{35} \times \frac{45}{42}.\] 
   \[ \frac{27}{34}\times \frac{51}{84}; 
 \phantom{meowmeow} \frac{76}{35}\times\frac{48}{49};
  \phantom{meowmeow}  \frac{38}{85}\times \frac{65}{133} \times \frac{119}{143}.\] 
  \[ \left(\frac{19}{12} + \frac5{21} + \frac{13}{28}
   \right)\times\frac{21}{32}; \phantom{meowmeowmeowmeow}
   \left(\frac{43}{28}-\frac{13}{21}\right)\times 
   \frac{16}{33} \]
     \[ \left(\frac{51}{56} + \frac8{21} + \frac{16}{48}
   \right)\times\frac{32}{65}; \phantom{meowmeowmeowmeow}
   \left(\frac{31}{20}-\frac{26}{45}\right)\times 
   \frac{36}{49} \]
   \[ \left( \frac45\right)^2 \times \left(\frac45\right)^3; \phantom{meowmeow} 
   \left( \frac23\right)^5 \times \left(\frac45\right)^3\times \left(\frac23\right);
   \phantom{meowmeow} \left[\left(\frac52\right)^3\right]^2\]
   \[ \left(\frac27\right)^2 \times \left(\frac27\right)^4\left(\frac56-\frac13\right) ; 
    \phantom{meowmeowmeowmeow}
    \left(\frac12\right)^4 \times \left(\frac12\right)^3 \left(\frac23+\frac34\right).\]
 \item Effectuer mentalement : 
 \[ 62 \times 5;  \phantom{meowmeowmeowmeow}
 126 \times 50;  \phantom{meowmeowmeowmeow}
 42 \times 15;\]
  \[ 38 \times 5;  \phantom{meowmeowmeowmeow}
 257 \times 50;  \phantom{meowmeowmeowmeow}
 72 \times 150;\] \[ 47 \times 5;  \phantom{meowmeowmeowmeow}
 184 \times 500;  \phantom{meowmeowmeowmeow}
 232 \times 15;\] 
 \[ 121 \times 5;  \phantom{meowmeowmeowmeow}
 365 \times 50;  \phantom{meowmeowmeowmeow}
 125 \times 150.\]
 \item Un champ est partagé en trois parties. La première est égale aux $\frac27$ de la surface totale,
 la seconde aux $\frac5{13}$ de la première.
 La différence entre les deux premières parties est 1~200 mètres carrés. Calculer la surface totale. 
 \item Un travail est exécuté par 3 ouvriers, le premier en a fait les $\frac25$, le second les $\frac56$ du travail effectué par le premier et le troisième le reste. Ce dernier a touché 2~100 F de moins que les autres ensemble. Calculer le prix total de ce travail. 
 \item Les $\frac34$ des $\frac45$ d'un nombre valent 108. Quel est ce nombre ? 
 \item En ajoutant à un nombre donné les $\frac45$ des 
 $\frac23$ de ce nombre, on trouve $322$. Quel est ce 
 nombre ? 
 \item On considère une suite de 4 nombres tels que chacun d'eux soit égal à la moitié du précédent. Trouver ces nombres, sachant que leur somme est 105. 
 \item Une personne perd les $\frac35$ de sa fortune ; elle regagne ensuite les $\frac47$ de ce qu'elle avait perdu et possède alors 39~000 F. Que possédait-elle primitivement ? 
 \item Une personne doit une certaine somme. Elle verse d'abord le quart de sa dette, puis les $\frac37$ de ce qui reste à payer, et se libère enfin en versant 63~000 F. Quelle était la dette de cette personne ? 
 
 \item Deux ouvriers ont travaillé le premier 18 jours, le second 20 jours ; le salaire journalier du premier est les $\frac45$ de celui du deuxième. 
 Sachant que ces deux ouvriers ont reçu ensemble 1~032 F, quel est le salaire journalier de chacun ? 

  \item Le bénéfice d'un commerçant est le $\frac{32}{100}$ du prix de vente. Ce bénéfice augmenterait de 3~600 F s'il était la moitié du prix d'achat. Déterminer le prix d'achat. 
  
  \item On retire d'une cuve les $\frac23$ de sa contenance moins 40 litres. On retire ensuite les $\frac25$ du reste ; il reste encore 84 litres. Quelle 
  est la capacité de la cuve ? 
  
  \item Par quelle fraction faut-il multiplier un nombre pour l'augmenter de ses $\frac25$, de ses $\frac34$, de ses $\frac23$ ? 
  \item Les $\frac34$ des $\frac56$ d'un nombre donné surpassent de 38 unités les $\frac23$ des $\frac7{10}$ de ce nombre. 
  \item On retranche d'un nombre ses $\frac25$,
  puis les $\frac57$ du reste. Quelle fraction du nombre reste-t-il ? 
  \item Les $\frac25$ des candidats à un concours ont été admissibles aux épreuves orales et un candidat sur huit a échoué à ses épreuves. Quel est le pourcentage 
  des candidats définitivement reçus au concours ? 
  Application : le nombre des candidats s'élève à 520. Trouver le nombre des admis.
  \item Partager une somme de 84~000 F entre cinq personnes de façon que la première reçoive les $\frac34$ de la part de la deuxième, 
  que la deuxième reçoive la moitié de la part de la troisième, 
  que celle-ci reçoive les $\frac23$ de la part de la quatrième et enfin que celle-ci reçoive les $\frac45$
  de la part de la cinquième. 
  \item Un joueur perd les $\frac27$ de la somme qu'il 
  possède, puis regagne les $\frac58$ de ce qu'il a perdu et se retire du jeu avec 125 F. Quel était son avoir primitif ? 
  \item Soient deux nombres dont l'un est $\frac34$ de 
  l'autre. On multiplie le premier par $\frac23$ et le second par $\frac78$ et en faisant le produit des deux résultats, on obtient 28. Quels sont ces deux nombres ? 
  \item Une personne dépense les $\frac23$ de son argent, puis elle gagne une somme égale aux $\frac37$
  de ce qui lui restait. Avec la somme qu'elle possède alors elle peut payer les $\frac25$ d'une pièce de drap de 31, 25 m valant 12 F le mètre. Combien avait-elle tout d'abord ? 
  \item Un ouvrier fait le quart d'un ouvrage en cinq jours ; un deuxième ouvrier fait les $\frac25$ du reste en douze jours. Combien les deux ouvriers travaillant ensemble mettront-ils de jour pour achever l'ouvrage ? Le premier ouvrier a touché 132 F pour son travail ; quelle somme touchera le second ouvrier sachant que pour chacun d'eux, le prix de la journée est le même ? 
 \end{enumerate}
