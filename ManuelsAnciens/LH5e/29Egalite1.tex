
\chapter{Les deux premiers cas d'égalité des triangles} 
\begin{enumerate}
\item Démontrer que dans deux triangles égaux $ABC$ et $A'B'C'$, les médianes $AM$ et $A'M'$ sont égales. (Comparer $ABM$ et $A'B'M'$.). Vérification graphique. 
\item Même problème pour les bissectrices intérieures $AD$ et $A'D'$.
\item On porte sur les cotés d'un angle $\widehat{xOy}$ respectivement deux longueurs égales $OA$ et $OB$ et on joint $A$ et $B$ à un point quelconque $M$ de la bissectrice de l'angle $\widehat{xOy}$.\begin{enumerate}
\item Comparer les triangles $AOM$ et $BOM$. Conséquences ? 
\item Démontrer que $AM=BM$ et que $[OM)$ est bissectrice de l'angle $\widehat{AMB}$.
\end{enumerate}
\item On considère un quadrilatère convexe $ABCD$ dans lequel la droite $(AC)$ est bissectrice intérieure des angles $\widehat{A}$ et $\widehat{C}$. 
\begin{enumerate}
\item Comparer les triangles $ABC$ et $ADC$. Conséquences ? 
\item On plie la figure suivant la droite $(AC)$. Que se passe-t-il ? En déduire que $(AC)$ est médiatrice du segment $[BD]$. 
\end{enumerate}
\item Soient $[AB]$ et $[CD]$ deux diamètres pris dans deux cercles de même centre $O$.\begin{enumerate}
\item Comparer les triangles $AOC$ et $BOD$ puis les triangles $AOD$ et $BOC$. 
\item Comparer les côtés opposés et les angles opposés du quadrilatère $ACDB$.
\end{enumerate}
\item On considère dans un cercle deux angles au centre égaux $\widehat{AOB}$ et
$\widehat{DOC}$ tous deux adjacents à l'angle $\widehat{BOD}$. 
\begin{enumerate}
\item Comparer les triangles $AOB$ et $COD$. Conséquences ? 
\item Comparer les triangles $AOD$ et $COB$. Conséquences ? Montrer que les angles $\widehat{BAD}$ et $\widehat{BCD}$ sont égaux ainsi que les angles $\widehat{ABC}$ et $\widehat{ADC}$. 
\end{enumerate}
\item Deux points $A$ et $B$ situés d'un même côté de la droite $(xy)$ se projettent
en $H$ et $K$ sur cette droite et on a $AH=BK$. \begin{enumerate}
\item Comparer les triangles $AHK$ et $BHK$. Conséquences ? 
\item Comparer les triangles $ABK$ et $BAH$ puis les angles $\widehat{KAB}$ et $\widehat{HBA}$. 
\end{enumerate}
\item Une droite $(xy)$ passe entre les points $A$ et $B$ et les distances $AH$ et $BK$ à cette droite sont égales. Soit $O$ le milieu de $[HK]$. \begin{enumerate}
\item Comparer les triangles $OAH$ et $OBK$. Conséquences ? 
\item Démontrer que les points $A$, $O$ et $B$ sont alignés et que $O$ est le milieu de $[AB]$. 
\end{enumerate}
\item Dans un quadrilatère convexe la diagonale $[AC]$ fait des angles égaux avec $[AD]$ et $[BC]$ d'une part et avec $[AB]$ et $[CD]$ d'autre part. Soit $O$ le milieu de $[AC]$. \begin{enumerate}
\item Comparer les triangles $ABC$ et $CDA$. En déduire une propriété des angles opposés et des côtés opposés du quadrilatère. 
\item Comparer les triangles $OAB$ et $OCD$ et montrer que $O$ est aussi milieu de $[BD]$. 
\end{enumerate}
\item Deux segments inégaux $[AB]$ et $[CD]$ ont même milieu $O$. \begin{enumerate}
\item Comparer les triangles $OAC$ et $OBD$ puis les triangles $AOD$ et $BOC$. Conséquences pour les côtés et les angles du quadrilatère $ACBD$. 
\item Une droite $(Ox)$ est perpendiculaire en $H$ à $(AC)$. Montrer qu'elle est également perpendiculaire en $K$ à $(BD)$ et que $O$ est le milieu de $[HK]$. 
\end{enumerate}
\item On considère un quadrilatère convexe $ABCD$ tel que $AB=CD$ et $\widehat{BAC}=\widehat{CAD}$. Soit $O$ le milieu de la diagonale $[AC]$.\begin{enumerate}
\item Comparer les triangles $OAB$ et $OCD$. Conséquences ? Que représente le point 
$O$ pour le segment $[BD]$ ? 
\item Une droite $(xOy)$ coupe $(AB)$ en $M$ et $(DC)$ en $N$. Comparer les triangles
$OAM$ et $OCN$. En déduire que $AM=CN$ et que $O$ est le milieu de $[MN]$. Comparer
les angles $\widehat{OMA}=\widehat{ONC}$. 
\end{enumerate}
\item On prolonge la médiane $[AM]$ du triangle $ABC$ d'une longueur $MD$ égale à $AM$. \begin{enumerate}
\item Comparer les triangles $MAC$ et $MDB$. Conséquences pour $AC$ et $BD$ ainsi que pour les angles $\widehat{MAC}$ et $\widehat{MDB}$ ?
\item Construire la figure sachant que $AM = 3\cm$, $AB=5\cm$ et $AC=4\cm$. 
\end{enumerate}
\item Utiliser les résultats de l'exercice précédent pour construire un triangle 
$ABC$ connaissant la médiane $AM = 36$ mm, $\widehat{MAB}= 72^o$, et $\widehat{MAC}= 54^o$. 
\item On considère deux angles égaux $\widehat{BAx}$ et $\widehat{ABy}$ situés
de part et d'autre de la droite $(AB)$. Une sécante issue du milieu $O$ de $[AB]$ coupe $[Ax)$ en $M$ et $[By)$ en $N$. \begin{enumerate}
\item Comparer les triangles $AOM$ et $BON$. Conséquences ? 
\item Comparer les triangles $AON$ et $BOM$ puis les segments $[AN]$ et $[BM]$.
Montrer que le milieu de $[AN]$ et le milieu de $[BM]$ sont sur une droite issue de 
$O$. 
\end{enumerate}
\item \begin{enumerate}
\item Construire un triangle $ABC$ tel que $\widehat{A}=42^o$, $AB=48$ mm, et $AC=36$ mm, puis, extérieurement au triangle, les triangles isocèles en $A$ $ACD$ et $ABE$ tels que $\widehat{CAD}= \widehat{BAE}= 58^o$.
\item Comparer les triangles $ABD$ et $AEC$ puis les segments $[BD]$ et $[EC]$. Ces
segments se coupent en $I$ : mesurer l'angle $\widehat{CID}$. 
\end{enumerate}
\item \begin{enumerate}
\item Construire un triangle $ABC$ tel que $BC=34$ mm, $\widehat{B}=82^o$ et $\widehat{C}= 68^o$, puis extérieurement au triangle $ABC$ construire les triangles 
$ABD$ et $ACE$ tels que $\widehat{ABD}= \widehat{ACE}=90^o$, $BD=BA$ et $CE=CA$. ainsi que le triangle $ICB$ égal au triangle $ABC$ ($IC=AB$, $IB=AC$). 
\item Comparer les triangles $BDI$ et $CIE$ puis les segments $[ID]$ et $[IE]$. 
Mesurer l'angle $\widehat{DIE}$. 
\end{enumerate}
\item On donne un triangle $ABC$ et on prolonge le côté $[AB]$ d'une longueur $BD=AB$, le côté $[AC]$ d'une longueur $CF=AC$ et la médiane $[AM]$ d'une longueur $MI=AM$.
\begin{enumerate}
\item Comparer les triangles $MAB$ et $MIC$ puis $MAC$ et $MIB$. Montrer que le quadrilatère $BACI$ a ses côtés opposés égaux et ses angles opposés égaux. 
\item Comparer les triangles $BDI$ et $CIF$ puis les segments $[ID]$ et $[IF]$. 
Mesurer l'angle $\widehat{DIF}$. Que déduit-on pour les $3$ points $I$, $D$ et $F$ ?
\end{enumerate}
\item On considère un quadrilatère concave $ABCD$ dans lequel la droite $(AC)$ est bissectrice de l'angle saillant $\widehat{BAD}$ et aussi de l'angle rentrant $\widehat{BCD}$. 
\begin{enumerate}
\item Comparer les triangles $ABC$ et $ADC$. Conséquences ? 
\item Le segment $[BD]$ coupe en $H$ le prolongement de $[AC]$. Comparer les triangles $HBC$ et $HDC$. Conséquences ? 
\item Montrer que la droite $(AC)$ est la médiatrice de $[BD]$. 
\end{enumerate}
\end{enumerate}