
 \chapter{Quotient de deux nombres à une approximation décimale donnée}
 
 \begin{enumerate}
 \item Calculer le quotient entier, le quotient à 0,1 près, le quotient à 0,01 près, et le quotient à 0,001 près de : 
 \begin{itemize}
 \item 39 par 7 ;
 \item 293,72 par 43 ; 
 \item 735,7 par 40,1. 
 \end{itemize}
 \item Le diviseur d'une division est 7,5 et le quotient à 0,001 près est 2,357. Que peut-être le dividende ? 
 \item Calculer à 1 décimètre près la largeur d'un rectangle dont la surface est 212 ares 7 centiares, sachant que sa longueur mesure 183, 7 mètres. 
 \item Calculer à 1 millimètre près les deux bases d'un trapèze dont la surface est 430 centimètres carrés, dont la hauteur mesure 147 millimètres, et sachant d'autre part que la différence des bases est 18 cm. 
 \item Couper par une corde de 27,55 m en 3 morceaux tels que le premier ait 2,50 m de plus que le second, et 1,25 de moins que le troisième.
 \item Un litre d'huile pèse 0,025 kg. Combien de bouteilles ayant une capacité de 0,83 litres pourra-t-on remplir avec 52 kg d'huile ? 
 \item Calculer le rayon du méridien terrestre sachant que sa longueur est de 20~004 kilomètres. De combien augmenterait ce rayon si la longueur du méridien terrestre augmentait de $1$ kilomètre ? 
 \item Calculer le quotient à un millième, près de : 
 \begin{itemize}
 \item $\frac7{11}$ par $\frac35$ ; 
 \item $12$ par $\frac57$; 
 \item $4,5$ par $\frac79$. 
 \end{itemize}
\item Calculer à 0,0001 près les quotients de 22 par 7 et de 355 par 113. 
\item Diviser 2 par 1,414, puis 3 par 1,732. 
\item Calculer à 0,00001 près le quotient de 1 par 3,1416. Le nombre 3,1416
étant une valeur approchée par excès de $\pi$, obtient-on ainsi une valeur approchée de $\frac1{\pi}$ par défaut ou par excès ?  
\item Calculer à un millionième près les quotients de 1 par 6, et de 1 par 3. Faire la somme des nombres obtenus, et la comparer à la somme des deux fractions $\frac16$ et $\frac13$. 
\item Calculer à un millionième près les quotients de 2 par 3 et de 1 par 6. Faire la différence des nombres obtenus, et la comparer à la différence des deux fractions $\frac23$ et $\frac16$. 
\item Dans une division, on a interverti les rôles du dividende et du diviseur et l'on a trouvé 0,857 comme quotient approché à 0,001 près par défaut. Calculer une valeur approchée par excès et une valeur approchée par 
défaut du quotient initial. Donner le quotient initial avec la meilleure
approximation possible. 
\item Trouver tous les nombres entiers qui, divisés par 658 donnent 0,87 comme quotient approché à $\frac1{100}$ près par défaut. 
\item Calculer à $\frac1{1~000}$ près les densités de l'oxygène et de l'azote
sachant qu'un litre d'air pesant $1,293$ gramme contient $0,209$ litre d'oxygène et $0,791$ litre d'azote et que dans un gramme d'air il y a 231 milligrammes d'oxygène et 769 milligrammes d'azote. 
\item Dans une installation de lumière électrique on a utilisé 678 mètres de 
fil de cuivre recouvert de caoutchouc. 
Quand on observe un morceau bien tendu de ce fil, on voit la surface extérieure d'un cylindre de caoutchouc dont le rayon est de 2 mm. Le fil de cuivre a un diamètre de 12 dixièmes de mm. On demande : 
\begin{itemize}
\item La masse exprimée en kg du cuivre contenu dans le fil employé. 
\item Le volume exprimé en cm${}^3$ du caoutchouc qui recouvre ce cuivre. 
\end{itemize}
\item Avec 51 kg de cuivre, on fabrique du fil électrique de deux diamètres différents : 12 et 16 dixièmes de mm. La longueur du fil fin est double de celle du fil qui est le plus gros. Quelles longueurs de chaque fil pourra-t-on obtenir ? La densité du cuivre est 8,8. On prendra $\frac{355}{113}$ comme valeur approchée de $\pi$. 
 \end{enumerate}