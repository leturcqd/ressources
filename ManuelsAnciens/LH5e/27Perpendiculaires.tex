
\chapter{Droites perpendiculaires} 
\begin{enumerate}
\item Vérification de l'angle droit d'une équerre.
\item Construire le complément d'un angle aigu : \begin{enumerate}
\item Par pliage ; 
\item à l'aide d'un rapporteur ; 
\item à l'aide de l'équerre.
\end{enumerate}
\item Déterminer la projection $H$ et la distance d'un point extérieur $O$ à une droite $(xy)$. \begin{enumerate}
\item Par pliage; \item à l'aide de l'équerre.
\end{enumerate}
\item Construire la médiatrice d'un segment donné $[AB]$. \begin{enumerate}
\item Par pliage; \item à l'aide d'un double décimètre et du rapporteur ou de l'équerre.
\end{enumerate}
\item Soient $I$ et $J$ deux points d'une droite $(xy)$ et un point extérieur $O$.
On trace les cercles de centres $I$ et $J$ passant par $O$. \begin{enumerate}
\item Montrer que les deux cercles passent également par le point $O'$ qui coïncide avec $O$ lorsqu'on plie la figure suivant $(xy)$.
\item En déduire une construction de la perpendiculaire menée de $O$ à $ (xy)$ et de la projection $H$ du point $O$ sur $(xy)$. 
\end{enumerate}
\item On prend trois points $A$, $B$, $C$ sur un même cercle. Mener d'un point $M$ de
ce cercle les perpendiculaires $(MP)$, $(MQ)$ et $(MR)$ aux trois droites $(BC)$, 
$(CA)$ et $(AB)$. Si la construction est précise, les trois points $P$, $Q$, $R$ sont alignés. Vérifiez-le. 
\item Soient trois points $A$, $B$, $C$ non alignés. Mener de chacun d'eux, à l'aide
de l'équerre, la perpendiculaire à la droite déterminée par les deux autres. Que constate-t-on ? On envisagera le cas où les trois angles sont aigus, et le cas où l'angle $\widehat{BAC}$ est obtus.
\item On considère un angle aigu $\widehat{AOB}$.\begin{enumerate}
\item Construire un angle aigu $\widehat{COD}$ tel que $[OC)$ soit perpendiculaire à $[OA)$ et $[OD)$ perpendiculaire à $[OB)$. 
\item Comparer les angles $\widehat{AOB}$ et $\widehat{COD}$ et énoncer le théorème
correspondant. 
\item Que peut-on dire des angles $\widehat{AOD}$ et $\widehat{BOC}$ ainsi que de 
leurs bissectrices ?
\end{enumerate}
\item Reprendre le problème précédent avec des angles $\widehat{AOB}$ et $\widehat{COD}$ obtus. 
\item On considère un angle obtus $\widehat{AOB}$ et on construit à l'intérieur de cet angle $[OC)$ perpendiculaire à $[OA)$ et $[OD)$ perpendiculaire à $[OB)$. \begin{enumerate}
\item Montrer que les angles $\widehat{AOB}$ et $\widehat{COD}$ sont supplémentaires.
\item Comment sont disposées les bissectrices des angles $\widehat{AOB}$ et $\widehat{COD}$ ? 
\item Calculer l'angle $\widehat{MON}$ des bissectrices des angles $\widehat{AOD}$ et $\widehat{BOC}$. 
\end{enumerate}
\item Soit une droite $(AB)$ et un point $O$ de cette droite. On construit d'un
même côté de cette droite deux angles $\widehat{AOC}$ et $\widehat{BOC}$ complémentaires ainsi que la perpendiculaire $[OE)$ à $(AB)$ en $O$. 
\begin{enumerate}
\item Démontrer que $(OC)$ et $(OD)$ sont perpendiculaires. 
\item Comparer les angles $\widehat{AOC}$ et $\widehat{EOD}$ puis les angles $\widehat{BOD}$ et $\widehat{EOC}$. 
\item Que peut-on dire des angles $\widehat{BOC}$ et $\widehat{DOE}$ ?
\end{enumerate}
\item Deux droites perpendiculaires $(xx')$ et $(yy')$ se coupent en $O$. Deux autres droites perpendiculaires $(uu')$ et $(vv')$ se coupent en $O$ de façon que $[Ou)$ soit dans l'angle $\widehat{xOy}$ et $[Ov)$ dans l'angle $\widehat{x'Oy}$. \begin{enumerate}
\item Trouver dans la figure les angles égaux à $\widehat{xOu}$. 
\item Déterminer les angles complémentaires, puis les angles supplémentaires de $\widehat{xOu}$. 
\end{enumerate}
\item Quatre angles consécutifs $\widehat{AOB}$, $\widehat{BOC}$, $\widehat{COD}$ et $\widehat{DOE}$ valent chacun $45^o$. 
\begin{enumerate}
\item Que peut-on dire des demi-droites $[OA)$ et $[OE)$, $[OB)$ et $[OD)$ ?
\item On construit les bissectrices des quatre angles initiaux. Montrer qu'elle sont deux à deux\footnote{Deux sont perpendiculaires à deux autres; non deux quelconques.} perpendiculaires. 
\end{enumerate}
\item On considère une feuille de papier $ABCD$ dont les angles $\widehat{A}$ et $\widehat{B}$ sont droits. Par le milieu $O$ de $[AB]$, on trace une demi-droite 
$[Ox)$.\begin{enumerate}
\item Plier la feuille de façon à amener les segments $[OA]$ et $[OB]$ sur $[Ox)$. 
Montrer que $A$ et $B$ viennent alors coïncider en $P$.
\item L'un des plis coupe $[AD]$ en $M$, l'autre coupe $[BC]$ en $N$. Montrer que $(Om)$ et $(ON)$ sont perpendiculaires.
\item Démontrer que les trois points $M$, $P$ et $N$ sont alignés sur la perpendiculaire en $P$ à $[Ox)$.
\end{enumerate}
\end{enumerate}