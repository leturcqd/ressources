
\chapter{Addition et soustraction des fractions}

\begin{enumerate}
\item Découper des bandes de papier AB de 225 mm de 
longueur. \begin{enumerate}
\item Construire les fractions suivantes de AB : 
\[ \frac35; \phantom{meow}\frac75; \phantom{meow}
\frac49; \phantom{meow}\frac79; \phantom{meow}\frac23.\]
\item Construire les fractions suivantes de $AB$ : 
\[\frac34+\frac23+\frac49; \phantom{meow}\frac79-\frac53.\]
\[\frac34+\left(\frac79-\frac23\right);  \phantom{meow} \frac75-\left(\frac13+\frac29\right);
 \phantom{meow}\frac75-\left(\frac13-\frac29\right).\]
 
 \item Vérifier que les fractions suivantes de AB sont égales : 
 \[\frac34+\frac23+\frac49 = \frac23+\frac34+\frac49 = \frac49 + \frac34 + \frac23.\]
 \end{enumerate}
 \item Calculer : 
 \[\frac13 + \frac23;  \phantom{meow}
 \frac34+\frac25; \phantom{meow}
 \frac23-\frac12;  \phantom{meow}
 \frac45-\frac23.\]
 \item Calculer : \[\left(4+\frac23\right) + \left(2+ \frac56\right);  \phantom{meow}
 \left(7 + \frac34\right) + \left(5 + \frac23\right)
 + \left(\frac13+\frac14\right).\]
 \item Calculer :\[ 17 - \left(\frac43 + \frac35\right) ;  \phantom{meow}
 \frac{15}4 + \left(\frac{17}9-\frac85\right); 
  \phantom{meow}
  \frac{13}{11} - \left(\frac79 - \frac23\right)\]

\item Calculer et écrire sous la forme $a+\frac{b}{c}$ avec $a$, $b$, $c$ entiers et $b<c$: 
\begin{enumerate}
\item $\frac12+\frac23+\frac34+\frac56$.
\item $\frac83+\frac{11}6+\frac{15}8$. 
\item $\frac{11}6+\frac3{14}+\frac8{21}$.
\item $\frac{16}{15}+\frac{25}{21}+\frac{6}{35}$. 
\item $\frac{22}{15}+\left(\frac{23}{42} + \frac{17}{35}\right).$ 
\item $\frac{50}{21} + \left(\frac{56}{33} + \frac{15}{77}\right)$.
\item $\frac{31}{15} + \left(\frac{26}{35} - \frac{8}{21}\right)$.
\item $\frac{91}{30} + \left(\frac{29}{42} - \frac{2}{35}\right)$.
\item $\frac{76}{15} - \left(\frac{13}{24} + \frac{21}{40}\right)$.
\item $\frac{125}{36} - \left(\frac{9}{20} - \frac{17}{45}\right)$.
\item $11 - \left(5\frac14 - 1 \frac58\right)$. 
\item $51\frac27 - \left(4\frac15 + 14\frac{5}8\right)$. 
\end{enumerate}
\item Extraire les entiers des fractions suivantes : 
\[ \frac{57}7\phantom{meow} \frac{235}{19}\phantom{meow}\frac{4~372}{53}\phantom{meow}\frac{7~979}{31}\phantom{meow}\frac{42~714}{3~333}.\]
\item Un marchand achète à la fois 5 bœufs, 7 vaches et 9 veaux. Un bœuf vaut 400 F de plus qu'une vache et 10 veaux valent autant que 3 vaches. Le marchand a payé en toute 10~820 F. Trouver le prix d'un bœuf, d'une vache, d'un veau. 
\item Trois personnes se partagent une certaine somme. 
La première reçoit les $\frac27$ plus 600 F, la deuxième les $\frac25$ plus 350 F, la 3e a 1~250 F. Quelle est la somme à partager ? Quelles sont les deux premiers parts ? 
\item Un joueur perd le $\frac13$, puis le $\frac14$ de son avoir. Il lui reste alors le $\frac15$ de son avoir primitif plus 260 F. Que possédait-il ? 
\item Trois personnes se partagent une certaine somme.
La première reçoit les $\frac25$, la deuxième les $\frac37$ et la troisième le reste. Quelles sont les 3 parts, sachant que la deuxième a 2~260F de plus que la première. 
\item Trouver les dimensions d'un rectangle dont le périmètre est 560 m, sachant que la largeur est le $\frac7{13}$ de la longueur. 
\item Partager une somme de $20~230$ F entre trois personnes de façon que la part de la deuxième soit 
les $\frac7{22}$ de celle de la première et celle de 
la troisième les $\frac{16}{33}$ de celle de la première. 
\item Partager 5~597 F entre deux personnes de façon
que la part de la première soit égale aux $\frac45$ de celle de la seconde plus 125 F. 
\item Une personne a dépensé dans un magasin les $\frac{5}{11}$ de ce qu'elle possédait. Il lui manque alors 3 F pour acheter 5 m d'étoffe à 9 F le mètre. Que possédait-elle primitivement ? 
\item Un ouvrier ferait un ouvrage en 5 jours, un autre le ferait en 6 jours. Quelle fractions de l'ouvrage font-ils en un jour lorsqu'ils travaillent ensemble ? 
\item Un robinet remplirait un bassin en 6 heures et un autre le viderait en 10 heures. Les deux robinets
étant ouverts, quelle est la fraction du bassin remplie en une heure ? 
\item Vénus et la Terre tournent autour du Soleil, la première en 225 jours, la seconde en 365 jours. Quelle est, de ces deux planètes, celle qui tourne la plus 
vite autour du soleil et quelle est la fraction de tour qu'elle fait de plus que l'autre en un jour ? 
\item Mars tourne autour du Soleil en 687 jours. Déterminer en fraction de tour, l'avance prise par 
la Terre sur Mars, en un jour. 
\item Un piéton marche pendant 3 heures : pendant la première, il parcourt 6 km $\frac58$,
pendant la deuxième, il fait 1 km $\frac14$ de moins que pendant la première, 
et pendant la troisième, $\frac45$ km de moins que pendant la seconde. Quel est le trajet parcouru par ce piéton ? 
\item Après avoir vendu les $\frac25$ puis les $\frac37$ d'une pièce d'étoffe, il en reste 18 m. Quelle était la longueur de la pièce ? 
\item Un héritage a été partagé entre 4 personnes. La première en a reçu le tiers, la seconde le quart, la troisième le cinquième, et la dernière a reçu 5~200 F. Quel est le montant de l'héritage ? 
\item Une propriété de 149,6 hectares est constituée pour les $\frac7{11}$ de terres cultivables, pour les $\frac4{17}$ en bois, et le reste en prairies. Les terres valent 20 F l'are et rapportent 7,5 \% ; 
les bois valent 24 F l'are et rapportent 6\% ; 
les prairies valent 14 F l'are et rapportent 2,5\%. 
Calculer le revenu de cette propriété.
\item D'une pièce de toile, on fait trois parts : la première est égale aux $\frac27$ plus 6 m ; la seconde
au tiers plus 7 m ; la dernière a pour longueur les 11 m restants. Calculer la longueur de chaque part. 
\item Calculer les notes d'un candidat qui a obtenu 
dans un examen 42 points $\frac34$. Sa note d'arithmétique surpasse de 1 point $\frac14$ celle d'histoire ; dans cette matière, il a obtenu 2 points $\frac34$ de plus qu'en orthographe. (Il n'y a pas d'autre discipline). 
\item Trois robinets remplissent un bassin : le premier a débité 54 litres $\frac14$, le deuxième 3 litres $\frac58$ de moins que le premier et le troisième 10 litres $\frac13$ de moins que les deux premiers réunis. Quelle est la capacité du bassin ? 
\item On remplit jusqu'aux $\frac38$ de sa hauteur un bassin ayant la forme d'un parallélépipède rectangle, dont la largeur est la moitié de la longueur. Puis on fait monter le niveau de 0,38 m en ajoutant 7,6 hectolitres. Le bassin est alors rempli aux $\frac57$ de sa hauteur. Calculer ses dimensions. 
\item Une fermière a vendu les $\frac25$ d'un panier d'œufs. Si elle ajoutait 46 à ce qui lui reste, le nombre des œufs qu'elle avait d'abord serait augmenté de son $\frac19$. Calculer ce nombre.
\item Soit la fraction $\frac58$ et la fraction $\frac7{10}$ obtenue en ajoutant 2 à chacun des termes. Comparer ces deux fractions en comparant leurs 
compléments à l'unité. 

Comparer de même les fractions $\frac{17}{35}$ et $\frac{17+n}{35+n}$ ainsi que les fractions $\frac{17}{35}$ et $\frac{17-n}{35-n}$, où $n$ est entier et inférieur à 17. 

\end{enumerate}