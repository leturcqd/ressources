
 \chapter{Fractions décimales, nombres décimaux}
 
 \begin{enumerate}
 \item Écrire sous forme décimale : 
 \[ \frac{75}{10};\phantom{meowmeow}\frac4{100};\phantom{meowmeow}\frac{12}{1~000};\phantom{meowmeow}\frac{25}{10~000};\phantom{meowmeow}\frac{23~752}{1~000};\phantom{meowmeow}\frac{47}{10~000}.\]
 \item Écrire sous forme de fractions les nombres suivants : 
 \[7,24; \phantom{meowmeow} 3,572;\phantom{meowmeow}
 0,0417;\phantom{meowmeow}5,00178;\phantom{meowmeow}
 0,0072.\]
 \item Combien peut-on écrire de nombres à 3 décimales compris entre 5,32 et 5,33 ? Combien peut-on écrire de nombres à 4 décimales compris entre 12,347 et 12,348 ? 
 \item Comparer les nombres 12,7 et 12,07. De façon générale, comment se modifie un nombre à une décimale lorsqu'on place un zéro entre la virgule et le chiffre décimal ?
 \item Quel est le nombre décimal de mètres contenus dans une longueur de 7 dam, 5 m, 3 dm et 8 cm ? 
 \item Effectuer de deux manières différentes les additions suivantes : 
 \[ 37,7 + (42+0,75+7,12);\] 
 \[ 13 + 10,5 + ( 43, 75 + 5,725) + (7 + 2,57) ; \]
 \[ 372, 5 + ( 5,703 + 12, 7 + 3,9).\]
 \item Calculer de deux façons différentes le résultat des opérations suivantes : 
 \[ 53 - ( 3,7 + 4,52 + 0,17);\]
 \[ 271,5 - (37,7 + 12 + 0,57 + 43,5).\]
 \item Calculer de deux façons différentes le résultat des opérations suivantes : 
 \[ 703,75 + (219,5-49,2);\]
 \[1~000 + (2,712-0,47).\]
 \item Calculer de deux façons différentes le résultat des opérations suivantes : 
 \[ 512,7-(57,25-43,5);\]
 \[47-(4,509-3,7).\]
 \item Calculer de deux façons : 
 \[ (0,75\times 27) + (0,75 \times 13); \]
 \[ (0,52\times 19) - (0,52 \times 17).\]
 \item Calculer : 
 \[ (0,25)^2; \phantom{meowmeow} (2,15)^3; \phantom{meowmeow}
 (2,45)^3 \times (2,45)^2; \phantom{meowmeow}
 \frac{(9,81)^7}{(9,81)^5}.\]
 \item Une vis avance de $\frac7{10}$ de millimètre en 13 tours. Combien doit-elle faire de tours pour avancer de 3,5 mm ? 
 \item Prendre le plus simplement possible les 0,9; les 0,99; les 0,999 du nombre 17,8. 
 \item Transformer en fractions décimales les fractions suivantes : 
 \[\frac32; \phantom{meowmeow}
 \frac54; \phantom{meowmeow}
 \frac45; \phantom{meowmeow}
 \frac{21}{25};\phantom{meowmeow}
 \frac78; \phantom{meowmeow}
 \frac{11}{250}.\]
 \item La fraction $\frac8{19}$ peut-elle se convertir en fraction décimale ?
 \item Par quel nombre décimal faut-il diviser un nombre pour réduire ce 
 nombre aux $\frac8{15}$ de sa valeur ? 
 \item Par quelle fraction décimale faut-il multiplier un nombre pour 
 le diminuer des $\frac6{100}$ de sa valeur ? 
 \item En déplaçant de deux rangs vers la gauche la virgule d'un nombre décimal, il diminue de 5~749,326. Quel est ce nombre ? 
 \item Un alliage d'argent et de cuivre provenant de la fonte de deux lingots
 pèse 3~800 grammes et a pour titre $\frac{775}{1~000}$. Quel est le titre du 
 premier lingot sachant que le second lingot avait pour poids 2 kg et pour titre $\frac{650}{1~000}$ ? 
 \item 
 \begin{enumerate}
 \item Quelle est la distance de Paris à Dijon sachant que sur une carte dont 
 l'échelle est un millionième, elle est représentée par une longueur de 31,5 cm ? 
 \item Par quelle longueur est-elle représentée sur une carte dont l'échelle est 4 dix-millionièmes ? 
 \item Quelle est l'échelle d'une carte sur laquelle la distance Paris-Dijon est représentée par 25,2 cm ? \end{enumerate}
 \item Effectuer mentalement : 
 \[ 54,45+ 3,95 ; \phantom{meowmeow} 12,55 + 5,75+ 4,70 ; \phantom{meowmeow} 
 8,72 + 15,9 + 5,6.\]
 \[ 30 - 22,85 ; \phantom{meowmeow} 75,7 - 13,45 ;\phantom{meowmeow} 134,4 - 12,18.\]
 \[ 19,8 \times 0,5 ; \phantom{meowmeow} 15,60\times 0,25 ; \phantom{meowmeow} 17,12 \times 0,75 ; \phantom{meowmeow} 14,32\times 0,125.\]
 \[ 7,3 \times 11; \phantom{meowmeow}3,15 \times 9 ; 
 \phantom{meowmeow} 5,1 \times 19 ; \phantom{meowmeow}
 3,4 \times 99 ; \phantom{meowmeow} 7 \times 5,95.\]
 \[ 24,7 \div 0,5 ; \phantom{meowmeow} 58 \div 0,25 ; \phantom{meowmeow}
 51,351\div 0,75; \phantom{meowmeow} 0,52 \div 0,125.\]
 \item Le nombre $x$ varie de dixième en dixième entre 2 et 5. Dresser le tableau de correspondance entre $x$ et les nombres $y$ suivants, puis construire le graphique qui en résulte : 
 \[ y = 3x; \phantom{meowmeow} y = 3x+4; \phantom{meowmeow} y=3x-1.\]
 \[ y = \frac{3x}5; \phantom{meowmeow}y = \frac{3x}5 + 2; \phantom{meowmeow}
 y = \frac{3x}5 - \frac12.\]
 \[ y = x^2; \phantom{meowmeow} y = x^3; \phantom{meowmeow} y = \frac45x^2.\]
 \[ y = 2x^2 - \frac15; \phantom{meowmeow} 
 y = x^3 + 1; \phantom{meowmeow} y = \frac45 x^2 + \frac12.\]
 
 \end{enumerate}