
\chapter*{Problèmes de révision } 
 \begin{enumerate}
 \item On donne un angle $\widehat{xOy}$ et un point $A$. Construire un triangle isocèle dont l'angle au sommet soit l'angle $\widehat{xOy}$ et tel que le côté opposé passe par $A$. 
 \item \begin{enumerate}
 \item Soit un triangle $ABC$ et $M$ le milieu de $[BC]$. La perpendiculaire au côté $[AB]$ issue de $M$ coupe ce côté en $H$. On prolonge $[HM]$ d'un segment $MK=HM$
 et on joint $[KC]$. Comparer les triangles $MBH$ et $MCK$. En déduire la valeur de l'angle $\widehat{MKC}$. 
 \item On donne un angle $\widehat{xAy}$ et un point $M$ intérieur à cet angle. Construire une droite passant par $M$ coupant $(Ax)$ en $B$ et $(Ay)$ en $C$ de façon que $M$ soit le milieu de $[BC]$. 
 \end{enumerate}
 \item \begin{enumerate}
 \item Construire deux droites $(D_1)$ et $(D_2)$ respectivement perpendiculaires
 en $A$ et $B$ au segment $[AB]$. Une droite $(xy)$ issue du milieu $O$ de $[AB]$ coupe $(D_1)$ en $P$ et $(D_2)$ en $Q$. Comparer les triangles $OAP$ et $OBQ$.
 \item Comparer les angles $\widehat{APO}$ et $\widehat{OQB}$. Que peut-on dire des angles que forme $(PQ)$ avec les droites $(D_1)$ et $(D_2)$ ?
 \end{enumerate}
 \item On donne un triangle isocèle $ABC$ de base $[BC]$. On construit les angles droits $\widehat{BAx}$ et $\widehat{CAy}$ adjacents à l'angle $\widehat{BAC}$. 
 Puis on porte sur $(Ax)$ et $(Ay)$ les segments $[AE]$ et $[AF]$ égaux à $[AB]$.
 Soit $M$ le milieu de $[BC]$ et $N$ le milieu de $[EF]$. \begin{enumerate}
 \item Comparer les triangles $MAE$ et $MAF$.
 \item Démontrer que les points $A$, $M$ et $N$ sont alignés.
 \end{enumerate}
 \item Soit un triangle $ABC$ rectangle en $A$, et $[AD]$ la hauteur relative à l'hypoténuse. On construit les symétriques $E$ et $F$ de $D$ par rapport aux droites $(AB)$ et $(AC)$.\begin{enumerate}
 \item Montrer que $D$, $E$, $F$, appartiennent à un même cercle de centre $A$.
 \item Montrer que les points $E$, $A$, $F$ sont alignés. Que représente $A$ pour le segment $[EF]$ ?
 \end{enumerate}
 \item D'un même côté d'un segment $[AB]$, on mène deux segments égaux $[AD]$ et $[BC]$ respectivement perpendiculaires à $[AB]$ en $A$ et en $B$.
 \begin{enumerate}
 \item Comparer les segments $[AC]$ et $[BD]$. 
 \item $[AC]$ et $[BD]$ se coupent en $O$. Démontrer que la médiatrice de $[AB]$ passe par $O$ et qu'elle est aussi médiatrice de $[CD]$.
 \end{enumerate}
 \item Deux segments $[AC]$ et $[DB]$ sont perpendiculaires et leur point commun
 $O$ est le milieu de chacun d'eux. \begin{enumerate}
 \item Montrer que le quadrilatère $ABCD$ a ses quatre côtés égaux et comparer les 
 quatre triangles $OAB$, $OCB$, $OCD$ et $OAD$.
 \item On mène $OP$ perpendiculaire à $[AD]$ en $P$. La droite $(PO)$ coupe $(BC)$ en 
 $R$. Comparer les triangles $ODP$ et $OBR$. En déduire que $(OR)$ est perpendiculaire à $(BC)$ et que $OP=OR$. 
 \item Montrer de même que la droite $(OQ)$ perpendiculaire en $Q$ à $(AB)$ est 
 perpendiculaire en $S$ à $(CD)$ et que $OP=OQ=OR=OS$. 
 \end{enumerate}
 \item Construire un angle $\widehat{AOB}=30^o$ puis deux angles $\widehat{AOC}$ et
 $\widehat{BOD}$ adjacents à $\widehat{AOB}$ et valant $60^o$ chacun. \begin{enumerate}
 \item Montrer que les angles $\widehat{AOB}$ et $\widehat{COD}$ ont même bissectrice.
 \item Montrer que les bissectrices $(ON)$ et $(OP)$ des angles $\widehat{AOC}$ et $\widehat{BOD}$ dont un angle droit.
 \end{enumerate}
 \item On donne un angle $\widehat{xOy}=45^o$ et un point $A$ intérieur à cet angle.
 Construire les points $B$ et $C$ symétriques de $A$ par rapport aux droites
 $(Ox)$ et $(Oy)$.\begin{enumerate}
 \item Évaluer l'angle $\widehat{BOC}$. Nature du triangle $BOC$.
 \item Démontrer que la médiatrice du segment $[BC]$ passe par $O$. Comparer l'angle
 formé par cette médiatrice et $(Oy)$ à l'angle $\widehat{AOx}$. 
 \end{enumerate}
 \item Trois segments $OA$, $OB$, $OC$ sont égaux et tels que les angles $\widehat{AOB}$, $\widehat{BOC}$, et $\widehat{COA}$ soient égaux :  \begin{enumerate}
 \item Quelle est la valeur commune de ces trois angles ? 
 \item Démontrer que le triangle $ABC$ est équilatéral. 
 \item Montrer que $O$ est équidistant des trois côtés du triangle $ABC$ et que
 le triangle ayant pour sommets les pieds des perpendiculaires menées de $O$ 
 aux côtés du triangle $ABC$ est lui-même équilatéral.
 \end{enumerate}
 \item Construire un triangle isocèle tel que $AB=AC=5$ cm et tel que l'angle
 $\widehat{BAC}=40^o$. La médiatrice de $[AC]$ coupe la droite $(BC)$ en $D$. On joint $[DA]$ et on le prolonge de $AE=BD$.\begin{enumerate}
 \item Montrer que $DAC$ est isocèle, puis établir l'égalité des angles $\widehat{CAE}$ et $\widehat{ABD}$. 
 \item Comparer les triangles $CAE$ et $ABD$. En déduire que le triangle $CDE$ est
 isocèle.
 \end{enumerate}
 \item Soit un triangle $ABC$ et la bissectrice intérieure $(Ax)$ de l'angle
 $\widehat{A}$. sur la demi-droite $[Ax)$, on construit les points $B'$ et $C'$ tels
 que $AB'=AB$ et $AC'=AC$. Les droites $(BC')$ et $(B'C)$ se coupent en $D$. \begin{enumerate}
 \item Comparer les triangles $ABC'$ et $ACB'$ puis les segments $[BC']$ et $[B'C]$.
 \item On mène $(AE)$ perpendiculaire en $E$ à $(BC')$ et $(AF)$ perpendiculaire en $F$ à $(B'C)$. \\ Demontrer que $AE=AF$ puis que $(DA)$ est une bissectrice du triangle $BCD$.
 \end{enumerate}
 \item On donne un triangle $ABC$. Sur le côté $[AB]$ on porte $AC'=AC$ et sur le côté $[AC]$ on porte $AB'=AB$. La bissectrice intérieure de l'angle $A$ coupe $[BC]$ en $D$.\begin{enumerate}
 \item Comparer les segments $[DC']$ et $[DC]$ puis les segments $[DB']$ et $[DB]$.
 \item Démontrer que les points $C'$, $D$ et $B'$ sont alignés.
 \end{enumerate}
 \item Soit $I$ le point commun aux bissectrices intérieures du triangle $ABC$. 
 On construit les segments $[IA']$, $[IB']$ et $[IC']$ respectivement perpendiculaires en $A'$, $B'$ et $C'$ aux côtés $[BC]$, $[CA]$ et $[AB]$. 
 \begin{enumerate}
 \item Démontrer que $(IA)$, $(IB)$ et $(IC)$ sont les médiatrices du triangle $A'B'C'$ et les bissectrices des angles $\widehat{B'IC'}$, $\widehat{C'IA'}$ et
 $\widehat{A'IB'}$.
 \item Montrer que les angles $\widehat{AIB'}$ et $\widehat{BIC}$ sont supplémentaires.
 \end{enumerate}
 \item On considère deux triangles isocèles $OAB$ et $OCD$ de sommet $O$ tels que les angles $\widehat{AOB}$ et $\widehat{COD}$ aient même bissectrice $(Ox)$. \begin{enumerate}
 \item Comparer les triangles $OAC$ et $OBD$. 
 \item Soit $M$ un point quelconque de $(Ox)$. Comparer les triangles $MAC$ et $MBD$.
 \end{enumerate}
 \item\begin{enumerate}
 \item Construire un triangle $ABC$ connaissant $AB= 8$ cm; $BC= 5$ cm et sachant que l'angle $\widehat{BAC}$ vaut $30^o$. Nombre de solutions ?
 \item Déduire de cette construction que deux triangles qui ont deux côtés respectivement égaux et un angle égal opposé à l'un de ces côtés ne sont pas nécessairement égaux.
\end{enumerate}  
\item On considère deux triangles $ABC$ et $A'B'C'$ tels que les angles $\widehat{A}$ et $\widehat{A'}$ soient supplémentaires, $AB=A'B'$ et $BC=B'C'$. On prolonge $[CA]$
d'une longueur $AD$ égale à $A'C'$.\begin{enumerate}
\item Comparer les triangles $A'B'C'$ et $ABD$. Nature du triangle $BCD$ ? 
\item Démontrer que les angles $\widehat{ACB}$ et $\widehat{A'C'B'}$ sont égaux.
\end{enumerate}
\item Un angle $\widehat{xOy}$ vaut $120^o$. D'un point $A$ de sa bissectrice on mène les perpendiculaires $(AM)$ à $(Ox)$ et $(AN)$ à $(Oy)$. La droite $(AM)$ coupe $(Oy)$ en $B$, et la droite $(AN)$ coupe $(Ox)$ en $C$. \begin{enumerate}
\item Comparer les triangles $OAM$ et $OBM$ ainsi que les triangles $OAN$ et $OCN$.
\item Démontrer que $OA=OB=OC$ puis que le triangle $ABC$ est équilatéral. \\ 
Que représente le point $O$ pour le triangle $ABC$ ?
\end{enumerate}
\item Soit un cercle de diamètre $[AB]$ de centre $O$. La médiatrice de $[OA]$ coupe
le cercle en $C$ et $D$ : la médiatrice de $[OB]$ coupe le cercle en $E$ et $F$ ($C$ et $E$ sont du même côté de $[AB]$). \begin{enumerate}
\item Montrer que les triangles $OAC$, $OAD$, $OBE$ et $OBF$ sont équilatéraux et égaux.
\item Montrer que les points $C$, $O$, $F$ d'une part et $D$, $O$, $E$ sont alignés.
\item Montrer que le diamètre du cercle $O$ perpendiculaire à $[AB]$ est médiatrice
des segments $[CE]$ et $[DF]$. 
\end{enumerate}
\item Construire la hauteur $[AH]$ et la médiane $[AM]$ du triangle $ABC$, puis
prolonger $[AH]$ d'un segment $HD=AH$ et $[AM]$ d'un segment $ME=AM$. Les droites $(BD)$ et $(CE)$ se coupent en $P$. \begin{enumerate}
\item Comparer les triangles $ABH$ et $DBH$, puis les triangles $MAB$ et $MEC$. 
\item Démontrer que le triangle $PBC$ est isocèle et que $(PM)$ est médiatrice des segments $[BC]$ et $[BE]$. 
\item On suppose de plus que $MA=MB$. Montrer que dans ce cas les cinq points $ABCDE$ sont sur un même cercle.
\end{enumerate}
\item Dans un quadrilatère $ABCD$ les angles $\widehat{ABC}$ et $\widehat{ADC}$ sont égaux et la diagonale $[AC]$ est bissectrice de l'angle $\widehat{BAD}$. On mène 
les perpendiculaires $(CH)$ et $(CK)$ à $(AB)$ et à $(AD)$. \begin{enumerate}
\item Comparer les triangles $AHC$ et $AKC$, puis les triangles $BHC$ et $DKC$.
\item Démontrer que $BC=CD$, $AB=AD$ et que $(AC)$ est bissectrice de l'angle
$\widehat{BCD}$.
\end{enumerate}
\item On considère un quadrilatère convexe $ABCD$ dans lequel la diagonale $[AC]$ 
fait des angles aigus égaux avec les côtés $[AB]$ et $[CD]$. On suppose en outre que les angles $\widehat{B}$ et $\widehat{D}$ du quadrilatère sont aigus et égaux. On mène les perpendiculaires $(AH)$ à $(CD)$ et $(CK)$ à $(AB)$.\begin{enumerate}
\item Comaprer les triangles rectangles $AKC$ et $CHA$, puis les triangles $BKC$ et $DHA$. \item Démontrer que $[AC]$ fait des angles égaux avec $(AD)$ et $(BC)$, que $AB=CD$ et que $AD=BC$.
\end{enumerate}
\item Construire un quadrilatère convexe $ABCD$ dans lequel $BC=3$ cm, $\widehat{ABD}=80^o$, $\widehat{BDC}=100^o$ et $AD=BC=4$ cm. On prolonge $[AB]$ d'une longueur $BE=CD$. \begin{enumerate}
\item Comparer les triangles $DBC$ et $DBE$. Conséquences ? 
\item Nature du triangle $DAE$ ? Démontrer que les angles $\widehat{BAD}$ et $\widehat{BCD}$ sont égaux.
\end{enumerate}
\item Dans le quadrilatère convexe $ABCD$ les diagonales $[AC]$ et $[BD]$ sont égales et les angles $\widehat{ABC}$ et $\widehat{BCD}$ sont supplémentaires. On prolonge $[AB]$ d'une longueur $BE=CD$. \begin{enumerate}
\item Comparer les triangles $BCD$ et $CBE$. Nature du triangle $CAE$ ? 
\item Démontrer que les angles $\widehat{BAC}$ et $\widehat{BDC}$ sont égaux.
\end{enumerate}
\item Soit un triangle $ABC$ tel que $AB>AC$. La bissectrice intérieure de l'angle
$\widehat{A}$ coupe en $D$ la médiatrice de $[BC]$. On construit le point $E$ de $[AB]$ tel que $AE=AC$. \begin{enumerate}
\item Que représente le point $D$ pour le triangle $BCE$ ? Nature du triangle $DBE$ ?
\item Comparer les triangles $ACD$ et $AED$ et montrer que les angles $ABD$ et $ACD$
sont supplémentaires. 
\end{enumerate}
\item Dans le quadrilatère convexe $ABCD$ les côtés $[BC]$ et $[CD]$ sont égaux 
et les angles $\widehat{B}$ et $\widehat{D}$ sont supplémentaires. Soient $H$ et $K$
les projections du point $C$ sur les droites $(AB)$ et $(AD)$.
\begin{enumerate}
\item Comparer les triangles $BCH$ et $DCK$. Conséquences ?
\item Montrer que $(AC)$ est bissectrice de l'angle $\widehat{BAD}$. Comparer $AH$
et $AK$ à la demi-somme de $AB$ et $AD$.
\end{enumerate}
\item Soit $O$ le centre du cercle circonscrit au triangle $ABC$. La médiatrice 
de $[BC]$ coupe en $I$ le côté $[AB]$ et en $J$ la droite $(AC)$.\begin{enumerate}
\item Comparer les triangles $IOB$ et $IOC$ et démontrer que les angles
$\widehat{IAO}$ et $\widehat{ICO}$ sont égaux.
\item Comparer les triangles $JOB$ et $JOC$ et démontrer que les angles $\widehat{JAO}$ et $\widehat{JBO}$ sont supplémentaires.
\end{enumerate}
\item La bissectrice extérieure de l'angle $\widehat{A}$ du triangle $ABC$ coupe en $M$ la médiatrice de $[BC]$. On prolonge $[BA]$ d'une longueur $AD=AC$.\begin{enumerate}
\item Comparer $MB$ et $MC$, puis les triangles $MAC$ et $MAD$.
\item Nature du triangle $MBD$ ? Démontrer l'égalité des angles $\widehat{ABM}$ et 
$\widehat{ACM}$.
\end{enumerate}
\item Dans le quadrilatère convexe $ABCD$ on a $AB>CD$, $AD=BC$ et d'autre part la 
diagonale $[AC]$ fait des angles aigus égaux avec $[AB]$ et $[CD]$. 
\begin{enumerate}
\item Construire le point $E$ de $[AB]$ tel que $AE=CD$ et comparer les triangles $ACD$ et $CAE$. Nature du triangle $BCE$ ? 
\item Démontrer que les angles $ABC$ et $ADC$ sont supplémentaires ? 
\end{enumerate}
\item Dans le triangle $ABC$, on considère un point intérieur $D$ tel que $AD=BC$ et tel que les angles $\widehat{ABC}$ et $\widehat{ADC}$ soient supplémentaires. On
prolonge $[AB]$ d'une longueur $BE=CD$.\begin{enumerate}
\item Comparer les triangles $ADC$ et $CBE$. Nature du triangle $CAE$ ? 
\item La droite $(CD)$ coupe $(AB)$ en $M$. Démontrer que $MA=MC$ et en déduire une construction géométrique du point $D$ en connaissant le triangle $ABC$. 
\end{enumerate}
\item Les côtés $[AB]$ et $[CD]$ d'un quadrilatère convexe $ABCD$ sont égaux. Les diagonales $[AC]$ et $[BD]$ se coupent en un point $O$ de telle sorte que $OA=OC$ et 
$OB>OD$. Les droites $(AB)$ et $(CD)$ se coupent en $I$. \begin{enumerate}
\item Construire le point $E$ du segment $[OB]$ tel que $OE=OD$ et comparer les triangles $OAE$ et $OCD$. Nature du triangle $ABE$ ? 
\item Comparer les angles $\widehat{ABO}$ et $\widehat{CDO}$, puis les segments $[IB]$ et $[ID]$.
\end{enumerate}
\item On considère un quadrilatère convexe $ABCD$ tel que $AB=CD$. Les diagonales 
$[AC]$ et $[BD]$ se coupent en $O$ tel que : $OA>OD$ et $OB=OC$. \begin{enumerate}
\item Soit $E$ le point du segment $[OA]$ tel que $OE=OD$. Comparer les triangles
$OBE$ et $OCD$.
\item Nature du triangle $BAE$ ? Comparer les angles $\widehat{BAC}$ et $\widehat{BDC}$.
\item Les droites $(BE)$ et $(CD)$ se coupent en $I$. Comparer $IB$ et $IC$, ainsi que $ID$ et $IE$ et montrer que $(IO)$ est bissectrice de l'angle $\widehat{BIC}$.
\end{enumerate}
\item Soit un triangle isocèle $OAB$ de base $AB<AO$. Le cercle de centre $A$ passant par $O$ recoupe la droite $(OB)$ en $E$. On prend un point $C$ sur le segment $[BE]$
et le point $D$ du segment $[OA]$ tel que $OD=EC$. \begin{enumerate}
\item Nature du triangle $AOE$ ? Comparer les triangles $AEC$ et $BOD$. 
\item Comparer les segments $[AC]$ et $[BD]$ ainsi que les angles $\widehat{ACB}$ et
$\widehat{ADB}$. 
\end{enumerate}
\item On désigne par $D$ et $E$ les milieux des côtés $[AB]$ et $[AC]$ du triangle $ABC$ et par $G$, $H$ et $K$ les projections sur la droite $(DE)$ des points $A$, $B$ et $C$. \begin{enumerate}
\item Comparer les triangles $AGD$ et $BHD$ ainsi que les triangles $AGE$ et $CKE$. 
En déduire que $BH=CK$ et que $HK=2DE$. 
\item Soit $M$ le milieu de $[BC]$ et $I$ le milieu de $[HK]$. Comparer les triangles $IBH$ et $ICK$. Nature du triangle $BIC$ ? Montrer que $(IM)$ est médiatrice de $[BC]$ et de $[HK]$. 
\end{enumerate}
\item Un \emph{trapèze isocèle} est un quadrilatère convexe $ABCD$ dont les \emph{bases} $AB$ et $CD$ admettent la même médiatrice $(xy)$. Démontrer que : 
\begin{enumerate}
\item Les angles adjacents à une même base sont égaux. 
\item Les côtés obliques $[AD]$ et $[BC]$ sont égaux et leurs prolongements se coupent en $I$ sur la droite $(xy)$ formant deux triangles $IAB$ et $ICD$ isocèles. 
\item Les diagonales $[AC]$ et $[BD]$ sont égales et se coupent en $J$ sur $(xy)$
formant deux triangles $JAB$ et $JCD$ isocèles.
\end{enumerate}
\item Démontrer qu'un quadrilatère convexe $ABCD$ est un trapèze isocèle (cf. exercice précédent), lorsque : \begin{enumerate}
\item les angles $\widehat{A}$ et $\widehat{B}$ sont égaux ainsi que les angles 
$\widehat{C}$ et $\widehat{D}$. 
\item Lorsque les côtés $[AD]$ et $[BC]$ sont égaux ainsi que les angles $\widehat{A}$ et $\widehat{B}$.
\item Lorsque les côtés $[AD]$ et $[BC]$ sont égaux ainsi que les diagonales $[AC]$ et $[BD]$.
\end{enumerate}
\item Le quadrilatère convexe $ABCD$ admet pour axe de symétrie la droite $(xy)$ médiatrice commune de $[AB]$ et $[CD]$ (trapèze isocèle). \begin{enumerate}
\item Démontrer que les médiatrices des côtés $[BC]$ et $[AD]$ se coupent en un point $O$ de $(xy)$, centre d'un cercle circonscrit au trapèze. 
\item Comparer sur ce cercle les arcs de même sens $\arc{DA}$ et $\arc{BC}$ ainsi que les arcs $\arc{DB}$ et $\arc{AC}$. 
\end{enumerate}
\item On considère un quadrilatère convexe $ABCD$ dont les sommets sont sur un cercle de centre $O$ de telle sorte que les côtés $[AD]$ et $[BC]$ soient égaux. \begin{enumerate}
\item Compare les angles au centre $\widehat{AOD}$ et $\widehat{BOC}$.
\item Démontrer que la bissectrice $(Ox)$ de l'angle $\widehat{AOB}$ est également bissectrice de l'angle $\widehat{AOB}$ est également bissectrice de l'angle $\widehat{COD}$ et médiatrice des deux côtés $[AB]$ et $[CE]$ du quadrilatère $ABCD$. 
\end{enumerate}
\item On désigne par $M$ et $N$ les milieux des côtés obliques $[AD]$ et $[BC]$ du trapèze isocèle $ABCD$ dont les bases $[AB]$ et $[CD]$ admettent pour médiatrice commune la droite $(xy)$. Le segment $[MN]$ coupe les diagonales $[AC]$ en $P$ et $[BD]$ en $Q$. \begin{enumerate}
\item Démontrer que $AM=CN$ et que les angles $\widehat{DMN}$ et $\widehat{CNM}$ sont égaux. 
\item On prolonge $[MP]$ d'une longueur $ME$ égale à $NP$. Comparer les triangles $AME$ et $CNP$. Nature du triangle $APE$ ? En déduire que les points $P$ et $Q$ sont les milieux des diagonales $[AC]$ et $[BD]$.
\end{enumerate}
\item Construire un triangle $ABC$ tel que $AB= 6$ cm, $BC=5$ cm, $AC=4$ cm, puis le point $D$ du côté $[AB]$ et le point $E$ du prolongement de $[CA]$ tels que $AD=AE=1$ cm. La droite $(DE)$ coupe $(BC)$ en $M$. \begin{enumerate}
\item Comparer les segments $[BD]$ et $[CE]$ ainsi que les angles $\widehat{BDM}$ 
et $\widehat{CEM}$.
\item On prolonge $[DM]$ d'une longueur $MP=DE$. Comparer les triangles $BDP$ et $CEM$. Nature du triangle $BMP$ ? Démontrer que $M$ est le milieu de $[BC]$. 
\end{enumerate}
\item Soient quatre points $A$, $B$,$C$, $D$ disposés dans cet ordre sur un cercle de centre $O$ et tels que $AC=BD$. On désigne par $E$ le point d'intersection de $[AC]$
et $[BD]$, par $I$ le milieu de $[AC]$ et par $J$ le milieu de $[BD]$. La droite $(IJ)$ coupe $(AB)$ en $M$.
\begin{enumerate}
\item Comparer les triangles $OAI$ et $OBJ$, puis les triangles $OEI$ et $OEJ$. 
Nature du triangle $EIJ$ ? 
\item On prolonge $[JI]$ d'une longueur $IN=JM$. Comparer les triangles $AIN$ et $BJM$. Nature du triangle $AMN$ ? En déduire que $M$ est le milieu de $[AB]$. 
\end{enumerate}
\item On considère un angle aigu $\widehat{BAC}$. Le point $B$ se projette en $E$
sur $[AC]$ et le point $C$ se projette en $F$ sur $[AB]$. Les segments $[BE]$ et $[CF]$ se coupent en $H$. Les symétriques de la droite $(EF)$ par rapport à $(BE)$
et par rapport à $(CF)$ se coupent en $D$. 
\begin{enumerate}
\item Montrer que les quatre points $A$, $B$, $C$ et $H$ sont équidistants des trois droites $(EF)$, $(FD)$ et $(DE)$. 
\item En déduire que $(AH)$ et $(BC)$ sont les bissectrices de l'angle $\widehat{D}$ du triangle $DEF$ et qu'elles sont perpendiculaires. Démontrer ainsi que les trois 
hauteurs du triangle $ABC$ sont concourantes en $H$. 
\end{enumerate}
 \end{enumerate}
 
 