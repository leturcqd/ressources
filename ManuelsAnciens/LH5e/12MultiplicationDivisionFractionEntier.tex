
\chapter{Multiplication et division d'une fraction par un nombre entier}
 
 \begin{enumerate}
 \item Découper des bandes de papier $AB$ de $120$ mm de long. Les partager en 12 parties égales. Vérifier 
 que les fractions suivantes de $AB$ sont égales : 
 \[ \frac5{12}\times 3 = \frac{15}{12}; \phantom{meow}\phantom{meow}\frac7{12}\times 6 = \frac72. \]
 \[ \frac53\div4=\frac5{12}; \phantom{meow}\phantom{meow}\frac43\div2=\frac23.\]
\item Effectuer les opérations suivantes et simplifier les résultats : 
\[ \frac56\times 8 ; \phantom{meow} \frac7{12}\times6; 
\phantom{meow}\frac{15}7\times21; \phantom{meow}
\frac{27}{13}\times 39.\] 
\[ \frac{45}{28} \times 12; \phantom{meow} \frac{39}{25} \times 20; \phantom{meow} 
\frac{17}{40}\times 8;\phantom{meow} \frac{49}{25}\times35 .\]
\[\frac{48}{35}\div30; \phantom{meow} \frac{33}{25}\div 15 ; \phantom{meow}\phantom{meow}
\frac{108}{55}\div 27 ; \phantom{meow} 
36 \div 54. \]
\[ \frac{84}{39}\div 36; \phantom{meow}
\frac{216}{35}\div 48 ; \phantom{meow}
\frac{360}{77}\div 135; \phantom{meow}
\frac{355}{113}\div 71.\]
\item Simplifier et calculer les expressions : 
\[ \frac{189\times 22}{126}; \phantom{meow}\phantom{meow} \frac{154\times 45}{525} ; \phantom{meow}\phantom{meow} \frac{168\times 63}{756}.\]
\[\frac{84\times 78}{1~512}\phantom{meow}\phantom{meow} \frac{432\times 168}{336} ; 
\phantom{meow}\phantom{meow} \frac{684\times 252}{11~340}.\]
\item Une personne dépense dans un magasin les $\frac34$ de ce qu'elle possède, puis dans un autre le tiers du reste. Il lui reste encore 72 F. Que possédait-il primitivement.
\item Partager une somme de 133~000 F entre 3 personnes de façon que la part de la 2e soit les $\frac56$ de celle de la première et celle de la troisième le tiers de celle de la deuxième. 
\item Un joueur perd les $\frac37$ de sa fortune, puis le $\frac13$ de sa fortune. Il lui reste le $\frac1{10}$ de ce qu'il a perdu plus 850 F. Quelle était sa fortune primitive ? 
\item Une balle élastique rebondit au tiers de la hauteur d'où elle est tombée. À quelle fraction de la hauteur primitive s'élève-t-elle après 4 bonds successifs ? 
\item On soutire chaque jour 3 L $\frac14$ d'une pièce de vin de 225 litres. Que reste-t-il après 60 jours ?
\item Quel est le poids de viande d'un bœuf de 450 kg contenant $\frac15$ d'os. 
\item Une personne exécute le cinquième d'un travail en 1 heure. Combien de temps lui faudra-t-il pour en 
faire les $\frac23$, les $\frac34$, les $\frac56$ ?
Quelle fraction en fait-elle en deux heures et demi ? 
\item On multiplie un nombre par $\frac58$ puis ce même nombre par $\frac7{12}$; on fait la somme des deux produits. Calculer le nombre sachant que cette somme surpasse de 135 unités le nombre lui-même. Faire la vérification. 
\item On partage les $\frac47$ d'une somme d'argent entre 6 personnes et le reste entre 4 personnes. 
Quelle fraction de la somme chaque personne a-t-elle 
reçue ? Quelles sont les parts, la somme s'élevant à 84~000 F ? 
\item Deux héritiers s'étant partagé une somme de 106~000 F ont dépensé, les premiers les $\frac67$ de sa part et le second les $\frac9{11}$ de la sienne. Le premier possède alors trois fois de plus que le second. Calculer les parts. 
\item Une ménagère a acheté une pièce de drap et une toile qui lui ont coûté ensemble 486 F. Elle a payé le drap 12 F le mètre et la toile 4,50 F le mètre. Elle emploie le $\frac45$ et les $\frac37$ de la pièce de toile et il se trouve que les restes des deux pièces ont des valeurs égales. Calculer combien chaque pièce contient de mètres. 
\item Un cultivateur laisse à ses deux fils un pré et une vigne estimés chacun 4~000 F l'hectare. Les $\frac58$ de la superficie du pré sont égaux aux $\frac67$ de la surface de la vigne, et la différence des surfaces est de 27,95 ares. 

On demande : \begin{itemize}
\item Quelle somme l'un des frères devra donner à l'autre pour que les deux parts soient égales ? 
\item Quelle est la surface du pré et quelle est la surface de la vigne ? 
\end{itemize}
\item Soit la fraction $\frac35$. 
\begin{enumerate}
\item Trouver une fraction égale à cette fraction dont le dénominateur soit 20. 
\item Trouver une fraction égale dont le numérateur soit 21. 
\item Trouver une fraction qui soit égale à la moité de la fraction proposée et dont le dénominateur soit 50. 
\end{enumerate} 
 \end{enumerate}
 