\chapter{Sommes de nombres entiers}
 \begin{enumerate}
\item Effectuer les additions suivantes : 
\[ 2\ 437 + 37\ 412 + 707 + 52\ 759 ;\]
\[ 3~127+ 25~742+790~395 + 42~759~375 ;\]
\[ 902~812+ 43 + 254 + 4~127 + 512~752.\]
\item De combien augmente une somme de trois nombres si on augmente le premier de $12$ unités, le deuxième de $3$ dizaines, et le troisième de $4$ centaines ?
\item De combien augmente une somme de trois nombres si on augmente le premier de $7$ dizaines, le deuxième de $25$ centaines, le troisième de $9$ mille ?
\item Calculer la somme des dix premiers nombres entiers. 
Calculer la somme des dix premiers nombres impairs.
\item Trouver trois nombres entiers consécutifs sachant que leur somme est $45$. 
\item Trouver quatre nombres entiers consécutifs sachant que leur somme est 498. 
\item En effectuant une addition de nombres entiers sans faire de retenues, on 
trouve dans chaque colonne, de droite à gauche, les sommes suivantes : 14, 11, 9. Quel est le résultat de l'addition ? 
\item Trois personnes se partagent une certaine somme. La première a $5~120$ F, la deuxième a 270 F de plus que la première. La troisième a autant que les deux autres ensemble. Quelle est la part de chacune ? la somme à partager ?
\item Dans un jeu de dominos, chaque pièce est formée par l'association d'un des chiffres de $0$ à $6$ avec lui-même ou avec un autre. \begin{enumerate}
\item Calculer le nombre de pièces différentes du jeu. Le comparer avec la somme des 7 premiers nombres entiers.
\item Combien de fois figure un chiffre donné dans l'ensemble du jeu ? 
\item Calculer le nombre total de points inscrits sur tous les dominos du jeu.
\end{enumerate}
\item \begin{enumerate}
Le carré ci-contre est dit « magique » car, en additionnant les nombres situés sur une même ligne horizontale, dans une même colonne verticale, ou bien sur une même diagonale, on obtient chaque fois le même résultat. Vérifiez-le. 
\[\displaystyle\begin{tabular}{|c|c|c|}
\hline 
8 & 1 & 6\\
\hline
3 & 5 & 7\\
\hline 
4 & 9 & 2\\
\hline
\end{tabular}
\]
\item On ajoute $4$ à chacun des nombres du carré magique. Montrer que l'on obtient encore un carré magique.
\item Quel nombre faut-il ajouter pour que la somme par ligne, colonne ou diagonale, soit égale à 54 ? Former ce carré.
\end{enumerate}
\item On considère les nombres de 1 à 12. \begin{enumerate}
\item De combien de manières peut-on les associer deux par deux de façon à obtenir une somme égale à 13 ? 
\item De combien de manières peut-on associer trois de ces nombres, distincts entre eux, de façon à obtenir une somme égale à 15 ? 
\end{enumerate}
\item \begin{enumerate}
\item Dessiner un carré partagé en $100$ petits carreaux disposés suivant 10
rangées horizontales de 10 carreaux chacune. 
Puis écrire sur la première rangée les nombres de 0 à 9, sur la deuxième, les nombres de 1 à 10, sur la troisième les nombres de 2 à 11, et ainsi de suite. 
On obtient une table d'addition. 
\item Vérifier que le nombre qui se trouve sur la ligne horizontale qui commence par $7$ et dans la colonne verticale qui commence par $5$ est égal
à 7+5. 
\item Calculer la somme des nombres situés dans chacune des lignes, puis la somme de tous les nombres inscrits dans la table. 
\end{enumerate}
\item Une ménagère achète 4 articles dans un magasin. Le deuxième coûte 25 F de plus que le premier, le troisième 50 F de plus que le second et le quatrième 75 F de plus que le troisième. Elle paie avec deux billets de 500 F
sur lesquels on lui rend un billet de 50 F, deux billets de 10 F, et un billet de 5 F. Calculer le prix de chaque article. 
\item 
\item Un particulier qui dispose de 27~000 F veut faire construire un pavillon. Il compte 7~000 F pour l'achat du terrain, 25~000 F pour la maçonnerie et la couverture, 8~000 F pour la menuiserie, 3~000 F pour l'eau, le gaz et l'électricité, 5~000 F pour le chauffage central, 2~500 F pour la peinture et 1~5000 F de frais accessoires. 
\begin{enumerate}
\item Trouver le prix de revient du pavillon.
\item Le particulier sollicite un emprunt du Crédit foncier pour la somme qui
lui manque. Il se libère en 5 ans en remboursant 1//5 de cet emprunt à la fin de chaque année. Trouver le montant exact de chacun de ces cinq versements, sachant qu'à la fin de chaque année il devra verser en même temps l'intérêt
à 8\% de la somme due au Crédit foncier pendant l'année écoulée.
\end{enumerate}
\item Effectuer de deux manières différentes les additions suivantes : 
\[ 37 + ( 43+25+12);\]
\[ 42 + 17 + (109+12) + (472+38);\]
\[ 375 + (515 + 127 + 39).\]
\item Exercices de calcul mental :
\[\begin{tabular}{cccc}
70 + 40 & 900 + 600 & 70 + 14 & 18+ 80\\
242 + 80 & 30 + 712 & 50 + 2~743 & 80 + 537\\
42 + 67 & 253 + 34 & 419 + 71 & 718 + 62 \\ 
24 + 35 & 347 + 25 & 525 + 263 & 342 + 675 
\end{tabular}\]
\item Découper trois segments dans une feuille de papier de longueurs 
respectives $a$, $b$ et $c$. Vérifier que : 
\begin{enumerate}
\item $a + (b + c) = a + b + c$.
\item $a + b + c = a + c + b = b + a + c = b + c + a = c + a + b = c + b + a$.
\end{enumerate}
 \item Au nombre entier $a$ compris entre $0$ et 10, on ajoute 5, soit b le nombre obtenu : \begin{enumerate}
 \item Établir le tableau de correspondance entre les nombres $a$ et $b$. 
 \item Construire le graphique correspondant. 
 \end{enumerate}
 \end{enumerate}