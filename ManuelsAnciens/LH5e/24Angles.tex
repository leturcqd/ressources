
\chapter{Angles}  
\begin{enumerate}
\item Transformer en grades les angles suivants et les construire : 
\begin{enumerate}
\item $45^o$; $30^o$; $60^o$. 
\item $120^o$; $150^o$; $135^o$. 
\item $40^o15'$; $36^o30'$.
\item $50^o17'$; $112^o17'$. 
\end{enumerate}
\item Transformer en degrés les angles suivants et les construire : 
\begin{enumerate}
\item $30$ gr; $50$ gr; $90$ gr. 
\item $120$ gr; $160$ gr; $190$ gr. 
\item $20,5$ gr; $37,7$ gr; $68,9$ gr. 
\item $42,25$ gr; $112,6$ gr; $148,4$ gr. 
\end{enumerate} 

\item Montrer sur des exemples que la différence de deux angles ne change pas si on leur ajoute (ou retranche) un même angle.

\item Découper dans une feuille de papier deux angles adjacents $\widehat{AOB}$ et $\widehat{BOC}$ mesurant respectivement $68^o$ et $42^o$. Construire par pliage leurs bissectrices $[OM)$ et $[ON)$. Mesurer l'angle $\widehat{MON}$ et le comparer à l'angle $\widehat{AOC}$. 

\item On considère dans cet ordre $4$ demi-droites $[OA)$, $[OB)$, $[OC)$ et $[OD)$. 
\begin{enumerate}
\item Sachant que $\widehat{AOB}= \widehat{COD} = 35^o$ et $\widehat{BOC}=48^o$, construire les quatre demi-droites. Calculer et comparer les angles $\widehat{AOC}$ et $\widehat{BOD}$. 
\item Soit $[OM)$ la bissectrice de l'angle $\widehat{BOC}$. Montrer que $[OM)$ est également la bissectrice de l'angle $\widehat{AOD}$. 
\end{enumerate}

\item Deux angles $\widehat{AOB}$ et $\widehat{BOC}$ sont adjacents 
et $[OM)$ est la bissectrice de l'angle $\widehat{BOC}$. \begin{enumerate}
\item Construire la figure sachant que $\widehat{AOB}=60^o$ et $\widehat{AOC}= 110^o$. Calculer la mesure de l'angle $\widehat{AOM}$. 
\item Si $\widehat{AOB}= \alpha$ et $\widehat{AOC}= \beta$, montrer que $\widehat{AOM}= \frac12(\alpha + \beta)$. 
\end{enumerate}

\item Les angles $\widehat{AOB}$ et $\widehat{AOC}$ sont adjacents et 
$[OM)$ est la bissectrice de l'angle $\widehat{BOC}$. 
\begin{enumerate}
\item Effectuer la construction de ces angles en prenant $\widehat{AOB}= 52^o$ et $\widehat{AOC}=108^o$. Mener $[OM)$ et calculer la mesure 
de l'angle $\widehat{AOM}$. 
\item On suppose $\widehat{AOB}= \alpha$ et $\widehat{AOC}= \beta$ avec $\beta>\alpha$. Montrer que $\widehat{AOM}= \frac12(\beta-\alpha)$. 
\end{enumerate}

\item Soient $[OM)$ et $[ON)$ les bissectrices des angles adjacents $\widehat{AOB}$ et $\widehat{AOC}$. 
\begin{enumerate}
\item Construire la figure pour $\widehat{AOB}= 72^o$ et $\widehat{AOC}= 48^o$. Calculer les mesures des angles $\widehat{BOC}$ et $\widehat{MON}$. Comparer ces mesures. 
\item Si $\widehat{AOB}= \alpha$ et $\widehat{AOC}= \beta$, montrer 
que $\widehat{BOC}= \alpha + \beta$ et $\widehat{MON}= \frac12(\alpha+\beta)$. 
\end{enumerate}

\item On considère deux angles adjacents $\widehat{AOB}$ et $\widehat{BOC}$. Soient $[OM)$ et $[ON)$ les bissectrices des angles $\widehat{AOB}$ et $\widehat{AOC}$.
\begin{enumerate}
\item On donne $\widehat{AOB}= 60^o$ et $\widehat{AOC}= 108^o$. Construire la figure et calculer les mesures des angles $\widehat{BOC}$ et $\widehat{MON}$. Comparer ces deux mesures. 
\item On suppose $\widehat{AOB}= \alpha$ et $\widehat{AOC}= \beta$, montrer que $\widehat{MON}= \frac12\widehat{BOC}= \frac12(\beta-\alpha)$. 
\end{enumerate}

\item Les bissectrices $[OM)$ et $[ON)$ des angles non-adjacents $\widehat{AOB}$ et $\widehat{AOC}$ font un angle de $36^o$ et l'angle $\widehat{AOB}$ mesure $64^o$. \begin{enumerate}
\item Construire la figure et calculer les angles $\widehat{AOM}$, $\widehat{AON}$ et $\widehat{AOC}$.
\item Comparer les angles $\widehat{BOC}$ et $\widehat{MON}$. En est-il toujours ainsi ? 
\end{enumerate}

\item On considère deux angles adjacents $\widehat{AOB}$ et $\widehat{AOC}$ dont les bissectrices $[OM)$ et $[ON)$ font un angle de $84^o$. \begin{enumerate}
\item Sachant que l'angle $\widehat{AOC}$ vaut $118^o$, construire la figure et calculer les mesures des angles $\widehat{AON}$, $\widehat{AOM}$ et $\widehat{AOB}$. 
\item Comparer les angles $\widehat{MON}$ et $\widehat{BOC}$. Généraliser.
\end{enumerate}

\item Autour d'un point $O$ sont construits cinq angles successivement adjacents $\widehat{AOB}$, $\widehat{BOC}$, $\widehat{COD}$, $\widehat{DOE}$, et $\widehat{EOA}$ recouvrant tout le plan. Ces angles
vérifient les relations : 
\[ \widehat{BOC}= 2\widehat{AOB}; \phantom{meo} \widehat{COD}= \widehat{AOB}+ \widehat{BOC}; \phantom{meo} \widehat{DOE}= 2\widehat{BOC}; \phantom{meo}; \widehat{EOA}= \widehat{BOC}+\widehat{COD}.\]
\begin{enumerate}
\item Calculer la mesure en degrés de chacun de ces angles. 
\item Calculer l'angle des bissectrices des angles $\widehat{AOB}$ et $\widehat{DOE}$. 
\end{enumerate}

\item\begin{enumerate}
\item Construire trois angles successivement adjacents : $\widehat{AOB}= 32^o$, $\widehat{BOC}= 72^o$, et $\widehat{COD}= 48^o$, puis les bissectrices $[OM)$, $[ON)$, $[OP)$ et $[OQ)$ des angles $\widehat{AOB}$, $\widehat{AOC}$, $\widehat{BOD}$ et $\widehat{COD}$. 
\item Calculer les angles $\widehat{MON}$ et $\widehat{POQ}$. Comparer
ces angles à l'angle $\widehat{BOC}$. 
\item Montrer que les angles $\widehat{MOQ}$ et $\widehat{NOP}$ ont la même bissectrice. 
\end{enumerate}

\item \begin{enumerate}
\item Construire un angle $\widehat{AOB}$ de $60^o$, sa bissectrice $[Ox)$, puis les angles droits $\widehat{AOC}$ et $\widehat{BOD}$ adjacents à l'angle $\widehat{AOB}$ et enfin les bissectrices $[Oy)$, $[Oz)$ et $[Ou)$ des angles $\widehat{AOC}$, $\widehat{BOD}$, et $\widehat{COD}$. 
\item Calculer la valeur des angles $\widehat{COD}$, $\widehat{xOy}$, et $\widehat{xOz}$. Montrer que $[Ox)$ est la bissectrice de l'angle $\widehat{yOz}$. 
\item Calculer les mesures des angles $\widehat{yOu}$, $\widehat{zOu}$, et $\widehat{xOu}$. Que peut-on dire des demi-droites $[Ox)$ et $[Ou]$ ? 
\end{enumerate}

\end{enumerate}