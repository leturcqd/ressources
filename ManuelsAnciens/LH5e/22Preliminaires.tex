
\chapter{Notions préliminaires} 
\begin{enumerate}
\item Vérification de la règle. 
\item Marquer deux points $A$ et $B$ sur une feuille de papier. Déterminer la droite $(AB)$ par pliage. 
\item Marquer trois points $A$, $B$, $C$ sur une feuille de papier. Déterminer les droites qui joignent deux de ces points. 
Même exercice avec quatre points $A$, $B$, $C$ et $D$. Combien de droites obtient-on dans ce cas ? Construire leurs points d'intersection ? 
\item Construire dans un plan $4$ droites se coupant deux à deux en des points distincts. Montrer que l'on obtient ainsi $6$ points d'intersection. Combien de droites nouvelles obtient-on en joignant ces points deux à deux ? 
\item On donne dans un plan $6$ points tels que $3$ quelconques d'entre eux ne soient pas alignés et on les joint deux à deux. Combien de droites obtient-on ? Montrer que le nombre de ces droites est égal à la somme des $5$ premiers nombres entiers ou au demi-produit de $6$ par $5$. Généraliser pour $n$ points. 
\item On considère deux droites $(Ox)$ et $(Oy)$ concourantes en $O$, situées dans un plan $P$. On joint par une droite un point $A$ de $(Ox)$ et un point $B$ de $(Oy)$. Montrer que tout point $M$ de la droite $AB$ est situé dans le plan $P$.\\ En déduire que si $A$ et $B$ se 
déplacent simultanément sur $(Ox)$ et $(Oy)$ la droite $(AB)$ engendre le point $P$.
\item On considère trois points $A$, $B$, $C$ d'une même droite et $3$ points $A'$, $B'$, $C'$ d'une seconde droite, distincte de la première. Les droites $(AB')$ et $(A'B)$ se coupent en $M$, les droites $(AC')$ et $(A'C)$ se coupent en $N$ et les droites $(BC')$ et $(B'C)$ se coupent en $P$. Vérifier que les trois points $M$, $N$, $P$ sont alignés.
\item Construire trois droites issues d'un même point $I$. Puis d'un point $O$ mener à ces trois droites deux sécantes. Soient $A$, $B$, $C$
les intersections des trois droites avec la première sécante ; $A'$, $B'$, $C'$ leurs intersections avec la seconde. Les droites $(AB')$ et 
$(A'B)$ se coupent en $M$, les droites $(AC')$ et $(A'C)$ en $N$ et les
droites $(BC')$ et $(B'C)$ en $P$. Vérifier que les $4$ points $M$, $N$, $O$ et $P$ sont alignés. 
\end{enumerate}