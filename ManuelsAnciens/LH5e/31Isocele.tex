
\chapter{Triangle isocèle}
\begin{enumerate}
\item Démontrer que lorsque la hauteur $[AH]$ d'un triangle $ABC$ est en même temps médiane, le triangle est isocèle.
\item Dans un triangle $ABC$, la hauteur $[AH]$ est en même temps bissectrice intérieure de l'angle $\widehat{A}$. Démontrer que le triangle est isocèle.
\item On considère un triangle $ABC$ dans lequel la médiane $[AM]$ est 
en même temps bissectrice intérieure de l'angle $\widehat{A}$. On prolonge $[AM]$ 
d'une longueur $MD=AM$. \begin{enumerate}
\item Comparer les triangles $AMD$ et $DMC$, puis les angles $\widehat{MAB}$ et 
$\widehat{MDC}$. 
\item En déduire que le triangle $ACD$ est isocèle, et qu'il en est de même du triangle $ABC$.
\end{enumerate}
\item Démontrer que dans un triangle isocèle $ABC$ en $A$ : \begin{enumerate}
\item Les médianes $[BM]$ et $[CN]$ sont égales. 
\item Les bissectrices intérieures $[BD]$ et $[CE]$ sont égales.
\end{enumerate}
\item Dans un triangle isocèle $OAB$ de sommet $O$, on prend un point $C$ sur $[OA]$
et un point $D$ sur $[OB]$ tels que $OC=OD$.\begin{enumerate}
\item Comparer les triangles $OAD$, $OBC$ puis les triangles $ABC$ et $BAD$. Conséquences ? 
\item Montrer que le point d'intersection $I$ de $[AD]$ et $[BC]$ appartient à l'axe
de symétrie du triangle $OAB$.
\end{enumerate}
\item Reprendre l'exercice précédent avec l'hypothèse $\widehat{OAD}=\widehat{OBC}$.
\item Dans le quadrilatère $ABCD$, les angles $\widehat{B}$ et $\widehat{D}$ sont supplémentaires et $AD=BC$. On prolonge $[AB]$ d'une longueur $BE=CD$.
\begin{enumerate}
\item Comparer les triangles $ACD$ et $EBC$. Nature du triangle $ACE$ ?
\item Démontrer que les angles $\widehat{BAC}$ et $\widehat{ACD}$ sont égaux.
\end{enumerate}
\item On construit extérieurement à l'angle $\widehat{A}$ du triangle $ABC$ le 
segment $[BD]$ perpendiculaire à $(BA)$ et égal à $[AC]$, puis le segments $[CE]$
perpendiculaire à $(CA)$ et égal à $[AB]$. 
\begin{enumerate}
\item Comparer les triangles $ABD$ et $ECA$. Conséquences ? 
\item Démontrer l'égalité des angles $\widehat{ADE}$ et $\widehat{AED}$.
\end{enumerate}
\item Sur les côtés $[BC]$, $[CA]$ et $[AB]$ du triangle équilatéral $ABC$, 
on prend respectivement $A'$, $B'$ et $C'$ tels que $BA'=CB'=AC'$.\begin{enumerate}
\item Comparer les triangles $AB'C'$, $BC'A'$ et $CA'B'$. 
\item Montrer que le triangle $A'B'C'$ est équilatéral.
\end{enumerate}
\item On considère un triangle isocèle $OAB$ de sommet $O$ et un point $C$ du côté $[OA]$. On prolonge $[OB]$ dune longueur $BD=AC$. Le segment $[CD]$ coupe $[AB]$ en
$M$. On prolonge $[BA]$ d'une longueur $AP=BM$. \begin{enumerate}
\item Comparer les triangles $APC$ et $BMD$. Conséquence pour $CP$ et $MD$ et pour
les angles $\widehat{CPA}$ et $\widehat{DMB}$ ? 
\item Nature du triangle $CMP$ ? Que représente $M$ pour le segment $[CD]$ ?
\end{enumerate}
\item \begin{enumerate} \item Dans un triangle rectangle $ABC$, l'hypoténuse $[BC]$ est le double du côté de l'angle droit $[AB]$. Montrer que l'angle $\widehat{B}$ du triangle est le double 
de l'angle $\widehat{C}$.
\item Énoncer et démontrer la réciproque. On prolongera $[BA]$ d'une longueur $AD=AB$.\end{enumerate}
\item Construire un triangle isocèle $OAB$ tel que $\widehat{A} = \widehat{B}= 68^o$. 
Le cercle de centre $A$ passant par $B$ recoupe $[OB]$ en $C$. On prolonge $[OA]$ 
d'une longueur $AD=OC$.\begin{enumerate}
\item Comparer les angles $\widehat{ACO}$ et $\widehat{BAD}$ puis les triangles $ACO$ et $BAD$. 
\item Nature du triangle $OBD$. Calculer la somme des angles $\widehat{OBD}$ et 
$\widehat{BAC}$. 
\end{enumerate}
\begin{enumerate}
\item Démontrer que si un diamètre d'un cercle de centre $O$ est perpendiculaire
à la corde $[AB]$ il passe par le milieu $H$ de cette corde et par les milieux $M$ 
et $N$ des arcs qu'elle sous-tend.
\item Comparer les triangles $MAH$ et $MHB$, puis $NHA$ et $NHB$ et montrer que 
$(MN)$ est la bissectrice des angles $\widehat{AMB}$ et $\widehat{ANB}$. 
\end{enumerate}
\item On construit extérieurement au triangle isocèle $ABC$ de sommet $A$ deux triangles égaux $ABD$ et $ACE$. \begin{enumerate}
\item Comparer les triangles $ABE$ et $ACD$ puis les triangles $BCD$ et $CBE$. En 
déduire l'égalité de $BE$ et $CD$. 
\item Montrer que $[BD]$ et $[CE]$ ainsi que $[CD]$ et $[BE]$ se coupent en $I$ 
et $J$ sur la hauteur $[AH]$ du triangle $ABC$. 
\end{enumerate}
\item Soit un triangle isocèle $ABC$ tel que l'angle au sommet $\widehat{BAC}=56^o$.
On construit un segment $[AD]$ perpendiculaire à $[AB]$ et du même côté de $(AB)$ que $[AC]$, puis le segment $[AE]$ égal à $[AD]$, perpendiculaire à $(AC)$ de telle sorte que l'angle droit $\widehat{CAE}$ soit adjacent à l'angle $\widehat{BAC}$. 
\begin{enumerate}
\item Comparer les triangles $ABD$ et $ACE$, les segments $[BD]$ et $[CE]$ et les
angles $\widehat{BAC}$ et $\widehat{DAE}$.
\item On mène les hauteurs $[AH]$ et $[AK]$ des triangles $ABC$ et $ADE$. Évaluer 
l'angle $HAK$. Les droites $(BD)$ et $(CE)$ se coupent en $I$ : mesurer l'angle $\widehat{BIC}$.
\end{enumerate}
\item Construire un quadrilatère $ABCD$ tel que $AD=52$ mm, $AB=CD=24$ mm et que les angles $\widehat{A}$ et $\widehat{D}$ soient égaux à $48^o$. Les droites $(AB)$ et 
$(CD)$ se coupent en $O$. 
\begin{enumerate}
\item Montrer que les triangles $AOD$ et $BOC$ sont isocèles et que la bissectrice de l'angle $\widehat{AOD}$ est aussi médiatrice de $[AD]$ et de $[BC]$.
\item Comparer les triangles $OAC$ et $OBD$. Que peut-on dire du point d'intersection
des segments $[AC]$ et $[BD]$ ? 
\end{enumerate}
\item Dans le quadrilatère $ABCD$ l'angle $\widehat{A}$ mesure $40^o$, les côtés $[AB]$ et $[AD]$ mesurent $3$ cm, et les angles $\widehat{B}$ et $\widehat{D}$ sont égaux à $52^o$. \begin{enumerate}
\item Construire le quadrilatère. Montrer que le triangle $BAD$ est isocèle puis, qu'il en est de même du triangle $BCD$.
\item Que représente la droite $(AC)$ pour le segment $[BD]$ et les angles $\widehat{A}$ et $\widehat{C}$ du quadrilatère. 
\end{enumerate}
\item Les côtés $[AB]$ et $[AD]$ du quadrilatère convexe $ABCD$ sont égaux et les angles $\widehat{B}$ et $\widehat{D}$ sont supplémentaires. On prolonge $[CB]$ d'une
longueur $BE=CD$. \begin{enumerate}
\item Comparer les triangles $ABE$ et $ADC$. Nature du triangle $ACE$ ? 
\item Comparer les angles $\widehat{CAE}$ et $\widehat{BAD}$ puis les angles $\widehat{ACB}$ et $\widehat{ACD}$. Que représente la diagonale $(CA)$ pour l'angle 
$\widehat{BCD}$ ? 
\end{enumerate}
\end{enumerate}