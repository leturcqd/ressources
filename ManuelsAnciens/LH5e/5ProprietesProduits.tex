
 \chapter{Propriétés des produits de deux nombres}
 
 \begin{enumerate}
 \item Que devient le produit de deux nombres entiers
 lorsqu'on augmente l'un des facteurs de 1. Lorsqu'on augmente l'un des facteurs de x ? (Exemple : $43 \times 24$.)
\item Que devient le produit de deux nombres lorsqu'on augmente chaque facteur de 1 ? On pourra faire une figure rectangulaire. Même question pour une augmentation de x. (Exemple : $92 \times 23$). 
\item Que devient le produit de deux nombres lorsqu'on
diminue l'un des facteurs de 1, et lorsqu'on diminue 
les deux facteurs de 1 ? Même question avec une 
diminution de x. (Exemple : $247 \times 38$.)
\item Trouver les dimensions d'un rectangle, sachant qu'en augmentant la longueur et la largeur de 1 m, la
surface augmente de 170 mètres carré, et sachant d'autre part que la longueur a 84 m de plus que la largeur. 
\item Le produit de deux nombres est 340. Si l'on ajoute 3 au multiplicateur, le produit devient 400. Quels sont ces deux nombres ? 
\item Le produit de deux nombres est 575. Si l'on retranche 5 au multiplicateur le produit devient 450.
Quels sont ces deux nombres ? 
\item Que devient la surface d'un rectangle lorsqu'on augmente sa longueur de 1 m et qu'on diminue sa largeur de 1 m ? Que devient le produit de deux 
nombres lorsqu'on augmente l'un des facteurs de 1 
et que l'on diminue l'autre de 1 ? (Exemple : $537 \times 215$.)
\item Trouver les dimensions d'un rectangle dont le 
périmètre est 704 m sachant qu'en augmentant sa longueur de 1 m et en diminuant sa largeur de 1 m sa
surface diminue de 73 mètres carré. 
\item Développer : 
\[
\begin{tabular}{lll}
3(x + 7) + 5(x + 1) + 7(x + 2) &\phantom{meowmeow}& 7(x + 5) - 3(x + 2)\\
12(x + 5) + 4(x - 7) &\phantom{meowmeow}& 17(x - 3) - 16(x - 4)\\
12(x + y) + 7(x + 1) + 13(y + 2) &\phantom{meowmeow}& 100(x + y) - 36(x - y).
\end{tabular}
\]
\item Calculez de deux façons différentes les sommes ou différences suivantes : 
 \[
\begin{tabular}{lll}
$(15 \times 13) + (15 \times 7) + (15 \times 20)$ &\phantom{meowmeow}& $ (7 \times 17) + (17 \times 13) + (17 \times 5) $ \\
$(75 \times 21) + (75 \times 19)$ &\phantom{meowmeow}& 
$(43 \times 104) - (43 \times 100)$\\
$(43 \times 75) - (75 \times 40)$ &\phantom{meowmeow}& $(52 \times 17) - (52 \times 15)$.
\end{tabular}
\]
\item Mettre $x$ en facteur commun dans les sommes ou différences suivantes : 
\[ 5x + 12x + 13x\]
\[19x - 15x\]
\[ax + bx + cx + dx\]
\[xy - xz\]
\item Trouver deux nombres dont la somme est 232 sachant que le premier est le triple du second.
\item Trouver deux nombres dont la différence est 432
sachant que le premier est égal au septuple du second.
\item Partager 125 billes entre 3 enfants de façon que la part du second dépasse de 15 billes le double de 
la part du premier et que la part du troisième soit inférieure de 10 billes au triple de la part du premier. 
\item Calculer la somme des nombres contenus dans chacune des lignes de la table de Pythagore suivante : 
\[\begin{tabular}{| c | c | c | c | c | c | c | c | c |}
\hline
1 & 2 & 3 & 4 & 5 & 6 & 7 & 8 & 9 \\ \hline
2 & 4 & 6 & 8 & 10 & 12 & 14 & 16 & 18 \\ \hline
3 & 6 & 9 & 12 & 15 & 18 & 21 & 24 & 27 \\ \hline
4 & 8 & 12 & 16 & 20 & 24 & 28 & 32 & 36 \\ \hline
5 & 10 & 15 & 20 & 25 & 30 & 35 & 40 & 45 \\ \hline
6 & 12 & 18 & 24 & 30 & 36 & 42 & 48 & 54 \\ \hline
7 & 14 & 21  & 28 & 35 & 42 & 49 & 56 & 63 \\ \hline
8 & 16 & 24 & 32 & 40 & 48 & 56 & 64 & 72 \\ \hline
9 & 18 & 27 & 36 & 45 & 54 & 63 & 72 & 81 \\ \hline \end{tabular}\]
Est-il nécessaire d'effectuer toutes les additions ? 
Calculer la somme de tous les nombres de la table. 
\item Trouver un nombre de deux chiffres sachant que la somme de ses chiffres est 11 et que lorsqu'on échange le chiffre des unités et celui des dizaines,
le nombre augmente de 27. 
\item Trouver les deux facteurs d'un produit tel que si on multiplie chaque facteur par 3 le produit augmente de 280
\item On multiplie un nombre de 3 chiffres par 7, le résultat par 11, puis le nouveau résultat par 13. On obtient finalement 843~843. Quel était le nombre initial ? 
\item Un capitaine fait ranger ses hommes en carré, et il lui reste dix hommes non placés. Sachant d'autre part qu'il lui manque quinze hommes pour placer un homme de plus sur le côté du carré, trouver l'effectif de la compagnie du capitaine. 
\item Montrer que pour multiplier entre eux deux nombres compris entre 10 et 20, il suffit d'ajouter à l'un les unités de l'autre, de multiplier le résultat par 10 et d'ajouter ensuite le produit des chiffres des unités. Vérifier pour $18 \times 15$. 
\item Montrer que : 
\[ 54 \times 26 = (6 \times 4) \text{ unités } + 
[(6 \times 5) + (2 \times 4) ]\text{ dizaines } + 
(2 \times 5) \text{ centaines } \]
Trouver à partir de ce résultat un procédé pour écrire le chiffre des unités, puis celui des dizaines, et le nombre des centaines du produit de deux facteurs de deux chiffres. 
\item Les murs d'une salle de manipulation de 
4,80 m de longueur sur 2,10 m de largeur sont recouverts de carreaux de faïence sur une hauteur de 1,35 m. Il y a une porte de 0,90 m de large et les carreaux ont 15 cm de côté. Calculer le nombre de carreaux utilisés et leur prix de revient à raison 
de 75 F le cent. 
\item Une personne a pris au cours d'un mois 24 repas
tantôt dans un restaurant, tantôt dans un autre. Dans 
le premier, le repas coûte 4,20 F et dans le second
3,80 F. Sachant que la note dans le second restaurant
dépasse de 19,20 F la note payée dans le premier, on demande combien cette personne a pris de repas dans chaque restaurant. 
\item \begin{enumerate}
\item Un tailleur a acheté 3 coupons de drap de 3,5 m
chacun à raison de 25 F le mètre pour le premier, 28 F
pour le deuxième et 32 F pour le troisième. Combien
a-t-il payé ? 
\item Le tailleur utilise chacun de ces coupons pour 
effectuer un costume sur mesures. Pour chacun il 
dépense 80 F de main-d'œuvre et 30 F de fournitures.
Les costumes sont facturés 250 F, 270 F, et 300 F. 
Combien le tailleur a-t-il gagné ? 
\end{enumerate}
\item Découper et peser des plaques rectangulaires de dimensions, $a$ et $c$, puis $b$ et $c$, puis 
$a+b$ et $c$, $a-b$ et $c$. En déduire que 
\[ ac + bc = (a + b)c\text{    et    } ac - bc = (a - b) c.\]
\item Construire un rectangle de dimensions $a + b$ 
et $c + d$. Montrer qu'on peut le découper en quatre
rectangles de dimensions respectives $a$ et $c$, 
$b$ et $c$, $a$ et $d$, $b$ et $d$. En déduire que 
\[ (a + b)(c + d) = ac + bc + ad + bd\]
\item Construire un rectangle de longueur $a$ et de
largeur $c$. Augmenter sa longueur de $b$ et 
diminuer sa largeur de $d$. Évaluer la surface du 
rectangle de dimensions $a + b$ et $c - d$ ainsi 
formé par rapport à celle des rectangles de dimensions
respectives $a$ et $c$; $b$ et $c$; $a$ et $d$ ; $b$
et $d$. En déduire que 
\[ (a + b)(c - d) = ac + bc - ad - bd\]
\item Construire un rectangle de longueur $a$, 
de largeur $c$. Retrancher $b$ à sa longueur et 
$d$ à sa largeur. Évaluer la surface du rectangle 
de dimensions $a - b$ et $c - d$ par rapport à 
celles des rectangles de dimensions respectives 
$a$ et $c$ ; $a$ et $d$ ; $b$ et $c$; $b$ et $d$. 
En déduire que : 
\[ (a - b)(c - d) = ac - bc - ad + bd\]

 \end{enumerate}