 
 \chapter{Différences de nombres entiers}
 \begin{enumerate}
 \item Que devient la différence de deux nombres.\begin{itemize}
 \item Si on augmente le premier terme de 12.
 \item Si on augmente le second terme de 12. 
 \item Si on augmente le premier terme de 12 et le second de 10. 
 \item Si on augmente le premier terme de 10 et le second de 12.
 \end{itemize}
 
 \item Calculer de deux façons différentes le résultat des opérations suivantes : 
 \[
 \begin{tabular}{l c l }
 2~315 - (37 + 452 + 17) & \ \ \ \ \ & 3~057 + (539 - 423) \\
 2~715 - (377 + 12 + 57 + 425) & \ \ \ \ \ & 70~375 + (2~195 - 492).
  \end{tabular}
 \]
 
  \item Calculer de deux façons différentes le résultat des opérations suivantes : 
 \[
 \begin{tabular}{l c l }
 4~039 - (3~215 - 2~237) & \ \ \ \ \ & 3~429 - (2~615 - 1~732) \\
 5~127 - (5~725 - 4~350) & \ \ \ \ \ & 6~847 - (3~240 - 2~428).
  \end{tabular}
 \]
 
 \item Supprimer les parenthèses en utilisant les propriétés des sommes et
 des différences dans les expressions suivantes : 
  \[
 \begin{tabular}{l c l }
 a + (b + c) + (d - e) & \ \ \ \ \ & a + (b + c) - (d - e) \\
 a - (b + c) + (d - e) & \ \ \ \ \ & a - (b + c) - (d - e).
  \end{tabular}
 \]
 
 \item Qu'obtient-on en ajoutant la somme de deux nombres et leur différence ?
 Qu'obtient-on si, de la somme de deux nombres, on retranche leur différence ?
 \item Trouver deux nombres, connaissant leur somme 342 et leur différence 88.
 \item Trouver deux nombres, connaissant leur somme 61~975 et leur différence
 2~047.
 \item Si Pierre donne 16 billes à Jean, ils en ont le même nombre.
 Combien Jean a-t-il de billes de plus que Pierre ? 
 \item Dans la soustraction 712 - 84, on oublie de faire les retenues.
  Trouver l'erreur commise sans faire l'opération.
 \item Trouver trois nombres dont la somme est 192, sachant que le deuxième
 surpasse le premier de 17 et que le troisième surpasse le deuxième de 23.
 \item Deux nombres ont pour différence 18. Si on les augmente tous deux 
 de 6, le premier devient le double du second. Trouver ces deux nombres. 
 \item Trouver trois nombres, sachant que la somme des deux premiers est 28,
 celle des deux derniers est 32, et celle du premier et du troisième est 30.
 \item Remplir les chiffres manquants dans les additions suivantes : \\
 $ 
 \begin{tabular}{cccc}
 . & . & . & 2 \\
  & 8 & 4 & . \\
  & 9 & 4 & 3 \\ 
  \hline 
  3 & 5 & 8 & 2
  \end{tabular}
$  \phantom{meowmeowmeow}
  $ \begin{tabular}{cccc}
 . & 7 & 3 & . \\
  & 7 & . & 2 \\
  2 & . & 5 & 4 \\ 
  \hline 
  7 & 8 & 7 & 7
  \end{tabular}
 $\phantom{meowmeowmeow}
  $ \begin{tabular}{ccccc}
& 2 & 3 & . & 7 \\
 &4 & 5 & 6 & . \\
 &. & . & 9 & 5 \\ 
  \hline 
  1&2 & 7 & 0 & 4
  \end{tabular}
 $
 
  \item Remplir les chiffres manquants dans les soustractions suivantes : \\
 $ 
 \begin{tabular}{ccc}
 7 & 9 & . \\
   . & . & 2 \\
  \hline 
  2 &2 & 6
  \end{tabular}
$  \phantom{meowmeowmeow}
  $ \begin{tabular}{cccc}
  . & 7 & . & . \\ 
  . & 7 & 9 & 8 \\
  \hline 
  3 & 8 & 3 & 5
  \end{tabular}
 $\phantom{meowmeowmeow}
  $ \begin{tabular}{cccc}
. & 8 & . & . \\
8 & . & 3 & 5 \\
  \hline 
  4 & 8 & 7 & 4
  \end{tabular}
 $
 \item Deux segments de droite ont une longueur totale de 118 cm. Le plus grand 
 a 12 cm de plus que l'autre. Quelle est la longueur de chaque segment ? 
 \item On veut partager une pièce d'étoffe de 60 m de long en 3 coupons de façon que le premier ait 5 m de plus que le second et 11 m de moins que le 
 troisième. Trouver les longueurs des trois coupons.
 \item Trois camarades font une excursion. Le premier paie le voyage : 
 3 billets à 2,25 F l'un. Le second paie les repas du midi : 3 déjeuners à 3
 F l'un plus 10\% de service. Le troisième paie 7,20 F pour les repas du soir. 
 Comment règleront-ils leurs comptes pour que les dépenses soient également partagées ?
 \item Plusieurs enfants se réunissent pour acheter un  ballon de football. Chacun d'eux doit payer 1,30 F. Mais au moment de l'achat trois d'entre eux sont absents, si bien que chacun des présents doit payer 1,60 F.
 Trouver le nombre total d'enfants, ainsi que le prix du ballon.
 \item Une ménagère décide d'utiliser ses économies du mois à l'achat 
 de mouchoirs. Elle pourrait acheter 15 mouchoirs d'ordinaire et il lui 
 resterait 2 F. Elle préfère dépenser 1 F de plus et faire 
 l'acquisition d'une douzaine de beaux mouchoirs coûtant 0,70 F de plus chacun. 
 De quelle somme disposait-elle, et quel prix a-t-elle payé chacun de ses mouchoirs ? 
 \item Un déjeuner à 8 F par personne réunit un certain nombre de convives.
 Trois de ces convives sont des invités et ne participent pas à la dépense,
 si bien que chacun des autres doit payer, y compris 10\% pour le service,
 11,20 F. Calculer le nombre total de convives.
 \item Effectuer mentalement les soustractions suivantes : 
 \[\begin{tabular}{rrr}
 237 - 187 & 871 - 791 & 4~783 - 4~573 \\
 217 - 29 & 712 - 89 & 7~813 - 59 \\
 701 - 439 & 802 - 547 & 1~003 - 719 \\ 
 2~754 - 781 & 3~232 - 2~192 & 7~833 - 5~935
\end{tabular}\]
 
 \item Découper deux segments $a$ et $b$ dans une feuille de papier.
 Vérifier que leur différence ne change pas lorsqu'on leur ajoute ou retranche un même segment de longueur c. 
 
 \item Découper trois segments $a$, $b$, et $c$ dans une feuille de papier tels que $b> c$ et $ b + c < a$. Vérifier que : 
 \[ a - (b+c) = a - b - c;\]
 \[ a + (b - c) = a + b - c ;\]
 \[ a - (b - c) = a - b + c.\] 
 
 \item Au nombre 12, on retranche le nombre entier $a$ compris entre 0 et 10. Soient b les nombres obtenus. 
 \begin{enumerate}
 \item Établir le tableau de correspondance entre $a$ et $b$. 
 \item Construire le graphique correspondant.
 \end{enumerate}
 
 \end{enumerate}
 