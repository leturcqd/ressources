\chapter{Symétrie par rapport à une droite}
\begin{enumerate}
\item Soient deux points $A$ et $A'$ symétriques par rapport à une droite $(xy)$ et 
deux points quelconques $I$ et $J$ de la droite $(xy)$. Montrer que les cercles de
centres $I$ et $J$ passant par $A$ se recoupent au point $A'$ En déduire une construction au compas du symétrique d'un point par rapport à la droite $(xy)$. 
\item \begin{enumerate}
\item Construire le symétrique d'un polygone $ABCDEF$ par rapport à une droite $(xy)$
en utilisant l'exercice précédent. 
\item Vérifier que $(BD)$ et son symétrique $(B'D')$ se coupent en $O$ sur $(xy)$ et
que l'angle $\widehat{BOB'}$ admet $(xy)$ pour bissectrice. 
\end{enumerate}
\item \begin{enumerate}
\item Construire le symétrique $\widehat{B'A'C'}$ d'un angle $\widehat{BAC}$ par rapport à la droite $(xy)$.
\item Dans quel cas la droite $(AB)$ est-elle confondue avec sa symétrique $(A'B')$ ?
\end{enumerate}
\item Dans un triangle $ABC$ on mène la bissectrice intérieure de l'angle $A$ coupant en $D$ le côté $[BC]$. (On suppose $AB<AC$). \begin{enumerate}
\item Construire le point $E$ symétrique de $B$ par rapport à $(AD)$. Que 
représente $(CE)$ pour les segments $[AB]$ et $[AC]$ ? 
\item Que représente la droite $(DA)$ pour l'angle $D$ du triangle $CDE$ ? 
\end{enumerate}
\item Reprendre l'exercice précédent avec $(AD)$ bissectrice extérieure de l'angle $\widehat{A}$. 
\item \begin{enumerate}
\item Construire le symétrique $A'B'C'$ du triangle $ABC$ par rapport à la droite $(xy)$. 
\item Les médianes $[BM]$ et $[CN]$ du triangle $ABC$ se coupent en $G$, les médianes
$[B'M']$ et $[C'N']$ du triangle $A'B'C'$ se coupent en $G'$. Que représente $(xy)$
pour le segment $[GG']$ ? 
\end{enumerate}
\item On considère un angle $\widehat{BAD}$ tel que $AB=AD$. On construit à l'intérieur de cet angle deux angles égaux $\widehat{ABx}$ et $\widehat{ADy}$.
\begin{enumerate}
\item Montrer que $[Bx)$ et $[Dy)$ sont symétriques par rapport à la bissectrice $[Ou)$ de l'angle $\widehat{BAD}$. Où se trouve le point d'intersection $C$ de ces
deux demi-droites ? 
\item Comparer les segments $[BC]$ et $[DC]$ et les angles $\widehat{ACB}$ et $\widehat{ACD}$.
\end{enumerate}
\item On prend les symétriques $B$ et $C$ d'un point $A$ intérieur à l'angle droit $\widehat{xOy}$ par rapport aux droites $[Ox)$ et $[Oy)$. \begin{enumerate}
\item Montrer que le point $O$ est le milieu du segment $[BC]$ et que le 
cercle de diamètre $[BC]$ passe par $A$. 
\item Comparer l'angle $\widehat{BAC}$ à la somme des angles $\widehat{B}$ et $\widehat{C}$ du triangle
$ABC$.
\end{enumerate}
\item Les deux points $A$ et $B$ sont d'un même côté de la droite $(xy)$. On construit le symétrique $A'$ de $A$ par rapport à $(xy)$. La droite $(A'B)$ coupe 
$(xy)$ en $M$. \begin{enumerate}
\item Montrer que $(MA)$ et $(MB)$ dont des angles égaux avec la droite $(xy)$.
\item Existe-t-il un autre point $N$ de la droite $(xy)$ tel que $(xy)$ soit
bissectrice extérieure de l'angle $\widehat{ANB}$ ? 
\end{enumerate}
\item On considère deux points $A$ et $B$ de part et d'autre de la droite $(xy)$
et on désigne par $A'$ le symétrique de $A$ par rapport à la droite $(xy)$.
La droite $(BA')$ coupe $(xy)$ en $M$.  \begin{enumerate}
\item Montrer que $(MA)$ et $(MB)$ dont des angles égaux avec la droite $(xy)$.
\item Existe-t-il un autre point $N$ de la droite $(xy)$ tel que $(xy)$ soit
bissectrice de l'angle $\widehat{ANB}$ ? 
\end{enumerate}
\item Soit un cercle de centre $O$ et de diamètre $[AB]$. On construit deux arcs
égaux $\arc{AC}$ et $\arc{AD}$. \begin{enumerate}
\item Montrer que $(AB)$ est la médiatrice de $[CD]$.
\item Démontrer que $AC=AD$, $BC=BD$ et que la droite $(AB)$ est bissectrice des 
angles $\widehat{COD}$, $\widehat{CAD}$ et $\widehat{CBD}$. 
\end{enumerate}
\item On construit d'un même côté d'un segment $[AB]$ deux angles égaux 
$\widehat{BAC}$ et $\widehat{ABD}$ tels que $AC=BD$. \begin{enumerate}
\item Montrer que la figure obtenue admet un axe de symétrie. En déduire l'égalité de 
$AD$ et de $BC$ et l'égalité entre les angles $\widehat{ACB}$ et $\widehat{ADB}$.
\item Retrouver ces propriétés en comparant les triangles $ABC$ et $BAD$ puis montrer
que $(AD)$ et $(BC)$ se coupent sur la médiatrice de $[AB]$. 
\end{enumerate}
\item Un quadrilatère $ABCD$ dont les diagonales se coupent en $O$ est tel que $OA=OB$ et $OC=OD$. \begin{enumerate}
\item Montrer que ce quadrilatère admet pour axe de symétrie la droite $(Ox)$ bissectrice des angles $\widehat{AOB}$ et $\widehat{COD}$. 
\item Montrer que $(AD)$ et $(BC)$ se coupent sur cet axe et démontrer $AD=BC$.
\end{enumerate}
\item On considère un angle $\widehat{xOy}$ et un point $A$ intérieur tel que $\widehat{AOx}=38^o$ et $\widehat{AOy}= 16^o$. \begin{enumerate}
\item Construire les symétriques $B$ et $C$ du point $A$ par rapport à $(Ox)$ et $(Oy)$, puis la bissectrice de l'angle $\widehat{BOC}$ sur laquelle on porte $OD=OA$. 
\item Montrer que l'angle $\widehat{BOC}$ est le double de l'angle $\widehat{xOy}$
et que les quatre points $A$, $B$, $C$ et $D$ sont sur un cercle de centre $O$. 
\item Montrer que $B$ et $C$ sont symétriques par rapport à $(OD)$ 
et que $A$ et $D$ sont symétriques par rapport à la droite $(Ou)$ bissectrice de l'angle $\widehat{xOy}$. 
\end{enumerate}
\end{enumerate}