
\chapter{Les fractions}
\begin{enumerate}
\item Découper une ficelle $AB$ en $2$, $4$, $8$ et $16$ segments égaux. Vérifier que : 
\[ AB \times \frac12 = AB \times \frac24 = AB \times \frac48 = AB \times \frac8{16}\]
puis que \[AB\times \frac58 > AB \times \frac38; \phantom{meowmeowmeow} AB \times \frac34 > AB \times
\frac38\]
\item Construire un angle, le partager en deux angles égaux, puis en 4 angles égaux, et en 8 angles égaux. Comparer les fractions correspondant à $\frac14$ et $\frac28$, puis celles correspondant à $\frac14$ et $\frac34$, puis celles correspondant à $\frac38$ et $\frac34$. 
\item Partager un cercle en 16 secteurs égaux. En déduire la comparaison des fractions suivantes : 
\[ \frac38\text{   et  } \frac6{16};\phantom{meowmeow}  \frac5{16}\text{   et  } \frac7{16};\phantom{meowmeow} 
 \frac58\text{   et  } \frac5{16}.\]
 \item Une longueur mesure  $360$ m. Quels sont les 
 produits de cette longueur par les fractions : 
 \[ \frac7{12}, \phantom{meow}\frac5{18},\phantom{meow}
 \frac{11}9,\phantom{meow}\frac{15}8, \phantom{meow}
 \frac9{10}\]
 \item Une balle élastique rebondit aux $\frac49$ de la hauteur où elle est tombée. On l'abandonne à une 
 hauteur de 1, 80 m au-dessus du sol. À quelle hauteur
 s'élève-t-elle après avoir rebondi 3 fois ? 
 \item La largeur d'un rectangle mesure 147 m ; elle est les $\frac7{11}$ de la longueur. Trouver la surface du rectangle. 
 \item Un tonneau est plein de vin ; on tire les $\frac79$ du tonneau et il reste encore $50$ litres de 
 vin. Quelle est la capacité du tonneau ? 
 \item Un alliage d'argent au titre de $\frac{875}{1~000}$ contient $1~750$ g de métal fin. Quel est le poids total de cet alliage ? 
 \item Un capital placé à $4$ pourcents produits un 
 intérêt de $724$ F. Quel est ce capital ? 
 \item Trouver une fraction égale à $\frac57$ ayant pour dénominateur $40$. 
 \item Trouver une fraction égale à $\frac{11}9$ 
 ayant pour dénominateur $63$. 
 \item Trouver une fraction égale à $\frac34$ 
 et telle que la somme de ses termes soit $21$.
 \item Trouver une fraction égale à $\frac85$ 
 et telle que la différence de ses termes soit $15$. 
 \item Une longueur $AB$ est mesurée par $\frac{13}{11}$ de mètre. Une longueur $CD$ est mesurée par $\frac5{11}$ de mètre. Quelle fraction de $AB$ représente $CD$ ? Quelle fraction de $CD$ représente 
$AB$ ? 
\item Quelle fraction de l'année représentent 5 jours ? 17 jours ? 265 jours ? 

Quelle fraction d'heure représentent 10 min ? 30 min ? 
7 min 21 s ? 13 s ? 

Quelle fraction du jour s'est-il écoulé lorsqu'il est 
$8$ h ? $7$ h du soir ? 

Quelle fraction de la semaine reste-t-il après 3 jours ? 

\item Quelle fraction\footnote{$1'$(une minute) est un soixantième de degré ; $1''$(une seconde) est un soixantième de minute.} de degré représentent les angles suivants : 
\[ 35',\phantom{meow}11', \phantom{meow}27', 
\phantom{meow}50',\phantom{meow}20'',\phantom{meow}
17'',\phantom{meow}45'', \phantom{meow}13''?\]

\item Un voyageur parcourt en chemin de fer 560 km. Quelle distance a-t-il parcourue lorsqu'il a accompli les $\frac58$ du parcours ? 
\item Les $\frac{15}{22}$ d'un nombre sont 855. Quel est ce nombre ? 
\item Trouver deux longueurs dont la somme a pour mesure 6,5 cm en sachant que l'une est les $\frac49$ de l'autre. 
\item Trouver toutes les fractions égales à $\frac7{10}$ dont le numérateur soit compris entre 400 et 500. 

\item Trouver toutes les fractions égales à $\frac4{11}$ dont le dénominateur soit compris entre 300 et 400. 
\item Comparer les fractions $\frac{11}{20}$ et 
$\frac{13}{17}$ en les comparant à une fraction auxiliaire qui ait le même numérateur que la seconde fraction et le même dénominateur que la seconde. 
\item Ranger les fractions suivantes dans l'ordre croissant (utiliser la méthode de l'exercice précédent) \[ \frac{73}{69}\phantom{meow}\frac{65}{72}
\phantom{meow}\frac{67}{70}\phantom{meow}\frac{323}{75}\]
\item Un cycliste prépare un voyage pour le lendemain :
il quittera la localité A pour aller à B où il 
s'arrêtera 2 h $\frac14$, puis terminera son excursion
en allant à C, où il désire arriver 25 minutes avant le passage du train de 17h 35 qu'il prendra pour le retour. \begin{enumerate}
\item À quelle heure doit-il partir de $A$ sachant que,
sur sa carte routière à l'échelle de 1 pour 200~000,
la distance de A à C mesure 40 cm $\frac12$, et que 
sa vitesse moyenne sera de 12 km à l'heure ? 
\item Calculer la distance BC sachant qu'elle est les $\frac45$ de la distance $AB$. 
\end{enumerate}
\item Un commerçant vend les $\frac56$ d'une pièce de toile et les $\frac34$ d'une pièce de soie soit, au total, 96 m de tissu. On demande quelle est la longueur de chacune des deux pièces sachant que les longueurs des coupons restants sont égales. On recommande de s'aider d'une figure. 
\end{enumerate}