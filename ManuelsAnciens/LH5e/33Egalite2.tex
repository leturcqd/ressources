
\chapter{Le troisième cas d'égalité des triangles}
\begin{enumerate}
\item Démontrer que lorsque deux triangles $ABC$ et $A'B'C'$ ont deux côtés respectivement égaux $AB=A'B'$ et $BC=B'C'$ ainsi que la médiane relative à l'un d'eux $AM = A'M'$, ces deux triangles sont égaux.
\item Démontrer que lorsque deux triangles $ABC$ et $A'B'C'$ ont deux côtés 
respectivement égaux $AB=A'B'$ et $AC=A'C'$ ainsi que la médiane relative au troisième 
$AM=A'M'$ ces deux triangles sont égaux (on prolongera $[AM]$ et $[A'M']$ d'une longueur égale $MD=M'D'=AM$ et on comparera d'abord $ABD$ et $A'B'D'$). 
\item Construire le centre $O$ du cercle circonscrit au triangle équilatéral $ABC$. 
\begin{enumerate}
\item Comparer les angles $\widehat{BOC}$, $\widehat{COA}$ et $\widehat{AOB}$ puis calculer leurs valeurs.
\item Montrer que les distances de $O$ aux trois côtés du triangle $ABC$ sont égales.
\end{enumerate}
\item Deux triangles $ABC$ et $A'B'C'$ sont égaux. Que peut-on dire des triangles obtenus en joignant dans chacun d'eux les pieds des hauteurs,des bissectrices, des médianes ?
\item Deux triangles équilatéraux $ABC$ et $A'B'C'$ sont tels que $AB=A'B'$. Comparer ces triangles ainsi que les rayons de leurs cercles circonscrits. 
\item Démontrer que dans un même cercle ou dans deux cercles égaux : \begin{enumerate}
\item Deux arcs égaux sont sous-tendus par des cordes égales. 
\item Deux cordes égales sous-tendent des arcs (inférieurs à un demi-cercle) égaux.
\end{enumerate}
\item Dans un quadrilatère $ABCD$, on a $AB=AD$ et $BC=DC$. 
\begin{enumerate}
\item Montrer que $(AC)$ est médiatrice du segment $[BD]$ et bissectrice des angles $\widehat{A}$ et $\widehat{C}$.
\item Soit $M$ le milieu de $[AC]$. Comparer $MB$ et $MD$ ainsi que $\widehat{AMB}$ et $\widehat{AMD}$.
\end{enumerate}
\item Le quadrilatère convexe $ABCD$ est tel que $AB=CD$ et $AD=BC$. \begin{enumerate}
\item Comparer les triangles $ABC$ et $CDA$, puis $ABD$ et $DBC$. Montrer qu'une diagonale forme des angles égaux avec deux côtés opposés. 
\item Les diagonales $[AC]$ et $[BD]$ se coupent en $O$. Que représente ce point pour
chacune d'elles ?
\end{enumerate}
\item Dans un quadrilatère convexe $ABCD$, on a $AD=BC$ et $AC=BD$. 
\begin{enumerate}
\item Comparer les triangles $ABC$ et $BAD$ puis $BCD$ et $ADC$. Montrer que l'un des côtés $[AB]$ ou $[CD]$ forme des angles égaux avec les deux diagonales.
\item Les diagonales se coupent en $I$. Nature des triangles $IAB$ et $ICD$ ? 
comparer les triangles $IAD$ et $IBC$. 
\end{enumerate}
\item Deux cercles de centre $O$ coupent une droite donnée $(xy)$ le premier en $A$ et $B$ le second en $C$ et $D$. Soit $H$ la projection de $O$ sur $(xy)$.
\begin{enumerate}
\item Nature des triangles $OAB$ et $OCD$ ? Que représente $H$ pour chacun des segments $[AB]$ et $[CD]$ ?
\item Comparer $AC$ et $BD$ puis $AD$ et $BC$ et enfin les triangles $OAC$ et $OBD$ et les triangles $OAD$ et $OBC$.
\end{enumerate}
\item \begin{enumerate}
\item Démontrer que deux triangles qui ont deux côtés respectivement égaux et même périmètre sont égaux.
\item Démontrer que deux triangles isocèles qui ont même base et même périmètre sont égaux.
\end{enumerate}
\item On donne un triangle $ABC$ tel que $AB<AC$. On prend sur le côté $[AC]$ le point $D$ tel que $AD=AB$. Le cercle de centre $A$ et de rayon $AC$ et le cercle de centre $D$ et de rayon $BC$ se coupent en $E$ et $F$. \begin{enumerate}
\item Montrer que l'un de ces points, $E$, par exemple, est sur la droite $(AB)$. 
\item Comment sont placés $E$ et $F$ par rapport à $[AC]$ ? Que représente $(AC)$ 
pour les angles $\widehat{BAF}$, $\widehat{ECF}$ et $\widehat{EDF}$ ? 
\end{enumerate}
\item On considère trois points $A$, $B$, $C$ dans cet ordre sur un cercle de centre $O$. La médiatrice de $[BC]$ coupe $[AC]$ en $M$. \begin{enumerate}
\item Comparer les triangles $OMB$ et $OMC$. Conséquence ? 
\item Démontrer que les angles $\widehat{OAM}$ et $\widehat{OBM}$ sont égaux. 
Que représente la droite $(MO)$ pour l'angle $\widehat{M}$ du triangle $MAB$ ?
\end{enumerate}
\item Démontrer que lorsque deux triangles $AOB$ et $OA'B'$ sont superposables par 
retournement, les médiatrices de $[AA']$ et $[BB']$ sont confondues. 
\item On considère deux segments égaux $[AB]$ et $[A'B']$. Construire le point $Q$ commun aux médiatrices de $[AA']$ et $[BB']$ et comparer les triangles $AOB$ et $OA'B'$. Montrer que ces deux triangles se superposent par glissement et comparer les angles $AOA'$ et $BOB'$. 
\end{enumerate}