
 \chapter{Problèmes de révision}
 \begin{enumerate}
 \item En vendant son blé 27 F le quintal, un paysan peut acheter une maison et il reste 675 F. En vendant le quintal de blé 24,75 F, il achète la maison et il lui reste 78,75 F. Quel est le prix de la maison ? 
 \item Deux ouvriers travaillent dans le même atelier. Le premier gagne 0,90 F par jour de plus que le second : il a travaillé 25 jours et le 
 second 23 jours. Le premier a gagné 47,70 F de plus que le second. Quel est le salaire journalier de chacun ? 
 \item On a, pour 60,20 F, acheté 4 kg de café, 5 kg de sucre, et 3 kg de chocolat. Trouver le prix au kilogramme des trois denrées, sachant qu'un kilogramme de chocolat coûte 1,20 F de moins qu'un kilogramme de café et 6,20 F de plus qu'un kilogramme de sucre. 
 \item Un cycliste roule à 32 km à l'heure et part 2 minutes avant un second cycliste lancé à sa poursuite et dont la vitesse est 36 km à l'heure. Au bout de combien de temps le second cycliste rejoindra-t-il 
 le premier ? 
 \item Un père a trois enfants âgés respectivement de 12, 10 et 8 ans. Le père a 36 ans. Dans combien d'années l'âge du père sera-t-il égal à
 la somme des âges de ses trois enfants ?
 \item Un marchand achète du vin qu'il revend avec un bénéfice égal aux $\frac{25}{100}$ du prix de vente. En gagnant 120 F de moins, il réaliserait un bénéfice égal au $\frac15$ du prix d'achat. Quel est
 son bénéfice ? 
 \item Une somme de 1~420 F est composée de 34 billets, les uns de 50 F, les autres de 10 F. Quel est le nombre de billets de chaque 
 sorte ? 
 \item La distance de deux villes est de 840 km. Une automobile parcourt cette distance en 13 h. Une partie du trajet est faite à la vitesse moyenne de 60 km à l'heure, l'autre à la vitesse moyenne de 80 km à l'heure. Quelles sont les deux parties de ce trajet ?
 \item  On achète une première fois 4 kg de café et 3 kg de sucre pour 33,25 F et une seconde fois 3 kg de café et 2 kg de sucre pour 24,70 F. Quel est le prix du kilogramme de café et celui du kilogramme de sucre ? 
 \item Deux cyclistes roulent sur une piste circulaire de 480 m de tour. Quand ils roulent dans le même sens, le premier dépasse le 
 second toutes les 3 minutes. Quand ils roulent en sens contraire, ils se croisent à intervalles réguliers de 24 secondes. Trouver la 
 vitesse de chaque cycliste. 
 \item Un train a mis 36 secondes à passer devant un observateur immobile. Sa longueur est 300 m. Quelle est sa vitesse ? Un second 
 train met 24 secondes pour croiser le premier et passe devant l'observateur immobile en 18 secondes. Trouver la longueur et la 
 vitesse de ce second train. 
 \item Un capital est placé à 6\% pendant 18 mois. S'il était placé à 5\% pendant 2 ans, les intérêts augmenteraient de 2~500 F. Quel 
 est ce capital ?
 \item Un cycliste et un piéton partent en même temps et dans le même sens de deux points A et B distants de 36 km. La vitesse du cycliste vaut 5 fois celle du piéton. À quelle distance de A le cycliste attendra-t-il le piéton ? À quelle distance de A était le cycliste lorsqu'il avait sur le piéton un retard de 10 km ?  
 \item Deux capitaux dont l'un est les $\frac34$ de l'autre sont placés pendant 18 mois au même taux 5 \%. La somme totale ainsi obtenue (capitaux et intérêts réunis) est 451~500 F. Quels sont ces deux capitaux ? 
 \item 15 litres de lait coupé d'au pèsent 15,150 kg. La densité du lait pur étant 1,03, combien ce lait contient-il de litres d'eau ? 
 \item Deux cyclistes partent en même temps de deux villes A et B distantes de 100 km et vont à la rencontre l'un de l'autre. Ils 
 se rencontrent au bout de 2 h. Si le cycliste qui part de A était parti 20 minutes avant l'autre, la rencontre aurait eu lieu
 $\frac{46}{25}$ d'heure après le départ du second cycliste. Trouver la vitesse de chacun d'eux. 
 \item Un bassin contient de l'eau jusqu'au $\frac6{11}$ de sa hauteur. On y verse encore de l'eau jusqu'à ce que le niveau atteigne les $\frac78$ de la hauteur. Calculer la hauteur du bassin sachant que le niveau s'est élevé de 58 cm.
 \item Un terrain a été partagé en trois parties : les $\frac25$ ont été plantés en vigne, le $\frac13$ a été ensemencé en blé et le reste en luzerne. Sachant qu'il y a 18 ares de différence entre la vigne et la luzerne, trouver la surface totale du terrain et
 la surface de chacune des parties. 
 \item Un travail a été exécuté par trois ouvrières. La première en fait les $\frac4{13}$, la deuxième les $\frac56$ de ce qu'a fait la 
 première, et la troisième fait le reste. Sachant que cette dernière a touché 75 F de moins que les deux autres réunies, on demande de calculer le prix du travail.
 \item Un champ est partagé en trois parties : la première est égale aux $\frac38$ du total, la deuxième aux $\frac9{11}$ de la première. Sachant que la dernière partie surpasse de 17 centiares la seconde, trouver la surface de chacune de ces trois parties.
 \item Une propriété comprend des bois, de la vigne et des prairies. La surface de la vigne est égale aux $\frac34$ de celle des bois, et celle des prairies est les $\frac6{13}$ du total. Trouver la surface des bois, celle de la vigne et celle des prairies, sachant que la surface des prairies dépasse de 3,14 ha celle des bois. 
 \item Une personne dépense le $\frac15$ de son argent dans un magasin, les $\frac37$ du reste dans un autre. Dans un troisième elle voudrait bien acheter $22$ m de toile à 30 F le mètre, mais il lui manque 52 F. Quelle somme avait-elle emportée ?
 \item Un marchand a vendu à un premier client $\frac15$ d'une pièce d'étoffe, puis à un deuxième $\frac13$ du reste, et à un troisième le quart du nouveau reste. Il lui reste 36 m d'étoffe. Quelle était la longueur de la pièce initiale ? 
 \item Trois enfants se partagent des fraises de la manière suivante. Le premier en prend le $\frac13$, le deuxième le $\frac13$ du reste, et le troisième le $\frac13$ du nouveau reste. Le reste final est enfin partagé également entre eux et chacun reçoit alors 40 fraises. Trouver le nombre de fraises total. 
 \item Un marchand vend une pièce d'étoffe. La première fois, il en vend les $\frac27$ plus 3 mètres à 32 F le mètre ; la seconde fois, il en vend les $\frac25$ moins 8 mètres à 30 F le mètre. Il reçoit ainsi 1~336 F. Quelle était la longueur de la pièce et combien de 
 mètres lui reste-t-il ? 
 \item Dans un théâtre il y a $\frac16$ des places à 4 F, $\frac14$ à 3 F, 450 places à 2,40 F et le reste à 1,60 F. La recette maximum
 possible est 3~180 F. La recette maximum possible est 3~180F. Trouver le nombre total de places. 
 \item Un marchand vend un lot de chemises. Il en vend $\frac14$ plus 2 avec un bénéfice de 7 F par chemise, puis les $\frac25$ moins 3 avec un bénéfice de 5 F, et le reste avec un bénéfice de 6 F. Son bénéfice total est de 356 F. Trouver le nombre de chemises vendues. 
 \item Dans une classe, le $\frac13$ des élèves est âgé de 11 ans, la moitié plus 3 est âgée de 12 ans et le reste est âgé de 13 ans. Sachant que les élèves ont à eux tous 352 ans, trouver le nombre des élèves de la classe. 
 \item Une certaine somme est partagée entre 3 personnes. La première en reçoit le tiers, la seconde les $\frac49$ moins 1~360 F et la troisième les $\frac27$ moins 2~120 F. Trouver la somme partagée et la part de chacune. 
 \item Partager un somme de 41~450 F entre 3 personnes de façon que la première reçoive 2~500 F de plus que la deuxième et 1~350 F de moins que la troisième. 
 \item Deux cyclistes partent de deux villes distantes de 72 km et se dirigent l'un vers l'autre. Le premier fait 24 km à l'heure et le
 deuxième 30 km à l'heure. Quelle sera la distance parcourue par chacun d'eux au moment de la rencontre ? 
 \item Un express part de Paris à 8 h 30 et se dirige vers Le Havre à 80 km à l'heure. À 8 h 48 un rapide part du Havre pour Paris et marche à 90 km à l'heure. Sachant que la distance Paris-Le Havre est égale à 228 km, trouver à quelle heure et à quelle distance de Paris aura lieu la rencontre. 
 \item Une personne veut consacrer 3 heures à une promenade. Elle part en automobile à la vitesse de 70 km à l'heure et revient à 
 pied à 5 km à l'heure. À quelle distance du point de départ devra-t-elle descendre de l'automobile ? 
 \item Un cycliste roule pendant 1 h 40 min, puis prend pendant 20 minutes un train dont la vitesse est les $\frac53$ de la sienne. Il a parcouru au total 60 km. Quelle est la vitesse du cycliste ? 
 \item Un premier cycliste part à 9 h du matin et fait 24 km à l'heure. Un second cycliste part à sa poursuite à 9 h 48 et fait d'abord 36 km à l'heure. Mais il s'arrête 9 minutes et ne fait ensuite que 30 km à l'heure. Il rejoint le premier cycliste à 12 h 15. Pendant combien de temps le second cycliste a-t-il roulé à 36 km à l'heure ?
 \item Un train part d'une ville A à 7 h. Il arrive en B à 11 h 30. Il fait les $\frac35$ du trajet à une vitesse de 84 km à l'heure. Dans la seconde partie du trajet, sa vitesse est réduite à 70 km à l'heure. Trouver la distance entre A et B. 
 \item Un cycliste met 4 h 30 pour faire le trajet aller et retour d'une ville A à une ville B distante de 60 km. Il fait 30 km à l'heure en terrain plat, 36 km à l'heure en descente et 20 km à l'heure en montée. Trouver la longueur du terrain plat entre A et B. 
 Sachant qu'il met 12 minutes de plus à l'aller qu'au retour, trouver la longueur des montées et descentes de A vers B. 
 \item Un piéton marche pendant 3 h 40. Il monte alors dans une automobile qui a une vitesse égale aux $\frac{40}3$ de la sienne et qui 
 le dépose au bout de 10 minutes à 26,5 km de son point de départ initial. Trouver la vitesse du piéton. 
 \item Deux capitaux égaux, placés le premier à 4,5\% pendant 16 mois, le second à 4\% pendant 18 mois ont rapporté 1~800 F d'intérêt
 total. Trouver leur valeur commune initiale. 
 \item Deux sommes égales sont placées l'une à 4\%, l'autre à 5\% pendant 2 ans et 8 mois. La seconde a rapporté 800 F de plus que la première. Quelle est le montant de chacune des sommes placées ? 
 \item On place le $\frac13$ d'un capital à 3\%, les $\frac25$ à 3,5\% et le reste à 4\%. Au bout de 15 mois, on a touché 2~600 F d'intérêt. Trouver la valeur de ce capital. 
 \item Un capital est partagé en trois parts. La première est les $\frac45$ de la deuxième, et les $\frac23$ de la troisième. La première part est placée à 4 \%, la deuxième à 5\%, la troisième à 5,5\%. L'intérêt annuel est de 11~100 F. Quel est ce capital ?
 \item Une personne place les $\frac23$ de son capital à 3,5 \%, les $\frac25$ du reste à 4\% et le reste à 4,5\%. Ce dernier placement lui est remboursé au bout de huit mois et reste 4 mois improductif. Le revenu total au bout de l'année est 78 F. Trouver le capital.
 \item Un certain capital est placé à 6\% pendant 10 mois. Un autre égal aux $\frac23$ du précédent est placé à 5\% pendant huit mois. Trouver les deux capitaux, sachant que la différence des intérêts est de 2~500 F. 
 \item Une personne avait placé les $\frac56$ de son capital à 4 \% et le reste à 5 \%. Elle prélève 60~000 F sur ce capital et place le reste à 4,5 \%. Son revenu diminue de 1~200 F par an. Quel était son capital primitif ? 
 \item Une personne possède 45~00 F. Elle emploie une partie de ce capital à l'achat d'une propriété. Elle place les $\frac23$ du reste à 4\% et le tiers restant à 5\%. Ces deux placements lui assurent un revenu de 780 F. Trouver le prix de la propriété.
 \item Un propriétaire veut construire une usine. Il emploie les $\frac3{10}$ de sa fortune à l'achat du terrain qui lui revient à 7,20 F le mètre carré. Il consacre les $\frac47$ du reste à la construction des bâtiments. Il place alors les $\frac23$ de l'argent disponible à 4\% et le reste à 4,5\%, ce qui lui procure un revenu annuel de 450 F. On demande la fortune totale du propriétaire et la surface du terrain acheté. 
 \item Une personne a placé une certaine somme à 6 \% à intérêts simples. Au bout de 5 ans et 4 mois, elle retire, capital et intérêts réunis, une somme de 409~200 F. Quelle était la somme placée ? 
 \item Une personne place son argent à intérêts simples ; les $\frac25$ sont placés à 4 \%, le $\frac13$ à 5 \%, et le reste à 4,5 \%. Elle retire son argent au bout de 2 ans et 7 mois et achète une propriété de 429 mètres carré à raison de 4,30 F le mètre carré. Il lui reste alors 163 F. Trouver son avoir primitif. 
 \item On place les $\frac25$ d'un capital de 150~000 F à un certain taux et le reste à un taux supérieur de 1 \%. L'intérêt annuel étant 6~150 F, trouver les taux des deux placements. 
 \item Un capital de 924~000 F est placé à un certain taux pendant 21 mois. Un autre capital de 660~000 F est placé à un taux inférieur de 1 \% au précédent, pendant 15 mois. La différence des intérêts étant 47~850 F, trouver les taux des deux placements. 
 \item On a placé au même taux 240~000 F pendant 2 ans et 250~000 F pendant 2 ans et demi. Le second capital a rapporté 5~800 F d'intérêts de plus que le premier. Calculer le taux de ces deux placements.
 \item Une personne avait placé deux capitaux, l'un de 320 F, l'autre de 480 F au même taux de 3 \%. Elle retire capitaux et intérêts réunis 834 F. Sachant que le premier capital de 320 F est resté placé 20 mois, on demande la durée du second placement.
 \item Une personne place 210 F à 3\% et 18 mois plus tard 560 F à 4,5\%. Au bout de combien de temps les deux placements auront-ils rapportés des intérêts égaux ? 
 \item Une personne place 240 F à 4\%. 18 mois plus tard elle place une seconde somme d'argent à 6\%. Calculer le montant de cette somme placée sachant que 6 mois après les deux placements ont rapporté le même intérêt. 
 \item Les $\frac35$ d'un capital ont été placés à 4\% et le reste à 5 \%. Au bout de deux ans et demi, les intérêts se sont élevés à 77 F. Calculer ce capital. 
 \item Une personne avait placé les $\frac35$ de son capital à 3 \% et le reste à 4 \% ; elle retire les deux parts, prélève 105~000 F et replace le reste à 5\% ; son revenu annuel se trouve ainsi augmenté de 6~830 F. Quel était le capital primitivement placé ?
 \item Une personne place les $\frac37$ de son capital à 3\%, et le reste est placé à un taux différent mais rapporte annuellement le même intérêt que le premier placement. Calculer ce taux. Calculer ensuite le capital, sachant que la personne perçoit annuellement 12 F d'intérêt de plus que si elle avait placé tout son capital à 5 \%. 
 \item Calculer le montant d'un capital placé à 4\% pendant 219 jours, sachant que si l'on comptait l'année de 365 jours dans le calcul de l'intérêt, on trouverait 200 F de moins que si l'on compte l'année de 360 jours que l'on utilise habituellement.
 \item Calculer les montants de deux capitaux ayant pour différence 200 F sachant que le moins élevé, placé pendant un an à 4,5\% et l'autre placé pendant 2 ans à 3,75 \% ont rapporté ensemble 63 F. 
 \item Un cycliste effectue une promenade de 3 heures. Il parcourt la moitié du trajet à 18 km à l'heure, le tiers à 20 km à l'heure, et le reste à 15 km à l'heure. Trouver la distance parcourue par le cycliste.
 \item Un express part à 9 h 15 d'une ville A et arrive à 16 h 25 en B.  Il a parcouru la moitié de son parcours à la vitesse de 75 km à l'heure, les $\frac25$ du trajet à la vitesse de 72 km à l'heure, et le reste à la vitesse de 80 km à l'heure. La durée totale des arrêts a été de 42 minutes. Calculer la distance AB. 
 \item Un alliage d'argent et de cuivre au titre de 0,720 pèse 500 grammes. Combien faudrait-il ajouter d'argent pour élever le titre à 0,800 ?
 \item On a un lingot d'or de 1~200 grammes au titre de $\frac{11}{12}$. Quelle quantité de cuivre faudrait-il y rajouter pour avoir un alliage au titre de 0,880 ?
 \item  Un alliage de cuivre et d'argent pesant 4,75 kg est constitué par des volumes égaux de ces deux métaux. Trouver le poids de cuivre et le poids d'argent que contient cet alliage en admettant que leurs poids spécifiques sont respectivement 9 et 10. 
 \item On a fondu ensemble deux lingots d'argent de titres différents contenant respectivement 250 grammes et 175 grammes d'argent fin et l'on a obtenu un alliage ayant pour titre 0,85. Quels étaient les titres de ces lingots sachant que le second a apporté 2 fois plus de cuivre que l'autre ? 
 \item Un épicier mélange du café à 8,40 F le kg et à 9,60 F le kg; Combien doit-il en prendre de chaque sorte sachant qu'il veut obtenir 30 kg de mélange revenant à 8,80 F le kg ? 
 \item Un négociant mélange trois sortes de café qu'il veut vendre 9,36 F le kg. Ce mélange comprend 64 kg de café, acheté 8,64 F le kg, et deux autres sortes achetées 7,56 F et 6,30 F le kg. Quel poids faut-il prendre de ces deux dernières sortes, sachant que le poids du café à 7,56 F doit être les $\frac34$ du poids du café à 6,30 F et que le négociant veut gagner 20\% sur le prix de revient. 
 \item Deux cyclistes éloignés de 66 km vont à la rencontre l'un de l'autre. Le premier qui fait 12 km à l'heure part une heure plus tôt que le second qui fait 15 km à l'heure. On demande au bout de combien 
 de temps ils se rencontreront et quels trajets ils auront parcourus. 
 \item Un piéton marchant à 5km à l'heure et un cycliste roulant à 15 km à l'heure partent d'une ville A en même temps et vont dans la même direction. Arrivé dans une ville B située à 12 km de A, le cycliste y reste 20 minutes puis revient en A. On demande à quelle distance de A il croisera le piéton. 
 \item Un cycliste part de Paris à 7 h et va à Fontainebleau où il 
 reste deux heures puis revient à Paris où il arrive à 18 heures. Calculer la distance Paris-Fontainebleau sachant qu'il a fait 15 km à l'heure à l'aller et 12 km à l'heure au retour. 
 \item Un cycliste monte une côte à la vitesse de 8 km à l'heure et la descend à 15 km à l'heure. Trouver la longueur de cette côte sachant qu'il met 35 minutes de plus pour monter que pour descendre. 
 \item La route allant d'une ville A à une ville B, distante de 60 km, comprend d'abord une montée, puis une partie horizontale de 20 km et enfin une descente. Un cycliste dont la vitesse en montée est de 8 km à l'heure, et en descente de 15 km à l'heure va de A à B et revient en A. On demande combien il y a de km de montée, et combien de km de descente de A vers B, sachant que le cycliste a mis 1 h 10 de plus pour aller que pour revenir. 
 \item Deux bassins contiennent déjà l'un 210 litres d'eau, et l'autre 100 litres. Pour les remplir, on ouvre 2 robinets, qui versent l'un 7 litres par minute dans le premier bassin, l'autre 8 litres par minute dans le second. Au bout de combien de temps le contenu du second bassin sera-t-il les $\frac47$ du contenu du premier ? 
 \item Un paysan doit labourer deux champs de même superficie. Il laboure le premier à raison de 8 ares par heure, et le second à raison de 10 ares par heure. Calculer la superficie de chaque champ sachant 
 que pour labourer le second, il met 8 heures de moins que pour labourer le premier.
  
 \end{enumerate}
 
 