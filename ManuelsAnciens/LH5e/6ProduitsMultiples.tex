
 \chapter{Produits de plusieurs facteurs}
\begin{enumerate}
\item Que devient le produit de deux nombres lorsqu'on
multiplie l'un des facteurs par 2 ; et lorsqu'on le multiplie plus généralement par un nombre $x$ ? 
\item Que devient le produit de deux nombres lorsqu'on
multiplie les deux facteurs par 2 ; et lorsqu'on les 
multiplie plus généralement par un nombre $x$ ?
\item Que devient la surface d'un carré lorsqu'on 
double son côté ? Même question pour la surface d'un
disque lorsqu'on double son rayon. 
\item Que devient le volume d'un cube quand on double
son arête ? Que devient le volume d'une sphère lorsqu'on double son rayon ? Que devient le volume d'un cylindre quand on double le rayon du disque de base et qu'on triple la hauteur ? 
\item Effectuer les produits suivants : 
\[ 712 \times 43 \times 51 \times 19\]
\[725 \times 41 \times 25 \times 725\]
\item Effectuer les produits suivants : 
\[ (4 \times 7 \times 12) \times (7 \times 13) \times 9\]
\[ 13 \times (43 \times 17)\]
\[ (25 \times 12 \times 13) \times 4\]
\item Réduire les opérations suivantes : 
\begin{enumerate}
\item $7x \times 5y \times 3z$
\item $4(3x + 2y)$
\item $7(5x - 2y)$
\item $7(2x + 5y) + 12(3x + y) + 4(x + 5y)$
\item $2a(3b - c) + 3b(c - 2a) + c(2a -3b)$
\item $5(3a + 2) + 3(5a -2) - 2(a + 2) - 3(a - 1)$
\end{enumerate}
\item Effectuer les opérations suivantes : 
\begin{enumerate}
\item $10^5\times 10^3$
\item $10^2\times 10^3 \times 10^4$
\item $(5^4)^2$
\item $(7^4 \times 7^2) + (5^4 \times 5^2) + 
(3^4 \times 3)$
\item $a^3(a^2 + 3) + 3a^2(a^3 + 5) + 2a^2( 2a^2 - 9)$
\item $5a^4(a^2 + 4) - 2a^2(2a^4 + 1) - a^3(a^3 - 7)$
\item $ab(a - b) + a(a^2 + b^2) - a^2$
\end{enumerate}
\item \begin{enumerate}
\item Calculer la somme des 7 premiers nombres impairs.
Généraliser ce résultat. 
\item En déduire que tout nombre impair est la 
différence des carrés de deux nombres consécutifs.
Décomposer ainsi 37.
\end{enumerate}
\item On écrit dans un tableau triangulaire la suite des nombres impairs comme suit : 
 \begin{tabular}{ccc}
1 & & \\
3 & 5 & \\
7 & 9 & 11  \\
\ldots & \ldots & \ldots
\end{tabular}
\begin{enumerate}
\item Écrire les dix premières lignes de ce tableau.
\item Combien de nombres a-t-on écrit ? Trouver la 
somme de ces nombres (on utilisera l'exercice précédent). 
\item Calculer la somme des nombres inscrits dans 
chaque ligne du tableau et en déduire la somme des 
cubes des dix premiers nombres entiers.
\end{enumerate}
\item Calculer de deux manières la somme 
$a(a - b) + b(a - b)$ et en déduire que $(a + b)(a 
- b) = a^2 - b^2$. \\
Application : La différence des surfaces de deux 
jardins carrés est de 1~152m${}^2$. Calculer les côtés de 
ces deux jardins sachant que leur différence 
est de 16 m. 
\item Un bloc de pierre taillé a 80 cm de longueur,
42 cm de largeur et 35 cm de hauteur. Sachant que 
le poids volumique de la pierre est 2,7, calculer 
le poids de ce bloc de pierre. 
\item Une colonne cylindrique en ciment armé a 0,80 m 
de diamètre et 3,50 m de hauteur. \begin{enumerate}
\item Calculer la surface latérale de cette colonne 
et le prix de la peinture nécessaire pour la recouvrir 
à raison de 2,50 F le m${}^2$. 
\item Calculer le volume de la colonne et son poids 
sachant qu'un dm${}^3$ de ciment armé pèse 2,9 kg. 
\end{enumerate}
\item Une borne en granit comprend une partie enterrée de 50 cm de largeur, 30 cm d'épaisseur et 60 cm de 
profondeur. La partie apparente a une épaisseur de 24 cm. Vue de face elle se compose d'un rectangle de 40
cm de base et 45 cm de hauteur surmonté d'un demi-cercle de 40 cm de diamètre. \begin{enumerate}
\item Calculer la surface extérieure apparente de 
la borne. 
\item Calculer son poids total, sachant que la densité 
du granit est de 2,7. 
\end{enumerate}
\item Un réservoir à mazout qui a la forme d'un cylindre horizontal de 3 m de long et de 1,60 m de 
diamètre a été fabriqué en tôle de 2 mm d'épaisseur.
\begin{enumerate}
\item Calculer le poids de la tôle utilisée sachant que sa densité est 7,8. 
\item Calculer la capacité en litres de ce réservoir et la dépense lorsqu'on en fait le plein avec du 
mazout à 0,25 F le litre. 
\end{enumerate}
\item Un bassin circulaire a 5 m de diamètre et 0,80 m 
de profondeur. On le fait cimenter entièrement à raison de 5 F le m${}^2$, et border à raison de 3 F le
mètre.\begin{enumerate}
\item Calculer la dépense. 
\item Un robinet qui débite 20 litres à la minute 
alimente ce bassin. Combien de temps faudra-t-il pour le remplir jusqu'à 10 cm du bord supérieur ? 
\end{enumerate}
\item Calcul mental : 
\[\begin{matrix}
63 \times 11 \phantom{meow}& 75 \times 11\phantom{meow} & 83 \times 21\phantom{meow} & 62 \times
 110\phantom{meow} \\
 63 \times 19 \phantom{meow}& 75 \times 99 \phantom{meow}& 83 \times 39 \phantom{meow}& 620 \times 190\phantom{meow}\\
 24 \times 15 \phantom{meow}& 17 \times 12 \phantom{meow}& 25 \times 35 \phantom{meow}& 43 \times 55 \phantom{meow}
 \end{matrix}\]
 \item Soit $x$ un nombre entier de 0 à 10. Établir les tableaux de correspondance entre $x$ et les nombres $y$ suivants. Construire ensuite le graphique 
 correspondant. \begin{enumerate}
 \item $y = x^2$
 \item $y = 2x^2$ 
 \item $y = 3x^2$ 
 \item $y = x^3$
 \item $y = 2x^3$
 \item $y = 3x^3$
 \end{enumerate}
 \item En s'inspirant des derniers exercices du 
 chapitre précédent, démontrer : 
 \[ (a + b)^2 = a^2 + 2ab + b^2\]
 \[ (a + b)(a - b) = a^2 - b^2\]
 \[ (a - b)^2 = a^2 - 2ab + b^2\]
\end{enumerate} 
 