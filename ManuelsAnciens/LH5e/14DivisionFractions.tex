
 \chapter{Division d'une fraction par une fraction}
 
 \begin{enumerate}
 \item \[ 3 \div \frac57;\phantom{meowmeow}
 12\div \frac74 ; \phantom{meowmeow}
 \frac{36}{55}\div\frac{9}{22}; \phantom{meowmeow}
 \frac{49}{81}\div \frac{21}{54}.\]
 
  \item \[ 21 \div \frac75;\phantom{meowmeow}
 36\div \frac{54}5 ; \phantom{meowmeow}
 \frac{48}{25}\div\frac{16}{15}; \phantom{meowmeow}
 \frac{15}{32}\div \frac{25}{24}.\]
 
  \item \[ \frac{52}{45} \div \frac{78}{81};\phantom{meowmeow}
 \frac{55}{126}\div\frac{44}{189} ; \phantom{meowmeow}
 \frac{51}{91}\div\frac{85}{156}; \phantom{meowmeow}
 \frac{63}{92}\div \frac{84}{115}.\]
 
 \item \[ \left( \frac{10}{21} +\frac{11}{28} + \frac5{12} \right) \div \frac{18}{35} ;
 \phantom{meowmeow}
 \left( \frac{32}{39} - \frac{21}{52} \right) \div \frac{15}{8}. \]
 
 \item \[ \left( \frac{23}{44} +\frac{34}{77} + \frac{11}{28} \right) \div \frac{38}{49};
 \phantom{meowmeow}
 \left( \frac{19}{28} - \frac{23}{70} \right) \div \frac{14}{45}. \]
 
  \item \[ 
\frac{\frac6{35}+\frac{19}{21}+\frac{16}{15}}{\frac{11}{21}+\frac3{14}+\frac{11}6};
 \phantom{meowmeow}
\frac{\frac8{21}+\left(\frac{11}{15}-\frac4{35}\right)}{\frac{17}{36}-\left(\frac9{20}-\frac{17}{45}\right)}.
 \]
  \item \[ \left(\frac59\right)^{10}\div\left(\frac59\right)^7;
 \phantom{meowmeow}
\left(\frac7{15}\right)^{43}\div\left(\frac7{15}\right)^{47};
 \phantom{meowmeow}
 \left(\frac2{13}\right)^{13}\div \left(\frac2{13}\right)^{23}.
 \]
 \item 
 \[ \left[\left(\frac38\right)^5\times\left(\frac38\right)^4\right]\div 
 \left(\frac38\right)^7 ; 
  \phantom{meowmeow}
  \left[\left(\frac56\right)^2\times\left(\frac56\right)^5\right]\div 
 \left(\frac56\right)^6.
 \]
 \item Simplifier : \[ \frac{a}{b}\div a ; \phantom{meowmeow}
 a\div\frac{a}{b}; \phantom{meowmeow}
 \frac1a\div\frac1b; \phantom{meowmeow}
 \frac{3a}b\div\frac{2a}{3b}.\]
 \item Simplifier : \[ \left(2+\frac1a\right)\div(2a+1); \phantom{meowmeow}
 \left(\frac{a}b + 1\right) \div \left(\frac{a}b-1\right).\]
 \item Simplifier : \[ \frac{a^{12}}{a^7}; \phantom{meowmeow}
 \frac{a^3\times a^5}{a^2}; \phantom{meowmeow}
 \frac{(a+b)^5}{a+b}; \phantom{meowmeow}
 \frac{(a-b)^4}{(a-b)^2} .\]
 \item Simplifier : \[ \frac{5a^3b^2}{15a^2b}; \phantom{meowmeow}
 \frac{108a^5b^3c^2}{36a^2b^2c}; \phantom{meowmeow}
 \frac{105 x y^4 z^3}{49x^2y^4z}.\]
 \item Simplifier : \[ \frac{a^2}{a^5}; \phantom{meowmeow}
 \frac{4a^2b^3}{8a^6b^2}; \phantom{meowmeow} \frac{121a^2b^3c}{77a^4b^4c}.\]
 \item En divisant $\frac47$ par une certaine fraction, le quotient est égal 
 à $\frac3{14}$. Quelle est la fraction diviseur ? 
 \item Les $\frac7{12}$ d'une fraction sont égaux à $\frac8{15}$. Quelle est
 cette fraction ? 
 \item Par quel nombre entier ou fractionnaire faut-il multiplier $\frac57$
 pour obtenir les $\frac34$ de $\frac{11}9$ ? 
 \item Que devient un produit de facteurs lorsqu'on divise l'un des facteurs
 par $\frac49$ ? 
 \item Trouver deux fractions dont la somme est $\frac{390}{187}$ et le 
 quotient $\frac{119}{11}$. 
 \item Trouver deux fractions dont la différence est $\frac{5}{21}$
 et le quotient $\frac{12}7$. 
 \item Un cycliste parcourt la distance $AB$ à la vitesse de $30$ km à l'heure et la distance $BA$ à la vitesse de $20$ km à l'heure. Quelle est sa
 vitesse moyenne pour le trajet aller et retour ? 
 \item Un cycliste se rend de $A$ à $B$ à la vitesse de $27$ km à l'heure.
 Il revient de $B$ à $A$ à la vitesse de $24$ km à l'heure. Trouver la 
 distance $AB$, sachant que la durée totale du voyage aller et retour est
 de 10 h 37 min 30 sec. (Indication : Réduire ce temps en une fraction d'heure.)
 \item Une équipe d'ouvriers ferait un travail en 5 jours et demi ; une seconde équipe le ferait en 4 jours $\frac23$. Combien les deux équipes, travaillant ensemble, mettront-elles de temps pour faire ce travail ? 
 \item Une somme de 21~700 F est partagée entre 2 personnes. La première ayant dépensé les $\frac57$ de sa part et la seconde les $\frac25$ de la sienne, il leur reste la même somme. Trouver les deux parts. 
 \item Les $\frac45$ d'un nombre valent les $\frac59$ d'un autre nombre. 
 Trouver ces deux nombres, sachant que leur différence est $33$.
 \item Un robinet remplit les $\frac27$ d'un bassin en 2h $\frac34$ ; combien
 de temps faut-il laisser couler le robinet pour que le bassin soit plein 
 aux $\frac45$ ? 
 \item Une équipe d'ouvriers ferait un travail en 7 jours, une autre équipe
 le ferait en 9 jours. On fait travailler ensemble les $\frac23$ de la 
 première équipe et la moitié de la seconde. Combien de temps faudra-t-il
 pour effectuer ce travail ? 
 \item Deux sommes égales sont placées, l'une à 5 \%, l'autre à 6 \% pendant 
 42 mois. La seconde rapporte 1~890 F de plus que la première. Quelle est
 la valeur commune de ces deux sommes ? 
 \item La durée de la révolution de la Terre autour du Soleil est 365 jours.
 En combien de temps la Terre effectue-t-elle les $\frac35$ de cette révolution ? 
 \item Les durées des révolutions de la Terre et de Vénus autour du Soleil
 sont respectivement 365 jours et 225 jours. Calculer l'intervalle des 
 temps qui, pour un observateur terrestre, séparent deux passages successifs
 de Vénus devant le Soleil. 
 \item La durée de la révolution de Jupiter autour du Soleil est environ 142 
 mois. Quel est l'intervalle des temps qui séparent deux oppositions successives du Soleil et de Jupiter (instants où le Soleil et Jupiter sont des directions opposées par rapport à la Terre). 
 \item Pour effectuer un parcours de $100$ km, un train à mis 1h 20. Quelle
 est sa vitesse horaire ? 
 \item Une vis avance de $\frac7{10}$ mm en 5 tours. Calculer le nombre de
 tours nécessaire pour avancer de 12 mm $\frac14$. 
 \item Une personne exécute les $\frac{3}{11}$ d'un travail en 3h $\frac58$.
 Quel sera le temps nécessaire pour en exécuter les $\frac8{11}$ ? 
 \item Un pré et une vigne ont une superficie totale de 51 ha. Les $\frac47$ de
 la superficie du pré sont égaux aux $\frac25$ de celle de la vigne. Calculer 
 la superficie du pré et de la vigne. 
 \item Trouver un nombre dont le quotient par $\frac7{11}$ surpasse ce nombre
 de 64. 
 \item En divisant $\frac47$ par une fraction on obtient un quotient qui est
 les $\frac23$ du dividende. Quel est le diviseur ? 
 \item Par quel nombre a-t-on divisé le nombre 36 lorsqu'on l'a augmenté de ses $\frac34$ ?
 \item Par quel nombre multiplie-t-on la fraction $\frac4{11}$ en augmentant
 de 2 son numérateur et en diminuant de 4 son dénominateur ? 
 \item Soient les fractions $\frac{a}b$ et $\frac{c}d$. Montrer qu'on ne change pas le quotient de ces deux fractions en les multipliant ou en les divisant toutes deux par un même nombre entier ou plus généralement par une même fraction. Vérifier sur des exemples. 
 
 \end{enumerate}
 