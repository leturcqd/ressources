
\chapter{Bissectrice d'un angle}
\begin{enumerate}
\item Construire sur une droite donnée $(D)$ un point $M$ équidistant de deux droites
$(xx')$ et $(yy')$.
\item Construire un point $M$ équidistant de deux droites $(xx')$ et $(yy')$ connaissant en outre sa distance à un point donné $O$. 
\item Construire un point $M$ équidistant de deux points donnés $A$ et $B$ et 
équidistant de deux droites données $(xx')$ et $(yy')$.
\item Un quadrilatère convexe $ABCD$ a deux angles opposés $\widehat{B}$ et $\widehat{D}$ droits et les côtés $[AB]$ et $[AD]$ égaux. Montrer que $CD=CB$
et que $(AC)$ est bissectrice des angles $\widehat{A}$ et $\widehat{C}$. 
\item Construire un triangle $ABC$ dont les bissectrices intérieures
se coupent en $I$ sachant que $\widehat{BIC}=124^o$, $IB=3$ cm et $IC = 4$ cm.
\item Démontrer qu'il y a quatre points $I$, $J$, $K$ et $L$ équidistants des trois droites $(BC)$, $(CA)$ et $(AB)$. Que représentent $(IJ)$, $(IK)$ et $(IL)$ pour
le triangle $JKL$ ?
\item Les angles $\widehat{ABC}$ et $\widehat{ADC}$ du quadrilatère convexe $ABCD$ 
sont supplémentaires et $BC=CD$. Le point $C$ se projette en $H$ sur $[AB]$ et en 
$K$ sur $[AD]$. \begin{enumerate}
\item Comparer les triangles $BCH$ et $CDK$ puis les longueurs $CH$ et $CK$.
\item Montrer que $(AC)$ est la bissectrice intérieure de l'angle $\widehat{BAD}$.
\end{enumerate}
\item Dans le quadrilatère convexe $ABCD$ on a $DB=DC$ et $\widehat{DBA}=\widehat{DCA}$. Le point $D$ se projette en $H$ sur $[AB]$ et en $K$ sur $[AC]$.
\begin{enumerate}
\item Comparer les triangles $BDH$ et $DCK$ puis les longueurs $DH$ et $DK$. 
\item Déterminer la somme des angles $\widehat{DAB}$ et $\widehat{DAC}$.
\end{enumerate}
\item Les bissectrices intérieures des angles $\widehat{B}$ et $\widehat{C}$ d'un triangle $ABC$ se coupent en $I$. On mène les perpendiculaires en $B$ à $(IB)$ et 
en $C$ à $(IC)$. Elles se coupent en $J$. \begin{enumerate}
\item Que représentent $(BJ)$ et $(CJ)$ pour les angles $\widehat{B}$ et $\widehat{C}$ du triangle ?
\item Montrer que les points $A$, $I$ et $J$ sont alignés.
\end{enumerate}
\item Un losange est un quadrilatère dont les quatre côtés sont égaux.\begin{enumerate}
\item Démontrer que les diagonales $[AC]$ et $[BD]$ sont médiatrices l'une de l'autre et bissectrices des angles intérieurs du losange. 
\item Montrer que leur point commun $O$ est équidistant des quatre côtés. 
\end{enumerate}
\item Soient $D$, $E$, $F$ les projections, sur les côtés $[BC]$, $[CA]$ et $[AB]$,
du point de concours $I$ des bissectrices intérieures du triangle $ABC$.
\begin{enumerate}
\item Montrer que $BD=BF$, $CE=CD$ et $AF=AE$. Que représente la somme $AE+BC$ pour le périmètre du triangle ? 
\item $BC= 48$ mm, $CA=40$ mm, et $AB=28$ mm. Calculer $AF$, $BD$ et $CE$.
\end{enumerate}
\item Dans un triangle $ABC$ tel que $AB>AC$ la médiatrice de $[BC]$ coupe en $M$
la bissectrice intérieure de l'angle $\widehat{A}$. Le point $M$ se projette en 
$H$ sur $[AB]$ et en $K$ sur $[AC]$. 
\begin{enumerate}
\item Comparer $AH$ et $AK$ puis $BH$ et $CK$. 
\item Démontrer que $AH=\frac12(AB+AC)$ et $BH=\frac12(AB-AC)$.
\end{enumerate}
\item Dans le même triangle $ABC$ qu'à l'exercice précédent, la médiatrice de $[BC]$ 
coupe en $M'$ la bissectrice extérieure de l'angle $\widehat{A}$. Le point $M'$ se projette en $H'$ sur $(AB)$ et en $K'$ sur $(AC)$. \begin{enumerate}
\item Comparer $AH'$ et $AK'$ puis $BH'$ et $CK'$.
\item Démontrer que $AH'=\frac12(AB-AC)$ et $BH'=\frac12(AB+AC)$ puis que $HH'=AC$. 
\end{enumerate}
\item On prolonge la base $[BC]$ du triangle $ABC$ de deux longueurs $BM=BA$ et $CN=CA$. Les bissectrices extérieures des angles $\widehat{B}$ et $\widehat{C}$ de ce triangle se coupent en $J$. \begin{enumerate}
\item Montrer que $M$ et $N$ sont les symétriques de $A$ par rapport à $(BJ)$ et $(CJ)$. Comparer $JA$, $JN$ et $JM$. Nature du triangle $JMN$. 
\item Comparer les angles $\widehat{BAJ}$ et $\widehat{CAJ}$. Démontrer ainsi que
\emph{dans un triangle les bissectrices extérieures de deux angles et la bissectrice
intérieure du troisième sont concourantes.}
\end{enumerate}
\item Un quadrilatère convexe $BICJ$ a ses angles opposés $\widehat{B}$ et $\widehat{C}$ droits. On construit les symétriques $(x'x)$ et $(y'y)$ de la droite 
$(BC)$ par rapport à chacune des droites $(IB)$ et $(IC)$. 
\begin{enumerate}
\item Montrer que $I$ et $J$ sont équidistants des trois droites $(BC)$, $(x'x)$ et 
$(y'y)$.
\item Démontrer que les trois droites $(IJ)$, $(x'x)$ et $(y'y)$ sont en générales 
concourantes en un même point $A$.
\end{enumerate}
\item On considère un triangle $ABC$ tel que $AB<AC$. La médiatrice de $[BC]$ coupe en $D$ le côté $[AC]$ et en $I$ et $J$ respectivement la bissectrice intérieure et la
bissectrice extérieure de l'angle $\widehat{BAC}$. \begin{enumerate}
\item Que représente $(IJ)$ pour l'angle $\widehat{ADB}$ ? Montrer que les points $I$ et $J$ sont équidistants des cotés du triangle $aBD$ et que $(BI)$ et $(BJ)$ sont 
les bissectrices extérieure et intérieure de l'angle $\widehat{ABD}$. 
\item Quelle est la valeur des angles $\widehat{IBJ}$ et $\widehat{ICJ}$ ? Démontrer que les angles $\widehat{ABJ}$ et $\widehat{ACJ}$ sont égaux et que les angles $\widehat{ABI}$ et $\widehat{ACI}$ sont supplémentaires.
\end{enumerate}
\end{enumerate}