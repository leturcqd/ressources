
\chapter{Segments de droite}
\begin{enumerate}
\item Construire quatre points $A$, $B$, $C$ et $D$ situés dans cet ordre sur une même droite $(xy)$ sachant que $AB=CD=3\cm$ et $BC = 5\cm$. Comparer ensuite les segments $[AC]$ et $[BD]$. 
\item Quatre points $A$, $B$, $C$, $D$ sont situés dans cet ordre sur une droite $(xy)$. Construite ces quatre points sachant que $AC=BD = 9 \cm$ et $BC = 7\cm$. Comparer les segments $[AB]$ et $[CD]$. Montrer que $[AD]$ et $[BC]$ ont milieu $O$. 
\item Quatre points $A$, $B$, $C$, $D$ sont situés dans cet ordre sur une droite $(xy)$. Construire ces quatre points sachant que $AD= 14\cm$, $BC = 10\cm$ et que les segments $[AD]$ et $[BC]$ ont même milieu $O$. Démontrer les égalités $AB=CD$ et $AC = BD$. 
\item On porte bout à bout des segments avec $AB = 10\cm$, $BC=3\cm$, $CD=7\cm$. Écrire les inégalités que vérifient $AB$ et $BC$, $AB$ et $CD$, $CD$ et $BC$. Quelle est la mesure de $[AD]$ en prenant $AB$ pour unité, puis $BC$ pour unité, puis $CD$ pour unité ? 
\item Trois points $O$, $A$, $B$ sont situés dans cet ordre sur une 
droite $(xy)$ et $M$ est le milieu de $[AB]$. \begin{enumerate}
\item On donne $OA = 4\cm$ et $OB = 10\cm$. Quelle est la longueur $OM$ ? 
\item On donne $OA = a$, et $OB = b$. Démontrer les égalités : 
\[ OM = OA + AM; \phantom{meow} OM = OB - AM;\phantom{meow} OM = \frac{a+b}2.\]
\end{enumerate}
\item Trois points $A$, $O$, $B$ sont situés dans cet ordre sur une droite $(xy)$ et 
$M$ désigne le milieu de $[AB]$. \begin{enumerate}
\item On donne $OA=5\cm$ et $OB = 11\cm$. Quelle est la longueur de $[OM]$ ? 
\item On donne $OA = a$, et $OB = b\ (b>a)$. Démontrer les égalités : 
\[ OM = AM - OA ; \phantom{meow} OM = OB-AM; \phantom{meow} OM = \frac{b-a}2.\]
\end{enumerate}
\item Trois points $B$, $A$, $C$ sont situés dans cet ordre sur une droite $(xy)$. On désigne par $I$ et $J$ les milieux respectifs de $[AB]$ et $[AC]$. 
\begin{enumerate}
\item On donne $AB= 7\cm$ et $AC = 5\cm$. Quelle est la longueur de $[IJ]$ ? Comparer la longueur trouvée à celle de $[BC]$ ? 
\item Calculer les longueurs $BC$ et $IJ$ connaissant $AB = a$ et $AC = b$. 
\end{enumerate}
\item Trois points $A$, $B$, $C$ sont situés dans cet ordre sur une droite $(xy)$. On désigne par $I$ et $J$ les milieux respectifs de $[AB]$ et $[AC]$. \begin{enumerate}
\item On donne $AB=9\cm$ et $AC=13\cm$. Quelle est la longueur de $[IJ]$ ? Comparer la longueur trouvée à celle de $[BC]$. 
\item Calculer les longueurs $BC$ et $IJ$ connaissant $AB = a$ et $AC = b$. 
\end{enumerate}

\end{enumerate} 