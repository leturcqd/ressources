\documentclass[12 pt]{report}
	\author{ }
	\usepackage[frenchb]{babel}
	\usepackage[utf8]{inputenc}  
	\usepackage[T1]{fontenc}
	\usepackage{amssymb}
	\usepackage[mathscr]{euscript}
	\usepackage{stmaryrd}
	\usepackage{amsmath}
	\usepackage{tikz}
	\usepackage[all,cmtip]{xy}
	\usepackage{amsthm}
	\usepackage{varioref}
\DeclareSymbolFont{yhlargesymbols}{OMX}{yhex}{m}{n}
\DeclareMathAccent{\arc}{\mathord}{yhlargesymbols}{"F3}
	\usepackage{geometry}
	\geometry{a4paper, head = 15pt, left = 1 cm, right = 1cm}
	\usepackage{lmodern}
	\usepackage{hyperref}
	\usepackage{array} 
	 \usepackage{float} 
	\theoremstyle{plain} 
	\newcounter{n}
	\setcounter{n}{1} 
	\newcommand{\cm}{\textbf{ cm}}
	\newcommand\ds\displaystyle
	\renewcommand{\it}{\item[$\mathbf{\then}.$]\stepcounter{n} }
	\newenvironment{calcul}{\item[$\mathbf{\then}.$] \stepcounter{n}\hfil $\displaystyle  }{.$\hfil   }
	\newcommand\syst[4]{
 \begin{calcul} \left\{ \begin{array}{l} #1 = #2, \\ #3 = #4.\end{array}\right\end{calcul}}
	\newcommand\intercalc{\end{calcul}\begin{calcul}}
	
	\title{\Large Exercices tirés du manuel de 3e de Pierre Chenevier, éd. Hachette\footnote{Vous pouvez me signaler les erreurs par courriel : \href{mailto:leturcq.d@orange.fr}{lien de contact.} 
	}
	}
	\date{1938}
	\begin{document}
	\maketitle 
	
	
	
 \part{Nombres entiers}
 \chapter{Numération}
 
 \begin{enumerate}
 \item Combien faut-il de mots différents pour nommer tous les nombres jusqu'à un million ? 
 \item On écrit les $237$ premiers nombres. Combien, au total, a-t-on écrit de chiffres ? Même question pour les nombres entre $94$ et $237$. 
 \item On écrit tous les nombres de deux chiffres. Combien en écrit-on ? 
 Combien de chiffres écrit-on au total ? Même question pour les nombres de trois chiffres. 
 \item Pour numéroter les pages d'un livre, on emploie $408$ caractères d'imprimerie. Quel est le nombre de pages de ce livre ?
 \item On écrit les $467$ premiers nombres. Combien de fois écrit-on le chiffre $3$ ? Combien de fois écrit-on le chiffre $5$ ? Combien de fois écrit-on le chiffre $8$ ? 
 \item Former tous les nombres de trois chiffres qui s'écrivent avec les chiffres $3$, $5$, $7$. Classer ces nombres dans l'ordre croissant.\\
 Même question pour les nombres de quatre chiffres qui s'écrivent avec $3$, $5$, $7$, $9$. 
 \item Combien faut-il de dizaines, de centaines, de mille pour former un million, un milliard, $35$ millions, $17$ milliards ?
 \item Dans un nombre de deux chiffres, le chiffre des dizaines est $7$, on place un zéro entre les deux chiffres de ce nombre. De combien augmente-t-on ainsi sa valeur ? \\
 Soit le nombre $672$. On intercale un zéro entre les chiffres $6$ et $7$ et un zéro entre les chiffres $7$ et $2$. De combien augmente-t-il ainsi ? 
 \item Quels sont les plus petit et le plus grand nombre de $4$ chiffres ? Combien y a-t-il de nombres ayant moins de $4$ chiffres ? moins de $5$ chiffres ? En déduire combien il existe de nombres de $4$ chiffres. Généraliser. 
 \item Écrire en chiffres romains les nombres suivants : 
 \[ 349 \ \ \ \ 654 \ \ \ \ 1\ 794 \ \ \ \ 2\ 497. \]
 Écrire en chiffres indo-arabes les nombres suivants : 
 \[ \rm CXLIX \ \ \ \ CDLXVII \ \ \ \ MCCXLIV \ \ \ \ MCDXCIV. \]
 
 \item Dans un nombre de deux chiffres, le chiffre des dizaines est le double du chiffre des unités, et la somme de ces deux chiffres est $12$. Trouver ce nombre. 
 \item Dans un nombre de trois chiffres, le chiffre des unités dépasse de $2$ celui des dizaines et ce dernier est le triple du chiffre des centaines. La somme des trois chiffres est $16$. Trouver ce nombre. 
 \item Une loterie comprend $5\ 000$ billets numérotés de $1$ à $5\ 000$. Les frais d'organisation s'élèvent à $33,50$ F. Tous les billets ont été vendus $1$ F l'un. Les billets se terminant par $27$ gagnent $10$ F. Tous les billets se terminant par $135$ gagnent $200$ F et le numéro $2\ 791$ gagne le gros lot, soit $1\ 000$ F. Quel est le bénéfice réalisé ? 
 \item On organise une loterie comprenant $1\ 000$ billets numérotés de $1$ à $1 000$ et qui sont tous vendus $0,50$ F chacun. Les frais d'organisation se montent à $50$ F. Les billets terminés par $7$ gagnent $1$ F, les billets terminés par $35$ gagnent $10$ F et le gros lot est gagné par le numéro $794$.
 Le bénéfice réalisé est de $150$ F. Quel est le montant du gros lot ? 
 
 \end{enumerate}
 
 \chapter{Sommes de nombres entiers}
 \begin{enumerate}
\item Effectuer les additions suivantes : 
\[ 2\ 437 + 37\ 412 + 707 + 52\ 759 ;\]
\[ 3~127+ 25~742+790~395 + 42~759~375 ;\]
\[ 902~812+ 43 + 254 + 4~127 + 512~752.\]
\item De combien augmente une somme de trois nombres si on augmente le premier de $12$ unités, le deuxième de $3$ dizaines, et le troisième de $4$ centaines ?
\item De combien augmente une somme de trois nombres si on augmente le premier de $7$ dizaines, le deuxième de $25$ centaines, le troisième de $9$ mille ?
\item Calculer la somme des dix premiers nombres entiers. 
Calculer la somme des dix premiers nombres impairs.
\item Trouver trois nombres entiers consécutifs sachant que leur somme est $45$. 
\item Trouver quatre nombres entiers consécutifs sachant que leur somme est 498. 
\item En effectuant une addition de nombres entiers sans faire de retenues, on 
trouve dans chaque colonne, de droite à gauche, les sommes suivantes : 14, 11, 9. Quel est le résultat de l'addition ? 
\item Trois personnes se partagent une certaine somme. La première a $5~120$ F, la deuxième a 270 F de plus que la première. La troisième a autant que les deux autres ensemble. Quelle est la part de chacune ? la somme à partager ?
\item Dans un jeu de dominos, chaque pièce est formée par l'association d'un des chiffres de $0$ à $6$ avec lui-même ou avec un autre. \begin{enumerate}
\item Calculer le nombre de pièces différentes du jeu. Le comparer avec la somme des 7 premiers nombres entiers.
\item Combien de fois figure un chiffre donné dans l'ensemble du jeu ? 
\item Calculer le nombre total de points inscrits sur tous les dominos du jeu.
\end{enumerate}
\item \begin{enumerate}
Le carré ci-contre est dit « magique » car, en additionnant les nombres situés sur une même ligne horizontale, dans une même colonne verticale, ou bien sur une même diagonale, on obtient chaque fois le même résultat. Vérifiez-le. 
\[\displaystyle\begin{tabular}{|c|c|c|}
\hline 
8 & 1 & 6\\
\hline
3 & 5 & 7\\
\hline 
4 & 9 & 2\\
\hline
\end{tabular}
\]
\item On ajoute $4$ à chacun des nombres du carré magique. Montrer que l'on obtient encore un carré magique.
\item Quel nombre faut-il ajouter pour que la somme par ligne, colonne ou diagonale, soit égale à 54 ? Former ce carré.
\end{enumerate}
\item On considère les nombres de 1 à 12. \begin{enumerate}
\item De combien de manières peut-on les associer deux par deux de façon à obtenir une somme égale à 13 ? 
\item De combien de manières peut-on associer trois de ces nombres, distincts entre eux, de façon à obtenir une somme égale à 15 ? 
\end{enumerate}
\item \begin{enumerate}
\item Dessiner un carré partagé en $100$ petits carreaux disposés suivant 10
rangées horizontales de 10 carreaux chacune. 
Puis écrire sur la première rangée les nombres de 0 à 9, sur la deuxième, les nombres de 1 à 10, sur la troisième les nombres de 2 à 11, et ainsi de suite. 
On obtient une table d'addition. 
\item Vérifier que le nombre qui se trouve sur la ligne horizontale qui commence par $7$ et dans la colonne verticale qui commence par $5$ est égal
à 7+5. 
\item Calculer la somme des nombres situés dans chacune des lignes, puis la somme de tous les nombres inscrits dans la table. 
\end{enumerate}
\item Une ménagère achète 4 articles dans un magasin. Le deuxième coûte 25 F de plus que le premier, le troisième 50 F de plus que le second et le quatrième 75 F de plus que le troisième. Elle paie avec deux billets de 500 F
sur lesquels on lui rend un billet de 50 F, deux billets de 10 F, et un billet de 5 F. Calculer le prix de chaque article. 
\item 
\item Un particulier qui dispose de 27~000 F veut faire construire un pavillon. Il compte 7~000 F pour l'achat du terrain, 25~000 F pour la maçonnerie et la couverture, 8~000 F pour la menuiserie, 3~000 F pour l'eau, le gaz et l'électricité, 5~000 F pour le chauffage central, 2~500 F pour la peinture et 1~5000 F de frais accessoires. 
\begin{enumerate}
\item Trouver le prix de revient du pavillon.
\item Le particulier sollicite un emprunt du Crédit foncier pour la somme qui
lui manque. Il se libère en 5 ans en remboursant 1//5 de cet emprunt à la fin de chaque année. Trouver le montant exact de chacun de ces cinq versements, sachant qu'à la fin de chaque année il devra verser en même temps l'intérêt
à 8\% de la somme due au Crédit foncier pendant l'année écoulée.
\end{enumerate}
\item Effectuer de deux manières différentes les additions suivantes : 
\[ 37 + ( 43+25+12);\]
\[ 42 + 17 + (109+12) + (472+38);\]
\[ 375 + (515 + 127 + 39).\]
\item Exercices de calcul mental :
\[\begin{tabular}{cccc}
70 + 40 & 900 + 600 & 70 + 14 & 18+ 80\\
242 + 80 & 30 + 712 & 50 + 2~743 & 80 + 537\\
42 + 67 & 253 + 34 & 419 + 71 & 718 + 62 \\ 
24 + 35 & 347 + 25 & 525 + 263 & 342 + 675 
\end{tabular}\]
\item Découper trois segments dans une feuille de papier de longueurs 
respectives $a$, $b$ et $c$. Vérifier que : 
\begin{enumerate}
\item $a + (b + c) = a + b + c$.
\item $a + b + c = a + c + b = b + a + c = b + c + a = c + a + b = c + b + a$.
\end{enumerate}
 \item Au nombre entier $a$ compris entre $0$ et 10, on ajoute 5, soit b le nombre obtenu : \begin{enumerate}
 \item Établir le tableau de correspondance entre les nombres $a$ et $b$. 
 \item Construire le graphique correspondant. 
 \end{enumerate}
 \end{enumerate}
 
 \chapter{Différences de nombres entiers}
 \begin{enumerate}
 \item Que devient la différence de deux nombres.\begin{itemize}
 \item Si on augmente le premier terme de 12.
 \item Si on augmente le second terme de 12. 
 \item Si on augmente le premier terme de 12 et le second de 10. 
 \item Si on augmente le premier terme de 10 et le second de 12.
 \end{itemize}
 
 \item Calculer de deux façons différentes le résultat des opérations suivantes : 
 \[
 \begin{tabular}{l c l }
 2~315 - (37 + 452 + 17) & \ \ \ \ \ & 3~057 + (539 - 423) \\
 2~715 - (377 + 12 + 57 + 425) & \ \ \ \ \ & 70~375 + (2~195 - 492).
  \end{tabular}
 \]
 
  \item Calculer de deux façons différentes le résultat des opérations suivantes : 
 \[
 \begin{tabular}{l c l }
 4~039 - (3~215 - 2~237) & \ \ \ \ \ & 3~429 - (2~615 - 1~732) \\
 5~127 - (5~725 - 4~350) & \ \ \ \ \ & 6~847 - (3~240 - 2~428).
  \end{tabular}
 \]
 
 \item Supprimer les parenthèses en utilisant les propriétés des sommes et
 des différences dans les expressions suivantes : 
  \[
 \begin{tabular}{l c l }
 a + (b + c) + (d - e) & \ \ \ \ \ & a + (b + c) - (d - e) \\
 a - (b + c) + (d - e) & \ \ \ \ \ & a - (b + c) - (d - e).
  \end{tabular}
 \]
 
 \item Qu'obtient-on en ajoutant la somme de deux nombres et leur différence ?
 Qu'obtient-on si, de la somme de deux nombres, on retranche leur différence ?
 \item Trouver deux nombres, connaissant leur somme 342 et leur différence 88.
 \item Trouver deux nombres, connaissant leur somme 61~975 et leur différence
 2~047.
 \item Si Pierre donne 16 billes à Jean, ils en ont le même nombre.
 Combien Jean a-t-il de billes de plus que Pierre ? 
 \item Dans la soustraction 712 - 84, on oublie de faire les retenues.
  Trouver l'erreur commise sans faire l'opération.
 \item Trouver trois nombres dont la somme est 192, sachant que le deuxième
 surpasse le premier de 17 et que le troisième surpasse le deuxième de 23.
 \item Deux nombres ont pour différence 18. Si on les augmente tous deux 
 de 6, le premier devient le double du second. Trouver ces deux nombres. 
 \item Trouver trois nombres, sachant que la somme des deux premiers est 28,
 celle des deux derniers est 32, et celle du premier et du troisième est 30.
 \item Remplir les chiffres manquants dans les additions suivantes : \\
 $ 
 \begin{tabular}{cccc}
 . & . & . & 2 \\
  & 8 & 4 & . \\
  & 9 & 4 & 3 \\ 
  \hline 
  3 & 5 & 8 & 2
  \end{tabular}
$  \phantom{meowmeowmeow}
  $ \begin{tabular}{cccc}
 . & 7 & 3 & . \\
  & 7 & . & 2 \\
  2 & . & 5 & 4 \\ 
  \hline 
  7 & 8 & 7 & 7
  \end{tabular}
 $\phantom{meowmeowmeow}
  $ \begin{tabular}{ccccc}
& 2 & 3 & . & 7 \\
 &4 & 5 & 6 & . \\
 &. & . & 9 & 5 \\ 
  \hline 
  1&2 & 7 & 0 & 4
  \end{tabular}
 $
 
  \item Remplir les chiffres manquants dans les soustractions suivantes : \\
 $ 
 \begin{tabular}{ccc}
 7 & 9 & . \\
   . & . & 2 \\
  \hline 
  2 &2 & 6
  \end{tabular}
$  \phantom{meowmeowmeow}
  $ \begin{tabular}{cccc}
  . & 7 & . & . \\ 
  . & 7 & 9 & 8 \\
  \hline 
  3 & 8 & 3 & 5
  \end{tabular}
 $\phantom{meowmeowmeow}
  $ \begin{tabular}{cccc}
. & 8 & . & . \\
8 & . & 3 & 5 \\
  \hline 
  4 & 8 & 7 & 4
  \end{tabular}
 $
 \item Deux segments de droite ont une longueur totale de 118 cm. Le plus grand 
 a 12 cm de plus que l'autre. Quelle est la longueur de chaque segment ? 
 \item On veut partager une pièce d'étoffe de 60 m de long en 3 coupons de façon que le premier ait 5 m de plus que le second et 11 m de moins que le 
 troisième. Trouver les longueurs des trois coupons.
 \item Trois camarades font une excursion. Le premier paie le voyage : 
 3 billets à 2,25 F l'un. Le second paie les repas du midi : 3 déjeuners à 3
 F l'un plus 10\% de service. Le troisième paie 7,20 F pour les repas du soir. 
 Comment règleront-ils leurs comptes pour que les dépenses soient également partagées ?
 \item Plusieurs enfants se réunissent pour acheter un  ballon de football. Chacun d'eux doit payer 1,30 F. Mais au moment de l'achat trois d'entre eux sont absents, si bien que chacun des présents doit payer 1,60 F.
 Trouver le nombre total d'enfants, ainsi que le prix du ballon.
 \item Une ménagère décide d'utiliser ses économies du mois à l'achat 
 de mouchoirs. Elle pourrait acheter 15 mouchoirs d'ordinaire et il lui 
 resterait 2 F. Elle préfère dépenser 1 F de plus et faire 
 l'acquisition d'une douzaine de beaux mouchoirs coûtant 0,70 F de plus chacun. 
 De quelle somme disposait-elle, et quel prix a-t-elle payé chacun de ses mouchoirs ? 
 \item Un déjeuner à 8 F par personne réunit un certain nombre de convives.
 Trois de ces convives sont des invités et ne participent pas à la dépense,
 si bien que chacun des autres doit payer, y compris 10\% pour le service,
 11,20 F. Calculer le nombre total de convives.
 \item Effectuer mentalement les soustractions suivantes : 
 \[\begin{tabular}{rrr}
 237 - 187 & 871 - 791 & 4~783 - 4~573 \\
 217 - 29 & 712 - 89 & 7~813 - 59 \\
 701 - 439 & 802 - 547 & 1~003 - 719 \\ 
 2~754 - 781 & 3~232 - 2~192 & 7~833 - 5~935
\end{tabular}\]
 
 \item Découper deux segments $a$ et $b$ dans une feuille de papier.
 Vérifier que leur différence ne change pas lorsqu'on leur ajoute ou retranche un même segment de longueur c. 
 
 \item Découper trois segments $a$, $b$, et $c$ dans une feuille de papier tels que $b> c$ et $ b + c < a$. Vérifier que : 
 \[ a - (b+c) = a - b - c;\]
 \[ a + (b - c) = a + b - c ;\]
 \[ a - (b - c) = a - b + c.\] 
 
 \item Au nombre 12, on retranche le nombre entier $a$ compris entre 0 et 10. Soient b les nombres obtenus. 
 \begin{enumerate}
 \item Établir le tableau de correspondance entre $a$ et $b$. 
 \item Construire le graphique correspondant.
 \end{enumerate}
 
 \end{enumerate}
 
 \chapter{Produits de deux nombres}
 
 \begin{enumerate}
 \item Dans un nombre entier de deux chiffres, on appelle $a$ le chiffre des dizaines, et $b$ celui des unités. Montrer que la valeur de ce nombre est $10a+b$.
 Même exercice pou un nombre de trois chiffres en désignant par $a$ le chiffre des centaines, $b$ celui des dizaines, et $c$ celui des unités.
 \item Un libraire achète cinq douzaines de livres à 
 24 F la douzaine et les revend 3 F pièce. Trouver le 
 bénéfice réalisé, sachant que l'éditeur donne 13 livres pour 12 au libraire. 
 \item La lumière parcourt 300~000 kilomètres par
 seconde. Évaluer la distance de la Terre au soleil,
 sachant que la lumière met $8$ min $30$ à parcourir
 cette distance.
 \item Trouver un nombre de $2$ chiffres sachant que la somme de ses chiffres est $12$, et qu'en 
 retranchant de ce nombre le nombre écrit dans l'ordre
 inverse on trouve 18. 
 \item Écrire plus simplement les sommes suivantes :
 \[ (a + b + c) + (a + b + c) + (a + b + c).\]
 \[(a - b) + (a - b) + (a - b) + (a - b).\]
 \item Le périmètre d'un rectangle est 386 m ; 
 la longueur a 23 m de plus que la largeur.
 Trouver la surface du rectangle.
 \item Dans la multiplication de 243 par 405, on ne tient pas compte du O au multiplicateur. Trouver, sans faire la multiplication, l'erreur ainsi commise. 
 \item En multipliant un nombre par 207, on oublie de 
 tenir compte du zéro du multiplicateur. On fait ainsi
 une erreur de 64~080. Retrouver le multiplicande\footnote{Dans un produit $a\times b$, $a$ est le \emph{multiplicande} et $b$ le \emph{multiplicateur}.} et le résultat correct de la multiplication.
 \item On considère le produit $56 \times 43$. On augmente le multiplicateur de $8$. Trouver sans effectuer les multiplications l'augmentation du produit. 
 \item Une mercière vend une première fois 52 mètres de drap à 36 francs le mètre, et une seconde fois 65 mètres de drap à 42 francs le mètre. Trouver, sans calculer les deux prix de vente, la différence entre ces deux prix. 
 \item Le produit de deux nombres est 109~450. Trouver ces deux nombres sachant que le multiplicateur a deux chiffres, que le chiffre de ses unités est 5 et que le premier produit partiel\footnote{La première ligne lorsque vous posez le produit.} de l'opération est 21~890. 
 \item On veut clore un jardin rectangulaire de 42 m de longueur et de 30 m de largeur à l'aide d'un grillage de 2 m de haut soutenu par des poteaux en 
 ciment distants de 2 m. Le grillage pèse 4 kg au mètre carré et revient à 72 F le quintal. Calculer la dépense sachant qu'un poteau coûte 4,50 F et qu'il faut ajouter une dépense supplémentaire de 25 F pour 
 le bâti de la porte d'entrée.
 \item Une école de trois classes brûle par jour et par classez deux seaux de charbon contenant 8 kg de 
 combustible. Calculer la dépense en une année sachant
 que l'on a chauffé pendant 25 semaines à raison de 
 5 jours par semaine et que le charbon utilisé revient à 180 F la tonne. 
 \item Une ruche produit en moyenne 10 kg de miel et 15 kg de cire. Le miel vaut 5,20 F le kg et la cire 3,60 F le kg. Calculer le rapport annuel d'un rucher 
 de 18 ruches sachant que les frais d'entretien s'élèvent au quart du produit total. 
 \item La toiture d'un hangar est composée de deux 
 trapèzes isocèles égaux dont les bases mesurent 
 10 m et 4 m et de deux triangles isocèles égaux de 6 m de base. La hauteur des trapèzes et des triangles est de 4,50 m. On recouvre la toiture de plaques de 
 fibrociment qui revient à 8 F le mètre carré. Calculer la dépense.\footnote{La surface d'un trapèze est donnée par la formule $\mathcal A = \text{moyenne des bases}\times \text{hauteur}$.}
 \item La façade d'un magasin a la forme d'un rectangle de 9 m de long et de 3,50 m de hauteur. Elle
 comprend trois baies vitrées. Chacune d'elles se compose d'un rectangle de 2 m de large et de 1,60 m de haut surmonté d'un demi-cercle de 2 m de diamètre. 
 \begin{enumerate}
 \item Faire le croquis de la façade en prenant 1 cm 
 pour 1 m, sachant qu'il y a un intervalle de 50 cm
 entre deux baies vitrées et que celle du milieu occupe le centre de la façade. 
 \item On fait recouvrir cette façade de plaques de marbre qui reviennent à 4,50 F le mètre carré. Calculer la dépense. 
 \end{enumerate}
\item \begin{enumerate}
\item Soit $a$ l'un des nombres entiers de 0 à 10. 
Établir les tableaux de correspondance entre $a$ et 
les nombres $b$, $c$ et $d$ tels que : 
\[ b= 3 a ; \phantom{meowmeowmeow} c = 3a + 2 ;
\phantom{meowmeowmeow} d = 3a + 5\]
\item Construire les graphiques correspondants. 
\end{enumerate}
\item \begin{enumerate}
\item Soit $a$ l'un des nombres entiers de 5 à 15. 
Établir les tableaux de correspondance entre $a$ et 
les nombres $b$, $c$ et $d$ tels que : 
\[ b= 2 a ; \phantom{meowmeowmeow} c = 2a + 4 ;
\phantom{meowmeowmeow} d = 2a - 5\]
\item Construire les graphiques correspondants. 
\end{enumerate}
\item Compléter les multiplications suivantes : 

  $ \begin{tabular}{cccccc}
   & & 4 & . & 5 & 3 \\
   & &   &   & . & 7  \\
   \hline 
   & . & . & 2 & . & . \\
   . & . &. &. & . &\\
   \hline 
   . &. &. &. & 1 &. 
  \end{tabular}
 $\phantom{meowmeow}
   $ \begin{tabular}{cccccc}
   & & & 9 & 7 & .\\
   & & & . & . &  7 \\
   \hline
   & & . & . &. & 2 \\
   & . & . & . & . & \\
   . & . & . & . & & \\
   \hline 
   . & . & 8 & 4 & 3 & . 
  \end{tabular}
 $
   $ \begin{tabular}{ccccc}
   & & . & 9 & 6 \\ 
   & & 2 & . & 8 \\
   \hline 
   & 3 & 1 & . & . \\
   . & . & . & & \\ 
   \hline 
   . & . & . & . &.
  \end{tabular}
 $

 \end{enumerate}
 
 \chapter{Propriétés des produits de deux nombres}
 
 \begin{enumerate}
 \item Que devient le produit de deux nombres entiers
 lorsqu'on augmente l'un des facteurs de 1. Lorsqu'on augmente l'un des facteurs de x ? (Exemple : $43 \times 24$.)
\item Que devient le produit de deux nombres lorsqu'on augmente chaque facteur de 1 ? On pourra faire une figure rectangulaire. Même question pour une augmentation de x. (Exemple : $92 \times 23$). 
\item Que devient le produit de deux nombres lorsqu'on
diminue l'un des facteurs de 1, et lorsqu'on diminue 
les deux facteurs de 1 ? Même question avec une 
diminution de x. (Exemple : $247 \times 38$.)
\item Trouver les dimensions d'un rectangle, sachant qu'en augmentant la longueur et la largeur de 1 m, la
surface augmente de 170 mètres carré, et sachant d'autre part que la longueur a 84 m de plus que la largeur. 
\item Le produit de deux nombres est 340. Si l'on ajoute 3 au multiplicateur, le produit devient 400. Quels sont ces deux nombres ? 
\item Le produit de deux nombres est 575. Si l'on retranche 5 au multiplicateur le produit devient 450.
Quels sont ces deux nombres ? 
\item Que devient la surface d'un rectangle lorsqu'on augmente sa longueur de 1 m et qu'on diminue sa largeur de 1 m ? Que devient le produit de deux 
nombres lorsqu'on augmente l'un des facteurs de 1 
et que l'on diminue l'autre de 1 ? (Exemple : $537 \times 215$.)
\item Trouver les dimensions d'un rectangle dont le 
périmètre est 704 m sachant qu'en augmentant sa longueur de 1 m et en diminuant sa largeur de 1 m sa
surface diminue de 73 mètres carré. 
\item Développer : 
\[
\begin{tabular}{lll}
3(x + 7) + 5(x + 1) + 7(x + 2) &\phantom{meowmeow}& 7(x + 5) - 3(x + 2)\\
12(x + 5) + 4(x - 7) &\phantom{meowmeow}& 17(x - 3) - 16(x - 4)\\
12(x + y) + 7(x + 1) + 13(y + 2) &\phantom{meowmeow}& 100(x + y) - 36(x - y).
\end{tabular}
\]
\item Calculez de deux façons différentes les sommes ou différences suivantes : 
 \[
\begin{tabular}{lll}
$(15 \times 13) + (15 \times 7) + (15 \times 20)$ &\phantom{meowmeow}& $ (7 \times 17) + (17 \times 13) + (17 \times 5) $ \\
$(75 \times 21) + (75 \times 19)$ &\phantom{meowmeow}& 
$(43 \times 104) - (43 \times 100)$\\
$(43 \times 75) - (75 \times 40)$ &\phantom{meowmeow}& $(52 \times 17) - (52 \times 15)$.
\end{tabular}
\]
\item Mettre $x$ en facteur commun dans les sommes ou différences suivantes : 
\[ 5x + 12x + 13x\]
\[19x - 15x\]
\[ax + bx + cx + dx\]
\[xy - xz\]
\item Trouver deux nombres dont la somme est 232 sachant que le premier est le triple du second.
\item Trouver deux nombres dont la différence est 432
sachant que le premier est égal au septuple du second.
\item Partager 125 billes entre 3 enfants de façon que la part du second dépasse de 15 billes le double de 
la part du premier et que la part du troisième soit inférieure de 10 billes au triple de la part du premier. 
\item Calculer la somme des nombres contenus dans chacune des lignes de la table de Pythagore suivante : 
\[\begin{tabular}{| c | c | c | c | c | c | c | c | c |}
\hline
1 & 2 & 3 & 4 & 5 & 6 & 7 & 8 & 9 \\ \hline
2 & 4 & 6 & 8 & 10 & 12 & 14 & 16 & 18 \\ \hline
3 & 6 & 9 & 12 & 15 & 18 & 21 & 24 & 27 \\ \hline
4 & 8 & 12 & 16 & 20 & 24 & 28 & 32 & 36 \\ \hline
5 & 10 & 15 & 20 & 25 & 30 & 35 & 40 & 45 \\ \hline
6 & 12 & 18 & 24 & 30 & 36 & 42 & 48 & 54 \\ \hline
7 & 14 & 21  & 28 & 35 & 42 & 49 & 56 & 63 \\ \hline
8 & 16 & 24 & 32 & 40 & 48 & 56 & 64 & 72 \\ \hline
9 & 18 & 27 & 36 & 45 & 54 & 63 & 72 & 81 \\ \hline \end{tabular}\]
Est-il nécessaire d'effectuer toutes les additions ? 
Calculer la somme de tous les nombres de la table. 
\item Trouver un nombre de deux chiffres sachant que la somme de ses chiffres est 11 et que lorsqu'on échange le chiffre des unités et celui des dizaines,
le nombre augmente de 27. 
\item Trouver les deux facteurs d'un produit tel que si on multiplie chaque facteur par 3 le produit augmente de 280
\item On multiplie un nombre de 3 chiffres par 7, le résultat par 11, puis le nouveau résultat par 13. On obtient finalement 843~843. Quel était le nombre initial ? 
\item Un capitaine fait ranger ses hommes en carré, et il lui reste dix hommes non placés. Sachant d'autre part qu'il lui manque quinze hommes pour placer un homme de plus sur le côté du carré, trouver l'effectif de la compagnie du capitaine. 
\item Montrer que pour multiplier entre eux deux nombres compris entre 10 et 20, il suffit d'ajouter à l'un les unités de l'autre, de multiplier le résultat par 10 et d'ajouter ensuite le produit des chiffres des unités. Vérifier pour $18 \times 15$. 
\item Montrer que : 
\[ 54 \times 26 = (6 \times 4) \text{ unités } + 
[(6 \times 5) + (2 \times 4) ]\text{ dizaines } + 
(2 \times 5) \text{ centaines } \]
Trouver à partir de ce résultat un procédé pour écrire le chiffre des unités, puis celui des dizaines, et le nombre des centaines du produit de deux facteurs de deux chiffres. 
\item Les murs d'une salle de manipulation de 
4,80 m de longueur sur 2,10 m de largeur sont recouverts de carreaux de faïence sur une hauteur de 1,35 m. Il y a une porte de 0,90 m de large et les carreaux ont 15 cm de côté. Calculer le nombre de carreaux utilisés et leur prix de revient à raison 
de 75 F le cent. 
\item Une personne a pris au cours d'un mois 24 repas
tantôt dans un restaurant, tantôt dans un autre. Dans 
le premier, le repas coûte 4,20 F et dans le second
3,80 F. Sachant que la note dans le second restaurant
dépasse de 19,20 F la note payée dans le premier, on demande combien cette personne a pris de repas dans chaque restaurant. 
\item \begin{enumerate}
\item Un tailleur a acheté 3 coupons de drap de 3,5 m
chacun à raison de 25 F le mètre pour le premier, 28 F
pour le deuxième et 32 F pour le troisième. Combien
a-t-il payé ? 
\item Le tailleur utilise chacun de ces coupons pour 
effectuer un costume sur mesures. Pour chacun il 
dépense 80 F de main-d'œuvre et 30 F de fournitures.
Les costumes sont facturés 250 F, 270 F, et 300 F. 
Combien le tailleur a-t-il gagné ? 
\end{enumerate}
\item Découper et peser des plaques rectangulaires de dimensions, $a$ et $c$, puis $b$ et $c$, puis 
$a+b$ et $c$, $a-b$ et $c$. En déduire que 
\[ ac + bc = (a + b)c\text{    et    } ac - bc = (a - b) c.\]
\item Construire un rectangle de dimensions $a + b$ 
et $c + d$. Montrer qu'on peut le découper en quatre
rectangles de dimensions respectives $a$ et $c$, 
$b$ et $c$, $a$ et $d$, $b$ et $d$. En déduire que 
\[ (a + b)(c + d) = ac + bc + ad + bd\]
\item Construire un rectangle de longueur $a$ et de
largeur $c$. Augmenter sa longueur de $b$ et 
diminuer sa largeur de $d$. Évaluer la surface du 
rectangle de dimensions $a + b$ et $c - d$ ainsi 
formé par rapport à celle des rectangles de dimensions
respectives $a$ et $c$; $b$ et $c$; $a$ et $d$ ; $b$
et $d$. En déduire que 
\[ (a + b)(c - d) = ac + bc - ad - bd\]
\item Construire un rectangle de longueur $a$, 
de largeur $c$. Retrancher $b$ à sa longueur et 
$d$ à sa largeur. Évaluer la surface du rectangle 
de dimensions $a - b$ et $c - d$ par rapport à 
celles des rectangles de dimensions respectives 
$a$ et $c$ ; $a$ et $d$ ; $b$ et $c$; $b$ et $d$. 
En déduire que : 
\[ (a - b)(c - d) = ac - bc - ad + bd\]

 \end{enumerate}
 \chapter{Produits de plusieurs facteurs}
 
\begin{enumerate}
\item Que devient le produit de deux nombres lorsqu'on
multiplie l'un des facteurs par 2 ; et lorsqu'on le multiplie plus généralement par un nombre $x$ ? 
\item Que devient le produit de deux nombres lorsqu'on
multiplie les deux facteurs par 2 ; et lorsqu'on les 
multiplie plus généralement par un nombre $x$ ?
\item Que devient la surface d'un carré lorsqu'on 
double son côté ? Même question pour la surface d'un
disque lorsqu'on double son rayon. 
\item Que devient le volume d'un cube quand on double
son arête ? Que devient le volume d'une sphère lorsqu'on double son rayon ? Que devient le volume d'un cylindre quand on double le rayon du disque de base et qu'on triple la hauteur ? 
\item Effectuer les produits suivants : 
\[ 712 \times 43 \times 51 \times 19\]
\[725 \times 41 \times 25 \times 725\]
\item Effectuer les produits suivants : 
\[ (4 \times 7 \times 12) \times (7 \times 13) \times 9\]
\[ 13 \times (43 \times 17)\]
\[ (25 \times 12 \times 13) \times 4\]
\item Réduire les opérations suivantes : 
\begin{enumerate}
\item $7x \times 5y \times 3z$
\item $4(3x + 2y)$
\item $7(5x - 2y)$
\item $7(2x + 5y) + 12(3x + y) + 4(x + 5y)$
\item $2a(3b - c) + 3b(c - 2a) + c(2a -3b)$
\item $5(3a + 2) + 3(5a -2) - 2(a + 2) - 3(a - 1)$
\end{enumerate}
\item Effectuer les opérations suivantes : 
\begin{enumerate}
\item $10^5\times 10^3$
\item $10^2\times 10^3 \times 10^4$
\item $(5^4)^2$
\item $(7^4 \times 7^2) + (5^4 \times 5^2) + 
(3^4 \times 3)$
\item $a^3(a^2 + 3) + 3a^2(a^3 + 5) + 2a^2( 2a^2 - 9)$
\item $5a^4(a^2 + 4) - 2a^2(2a^4 + 1) - a^3(a^3 - 7)$
\item $ab(a - b) + a(a^2 + b^2) - a^2$
\end{enumerate}
\item \begin{enumerate}
\item Calculer la somme des 7 premiers nombres impairs.
Généraliser ce résultat. 
\item En déduire que tout nombre impair est la 
différence des carrés de deux nombres consécutifs.
Décomposer ainsi 37.
\end{enumerate}
\item On écrit dans un tableau triangulaire la suite des nombres impairs comme suit : 
 \begin{tabular}{ccc}
1 & & \\
3 & 5 & \\
7 & 9 & 11  \\
\ldots & \ldots & \ldots
\end{tabular}
\begin{enumerate}
\item Écrire les dix premières lignes de ce tableau.
\item Combien de nombres a-t-on écrit ? Trouver la 
somme de ces nombres (on utilisera l'exercice précédent). 
\item Calculer la somme des nombres inscrits dans 
chaque ligne du tableau et en déduire la somme des 
cubes des dix premiers nombres entiers.
\end{enumerate}
\item Calculer de deux manières la somme 
$a(a - b) + b(a - b)$ et en déduire que $(a + b)(a 
- b) = a^2 - b^2$. \\
Application : La différence des surfaces de deux 
jardins carrés est de 1~152m${}^2$. Calculer les côtés de 
ces deux jardins sachant que leur différence 
est de 16 m. 
\item Un bloc de pierre taillé a 80 cm de longueur,
42 cm de largeur et 35 cm de hauteur. Sachant que 
le poids volumique de la pierre est 2,7, calculer 
le poids de ce bloc de pierre. 
\item Une colonne cylindrique en ciment armé a 0,80 m 
de diamètre et 3,50 m de hauteur. \begin{enumerate}
\item Calculer la surface latérale de cette colonne 
et le prix de la peinture nécessaire pour la recouvrir 
à raison de 2,50 F le m${}^2$. 
\item Calculer le volume de la colonne et son poids 
sachant qu'un dm${}^3$ de ciment armé pèse 2,9 kg. 
\end{enumerate}
\item Une borne en granit comprend une partie enterrée de 50 cm de largeur, 30 cm d'épaisseur et 60 cm de 
profondeur. La partie apparente a une épaisseur de 24 cm. Vue de face elle se compose d'un rectangle de 40
cm de base et 45 cm de hauteur surmonté d'un demi-cercle de 40 cm de diamètre. \begin{enumerate}
\item Calculer la surface extérieure apparente de 
la borne. 
\item Calculer son poids total, sachant que la densité 
du granit est de 2,7. 
\end{enumerate}
\item Un réservoir à mazout qui a la forme d'un cylindre horizontal de 3 m de long et de 1,60 m de 
diamètre a été fabriqué en tôle de 2 mm d'épaisseur.
\begin{enumerate}
\item Calculer le poids de la tôle utilisée sachant que sa densité est 7,8. 
\item Calculer la capacité en litres de ce réservoir et la dépense lorsqu'on en fait le plein avec du 
mazout à 0,25 F le litre. 
\end{enumerate}
\item Un bassin circulaire a 5 m de diamètre et 0,80 m 
de profondeur. On le fait cimenter entièrement à raison de 5 F le m${}^2$, et border à raison de 3 F le
mètre.\begin{enumerate}
\item Calculer la dépense. 
\item Un robinet qui débite 20 litres à la minute 
alimente ce bassin. Combien de temps faudra-t-il pour le remplir jusqu'à 10 cm du bord supérieur ? 
\end{enumerate}
\item Calcul mental : 
\[\begin{matrix}
63 \times 11 \phantom{meow}& 75 \times 11\phantom{meow} & 83 \times 21\phantom{meow} & 62 \times
 110\phantom{meow} \\
 63 \times 19 \phantom{meow}& 75 \times 99 \phantom{meow}& 83 \times 39 \phantom{meow}& 620 \times 190\phantom{meow}\\
 24 \times 15 \phantom{meow}& 17 \times 12 \phantom{meow}& 25 \times 35 \phantom{meow}& 43 \times 55 \phantom{meow}
 \end{matrix}\]
 \item Soit $x$ un nombre entier de 0 à 10. Établir les tableaux de correspondance entre $x$ et les nombres $y$ suivants. Construire ensuite le graphique 
 correspondant. \begin{enumerate}
 \item $y = x^2$
 \item $y = 2x^2$ 
 \item $y = 3x^2$ 
 \item $y = x^3$
 \item $y = 2x^3$
 \item $y = 3x^3$
 \end{enumerate}
 \item En s'inspirant des derniers exercices du 
 chapitre précédent, démontrer : 
 \[ (a + b)^2 = a^2 + 2ab + b^2\]
 \[ (a + b)(a - b) = a^2 - b^2\]
 \[ (a - b)^2 = a^2 - 2ab + b^2\]
\end{enumerate} 
 
 
 \chapter{Division des nombres entiers}
 \begin{enumerate}
 \item Trouver tous les nombres entiers dont le 
 produit par 62 est inférieur à 685. 
 \item Montrer que le nombre des chiffres du quotient
 dans une division est égal au plus petit nombre de zéros qu'il 
 faut écrire à la droite du diviseur pour 
 obtenir un nombre supérieur au dividende.
 \item Montrer que, dans une division, le dividende est
 supérieur au double du reste. 
 \item Dans une division, le diviseur est 9. 
 Quels sont les restes possibles ? 
 \item Trouver les nombres qui, divisés par 13, 
 donnent un quotient et un reste égaux entre eux. 
 \item Quels sont les nombres qui, divisés par 7, 
 donnent un quotient égal à la moitié du reste ? 
 \item Quels sont les nombres qui, divisés par 5, 
 donnent un quotient égal au triple du reste. 
 \item Trouver tous les couples de nombres entiers $x$
 et $y$ qui satisfont à la relation suivante :
 \[ 287 = 17x + y\]
 \item Le quotient d'une division est 5 et le reste 32
 Trouver la plus petite valeur du diviseur et du dividende. Le dividende étant inférieur à 225, quelles 
 sont les valeurs possibles pour le dividende et le
 diviseur ? 
 \item Trouver un nombre terminé par deux zéros qui,
 divisé par 67, donne pour quotient 129. 
 \item Trouver deux nombres connaissant leur somme, 
 958 et sachant qu'en divisant le premier par le second on trouve 3 comme quotient et 98 comme reste.
 \item Trouver deux nombres connaissant leur différence, 291, et en sachant qu'en divisant le 
 premier par le second on trouve 13 pour quotient et 15 pour reste. 
 \item Effectuer la division de 272 par 57. 
 De combien peut-on augmenter le dividende sans 
 changer le quotient ? De combien peut-on diminuer le 
 dividende sans changer le quotient ? Généraliser lorsque le dividende et le diviseur sont deux nombres
 donnés $a$ et $b$, et $q$ et $r$ le quotient et le 
 reste de leur division. 
 \item Le quotient d'une division est 5, le reste 28. 
 En additionnant le dividende, le diviseur, le
 quotient et le reste, on trouve 283. Trouver 
 le dividende et le diviseur.
 \item On considère la division de 272 par 57. Montrer que le quotient ne change pas lorsqu'on multiplie le dividende et le diviseur par un même nombre. Que devient le reste ? 
 \item On considère la division de 236 par 36. Montrer que le quotient ne change pas lorsqu'on divise le dividende et le diviseur par un même nombre. Que devient le reste ? 
 \item Dans une division, le quotient est 21 et le
 reste est 8. Si on ajoute 27 au dividende
 sans changer le diviseur, le quotient est 22 et le
 reste est nul. Trouver le dividende et le diviseur 
 initiaux. 
 \item On augmente le dividende d'une division de 
 35 et le diviseur de 5. Il se trouve que ni le 
 quotient, ni le reste ne change. Quel est le quotient
 ?
 \item On dispose d'un certain nombre de billes. 
 En les rangeant par dizaines, il en reste 8. Mais il manque 5 billes pour pouvoir en ajouter une de plus par groupe. Trouver le nombre de billes. 
 \item On dispose de 225 g d'argent avec lequel on se propose de faire frapper des médailles au titre\footnote{Cela signifie qu'il y a 0,9 gramme d'argent dans chaque gramme de la médaille.} de 
 0,900 et pesant 15 g chacune. Combien pourra-t-on 
 en fabriquer ? 
 \item Une pièce de drap de 36 m de long et coûtant 
 23 F le mètre a été utilisée pour confectionner
 des costumes. On compte pour 3,20 m de tissu par 
 costume et 85 F de frais de main-d'œuvre et de fournitures. Les costumes sont vendus 189 F. Calculer 
 le bénéfice réalisé par le fabricant. 
 \item Une tente qui a pour base un rectangle de 6 m sur 2 m est fermée à ses extrémités par deux triangles isocèles verticaux de 2 m de base et de 1,15 m de 
 hauteur. Latéralement, elle se compose de deux parties inclinées rectangulaires. 
 \begin{enumerate}
 \item Faire un dessin à main levée.
 \item Calculer le volume intérieur de cette tente.  \item Combien d'hommes pourra-t-on
 y abriter si l'on veut que chacun dispose de 0,7 m${}^3$ ?
 \end{enumerate}
 \item Un cultivateur a fait venir en gare un wagon 
 d'engrais. Ce wagon mesure 6 m de long, 2,50 m de large et est chargé sur une hauteur de 80 cm. 
 L'engrais pèse 130 kg à l'hectolitre. Le cultivateur dispose d'un tombereau qui peut supporter 2,4 tonnes. 
 \begin{enumerate}
 \item Combien de voyages seront nécessaires pour enlever
 tout l'engrais ?
 \item  Afin de ménager son attelage, le cultivateur décide de faire un voyage de plus et de répartir la charge également sur les différents voyages. Quel masse charge-t-on à chaque voyage ? 
 \end{enumerate}
 \item Deux caisses contiennent chacune 145 oranges. On retire 25 oranges de la première caisse pour les mettre dans la deuxième. 
 \begin{enumerate}
 \item Combien la deuxième caisse contient-elle alors d'oranges de plus que la première ? 
 \item On répartit les oranges de chacune des caisses 
 dans des caissettes qui en contiennent chacune 25. 
 Combien de caissettes pourra-t-on remplir avec chaque
 caisse ? Pourrait-on, en réunissant les oranges restant dans les deux caisses, remplir une caissette
 de plus ? Y aurait-il encore du reste ? 
 \item Quel est le plus petit nombre d'oranges qu'il eût suffi d'ajouter à chacune des caisses initiales pour que la répartition en caissettes, effectuée après l'opération du (a), se fasse sans reste ? 
 \end{enumerate}
 \item Compléter les divisions suivantes. 
 
 \end{enumerate}
 
 \chapter{Caractères de divisibilité}
 
\begin{enumerate}
\item  $a$ désignant un nombre entier, montrer qu'un nombre pair peut s'écrire $2a$, qu'un nombre impair peut s'écrire $2a+1$ ou $2a-1$. Vérifier que la somme ou la différence de deux nombres impairs est un nombre pair.  
\item Montrer que tout nombre qui n'est pas multiple de $3$ peut s'écrire $3a+1$ ou $3a-1$, $a$ désignant un nombre entier.
Vérifier que le produit de trois nombres entiers consécutifs est toujours multiple de $3$. 
\item Montrer que tout nombre qui n'est pas multiple de $5$ peut s'écrire $5a+1$, $5a-1$, $5a+2$ ou $5a-2$, $a$ désignant un nombre entier. Que peut-on dire du produit de cinq nombres entiers consécutifs ? 
\item Vérifier qu'un nombre entier est divisible par $11$ lorsque la différence entre la somme des chiffres
de rang impair et la somme des chiffres de rang pair 
est multiple de $11$. 
\item Déterminer les lettres $x$ et $y$ pour que le nombre qui s'écrit : 
\begin{enumerate}
\item $5x9$ soit divisible par $9$ ; 
\item $7x6$ soit divisible par $3$ ; 
\item $1~3x4$ soit divisible par $9$ ; 
\item $6~x5y$ soit divisible par $2$ et par $9$ ; 
\item $7~5xy$ soit divisible par $5$ et par $9$. 
\end{enumerate}
\item Deux nombres sont composés des mêmes chiffres écrits dans un ordre différent. Montrer que leur différence est un multiple de $9$. Exemple : $2~468$
et $6~482$. 
\item On échange le chiffre des dizaines et celui des unités d'un nombre de deux chiffres. Montrer que la différence des deux nombres est le produit par $9$
de la différence de leur deux chiffres. 

\emph{Application.} Trouver un nombre de deux chiffres sachant que la somme de ses chiffres est $11$ et que l'échange de ses deux chiffres le fait augmenter de $63$. 
\item Montrer que toute puissance de $1~000$ est un 
multiple de $37$ augmenté de $1$. En déduire que le reste de la division de $254~438~906$ par $37$ est le même que celui de la division de $254+438+906$ par $37$. 
\item Déterminer le plus grand nombre de $3$ chiffres, puis le plus grand nombre de $4$ chiffres terminés par 
un $5$ et divisibles par $9$. 
\item Dans les opérations suivantes une erreur a été
commise. Expliquer pourquoi cette erreur n'est pas 
mise en évidence par la preuve par $9$ de ces opérations.

  $ \begin{tabular}{cccc}
    & 2 & 5 & 7 \\ 
    &  & 7 & 8 \\
   \hline 
    2 & 0 & 5 & 6 \\
   1 & 7 & 9 & 9  \\ 
   \hline 
   3 & 8 & 5 & 5 
  \end{tabular}
 $\phantom{meooooow}
  $ \begin{tabular}{cccc}
    & 2 & 1 & 3 \\ 
    & 3 & 0 & 5 \\
   \hline 
    1 & 0 & 6 & 5 \\
   6 & 3 & 9 &   \\ 
   \hline 
   7 & 4 & 5 & 5 
  \end{tabular}
 $\phantom{meooooow}
  $ \begin{tabular}{cccc|cc}
  7 & 2 & 3 & 8 & 2 & 4\\
  & 1 & 2 & 8 & 3 & 5 \\
  & &  & 8  & &  
  
  \end{tabular}
 $
 \item La somme des chiffres d'un nombre inconnu est égale à $23$. La division de ce nombre par $9$ donne $96$ pour quotient. Trouver ce nombre. 
 \item On sait que les seuls nombres divisibles par $25$ sont les nombres terminés par $00$, $25$, $50$ ou 
 $75$. 
 
 En déduire que tous les nombres de $3$ chiffres divisibles à la fois par $9$ et par $25$. Les comparer au plus petit d'entre eux. 
 \item Un commerçant a vendu un certain nombre d'articles à $13$ F et à $18$ F pour une somme totale de 282 F. Montrer que le nombre d'articles vendus à 13 F est obligatoirement un multiple de 2 et un multiple de 3. En déduire le nombre d'articles de chaque sorte
 vendu. 
 \item Une somme de $5,10$ F est formée par des pièces de 50 centimes, et des pièces de 20 centimes : 
 \begin{enumerate}
 \item[a)]Trouver le nombre de pièces de chaque sorte
 sachant qu'il y en a 15 en tout.
 \item[b)] Y a-t-il d'autres façons de former une somme de $5,10$ F avec des pièces de $50$ centimes et de 20 centimes ? Si oui, les déterminer. 
 \end{enumerate}
 \item Une somme de $1~382$ F est formée par des billets de $50$ F, 10 F, et 5 F, et des pièces de 1 F. 
 Le nombre de billets de 5 F est le triple de celui des pièces de 1 F ; le nombre des billets de 10 F 
 dépasse de 5 celui des pièces de 1 F ; le nombre de 
 billets de 50 F dépasse de 2 celui des billets de 5 F. Calculer le nombre de billets ou de pièces de chaque sorte. 
 \item Établir la liste des diviseurs des deux nombres suivants, puis celle de leurs diviseurs communs : 
 
\begin{center}\begin{tabular}{ccc}
63 et 171. & 84 et 180. & 60 et 105.\\
120 et 216. & 126 et 210. & 108 et 252. \\ 
100 et 140. & 140 et 175. & 132 et 198. \\
112 et 231. & 95 et 225. & 1~815 et 2~385. \\
45 ; 108 et 135. &  & 55 ; 121 et 165.\\
252 ; 315 et 441. & & 378 ; 432 et 648. 
\end{tabular} \end{center}

\item Établir la liste des dix premiers multiples des
nombres suivants, puis celle de leurs cinq premiers
multiples communs. 
\begin{center}\begin{tabular}{ccc} 
36 et 54. & 42 et 54. & 66 et 110. \\
40; 45 et 72. & & 91 ; 117 et 273.\end{tabular}
\end{center}
\end{enumerate}
\part{Nombres fractionnaires}
\chapter{Les fractions}
\begin{enumerate}
\item Découper une ficelle $AB$ en $2$, $4$, $8$ et $16$ segments égaux. Vérifier que : 
\[ AB \times \frac12 = AB \times \frac24 = AB \times \frac48 = AB \times \frac8{16}\]
puis que \[AB\times \frac58 > AB \times \frac38; \phantom{meowmeowmeow} AB \times \frac34 > AB \times
\frac38\]
\item Construire un angle, le partager en deux angles égaux, puis en 4 angles égaux, et en 8 angles égaux. Comparer les fractions correspondant à $\frac14$ et $\frac28$, puis celles correspondant à $\frac14$ et $\frac34$, puis celles correspondant à $\frac38$ et $\frac34$. 
\item Partager un cercle en 16 secteurs égaux. En déduire la comparaison des fractions suivantes : 
\[ \frac38\text{   et  } \frac6{16};\phantom{meowmeow}  \frac5{16}\text{   et  } \frac7{16};\phantom{meowmeow} 
 \frac58\text{   et  } \frac5{16}.\]
 \item Une longueur mesure  $360$ m. Quels sont les 
 produits de cette longueur par les fractions : 
 \[ \frac7{12}, \phantom{meow}\frac5{18},\phantom{meow}
 \frac{11}9,\phantom{meow}\frac{15}8, \phantom{meow}
 \frac9{10}\]
 \item Une balle élastique rebondit aux $\frac49$ de la hauteur où elle est tombée. On l'abandonne à une 
 hauteur de 1, 80 m au-dessus du sol. À quelle hauteur
 s'élève-t-elle après avoir rebondi 3 fois ? 
 \item La largeur d'un rectangle mesure 147 m ; elle est les $\frac7{11}$ de la longueur. Trouver la surface du rectangle. 
 \item Un tonneau est plein de vin ; on tire les $\frac79$ du tonneau et il reste encore $50$ litres de 
 vin. Quelle est la capacité du tonneau ? 
 \item Un alliage d'argent au titre de $\frac{875}{1~000}$ contient $1~750$ g de métal fin. Quel est le poids total de cet alliage ? 
 \item Un capital placé à $4$ pourcents produits un 
 intérêt de $724$ F. Quel est ce capital ? 
 \item Trouver une fraction égale à $\frac57$ ayant pour dénominateur $40$. 
 \item Trouver une fraction égale à $\frac{11}9$ 
 ayant pour dénominateur $63$. 
 \item Trouver une fraction égale à $\frac34$ 
 et telle que la somme de ses termes soit $21$.
 \item Trouver une fraction égale à $\frac85$ 
 et telle que la différence de ses termes soit $15$. 
 \item Une longueur $AB$ est mesurée par $\frac{13}{11}$ de mètre. Une longueur $CD$ est mesurée par $\frac5{11}$ de mètre. Quelle fraction de $AB$ représente $CD$ ? Quelle fraction de $CD$ représente 
$AB$ ? 
\item Quelle fraction de l'année représentent 5 jours ? 17 jours ? 265 jours ? 

Quelle fraction d'heure représentent 10 min ? 30 min ? 
7 min 21 s ? 13 s ? 

Quelle fraction du jour s'est-il écoulé lorsqu'il est 
$8$ h ? $7$ h du soir ? 

Quelle fraction de la semaine reste-t-il après 3 jours ? 

\item Quelle fraction\footnote{$1'$(une minute) est un soixantième de degré ; $1''$(une seconde) est un soixantième de minute.} de degré représentent les angles suivants : 
\[ 35',\phantom{meow}11', \phantom{meow}27', 
\phantom{meow}50',\phantom{meow}20'',\phantom{meow}
17'',\phantom{meow}45'', \phantom{meow}13''?\]

\item Un voyageur parcourt en chemin de fer 560 km. Quelle distance a-t-il parcourue lorsqu'il a accompli les $\frac58$ du parcours ? 
\item Les $\frac{15}{22}$ d'un nombre sont 855. Quel est ce nombre ? 
\item Trouver deux longueurs dont la somme a pour mesure 6,5 cm en sachant que l'une est les $\frac49$ de l'autre. 
\item Trouver toutes les fractions égales à $\frac7{10}$ dont le numérateur soit compris entre 400 et 500. 

\item Trouver toutes les fractions égales à $\frac4{11}$ dont le dénominateur soit compris entre 300 et 400. 
\item Comparer les fractions $\frac{11}{20}$ et 
$\frac{13}{17}$ en les comparant à une fraction auxiliaire qui ait le même numérateur que la seconde fraction et le même dénominateur que la seconde. 
\item Ranger les fractions suivantes dans l'ordre croissant (utiliser la méthode de l'exercice précédent) \[ \frac{73}{69}\phantom{meow}\frac{65}{72}
\phantom{meow}\frac{67}{70}\phantom{meow}\frac{323}{75}\]
\item Un cycliste prépare un voyage pour le lendemain :
il quittera la localité A pour aller à B où il 
s'arrêtera 2 h $\frac14$, puis terminera son excursion
en allant à C, où il désire arriver 25 minutes avant le passage du train de 17h 35 qu'il prendra pour le retour. \begin{enumerate}
\item À quelle heure doit-il partir de $A$ sachant que,
sur sa carte routière à l'échelle de 1 pour 200~000,
la distance de A à C mesure 40 cm $\frac12$, et que 
sa vitesse moyenne sera de 12 km à l'heure ? 
\item Calculer la distance BC sachant qu'elle est les $\frac45$ de la distance $AB$. 
\end{enumerate}
\item Un commerçant vend les $\frac56$ d'une pièce de toile et les $\frac34$ d'une pièce de soie soit, au total, 96 m de tissu. On demande quelle est la longueur de chacune des deux pièces sachant que les longueurs des coupons restants sont égales. On recommande de s'aider d'une figure. 
\end{enumerate}

\chapter*{Simplification des fractions}

\begin{enumerate}
\item Simplifier les fractions 
\[ \frac{77}{121}; \phantom{meow}
\frac{156}{208};\phantom{meow}
\frac{225}{375};\phantom{meow}
\frac{125}{1~000};\phantom{meow}
\frac{130}{273}; \phantom{meow}
\frac{2~352}{5~376}; \phantom{meow}
\frac{30~752}{37~800}.\]
\item Simplifier les fractions : 
\[\frac{18\times 35\times77}{66\times 21\times9}; 
\phantom{meow} \frac{45\times 38\times 34\times 100}{25\times 95\times 17};\phantom{meow}\frac{3\times 5^2\times7^4}{3^2\times5^4\times 7};\phantom{meow}\frac{2^5\times3^2\times11}{2^5\times3^3\times17}.\]
\item Réduire au même dénominateur les fractions : 
\[ \frac{17}{25}\text{  et  }\frac{19}{45};\phantom{meow}
\frac{5}{26}\text{  et  }\frac{7}{39};\phantom{meow}
\frac1{350}\text{  et  }\frac1{420};\phantom{meow}
\frac{13}{77}\text{  et  }\frac{9}{66}.\]
\item Réduire au même dénominateur les fractions : 
\[\frac{18}{36}, \frac{19}{38}\text{  et  }\frac{17}{40}; \phantom{meow} 
\frac{11}{20}, \frac{34}{60} \text{  et  }\frac{128}{120}; \phantom{meow}
\frac{15}{17}, \frac{11}{60} et \frac{15}{66}; \phantom{meow}
\frac58, \frac79, \frac3{24} et \frac5{18}.\]
\item Comparer les fractions : \[\frac{15}{16}\text{  et  }\frac{14}{15}; \phantom{meow}\frac{13}{21}\text{  et  }\frac{15}{28}; \phantom{meow}\frac{84}{115}\text{  et  }\frac{36}{43}\]
\item Ranger par ordre de grandeur les fractions et les nombres : \[\frac74 \phantom{meow}\frac78\phantom{5}{12}\phantom{meow}\frac{23}{45}
\phantom{meow}\frac{27}{50}\phantom{meow}2\phantom{meow}\frac{39}{12}\phantom{meow}3.\]
\item \begin{enumerate}
\item Réduire au même dénominateur les fractions : 
\[\frac12\phantom{meow}\frac23\phantom{meow}\frac34
\phantom{meow}\frac45\phantom{meow}\frac56\phantom{meow}\frac67.\]
\item Les réduire au même numérateur ; 
\item les comparer. 
\end{enumerate}
\item Trouver toutes les fractions égales à $\frac{60}{72}$ et dont les termes soient plus petits que ceux
de cette fraction. 
\item Montrer qu'en ajoutant aux deux termes de la fraction $\frac49$ les produits de $4$ et de $9$ par un même nombre entier, on obtient un fraction égale à $\frac49$. Généraliser. 
\item Démontrer que les fractions $\frac{141}{329}$ 
et $\frac{111}{259}$ sont égales. Simplifier ces deux fractions, puis comparer la fraction irréductible 
trouvée à chacune des deux fractions : 
\[\frac{141-111}{329-259}\text{   et   }\frac{141+111}{329+259}.\]
\item Deux roues font, l'une 17 tours en 5 secondes et l'autre 7 tours en 2 secondes. Quelle est celle qui
tourne le plus vite ? 
\item La capacité d'un flacon est égale aux $\frac3{16}$ de la capacité d'un vase A et aux $\frac4{21}$ de celle d'un vase B. Comparer les capacités des deux vases. Quelle fraction de A représente B ? 
\item Une pièce de toile a une longueur double de celle d'une pièce de soie. On vend le tiers de la pièce de toile et les $\frac47$ de la pièce de soie. Comparer les longueurs des coupons vendus. 
\item Deux réservoirs identiques sont emplis d'eau l'un aux $\frac47$, l'autre aux $\frac59$. Quel est celui qui contient le plus d'eau ? Sachant que l'un d'eux contient $5$ litres d'eau de plus que l'autre, trouver la capacité de l'un de ces réservoirs et le contenu de chacun. 
\item Deux angles A et B représentent, l'un 
les $\frac35$, l'autre les $\frac47$ d'un angle C. 
\begin{enumerate}
\item Quelle fraction de l'angle A représente l'angle B ? 
\item Sachant que la différence entre les angles A et B est égale à $3^o45'$ calculer les valeurs des trois angles A, B et C. 
\end{enumerate}
\item On considère les fractions $\frac{a}{b}$ et $\frac{a+1}{b+1}$. 

Montrer que la seconde est supérieure à la première si $a< b$. 

Montrer que la seconde est supérieure à la première si $a > b$. 

Plus généralement, comparer les fractions $\frac{a}{b}$ et $\frac{a+n}{b+n}$; puis $\frac{a}{b}$ et $\frac{a-n}{b-n}$ (n étant inférieur à a et b). En supposant : a inférieur à b, a supérieur à b. 

\item On a vendu trois coupons d'une même pièce d'étoffe. Le premier représente le $\frac13$ de la pièce, le second les $\frac5{18}$, et les $\frac{7}{24}$. \begin{enumerate}
\item Classer les 3 coupons par ordre de grandeur. 
\item Sachant que le 3${}^e$ coupon mesure 10,50 m, 
calculer la longueur de la pièce et celle de chacun des deux autres coupons. 
\end{enumerate}
\item Un marchand achète deux lots de marchandises qu'il paie 44~100 F chacun. Il revend le premier lot 56~920 F et il fait sur le second un bénéfice de 16\% sur le prix de vente. \begin{enumerate}
\item Quelle fraction d'un prix d'achat représente chacun des bénéfices réalisés ? 
\item Calculer le prix de vente du deuxième lot et 
le bénéfice sur ce lot. 
\end{enumerate} 
\item Un négociant calcule un prix de vente. Il hésite entre un bénéfice de 18\% sur le prix d'achat et un bénéfice de 15\% sur le prix de vente. \begin{enumerate}
\item Quelles fractions du prix d'achat représentent chacun des bénéfices envisagés ? Quel est le plus important ? 
\item Calculer la différence entre ces deux bénéfices 
sachant que le prix d'achat est de 76~000 F. 
\end{enumerate}
\item \begin{enumerate}
\item On donne la fraction $\frac25$. On ajoute 14 au numérateur et 35 au dénominateur. Comparer la fraction obtenue à la fraction donnée. 
\item Quels nombres faut-il ajouter au numérateur et au dénominateur de la fraction $\frac25$ pour obtenir une fraction égale dont le dénominateur soit 150 ?
\end{enumerate}
\end{enumerate}

\chapter{Addition et soustraction des fractions}

\begin{enumerate}
\item Découper des bandes de papier AB de 225 mm de 
longueur. \begin{enumerate}
\item Construire les fractions suivantes de AB : 
\[ \frac35; \phantom{meow}\frac75; \phantom{meow}
\frac49; \phantom{meow}\frac79; \phantom{meow}\frac23.\]
\item Construire les fractions suivantes de $AB$ : 
\[\frac34+\frac23+\frac49; \phantom{meow}\frac79-\frac53.\]
\[\frac34+\left(\frac79-\frac23\right);  \phantom{meow} \frac75-\left(\frac13+\frac29\right);
 \phantom{meow}\frac75-\left(\frac13-\frac29\right).\]
 
 \item Vérifier que les fractions suivantes de AB sont égales : 
 \[\frac34+\frac23+\frac49 = \frac23+\frac34+\frac49 = \frac49 + \frac34 + \frac23.\]
 \end{enumerate}
 \item Calculer : 
 \[\frac13 + \frac23;  \phantom{meow}
 \frac34+\frac25; \phantom{meow}
 \frac23-\frac12;  \phantom{meow}
 \frac45-\frac23.\]
 \item Calculer : \[\left(4+\frac23\right) + \left(2+ \frac56\right);  \phantom{meow}
 \left(7 + \frac34\right) + \left(5 + \frac23\right)
 + \left(\frac13+\frac14\right).\]
 \item Calculer :\[ 17 - \left(\frac43 + \frac35\right) ;  \phantom{meow}
 \frac{15}4 + \left(\frac{17}9-\frac85\right); 
  \phantom{meow}
  \frac{13}{11} - \left(\frac79 - \frac23\right)\]

\item Calculer et écrire sous la forme $a+\frac{b}{c}$ avec $a$, $b$, $c$ entiers et $b<c$: 
\begin{enumerate}
\item $\frac12+\frac23+\frac34+\frac56$.
\item $\frac83+\frac{11}6+\frac{15}8$. 
\item $\frac{11}6+\frac3{14}+\frac8{21}$.
\item $\frac{16}{15}+\frac{25}{21}+\frac{6}{35}$. 
\item $\frac{22}{15}+\left(\frac{23}{42} + \frac{17}{35}\right).$ 
\item $\frac{50}{21} + \left(\frac{56}{33} + \frac{15}{77}\right)$.
\item $\frac{31}{15} + \left(\frac{26}{35} - \frac{8}{21}\right)$.
\item $\frac{91}{30} + \left(\frac{29}{42} - \frac{2}{35}\right)$.
\item $\frac{76}{15} - \left(\frac{13}{24} + \frac{21}{40}\right)$.
\item $\frac{125}{36} - \left(\frac{9}{20} - \frac{17}{45}\right)$.
\item $11 - \left(5\frac14 - 1 \frac58\right)$. 
\item $51\frac27 - \left(4\frac15 + 14\frac{5}8\right)$. 
\end{enumerate}
\item Extraire les entiers des fractions suivantes : 
\[ \frac{57}7\phantom{meow} \frac{235}{19}\phantom{meow}\frac{4~372}{53}\phantom{meow}\frac{7~979}{31}\phantom{meow}\frac{42~714}{3~333}.\]
\item Un marchand achète à la fois 5 bœufs, 7 vaches et 9 veaux. Un bœuf vaut 400 F de plus qu'une vache et 10 veaux valent autant que 3 vaches. Le marchand a payé en toute 10~820 F. Trouver le prix d'un bœuf, d'une vache, d'un veau. 
\item Trois personnes se partagent une certaine somme. 
La première reçoit les $\frac27$ plus 600 F, la deuxième les $\frac25$ plus 350 F, la 3e a 1~250 F. Quelle est la somme à partager ? Quelles sont les deux premiers parts ? 
\item Un joueur perd le $\frac13$, puis le $\frac14$ de son avoir. Il lui reste alors le $\frac15$ de son avoir primitif plus 260 F. Que possédait-il ? 
\item Trois personnes se partagent une certaine somme.
La première reçoit les $\frac25$, la deuxième les $\frac37$ et la troisième le reste. Quelles sont les 3 parts, sachant que la deuxième a 2~260F de plus que la première. 
\item Trouver les dimensions d'un rectangle dont le périmètre est 560 m, sachant que la largeur est le $\frac7{13}$ de la longueur. 
\item Partager une somme de $20~230$ F entre trois personnes de façon que la part de la deuxième soit 
les $\frac7{22}$ de celle de la première et celle de 
la troisième les $\frac{16}{33}$ de celle de la première. 
\item Partager 5~597 F entre deux personnes de façon
que la part de la première soit égale aux $\frac45$ de celle de la seconde plus 125 F. 
\item Une personne a dépensé dans un magasin les $\frac{5}{11}$ de ce qu'elle possédait. Il lui manque alors 3 F pour acheter 5 m d'étoffe à 9 F le mètre. Que possédait-elle primitivement ? 
\item Un ouvrier ferait un ouvrage en 5 jours, un autre le ferait en 6 jours. Quelle fractions de l'ouvrage font-ils en un jour lorsqu'ils travaillent ensemble ? 
\item Un robinet remplirait un bassin en 6 heures et un autre le viderait en 10 heures. Les deux robinets
étant ouverts, quelle est la fraction du bassin remplie en une heure ? 
\item Vénus et la Terre tournent autour du Soleil, la première en 225 jours, la seconde en 365 jours. Quelle est, de ces deux planètes, celle qui tourne la plus 
vite autour du soleil et quelle est la fraction de tour qu'elle fait de plus que l'autre en un jour ? 
\item Mars tourne autour du Soleil en 687 jours. Déterminer en fraction de tour, l'avance prise par 
la Terre sur Mars, en un jour. 
\item Un piéton marche pendant 3 heures : pendant la première, il parcourt 6 km $\frac58$,
pendant la deuxième, il fait 1 km $\frac14$ de moins que pendant la première, 
et pendant la troisième, $\frac45$ km de moins que pendant la seconde. Quel est le trajet parcouru par ce piéton ? 
\item Après avoir vendu les $\frac25$ puis les $\frac37$ d'une pièce d'étoffe, il en reste 18 m. Quelle était la longueur de la pièce ? 
\item Un héritage a été partagé entre 4 personnes. La première en a reçu le tiers, la seconde le quart, la troisième le cinquième, et la dernière a reçu 5~200 F. Quel est le montant de l'héritage ? 
\item Une propriété de 149,6 hectares est constituée pour les $\frac7{11}$ de terres cultivables, pour les $\frac4{17}$ en bois, et le reste en prairies. Les terres valent 20 F l'are et rapportent 7,5 \% ; 
les bois valent 24 F l'are et rapportent 6\% ; 
les prairies valent 14 F l'are et rapportent 2,5\%. 
Calculer le revenu de cette propriété.
\item D'une pièce de toile, on fait trois parts : la première est égale aux $\frac27$ plus 6 m ; la seconde
au tiers plus 7 m ; la dernière a pour longueur les 11 m restants. Calculer la longueur de chaque part. 
\item Calculer les notes d'un candidat qui a obtenu 
dans un examen 42 points $\frac34$. Sa note d'arithmétique surpasse de 1 point $\frac14$ celle d'histoire ; dans cette matière, il a obtenu 2 points $\frac34$ de plus qu'en orthographe. (Il n'y a pas d'autre discipline). 
\item Trois robinets remplissent un bassin : le premier a débité 54 litres $\frac14$, le deuxième 3 litres $\frac58$ de moins que le premier et le troisième 10 litres $\frac13$ de moins que les deux premiers réunis. Quelle est la capacité du bassin ? 
\item On remplit jusqu'aux $\frac38$ de sa hauteur un bassin ayant la forme d'un parallélépipède rectangle, dont la largeur est la moitié de la longueur. Puis on fait monter le niveau de 0,38 m en ajoutant 7,6 hectolitres. Le bassin est alors rempli aux $\frac57$ de sa hauteur. Calculer ses dimensions. 
\item Une fermière a vendu les $\frac25$ d'un panier d'œufs. Si elle ajoutait 46 à ce qui lui reste, le nombre des œufs qu'elle avait d'abord serait augmenté de son $\frac19$. Calculer ce nombre.
\item Soit la fraction $\frac58$ et la fraction $\frac7{10}$ obtenue en ajoutant 2 à chacun des termes. Comparer ces deux fractions en comparant leurs 
compléments à l'unité. 

Comparer de même les fractions $\frac{17}{35}$ et $\frac{17+n}{35+n}$ ainsi que les fractions $\frac{17}{35}$ et $\frac{17-n}{35-n}$, où $n$ est entier et inférieur à 17. 

\end{enumerate}

\chapter{Multiplication et division d'une fraction par un nombre entier}
 
 \begin{enumerate}
 \item Découper des bandes de papier $AB$ de $120$ mm de long. Les partager en 12 parties égales. Vérifier 
 que les fractions suivantes de $AB$ sont égales : 
 \[ \frac5{12}\times 3 = \frac{15}{12}; \phantom{meow}\phantom{meow}\frac7{12}\times 6 = \frac72. \]
 \[ \frac53\div4=\frac5{12}; \phantom{meow}\phantom{meow}\frac43\div2=\frac23.\]
\item Effectuer les opérations suivantes et simplifier les résultats : 
\[ \frac56\times 8 ; \phantom{meow} \frac7{12}\times6; 
\phantom{meow}\frac{15}7\times21; \phantom{meow}
\frac{27}{13}\times 39.\] 
\[ \frac{45}{28} \times 12; \phantom{meow} \frac{39}{25} \times 20; \phantom{meow} 
\frac{17}{40}\times 8;\phantom{meow} \frac{49}{25}\times35 .\]
\[\frac{48}{35}\div30; \phantom{meow} \frac{33}{25}\div 15 ; \phantom{meow}\phantom{meow}
\frac{108}{55}\div 27 ; \phantom{meow} 
36 \div 54. \]
\[ \frac{84}{39}\div 36; \phantom{meow}
\frac{216}{35}\div 48 ; \phantom{meow}
\frac{360}{77}\div 135; \phantom{meow}
\frac{355}{113}\div 71.\]
\item Simplifier et calculer les expressions : 
\[ \frac{189\times 22}{126}; \phantom{meow}\phantom{meow} \frac{154\times 45}{525} ; \phantom{meow}\phantom{meow} \frac{168\times 63}{756}.\]
\[\frac{84\times 78}{1~512}\phantom{meow}\phantom{meow} \frac{432\times 168}{336} ; 
\phantom{meow}\phantom{meow} \frac{684\times 252}{11~340}.\]
\item Une personne dépense dans un magasin les $\frac34$ de ce qu'elle possède, puis dans un autre le tiers du reste. Il lui reste encore 72 F. Que possédait-il primitivement.
\item Partager une somme de 133~000 F entre 3 personnes de façon que la part de la 2e soit les $\frac56$ de celle de la première et celle de la troisième le tiers de celle de la deuxième. 
\item Un joueur perd les $\frac37$ de sa fortune, puis le $\frac13$ de sa fortune. Il lui reste le $\frac1{10}$ de ce qu'il a perdu plus 850 F. Quelle était sa fortune primitive ? 
\item Une balle élastique rebondit au tiers de la hauteur d'où elle est tombée. À quelle fraction de la hauteur primitive s'élève-t-elle après 4 bonds successifs ? 
\item On soutire chaque jour 3 L $\frac14$ d'une pièce de vin de 225 litres. Que reste-t-il après 60 jours ?
\item Quel est le poids de viande d'un bœuf de 450 kg contenant $\frac15$ d'os. 
\item Une personne exécute le cinquième d'un travail en 1 heure. Combien de temps lui faudra-t-il pour en 
faire les $\frac23$, les $\frac34$, les $\frac56$ ?
Quelle fraction en fait-elle en deux heures et demi ? 
\item On multiplie un nombre par $\frac58$ puis ce même nombre par $\frac7{12}$; on fait la somme des deux produits. Calculer le nombre sachant que cette somme surpasse de 135 unités le nombre lui-même. Faire la vérification. 
\item On partage les $\frac47$ d'une somme d'argent entre 6 personnes et le reste entre 4 personnes. 
Quelle fraction de la somme chaque personne a-t-elle 
reçue ? Quelles sont les parts, la somme s'élevant à 84~000 F ? 
\item Deux héritiers s'étant partagé une somme de 106~000 F ont dépensé, les premiers les $\frac67$ de sa part et le second les $\frac9{11}$ de la sienne. Le premier possède alors trois fois de plus que le second. Calculer les parts. 
\item Une ménagère a acheté une pièce de drap et une toile qui lui ont coûté ensemble 486 F. Elle a payé le drap 12 F le mètre et la toile 4,50 F le mètre. Elle emploie le $\frac45$ et les $\frac37$ de la pièce de toile et il se trouve que les restes des deux pièces ont des valeurs égales. Calculer combien chaque pièce contient de mètres. 
\item Un cultivateur laisse à ses deux fils un pré et une vigne estimés chacun 4~000 F l'hectare. Les $\frac58$ de la superficie du pré sont égaux aux $\frac67$ de la surface de la vigne, et la différence des surfaces est de 27,95 ares. 

On demande : \begin{itemize}
\item Quelle somme l'un des frères devra donner à l'autre pour que les deux parts soient égales ? 
\item Quelle est la surface du pré et quelle est la surface de la vigne ? 
\end{itemize}
\item Soit la fraction $\frac35$. 
\begin{enumerate}
\item Trouver une fraction égale à cette fraction dont le dénominateur soit 20. 
\item Trouver une fraction égale dont le numérateur soit 21. 
\item Trouver une fraction qui soit égale à la moité de la fraction proposée et dont le dénominateur soit 50. 
\end{enumerate} 
 \end{enumerate}
 
 \chapter{Multiplication des fractions}
 
 \begin{enumerate}
 \item Découper une bande de papier de 20 cm de long. 
 En prendre les $\frac35$. Construire les $\frac78$ de la bande obtenue. Vérifier que la longueur finale est les $\frac{21}{40}$ de la bande primitive. 
 \item Effectuer les produits suivants et simplifier les résultats : 
 \[ \frac47\times \frac59\times \frac3{10}; 
 \phantom{meowmeow} \frac74\times5\times\frac8{21};
  \phantom{meowmeow}  \frac34\times \frac45\times \frac56.\] 
   \[ \frac{45}{64}\times\frac{24}{27}; 
 \phantom{meowmeow} \frac{35}{36}\times\frac{48}{49};
  \phantom{meowmeow}  \frac{21}{36}\times\frac{24}{35} \times \frac{45}{42}.\] 
   \[ \frac{27}{34}\times \frac{51}{84}; 
 \phantom{meowmeow} \frac{76}{35}\times\frac{48}{49};
  \phantom{meowmeow}  \frac{38}{85}\times \frac{65}{133} \times \frac{119}{143}.\] 
  \[ \left(\frac{19}{12} + \frac5{21} + \frac{13}{28}
   \right)\times\frac{21}{32}; \phantom{meowmeowmeowmeow}
   \left(\frac{43}{28}-\frac{13}{21}\right)\times 
   \frac{16}{33} \]
     \[ \left(\frac{51}{56} + \frac8{21} + \frac{16}{48}
   \right)\times\frac{32}{65}; \phantom{meowmeowmeowmeow}
   \left(\frac{31}{20}-\frac{26}{45}\right)\times 
   \frac{36}{49} \]
   \[ \left( \frac45\right)^2 \times \left(\frac45\right)^3; \phantom{meowmeow} 
   \left( \frac23\right)^5 \times \left(\frac45\right)^3\times \left(\frac23\right);
   \phantom{meowmeow} \left[\left(\frac52\right)^3\right]^2\]
   \[ \left(\frac27\right)^2 \times \left(\frac27\right)^4\left(\frac56-\frac13\right) ; 
    \phantom{meowmeowmeowmeow}
    \left(\frac12\right)^4 \times \left(\frac12\right)^3 \left(\frac23+\frac34\right).\]
 \item Effectuer mentalement : 
 \[ 62 \times 5;  \phantom{meowmeowmeowmeow}
 126 \times 50;  \phantom{meowmeowmeowmeow}
 42 \times 15;\]
  \[ 38 \times 5;  \phantom{meowmeowmeowmeow}
 257 \times 50;  \phantom{meowmeowmeowmeow}
 72 \times 150;\] \[ 47 \times 5;  \phantom{meowmeowmeowmeow}
 184 \times 500;  \phantom{meowmeowmeowmeow}
 232 \times 15;\] 
 \[ 121 \times 5;  \phantom{meowmeowmeowmeow}
 365 \times 50;  \phantom{meowmeowmeowmeow}
 125 \times 150.\]
 \item Un champ est partagé en trois parties. La première est égale aux $\frac27$ de la surface totale,
 la seconde aux $\frac5{13}$ de la première.
 La différence entre les deux premières parties est 1~200 mètres carrés. Calculer la surface totale. 
 \item Un travail est exécuté par 3 ouvriers, le premier en a fait les $\frac25$, le second les $\frac56$ du travail effectué par le premier et le troisième le reste. Ce dernier a touché 2~100 F de moins que les autres ensemble. Calculer le prix total de ce travail. 
 \item Les $\frac34$ des $\frac45$ d'un nombre valent 108. Quel est ce nombre ? 
 \item En ajoutant à un nombre donné les $\frac45$ des 
 $\frac23$ de ce nombre, on trouve $322$. Quel est ce 
 nombre ? 
 \item On considère une suite de 4 nombres tels que chacun d'eux soit égal à la moitié du précédent. Trouver ces nombres, sachant que leur somme est 105. 
 \item Une personne perd les $\frac35$ de sa fortune ; elle regagne ensuite les $\frac47$ de ce qu'elle avait perdu et possède alors 39~000 F. Que possédait-elle primitivement ? 
 \item Une personne doit une certaine somme. Elle verse d'abord le quart de sa dette, puis les $\frac37$ de ce qui reste à payer, et se libère enfin en versant 63~000 F. Quelle était la dette de cette personne ? 
 
 \item Deux ouvriers ont travaillé le premier 18 jours, le second 20 jours ; le salaire journalier du premier est les $\frac45$ de celui du deuxième. 
 Sachant que ces deux ouvriers ont reçu ensemble 1~032 F, quel est le salaire journalier de chacun ? 

  \item Le bénéfice d'un commerçant est le $\frac{32}{100}$ du prix de vente. Ce bénéfice augmenterait de 3~600 F s'il était la moitié du prix d'achat. Déterminer le prix d'achat. 
  
  \item On retire d'une cuve les $\frac23$ de sa contenance moins 40 litres. On retire ensuite les $\frac25$ du reste ; il reste encore 84 litres. Quelle 
  est la capacité de la cuve ? 
  
  \item Par quelle fraction faut-il multiplier un nombre pour l'augmenter de ses $\frac25$, de ses $\frac34$, de ses $\frac23$ ? 
  \item Les $\frac34$ des $\frac56$ d'un nombre donné surpassent de 38 unités les $\frac23$ des $\frac7{10}$ de ce nombre. 
  \item On retranche d'un nombre ses $\frac25$,
  puis les $\frac57$ du reste. Quelle fraction du nombre reste-t-il ? 
  \item Les $\frac25$ des candidats à un concours ont été admissibles aux épreuves orales et un candidat sur huit a échoué à ses épreuves. Quel est le pourcentage 
  des candidats définitivement reçus au concours ? 
  Application : le nombre des candidats s'élève à 520. Trouver le nombre des admis.
  \item Partager une somme de 84~000 F entre cinq personnes de façon que la première reçoive les $\frac34$ de la part de la deuxième, 
  que la deuxième reçoive la moitié de la part de la troisième, 
  que celle-ci reçoive les $\frac23$ de la part de la quatrième et enfin que celle-ci reçoive les $\frac45$
  de la part de la cinquième. 
  \item Un joueur perd les $\frac27$ de la somme qu'il 
  possède, puis regagne les $\frac58$ de ce qu'il a perdu et se retire du jeu avec 125 F. Quel était son avoir primitif ? 
  \item Soient deux nombres dont l'un est $\frac34$ de 
  l'autre. On multiplie le premier par $\frac23$ et le second par $\frac78$ et en faisant le produit des deux résultats, on obtient 28. Quels sont ces deux nombres ? 
  \item Une personne dépense les $\frac23$ de son argent, puis elle gagne une somme égale aux $\frac37$
  de ce qui lui restait. Avec la somme qu'elle possède alors elle peut payer les $\frac25$ d'une pièce de drap de 31, 25 m valant 12 F le mètre. Combien avait-elle tout d'abord ? 
  \item Un ouvrier fait le quart d'un ouvrage en cinq jours ; un deuxième ouvrier fait les $\frac25$ du reste en douze jours. Combien les deux ouvriers travaillant ensemble mettront-ils de jour pour achever l'ouvrage ? Le premier ouvrier a touché 132 F pour son travail ; quelle somme touchera le second ouvrier sachant que pour chacun d'eux, le prix de la journée est le même ? 
 \end{enumerate}

 
 \chapter{Division d'une fraction par une fraction}
 
 \begin{enumerate}
 \item \[ 3 \div \frac57;\phantom{meowmeow}
 12\div \frac74 ; \phantom{meowmeow}
 \frac{36}{55}\div\frac{9}{22}; \phantom{meowmeow}
 \frac{49}{81}\div \frac{21}{54}.\]
 
  \item \[ 21 \div \frac75;\phantom{meowmeow}
 36\div \frac{54}5 ; \phantom{meowmeow}
 \frac{48}{25}\div\frac{16}{15}; \phantom{meowmeow}
 \frac{15}{32}\div \frac{25}{24}.\]
 
  \item \[ \frac{52}{45} \div \frac{78}{81};\phantom{meowmeow}
 \frac{55}{126}\div\frac{44}{189} ; \phantom{meowmeow}
 \frac{51}{91}\div\frac{85}{156}; \phantom{meowmeow}
 \frac{63}{92}\div \frac{84}{115}.\]
 
 \item \[ \left( \frac{10}{21} +\frac{11}{28} + \frac5{12} \right) \div \frac{18}{35} ;
 \phantom{meowmeow}
 \left( \frac{32}{39} - \frac{21}{52} \right) \div \frac{15}{8}. \]
 
 \item \[ \left( \frac{23}{44} +\frac{34}{77} + \frac{11}{28} \right) \div \frac{38}{49};
 \phantom{meowmeow}
 \left( \frac{19}{28} - \frac{23}{70} \right) \div \frac{14}{45}. \]
 
  \item \[ 
\frac{\frac6{35}+\frac{19}{21}+\frac{16}{15}}{\frac{11}{21}+\frac3{14}+\frac{11}6};
 \phantom{meowmeow}
\frac{\frac8{21}+\left(\frac{11}{15}-\frac4{35}\right)}{\frac{17}{36}-\left(\frac9{20}-\frac{17}{45}\right)}.
 \]
  \item \[ \left(\frac59\right)^{10}\div\left(\frac59\right)^7;
 \phantom{meowmeow}
\left(\frac7{15}\right)^{43}\div\left(\frac7{15}\right)^{47};
 \phantom{meowmeow}
 \left(\frac2{13}\right)^{13}\div \left(\frac2{13}\right)^{23}.
 \]
 \item 
 \[ \left[\left(\frac38\right)^5\times\left(\frac38\right)^4\right]\div 
 \left(\frac38\right)^7 ; 
  \phantom{meowmeow}
  \left[\left(\frac56\right)^2\times\left(\frac56\right)^5\right]\div 
 \left(\frac56\right)^6.
 \]
 \item Simplifier : \[ \frac{a}{b}\div a ; \phantom{meowmeow}
 a\div\frac{a}{b}; \phantom{meowmeow}
 \frac1a\div\frac1b; \phantom{meowmeow}
 \frac{3a}b\div\frac{2a}{3b}.\]
 \item Simplifier : \[ \left(2+\frac1a\right)\div(2a+1); \phantom{meowmeow}
 \left(\frac{a}b + 1\right) \div \left(\frac{a}b-1\right).\]
 \item Simplifier : \[ \frac{a^{12}}{a^7}; \phantom{meowmeow}
 \frac{a^3\times a^5}{a^2}; \phantom{meowmeow}
 \frac{(a+b)^5}{a+b}; \phantom{meowmeow}
 \frac{(a-b)^4}{(a-b)^2} .\]
 \item Simplifier : \[ \frac{5a^3b^2}{15a^2b}; \phantom{meowmeow}
 \frac{108a^5b^3c^2}{36a^2b^2c}; \phantom{meowmeow}
 \frac{105 x y^4 z^3}{49x^2y^4z}.\]
 \item Simplifier : \[ \frac{a^2}{a^5}; \phantom{meowmeow}
 \frac{4a^2b^3}{8a^6b^2}; \phantom{meowmeow} \frac{121a^2b^3c}{77a^4b^4c}.\]
 \item En divisant $\frac47$ par une certaine fraction, le quotient est égal 
 à $\frac3{14}$. Quelle est la fraction diviseur ? 
 \item Les $\frac7{12}$ d'une fraction sont égaux à $\frac8{15}$. Quelle est
 cette fraction ? 
 \item Par quel nombre entier ou fractionnaire faut-il multiplier $\frac57$
 pour obtenir les $\frac34$ de $\frac{11}9$ ? 
 \item Que devient un produit de facteurs lorsqu'on divise l'un des facteurs
 par $\frac49$ ? 
 \item Trouver deux fractions dont la somme est $\frac{390}{187}$ et le 
 quotient $\frac{119}{11}$. 
 \item Trouver deux fractions dont la différence est $\frac{5}{21}$
 et le quotient $\frac{12}7$. 
 \item Un cycliste parcourt la distance $AB$ à la vitesse de $30$ km à l'heure et la distance $BA$ à la vitesse de $20$ km à l'heure. Quelle est sa
 vitesse moyenne pour le trajet aller et retour ? 
 \item Un cycliste se rend de $A$ à $B$ à la vitesse de $27$ km à l'heure.
 Il revient de $B$ à $A$ à la vitesse de $24$ km à l'heure. Trouver la 
 distance $AB$, sachant que la durée totale du voyage aller et retour est
 de 10 h 37 min 30 sec. (Indication : Réduire ce temps en une fraction d'heure.)
 \item Une équipe d'ouvriers ferait un travail en 5 jours et demi ; une seconde équipe le ferait en 4 jours $\frac23$. Combien les deux équipes, travaillant ensemble, mettront-elles de temps pour faire ce travail ? 
 \item Une somme de 21~700 F est partagée entre 2 personnes. La première ayant dépensé les $\frac57$ de sa part et la seconde les $\frac25$ de la sienne, il leur reste la même somme. Trouver les deux parts. 
 \item Les $\frac45$ d'un nombre valent les $\frac59$ d'un autre nombre. 
 Trouver ces deux nombres, sachant que leur différence est $33$.
 \item Un robinet remplit les $\frac27$ d'un bassin en 2h $\frac34$ ; combien
 de temps faut-il laisser couler le robinet pour que le bassin soit plein 
 aux $\frac45$ ? 
 \item Une équipe d'ouvriers ferait un travail en 7 jours, une autre équipe
 le ferait en 9 jours. On fait travailler ensemble les $\frac23$ de la 
 première équipe et la moitié de la seconde. Combien de temps faudra-t-il
 pour effectuer ce travail ? 
 \item Deux sommes égales sont placées, l'une à 5 \%, l'autre à 6 \% pendant 
 42 mois. La seconde rapporte 1~890 F de plus que la première. Quelle est
 la valeur commune de ces deux sommes ? 
 \item La durée de la révolution de la Terre autour du Soleil est 365 jours.
 En combien de temps la Terre effectue-t-elle les $\frac35$ de cette révolution ? 
 \item Les durées des révolutions de la Terre et de Vénus autour du Soleil
 sont respectivement 365 jours et 225 jours. Calculer l'intervalle des 
 temps qui, pour un observateur terrestre, séparent deux passages successifs
 de Vénus devant le Soleil. 
 \item La durée de la révolution de Jupiter autour du Soleil est environ 142 
 mois. Quel est l'intervalle des temps qui séparent deux oppositions successives du Soleil et de Jupiter (instants où le Soleil et Jupiter sont des directions opposées par rapport à la Terre). 
 \item Pour effectuer un parcours de $100$ km, un train à mis 1h 20. Quelle
 est sa vitesse horaire ? 
 \item Une vis avance de $\frac7{10}$ mm en 5 tours. Calculer le nombre de
 tours nécessaire pour avancer de 12 mm $\frac14$. 
 \item Une personne exécute les $\frac{3}{11}$ d'un travail en 3h $\frac58$.
 Quel sera le temps nécessaire pour en exécuter les $\frac8{11}$ ? 
 \item Un pré et une vigne ont une superficie totale de 51 ha. Les $\frac47$ de
 la superficie du pré sont égaux aux $\frac25$ de celle de la vigne. Calculer 
 la superficie du pré et de la vigne. 
 \item Trouver un nombre dont le quotient par $\frac7{11}$ surpasse ce nombre
 de 64. 
 \item En divisant $\frac47$ par une fraction on obtient un quotient qui est
 les $\frac23$ du dividende. Quel est le diviseur ? 
 \item Par quel nombre a-t-on divisé le nombre 36 lorsqu'on l'a augmenté de ses $\frac34$ ?
 \item Par quel nombre multiplie-t-on la fraction $\frac4{11}$ en augmentant
 de 2 son numérateur et en diminuant de 4 son dénominateur ? 
 \item Soient les fractions $\frac{a}b$ et $\frac{c}d$. Montrer qu'on ne change pas le quotient de ces deux fractions en les multipliant ou en les divisant toutes deux par un même nombre entier ou plus généralement par une même fraction. Vérifier sur des exemples. 
 
 \end{enumerate}
 
 \chapter{Fractions décimales, nombres décimaux}
 
 \begin{enumerate}
 \item Écrire sous forme décimale : 
 \[ \frac{75}{10};\phantom{meowmeow}\frac4{100};\phantom{meowmeow}\frac{12}{1~000};\phantom{meowmeow}\frac{25}{10~000};\phantom{meowmeow}\frac{23~752}{1~000};\phantom{meowmeow}\frac{47}{10~000}.\]
 \item Écrire sous forme de fractions les nombres suivants : 
 \[7,24; \phantom{meowmeow} 3,572;\phantom{meowmeow}
 0,0417;\phantom{meowmeow}5,00178;\phantom{meowmeow}
 0,0072.\]
 \item Combien peut-on écrire de nombres à 3 décimales compris entre 5,32 et 5,33 ? Combien peut-on écrire de nombres à 4 décimales compris entre 12,347 et 12,348 ? 
 \item Comparer les nombres 12,7 et 12,07. De façon générale, comment se modifie un nombre à une décimale lorsqu'on place un zéro entre la virgule et le chiffre décimal ?
 \item Quel est le nombre décimal de mètres contenus dans une longueur de 7 dam, 5 m, 3 dm et 8 cm ? 
 \item Effectuer de deux manières différentes les additions suivantes : 
 \[ 37,7 + (42+0,75+7,12);\] 
 \[ 13 + 10,5 + ( 43, 75 + 5,725) + (7 + 2,57) ; \]
 \[ 372, 5 + ( 5,703 + 12, 7 + 3,9).\]
 \item Calculer de deux façons différentes le résultat des opérations suivantes : 
 \[ 53 - ( 3,7 + 4,52 + 0,17);\]
 \[ 271,5 - (37,7 + 12 + 0,57 + 43,5).\]
 \item Calculer de deux façons différentes le résultat des opérations suivantes : 
 \[ 703,75 + (219,5-49,2);\]
 \[1~000 + (2,712-0,47).\]
 \item Calculer de deux façons différentes le résultat des opérations suivantes : 
 \[ 512,7-(57,25-43,5);\]
 \[47-(4,509-3,7).\]
 \item Calculer de deux façons : 
 \[ (0,75\times 27) + (0,75 \times 13); \]
 \[ (0,52\times 19) - (0,52 \times 17).\]
 \item Calculer : 
 \[ (0,25)^2; \phantom{meowmeow} (2,15)^3; \phantom{meowmeow}
 (2,45)^3 \times (2,45)^2; \phantom{meowmeow}
 \frac{(9,81)^7}{(9,81)^5}.\]
 \item Une vis avance de $\frac7{10}$ de millimètre en 13 tours. Combien doit-elle faire de tours pour avancer de 3,5 mm ? 
 \item Prendre le plus simplement possible les 0,9; les 0,99; les 0,999 du nombre 17,8. 
 \item Transformer en fractions décimales les fractions suivantes : 
 \[\frac32; \phantom{meowmeow}
 \frac54; \phantom{meowmeow}
 \frac45; \phantom{meowmeow}
 \frac{21}{25};\phantom{meowmeow}
 \frac78; \phantom{meowmeow}
 \frac{11}{250}.\]
 \item La fraction $\frac8{19}$ peut-elle se convertir en fraction décimale ?
 \item Par quel nombre décimal faut-il diviser un nombre pour réduire ce 
 nombre aux $\frac8{15}$ de sa valeur ? 
 \item Par quelle fraction décimale faut-il multiplier un nombre pour 
 le diminuer des $\frac6{100}$ de sa valeur ? 
 \item En déplaçant de deux rangs vers la gauche la virgule d'un nombre décimal, il diminue de 5~749,326. Quel est ce nombre ? 
 \item Un alliage d'argent et de cuivre provenant de la fonte de deux lingots
 pèse 3~800 grammes et a pour titre $\frac{775}{1~000}$. Quel est le titre du 
 premier lingot sachant que le second lingot avait pour poids 2 kg et pour titre $\frac{650}{1~000}$ ? 
 \item 
 \begin{enumerate}
 \item Quelle est la distance de Paris à Dijon sachant que sur une carte dont 
 l'échelle est un millionième, elle est représentée par une longueur de 31,5 cm ? 
 \item Par quelle longueur est-elle représentée sur une carte dont l'échelle est 4 dix-millionièmes ? 
 \item Quelle est l'échelle d'une carte sur laquelle la distance Paris-Dijon est représentée par 25,2 cm ? \end{enumerate}
 \item Effectuer mentalement : 
 \[ 54,45+ 3,95 ; \phantom{meowmeow} 12,55 + 5,75+ 4,70 ; \phantom{meowmeow} 
 8,72 + 15,9 + 5,6.\]
 \[ 30 - 22,85 ; \phantom{meowmeow} 75,7 - 13,45 ;\phantom{meowmeow} 134,4 - 12,18.\]
 \[ 19,8 \times 0,5 ; \phantom{meowmeow} 15,60\times 0,25 ; \phantom{meowmeow} 17,12 \times 0,75 ; \phantom{meowmeow} 14,32\times 0,125.\]
 \[ 7,3 \times 11; \phantom{meowmeow}3,15 \times 9 ; 
 \phantom{meowmeow} 5,1 \times 19 ; \phantom{meowmeow}
 3,4 \times 99 ; \phantom{meowmeow} 7 \times 5,95.\]
 \[ 24,7 \div 0,5 ; \phantom{meowmeow} 58 \div 0,25 ; \phantom{meowmeow}
 51,351\div 0,75; \phantom{meowmeow} 0,52 \div 0,125.\]
 \item Le nombre $x$ varie de dixième en dixième entre 2 et 5. Dresser le tableau de correspondance entre $x$ et les nombres $y$ suivants, puis construire le graphique qui en résulte : 
 \[ y = 3x; \phantom{meowmeow} y = 3x+4; \phantom{meowmeow} y=3x-1.\]
 \[ y = \frac{3x}5; \phantom{meowmeow}y = \frac{3x}5 + 2; \phantom{meowmeow}
 y = \frac{3x}5 - \frac12.\]
 \[ y = x^2; \phantom{meowmeow} y = x^3; \phantom{meowmeow} y = \frac45x^2.\]
 \[ y = 2x^2 - \frac15; \phantom{meowmeow} 
 y = x^3 + 1; \phantom{meowmeow} y = \frac45 x^2 + \frac12.\]
 
 \end{enumerate}
 
 \chapter{Quotient de deux nombres à une approximation décimale donnée}
 
 \begin{enumerate}
 \item Calculer le quotient entier, le quotient à 0,1 près, le quotient à 0,01 près, et le quotient à 0,001 près de : 
 \begin{itemize}
 \item 39 par 7 ;
 \item 293,72 par 43 ; 
 \item 735,7 par 40,1. 
 \end{itemize}
 \item Le diviseur d'une division est 7,5 et le quotient à 0,001 près est 2,357. Que peut-être le dividende ? 
 \item Calculer à 1 décimètre près la largeur d'un rectangle dont la surface est 212 ares 7 centiares, sachant que sa longueur mesure 183, 7 mètres. 
 \item Calculer à 1 millimètre près les deux bases d'un trapèze dont la surface est 430 centimètres carrés, dont la hauteur mesure 147 millimètres, et sachant d'autre part que la différence des bases est 18 cm. 
 \item Couper par une corde de 27,55 m en 3 morceaux tels que le premier ait 2,50 m de plus que le second, et 1,25 de moins que le troisième.
 \item Un litre d'huile pèse 0,025 kg. Combien de bouteilles ayant une capacité de 0,83 litres pourra-t-on remplir avec 52 kg d'huile ? 
 \item Calculer le rayon du méridien terrestre sachant que sa longueur est de 20~004 kilomètres. De combien augmenterait ce rayon si la longueur du méridien terrestre augmentait de $1$ kilomètre ? 
 \item Calculer le quotient à un millième, près de : 
 \begin{itemize}
 \item $\frac7{11}$ par $\frac35$ ; 
 \item $12$ par $\frac57$; 
 \item $4,5$ par $\frac79$. 
 \end{itemize}
\item Calculer à 0,0001 près les quotients de 22 par 7 et de 355 par 113. 
\item Diviser 2 par 1,414, puis 3 par 1,732. 
\item Calculer à 0,00001 près le quotient de 1 par 3,1416. Le nombre 3,1416
étant une valeur approchée par excès de $\pi$, obtient-on ainsi une valeur approchée de $\frac1{\pi}$ par défaut ou par excès ?  
\item Calculer à un millionième près les quotients de 1 par 6, et de 1 par 3. Faire la somme des nombres obtenus, et la comparer à la somme des deux fractions $\frac16$ et $\frac13$. 
\item Calculer à un millionième près les quotients de 2 par 3 et de 1 par 6. Faire la différence des nombres obtenus, et la comparer à la différence des deux fractions $\frac23$ et $\frac16$. 
\item Dans une division, on a interverti les rôles du dividende et du diviseur et l'on a trouvé 0,857 comme quotient approché à 0,001 près par défaut. Calculer une valeur approchée par excès et une valeur approchée par 
défaut du quotient initial. Donner le quotient initial avec la meilleure
approximation possible. 
\item Trouver tous les nombres entiers qui, divisés par 658 donnent 0,87 comme quotient approché à $\frac1{100}$ près par défaut. 
\item Calculer à $\frac1{1~000}$ près les densités de l'oxygène et de l'azote
sachant qu'un litre d'air pesant $1,293$ gramme contient $0,209$ litre d'oxygène et $0,791$ litre d'azote et que dans un gramme d'air il y a 231 milligrammes d'oxygène et 769 milligrammes d'azote. 
\item Dans une installation de lumière électrique on a utilisé 678 mètres de 
fil de cuivre recouvert de caoutchouc. 
Quand on observe un morceau bien tendu de ce fil, on voit la surface extérieure d'un cylindre de caoutchouc dont le rayon est de 2 mm. Le fil de cuivre a un diamètre de 12 dixièmes de mm. On demande : 
\begin{itemize}
\item La masse exprimée en kg du cuivre contenu dans le fil employé. 
\item Le volume exprimé en cm${}^3$ du caoutchouc qui recouvre ce cuivre. 
\end{itemize}
\item Avec 51 kg de cuivre, on fabrique du fil électrique de deux diamètres différents : 12 et 16 dixièmes de mm. La longueur du fil fin est double de celle du fil qui est le plus gros. Quelles longueurs de chaque fil pourra-t-on obtenir ? La densité du cuivre est 8,8. On prendra $\frac{355}{113}$ comme valeur approchée de $\pi$. 
 \end{enumerate}
 
 \part{Arithmétique littérale}
 
 \chapter{Somme généralisées}
 \begin{enumerate}
 \item Supprimer les parenthèses dans les sommes suivantes sans effectuer les opérations : 
 \[ 12 + (11 - 7 + 3) ; \phantom{meowmeow} 15 - (13 - 5 + 2).\]
 \[ 14 - ( 5 - 3 + 7); \phantom{meowmeow} 19 + (12 - 8 - 1). \]
 \[ 17 + ( 8 - 5 + 4) - (13 -3); \phantom{meowmeow} 14 + (9 - 4) - (11 + 2 - 7).\]
 \[(9 + 4 - 5) + (7 - 2 - 3); \phantom{meowmeow} 13 - (6 + 3 - 4) + ( 8 - 5)\]
 \[ 12 + ( x - 7 + 5); \phantom{meowmeow} m - (n - 4) + (p - 9).\]
 \[ a - (b + c - 8); \phantom{meowmeow} 25 + (a - 7)  - (b + 5).\]
 \[ 12 - (x - 5) + (x - 9); \phantom{meowmeow} 17 - (x + 3) + (y - 10).\]
 \item Réduire les sommes suivantes : 
 \[ (x + 13) + (x - 9) - (x - 6); \phantom{meowmeow} 
 (a + 10) - (a - 5) + (a - 9). \]
 \[ x - (x - 7)  + (x - 9); \phantom{meowmeow} 
 a + (a - b + 7) - (a + b). \]
 \[ x + 13 - (x + y + 10) + (x - y); \phantom{meowmeow} 
 m - n - (7 - m) + (n - 9).\]
 \item Mettre sous forme de produits les expressions suivantes : 
\[ 15 x -10 a + 25; \phantom{meowmeow} 32 a - 12 b + 16; \phantom{meowmeow}
14a + 7b - 21.\]
\[ 3ax-2bx + 5x; \phantom{meowmeow} ax + ay - 3a; \phantom{meowmeow}
5ax - 15 bx - 10x.\]
\[ 9ax + 6bx - 15x; \phantom{meowmeow} 6ax - 9ay - 15 a; \phantom{meowmeow}
\frac52 ax - \frac{15}2bx + \frac{25}2 x. \]
\item Réduire les sommes suivantes : 
\[25x - 13 x + 4x - 7x; \phantom{meowmeow} 13 ab - 11 ab + 3ab. \]
\[ 7x - 3 + 4x +7 - 6x; \phantom{meowmeow} 5x + 7 - 6x - 3 - 2x + 5.\]
\[ 17 ax - 5ax - 9ax + 4ax ; \phantom{meowmeow} 8,4x + 6 - 5,2x + 5,6 - 2.\]
\item Réduire les expressions suivantes : 
\[ (6x - 8) + (7x - 5) - (4x + 10); \phantom{meowmeow} (3x -7) - (7x - 15)
+ (2x + 3).\]
\[ (3x + 2a) - (4x + 6a) + (2x + 5a); \phantom{meowmeow} (2x - y) - (x - 5z)
- (4z - 2y). \]
\[ x + (6z - 2y) - (x - 3y + 5z); \phantom{meowmeow} (3x - y - z) - (x - 3y + z) - (x + y - 3z).\]
\[ 4(2x - 3) - 2(3x - 7); \phantom{meowmeow} 3(4x - 2)- 5(3x - 2) - 2(x -1).\]
\[2(3x + 4y - 5) - 3(x + 2y - 3) ; \phantom{meowmeow} 2x(a + 3) - a(2x +  5)
+ 3(2x -a).\]
\[7(x - 2y + 8) - 5(x + 4y + 5); \phantom{meowmeow} 8(3a + 5b - 9)
- 5(4a + 7b - 12). \]
\item Effectuer :
\[ (5 + 7)\times(8 + 3); \phantom{meowmeow} (12 + 3 + 7) \times (6 + 3).\]
\[ (9 - 5) \times (7 + 2) ; \phantom{meowmeow} (14 - 9)\times (9 - 2).\]
\[ (12 - 5 + 7) \times (17 - 9 + 1) ; \phantom{meowmeow} (15 - 7 + 3 - 9) \times (21 + 4 - 8). \]
\[ (2a + 3b - 5) \times (x + 3); \phantom{meowmeow} (4x - 5y + 8) \times (3a -1).\]
\[ (12a - 5b + 4c) \times (3x + 2y - z).\]
\item Effectuer et réduire s'il y a lieu : 
\[ (a + 3)\times(b -5)+ (x + 2)\times(y + 6) - (a + 4)\times b - xy - 2y.\]
\[(7a - 5)\times(x - 2) - (8 - a)\times(4 - x).\]
\[(3x + 5y)\times 6 - (4 - x)(9 - y) + xy + 20.\]
\[ (2a + 3)(a - 3) + (a + 5)(a - 2) - 3(a + 1)(a - 1).\]
 \end{enumerate}
 
 \chapter{Résolution littérale de problèmes}
 \begin{enumerate}
 \item Trouver un nombre dont le produit par 11 surpasse de 108 le produit de 
 ce même nombre par 8. 
 \item Le produit d'un nombre par 9 diminué de 23 surpasse de 19 le produit de ce même nombre par 6. Quel est ce nombre ? 
 \item Un père a 30 ans, son fils a 4 ans. Dans combien d'années l'âge du père sera-t-il le triple de l'âge de son fils ?
 \item Un père âgé de 37 ans a trois enfants âgés respectivement de 3, 7 et 11 ans. Dans combien d'années l'âge du père sera-t-il égal à la somme des âges de ses enfants ? 
 \item Deux trains séparés par une distance de 310 km marchent à la rencontre l'un de l'autre. Le premier fait 90 km à l'heure et le deuxième 65 km à l'heure. Dans combien de temps se fera la rencontre ? 
 \item Un cycliste qui roule à la vitesse de 30 km a 60 km d'avance sur un motocycliste qui le suit à la vitesse de 50 km à l'heure. On demande combien de temps il faudra à ce dernier pour rejoindre le cycliste. 
 \item Un épicier achète de l'huile 2,10 F le litre. Il la revend 2,70 F le litre et fait ainsi un bénéfice de 138 F. Quelle quantité d'huile avait-il achetée ? 
 \item Deux pièces de la même étoffe valent l'une 2~340 F, l'autre 5~040 F. Sachant que la première mesure 3 mètres de moins que la seconde, trouver le
 prix du mètre de cette étoffe.
 \item Un marchand a acheté 5 pièces de vin pour 575 F. En revendant ce 
 vin 805 F, il fait un bénéfice de 0,20 F par litre. Trouver la contenance de chaque pièce. 
 \item Un libraire achète des casiers 8 F la douzaine. Il les vend 1 F pièce et gagne ainsi 480 F. Combien de douzaines de cahiers a-t-il vendues ? 
 \item Un chemisier achète des chemises 108 F la douzaine, mais il en 
 reçoit une treizième par douzaine achetée. Il les revend 12 F la pièce et
 fait ainsi un bénéfice de 192 F. Combien de douzaines de chemises a-t-il acheté ?
 \item Une pièce d'étoffe vaut 675 F. On la diminue de 7 mètres. Elle ne vaut plus alors que 486 F. Quelle est la valeur du mètre de cette étoffe ?
 \item La somme de deux nombres impairs consécutifs est égale à 124. Quels sont ces deux nombres ? 
 \item Trouver deux nombres sachant que le premier surpasse de 8 le deuxième et que, si on l'augmente de 14, il est égal au triple du second. 
 \item La somme de trois nombres est égale à 388. Trouver ces trois nombres
 sachant que le premier surpasse de 11 le second et est inférieur de 15 
 au troisième. 
 \end{enumerate}
 
 \chapter{Égalités et équations}
 \begin{enumerate}
 \item Vérifier en calculant séparément la valeur de chacun des deux membres les égalités suivantes : 
 \[ 7(8 - 5) - 4(13 - 8) = 18 \left(4 - \frac72\right) - 2(7 - 3).\]
 \[ 15 - [13 - (9 - 5)] = 3[7 - (8 - 3)].\]
 \[ 6(2\times7 - 5) + 49 = 105 - 10(7 - 4) + 7(9 - 5).\]
 \[ 11(3 \times 5 - 4) + 7(8 - 5) = 9(13 - 7) + 10(8 + 5) - 42.\]
 \[45 - 7(13 - 8) + 5\times3 = (13 - 9 + 1)(19 - 14).\]
 \item Montrer que les égalités suivantes sont vérifiées si l'on donne à la lettre $a$ la valeur 4. (Remplacer $a$ par $4$ et calculer la valeur des deux membres.)
 \[ 231 - 12a - 81 + 56a = 31a + 242 - 10a.\]
 \[ 4(a + 3) - 3(a + 5) = 2(a -2) - 3(a -3).\]
 \[\frac{5a}4 + \frac72 - \frac{7a}8 = \frac{a}4 + 4.\]
 \[ \frac{3a - 2}2 - \frac{2a - 4}3 = \frac{4a - 1}3 - \frac{a + 4}6.\]
  \item Vérifier pour $a=5$ et $b=3$, les égalités suivantes : 
  \[ 2a + 3b - 6 = 4ab - 7a - 4b.\]
  \[ 5(8a - 5b) + 8 = 7(11b - 4a + 6).\]
  \[ 9(a + 6) - 4(11 - 2b) = 5(3a + 4) - 8(3b - 7).\]
  \[ \frac{2(2a + b)}{13} + \frac{a + 3}2 + \frac{b + 2}5 = \frac{ab}5 - (a - 5) + 4.\]
  \item Réduire et simplifier les égalités suivantes : 
  \[ 9a - 5b - 5 = 6a - 8b + 1.\]
  \[ 6(2a + b) - 5(a - 4) = 7(a + b) + 15.\]
  \[ \frac{a - 3}4 + \frac{3(b - 5)}2 - 3a = 4 - \frac{3a -4}2 + \frac{b - 7}4.\]
  \[ \frac{3a + 2b}2 - \frac{5a + 4b}8 = \frac{7a + b}4 + 3(a -2b) + 4.\]
  \item Résoudre les équations suivantes :  
  \begin{center}
  \begin{tabular}{l p{.5cm}  l}
  $5x + 4x - 7x = 14 - 11 + 5$ && $ 9 x - 5x + 2x = 33 - 13 + 22 $ \\
  $ 19x - 11x - 5x = 57 - 29 + 11$ && $ 6x - x + 4x = 454 - 147 - 181$\\
  $24x - 11x + 7x - 8x = 472 - 57 - 42 - 25$ && $152x - 73x - 56x + 86x = 336 + 76 - 187 + 102$ \\
  $9x - 5 - 6x = 14 + 8$ && $10x + 14 - 6x = 23 + 11$\\
  $ 11x - 25 = 5x + 59$&&$ 13x - 52 = 43 - 6x$\\
  $ 2x + 7x + 23 = 12x - 22$ && $ 42x + 41 - 17x + 23 = 182 + 13x$ \\
  $ 37x - 160 + 24x = 44x - 41$ && $ 235 - 23x - 87 + 112x = 67x + 242 - 25x$ \\
  $3x + 1 - 2,4x = 11,6 - 3,7x - 2 $ && $ 13,5 + 3 - 4x = 13,8 + 2,5 x - 5,2$
   \end{tabular}
   \end{center}  

 \begin{tabular}{l p{.5cm}  l}  
  $ 28,8x - 2,45 + 18,3x = 34,4 - 19,9x$ && $27,6x - 17,2 - 14,83x = 90 - 30,23x - 5,29 $ \\
 $ 2(x - 1) + 7 = 5(x - 3) - 4$ && $  3(x + 1) + 6 = 2(4x - 7) - 12$\\
   $ 8(2x + 3) - 7 = 3(7x - 5) + 17$ && $ 16 - 2(x + 5) = 3(3x + 1) - 31$ \\
   $ 4(x + 3) - 2(x - 2) = 3(x + 5) - (3x - 9)$ && $5(5x + 7) - 3(3x + 5) = 2(4x + 9) + 7(x + 1)$ \\
   $  3(2x + 5) + 5(x + 3) = 4(4x + 7) - (x + 6)$ && $ 9( x + 4) - 4(12 - x) = 5(3x - 2) - 8(x - 5)$ \\
   $17(x - 7) - 3(x + 10) = 5(2x + 3) - 8(x + 4)$ && $ 13(x - 8) + 7(2x -19) = 8(x + 5) - 5(19 - x)$ \\
  $ \frac52 x + 3 - \frac{7x}4 = x + \frac94$&&
  $ \frac{7x}4 - 2 - \frac{x}2 = \frac{2x}3 + 12$\\
  $ \frac{x}4 + \frac32 - \frac{x}3 = \frac{x}6 - 1$&&
  $ x + \frac12 - \frac{x}6 = 16 - \frac{2x}6 - \frac13$\\
  $ \frac{x}3 - \frac{x}4 + 2 = \frac{x}6 + \frac{10}3 - \frac{x}4$&&
  $ \frac{5x}2 - \frac{18}5 - \frac{3x}4 = \frac85x + \frac94$\\
  $ \frac{2x}3 + 4 - \frac{2x}5 = \frac{x}2 - \frac{x}3 + \frac{11}2$&&
  $ \frac{3x}7 - \frac{2x}{15} + 3 = \frac{x}3 + \frac{11}5 $\\
  $ \frac{x}2 + \frac52 - \frac{x}3 = \frac{2x}3 + \frac{x}6 - \frac56$&&
  $ \frac{5x}4 - \frac12 - \frac{7x}8 = \frac{x}2 - 1$\\
  $ \frac{x + 2}2 - \frac{x}4 = 3 $&&
  $ \frac{x+5}2 - \frac{x}3 = \frac{x}4 + \frac{25}{12}$\\
  $\frac{x + 5}4 - \frac{x - 3}6 = \frac{x}3$&&
  $ x - \frac{x +1}3 = \frac{2x+1}5$ \\
  $\frac{3x - 1}2 + \frac{x-2}3 = 8$&&
  $ \frac{2x+3}5 - \frac{13-x}2 = 4$\\
  $ \frac{3x-7}2 + \frac{x-1}3 = 20$&&
  $ \frac{x+3}4 - \frac{x-2}3 = \frac{x+1}6 $\\
  $ \frac{5x-3}4 - \frac{3x-5}8 = 6 $&&
  $ \frac{3x+5}2 + \frac{x+3}3 = 9$\\
  $ \frac{5x+3}4 - \frac{7x+4}8 = \frac{x}2$&&
  $ \frac{5x-7}4 = \frac{7x-10}8 + \frac{x-2}2$\\
  $ \frac{7x-3}2 - \frac{3x+4}4 = \frac{8x-5}3$&&
  $ \frac{7x+2}2 - \frac{8x+2}3 = \frac{3x+2}4 $\\
  $ \frac{2x+7}5 - \frac{11-x}2 = 4 $&&
  $ (x+1)-\frac{x+2}3 = \frac{2x+3}5$\\
  $ \frac{5x+1}2 - \frac{3x+1}4 = \frac{8x+5}5 $&&
  $ \frac{15x+7}4 - \frac{9x+5}8 = \frac{3(4x+3)}5$\\
  $	\frac{8x+9}5 + \frac{3x-1}4 = \frac{5x+1}2 $&&
  $ \frac{2x+3}3 - \frac{x-3}6 = \frac{x+13}4$\\
  $(x - 2) - \frac{x-5}6 = 16 - \frac{2x - 7}9$&&
  $ \frac{5x-3}2 - \frac{3x-1}4 = \frac{2(4x+1)}5$\\
  $ \frac{2x+1}3 + \frac{3x+1}4 = 28 - \frac{5x-2}7 $&&
  $ \frac{5x+1}8 - \frac{x-1}3 = \frac{4(2x-3)}9 $\\
  $ \frac{7x+2}2 - \frac{3x+10}4 = \frac{8x+5}3 - \frac52 $&&
  $ \frac{7x - 5}2 - \frac{8x-6}3 = \frac{3x+7}4 - 2$\\
  $ \frac{x+6}2 - \frac{x}3 = \frac{2x+7}3 - \frac{x+4}6$&&
  $\frac{5x-4}2 + \frac{7x-16}4= \frac{3(x+12)}4 - 1 $\\
  $ \frac{7 - 3x}{12} + \frac34 = 2(x-2) + \frac{5(5-2x)}6 $&&
  $\frac{3(x+3)}4 + \frac12 = \frac{5x+9}3 - \frac{7x-9}4 $\\
  $ \frac{8x+2}5 - 2 = \frac{5x-3}2 + \frac{3x-1}4$&&
  $ \frac{4x+7}5 - \frac{5(x-1)}6 = \frac{14}3 - \frac{2x-7}9$\\
  $ \frac{5x+7}4 - \frac{3x+5}8 = \frac{4x+9}5 $&&
  $ \frac{5x-1}8 - \frac{2x+7}{11} = \frac{4x-7}9$\\
  $ \frac{4(2x+3)}5 - \frac{3x+2}7 = \frac{7x-11}4 - \frac32 $&&
  $ \frac{7x-1}8 + \frac{2x+9}3 = \frac{2(6x+1)}7 $\\
  $ \frac{5x+3}7 - \frac{4x-3}9 = \frac{3x+8}{11} $&&
  $ \frac{3(x+2)}{13} + \frac{2x-3}5 = \frac{5(x-3)}7 $\\
  $ \frac{4x-1}3 - \frac{7x-3}{11} = \frac{5x-2}7 $ &&
  $ \frac{5x-7}8 + \frac{2(x+2)}7 = \frac{4x+9}5$\end{tabular}
  
  
 
 
 \end{enumerate}
 
 \chapter{Application aux problèmes}
 \begin{enumerate}
 \item Trouver un nombre dont la somme des quotients par 
 4 et par 7 soit égale à 55. 
 \item Trouver un nombre dont la somme des quotients par 4, 6 et 8 soit égale à 78.
 \item Les trois quarts d'un nombre surpassent les deux tiers de ce nombre de 17. Trouve ce nombre.
 \item Trouver un nombre dont les $\frac45$ surpassent de 14 les $\frac23$ de ce même nombre. 
 \item En retranchant 18 aux $\frac34$ d'un nombre on trouve le même résultat que si l'on avait ajouté 7 au 
 $\frac13$ de ce nombre. Trouver ce nombre.
 \item Trouver un nombre tel que si on en ajoute les $\frac9{14}$ aux $\frac23$ le résultat surpasse de $50$ les $\frac56$ de ce même nombre.
\item Un marchand a acheté deux tonneaux de vin de même 
 contenance pour 270 F. Lors de la mise en bouteille, il 
 en perd 15 litres du premier tonneau qu'il revend 0,80 F 
 le litre et 25 litres du second qu'il revend 0,90 F le 
 litre. Il fait ainsi un bénéfice de 78 G. Trouver la 
 capacité de chacun des tonneaux. 
\item Une libraire fait éditer trois livres à un même 
 nombre d'exemplaires chacun ; le premier est vendu 3,60 
 F, le deuxième 3 F et le troisième 4,50 F. Il reste 1~300 
 exemplaires invendus du premier, 980 du deuxième, et 640 
 du troisième. Sachant que la vente a rapporté 45~000 F, 
 on demande à combien d'exemplaires chacun de ces livres a 
 été édité.
 \item Un marchand achète deux pièces du même drap, l'une 
 de 64 mètres, l'autre de 50 mètres. Il fait un bénéfice 
 de 96 F par mètre sur la première et de 105 F par mètre 
 sur la seconde et retire ainsi de sa vente 51~750 F. 
 Combien a-t-il payé le mètre de drap ? 
 \item Si un vigneron vend son vin 68 F l'hectolitre, il 
 lui restera 320 F après avoir acheté un champ. Mais s'il 
 ne le vend que 60 F l'hectolitre, il lui manquera 48 F. 
 Combien d'hectolitres de vin a-t-il récoltés ? 
 \item Un ouvrier calculer qu'il dépense par mois les 
 $\frac23$ de son salaire plus 48 F. Sachant qu'il 
 économise ainsi 102 F, trouver son salaire mensuel. 
 \item Un fermier espère payer son propriétaire avec 
 le prix de sa récolte de blé. S'il la vend 26,50 F le 
 quintal, il lui restera 195 F. S'il ne la vend que 25 F, 
 il lui manquera 225 F. Combien de quintaux de blé a-t-il 
 récoltés ? 
 \item Un marchand a acheté des moutons 93 F l'un et les 
 revend 105 F au marché de la Villette à Paris. Le 
 transport lui revient à 97,50 F et l'un des moutons 
 est mort en route ; il gagne malgré tout 241,50 F. 
 Combien de moutons avait-il acheté ? 
 \item Un litre de lait pur pèse 1,033 kg. Une laitière a acheté 50 litres de lait et trouve qu'ils ne pèsent que 51,485 kg. Quelle quantité d'eau contient ce lait ? 
 \item On a acheté une pièce d'étoffe à raison de 20 F les 3 mètres. On la revend à raison de 60 F les 7 mètres. On fait un bénéfice de 100 F. Quelle est la longueur de 
 la pièce ? 
 \item On veut placer des élèves dans une salle de projection. En mettant dix élèves par banc, il y en a 11
 qui ne sont pas placés. En mettant 11 élèves par banc, il reste alors 7 places disponibles. Quel est le nombre 
 d'élèves ? 
 \item Un groupe d'enfants doit faire une excursion qui 
 revient à 150 F chacun. Au moment du départ, trois d'entre eux sont absents et chacun des autres doit payer 
 15 F en plus. Quel était le prix de revient total de 
 l'excursion. 
 \item Un enfant veut placer ses billes en tas égaux contenant chacun 12 billes. Il lui reste alors 16 billes.
 En mettant 3 billes de plus par tas, il lui manque 5 billes. Combien de billes possède-t-il ? 
 \item Un chemisier a acheté des chemises à 9 F la pièce. Il en vend la moitié à 12 F pièce, le tiers à 11,40 F et le reste à 9,60 F. Il fait ainsi un bénéfice de 115,20 F. Combien de chemises a-t-il vendues ? 
 \item Un marchand a acheté une pièce d'étoffe à 7,20 F 
 le mètre. Il en revend 13 mètres à 9 F puis les $\frac35$ du reste à 9,60 F le mètre et le nouveau reste à 7,50 F le mètre. Il fait ainsi un bénéfice de 109,20 F. Trouver la longueur de cette pièce. 
 \item On mélange du café à 7,20 F le kg avec du café à 8,40 F le kg. Quelle quantité du premier doit-on prendre pour 10 kg du second, si l'on veut obtenir un mélange qui revienne à 8 F le kg. 
 \item  Deux trains partent en même temps, le premier de Paris, le second de Tours, se dirigeant l'un vers l'autre. Le premier marche à 90 km à l'heure, et le deuxième à 80 km à l'heure. Sachant que la distance Paris-Tours est de 238 km, on demande à quelle distance de Paris ils vont se croiser. 
 \item Un express part de Paris à 8h30 et se dirige vers le Mans à la vitesse de 75 km à l'heure. À 9h01, un rapide part du Mans vers Paris et marche à 84 km à l'heure. La distance Paris-Le Mans étant de 211 km, trouver l'heure de la rencontre. 
 \item Deux nombres ont pour différence 25 et leur somme est égale à 109. Quels sont ces deux nombres ? 
 \item La différence de deux nombres est 14, et le double
 du plus grand surpasse de 5 le triple du plus petit. Quels sont ces deux nombres ? 
 \item Trouver trois nombres impairs consécutifs, sachant que leur somme est égale à 141. 
 \item Trouver les dimensions d'un rectangle sachant que le périmètre est égal à 272 mètres et que la longueur est les $\frac53$ de la largeur. 
 \item La longueur d'un champ rectangulaire est inférieure de 15 mètres au double de la largeur. Trouver sa surface sachant que le demi-périmètre est égal à 186 mètres. 
 \item Trois enfants ont ensemble 33 ans. L'âge du premier dépasse de 2 ans l'âge du deuxième et est le double de l'âge du troisième. Trouver l'âge de chacun d'eux. 
 \item La somme de deux nombres est 496. En les divisant l'un par l'autre, on trouve 6 pour quotient entier et 48 pour le reste. Trouver ces deux nombres. 
 \item La différence de deux nombres est 516. Le quotient entier de ces deux nombres est 13 et le reste de leur division est 24. Trouver ces deux nombres.
 \item Deux enfants ont ensemble 105 billes. Si le premier en avait 15 de plus, il en aurait trois fois plus que le second. Combien de billes ont-ils chacun ? 
 \item On écrit 3 nombres à la suite l'un de l'autre.
 Chacun d'eux surpasse de 5 le double de celui qui le 
 précède. Sachant que leur somme est égale à 160, trouver ces trois nombres. 
 \item La somme de deux nombres est 162. En ajoutant 13 à chacun d'eux, l'un d'eux devient le triple de l'autre. Trouver ces deux nombres. 
 \item Partager une somme de 1~850 F entre 3 personnes,
 sachant que la première reçoit 250 F de moins que la deuxième et deux fois moins que la troisième.
 \item Un cultivateur vend pour 994 F du blé à 26 F le quintal et de l'avoine à 15 F le quintal. Sachant que le poids de l'avoine est le triple de celui du blé, combien de quintaux de chaque sorte a-t-il vendus ? 
 \item Un cycliste qui fait 30 km à l'heure rejoint au bout de 1 h 20 min un piéton qui marche à 6 km à l'heure. 
 Quelle était l'avance du piéton au moment du départ du 
 cycliste ? 
 \item Une usine emploie 171 ouvriers. Le nombre de femmes est le tiers de celui des hommes et celui des enfants\footnote{Ahem.} la moitié de celui des femmes. Trouver le nombre d'hommes, de femmes, et d'enfants employés à l'usine.
 \item Un mètre de drap coûte 7,20 F de plus qu'un mètre de toile. Sachant que 10 m de drap et 12 m de toile coûtent ensemble 256,80 F, trouver le prix du mètre de chacune des deux étoffes.
 \item Un épicier vend 1 kg de café et 3 kg de sucre pour 13,20 F, puis une autre fois 9 kg de café et 13 kg de sucre pour 102 F. Quel est le prix du kg de café et celui 
 du kg de sucre ? 
 \item Une fermière vend 3 canards et 4 poulets pour 75 F. Sachant qu'un canard et un poulet valent 21 F, trouver
 le prix d'un canard et celui d'un poulet.
 \item Deux ouvriers gagnent à eux deux 30 F par jour. En un mois le premier a travaillé 24 jours et le deuxième 20 
 jours et ils ont reçu à eux deux 656 F. Quel est le salaire journalier de chacun d'eux ? 
 \item Un fabricant a vendu 5 mètres de toile et 10 mètres de drap pour 210 F; puis une autre fois 27 mètres de toile et 23 mètres de drap pour 631,80 F. Trouver le prix d'un mètre de toile et celui d'un mètre de drap.
 \item Un épicier vend 15 litres de liqueurs pour 96 F. L'une de ces liqueurs est vendue 5,40 F le litre, et l'autre 6,90 F le litre. Combien de litres de chaque sorte a-t-il vendus ? 
 \item Pour payer une somme de 890 F, on a donné 46 pièces, les unes de 20 F, les autres de 10 F. Combien de pièces de chaque sorte a-t-on données ? 
 \item Un négociant a vendu 37 quintaux de blé, les uns à 25 F, les autres à 27 F. Il a retiré 975 F de sa vente. Calculer le nombre de 
 quintaux de chaque sorte. 
 Une somme de 329~000 F doit être partagée entre trois personnes. La
 part de la deuxième est les $\frac43$ de celle de la première et celle
 de la troisième est la moitié de celle de la première plus 23~000 F.
 Calculer les trois parts.
 \item Une salle de cinéma comprend des places à 3 F, à 2,40 F, et à 2 F. Il y a deux fois plus de places à 2,40 F que de places à 3 F et le 
 nombre de places à 2 F est les $\frac5{11}$ du nombre total. La 
 salle complète fournit une recette de 1~792 F. Trouver le nombre 
 total de places.
 
 \end{enumerate}
 
 \chapter{Problèmes de révision}
 \begin{enumerate}
 \item En vendant son blé 27 F le quintal, un paysan peut acheter une maison et il reste 675 F. En vendant le quintal de blé 24,75 F, il achète la maison et il lui reste 78,75 F. Quel est le prix de la maison ? 
 \item Deux ouvriers travaillent dans le même atelier. Le premier gagne 0,90 F par jour de plus que le second : il a travaillé 25 jours et le 
 second 23 jours. Le premier a gagné 47,70 F de plus que le second. Quel est le salaire journalier de chacun ? 
 \item On a, pour 60,20 F, acheté 4 kg de café, 5 kg de sucre, et 3 kg de chocolat. Trouver le prix au kilogramme des trois denrées, sachant qu'un kilogramme de chocolat coûte 1,20 F de moins qu'un kilogramme de café et 6,20 F de plus qu'un kilogramme de sucre. 
 \item Un cycliste roule à 32 km à l'heure et part 2 minutes avant un second cycliste lancé à sa poursuite et dont la vitesse est 36 km à l'heure. Au bout de combien de temps le second cycliste rejoindra-t-il 
 le premier ? 
 \item Un père a trois enfants âgés respectivement de 12, 10 et 8 ans. Le père a 36 ans. Dans combien d'années l'âge du père sera-t-il égal à
 la somme des âges de ses trois enfants ?
 \item Un marchand achète du vin qu'il revend avec un bénéfice égal aux $\frac{25}{100}$ du prix de vente. En gagnant 120 F de moins, il réaliserait un bénéfice égal au $\frac15$ du prix d'achat. Quel est
 son bénéfice ? 
 \item Une somme de 1~420 F est composée de 34 billets, les uns de 50 F, les autres de 10 F. Quel est le nombre de billets de chaque 
 sorte ? 
 \item La distance de deux villes est de 840 km. Une automobile parcourt cette distance en 13 h. Une partie du trajet est faite à la vitesse moyenne de 60 km à l'heure, l'autre à la vitesse moyenne de 80 km à l'heure. Quelles sont les deux parties de ce trajet ?
 \item  On achète une première fois 4 kg de café et 3 kg de sucre pour 33,25 F et une seconde fois 3 kg de café et 2 kg de sucre pour 24,70 F. Quel est le prix du kilogramme de café et celui du kilogramme de sucre ? 
 \item Deux cyclistes roulent sur une piste circulaire de 480 m de tour. Quand ils roulent dans le même sens, le premier dépasse le 
 second toutes les 3 minutes. Quand ils roulent en sens contraire, ils se croisent à intervalles réguliers de 24 secondes. Trouver la 
 vitesse de chaque cycliste. 
 \item Un train a mis 36 secondes à passer devant un observateur immobile. Sa longueur est 300 m. Quelle est sa vitesse ? Un second 
 train met 24 secondes pour croiser le premier et passe devant l'observateur immobile en 18 secondes. Trouver la longueur et la 
 vitesse de ce second train. 
 \item Un capital est placé à 6\% pendant 18 mois. S'il était placé à 5\% pendant 2 ans, les intérêts augmenteraient de 2~500 F. Quel 
 est ce capital ?
 \item Un cycliste et un piéton partent en même temps et dans le même sens de deux points A et B distants de 36 km. La vitesse du cycliste vaut 5 fois celle du piéton. À quelle distance de A le cycliste attendra-t-il le piéton ? À quelle distance de A était le cycliste lorsqu'il avait sur le piéton un retard de 10 km ?  
 \item Deux capitaux dont l'un est les $\frac34$ de l'autre sont placés pendant 18 mois au même taux 5 \%. La somme totale ainsi obtenue (capitaux et intérêts réunis) est 451~500 F. Quels sont ces deux capitaux ? 
 \item 15 litres de lait coupé d'au pèsent 15,150 kg. La densité du lait pur étant 1,03, combien ce lait contient-il de litres d'eau ? 
 \item Deux cyclistes partent en même temps de deux villes A et B distantes de 100 km et vont à la rencontre l'un de l'autre. Ils 
 se rencontrent au bout de 2 h. Si le cycliste qui part de A était parti 20 minutes avant l'autre, la rencontre aurait eu lieu
 $\frac{46}{25}$ d'heure après le départ du second cycliste. Trouver la vitesse de chacun d'eux. 
 \item Un bassin contient de l'eau jusqu'au $\frac6{11}$ de sa hauteur. On y verse encore de l'eau jusqu'à ce que le niveau atteigne les $\frac78$ de la hauteur. Calculer la hauteur du bassin sachant que le niveau s'est élevé de 58 cm.
 \item Un terrain a été partagé en trois parties : les $\frac25$ ont été plantés en vigne, le $\frac13$ a été ensemencé en blé et le reste en luzerne. Sachant qu'il y a 18 ares de différence entre la vigne et la luzerne, trouver la surface totale du terrain et
 la surface de chacune des parties. 
 \item Un travail a été exécuté par trois ouvrières. La première en fait les $\frac4{13}$, la deuxième les $\frac56$ de ce qu'a fait la 
 première, et la troisième fait le reste. Sachant que cette dernière a touché 75 F de moins que les deux autres réunies, on demande de calculer le prix du travail.
 \item Un champ est partagé en trois parties : la première est égale aux $\frac38$ du total, la deuxième aux $\frac9{11}$ de la première. Sachant que la dernière partie surpasse de 17 centiares la seconde, trouver la surface de chacune de ces trois parties.
 \item Une propriété comprend des bois, de la vigne et des prairies. La surface de la vigne est égale aux $\frac34$ de celle des bois, et celle des prairies est les $\frac6{13}$ du total. Trouver la surface des bois, celle de la vigne et celle des prairies, sachant que la surface des prairies dépasse de 3,14 ha celle des bois. 
 \item Une personne dépense le $\frac15$ de son argent dans un magasin, les $\frac37$ du reste dans un autre. Dans un troisième elle voudrait bien acheter $22$ m de toile à 30 F le mètre, mais il lui manque 52 F. Quelle somme avait-elle emportée ?
 \item Un marchand a vendu à un premier client $\frac15$ d'une pièce d'étoffe, puis à un deuxième $\frac13$ du reste, et à un troisième le quart du nouveau reste. Il lui reste 36 m d'étoffe. Quelle était la longueur de la pièce initiale ? 
 \item Trois enfants se partagent des fraises de la manière suivante. Le premier en prend le $\frac13$, le deuxième le $\frac13$ du reste, et le troisième le $\frac13$ du nouveau reste. Le reste final est enfin partagé également entre eux et chacun reçoit alors 40 fraises. Trouver le nombre de fraises total. 
 \item Un marchand vend une pièce d'étoffe. La première fois, il en vend les $\frac27$ plus 3 mètres à 32 F le mètre ; la seconde fois, il en vend les $\frac25$ moins 8 mètres à 30 F le mètre. Il reçoit ainsi 1~336 F. Quelle était la longueur de la pièce et combien de 
 mètres lui reste-t-il ? 
 \item Dans un théâtre il y a $\frac16$ des places à 4 F, $\frac14$ à 3 F, 450 places à 2,40 F et le reste à 1,60 F. La recette maximum
 possible est 3~180 F. La recette maximum possible est 3~180F. Trouver le nombre total de places. 
 \item Un marchand vend un lot de chemises. Il en vend $\frac14$ plus 2 avec un bénéfice de 7 F par chemise, puis les $\frac25$ moins 3 avec un bénéfice de 5 F, et le reste avec un bénéfice de 6 F. Son bénéfice total est de 356 F. Trouver le nombre de chemises vendues. 
 \item Dans une classe, le $\frac13$ des élèves est âgé de 11 ans, la moitié plus 3 est âgée de 12 ans et le reste est âgé de 13 ans. Sachant que les élèves ont à eux tous 352 ans, trouver le nombre des élèves de la classe. 
 \item Une certaine somme est partagée entre 3 personnes. La première en reçoit le tiers, la seconde les $\frac49$ moins 1~360 F et la troisième les $\frac27$ moins 2~120 F. Trouver la somme partagée et la part de chacune. 
 \item Partager un somme de 41~450 F entre 3 personnes de façon que la première reçoive 2~500 F de plus que la deuxième et 1~350 F de moins que la troisième. 
 \item Deux cyclistes partent de deux villes distantes de 72 km et se dirigent l'un vers l'autre. Le premier fait 24 km à l'heure et le
 deuxième 30 km à l'heure. Quelle sera la distance parcourue par chacun d'eux au moment de la rencontre ? 
 \item Un express part de Paris à 8 h 30 et se dirige vers Le Havre à 80 km à l'heure. À 8 h 48 un rapide part du Havre pour Paris et marche à 90 km à l'heure. Sachant que la distance Paris-Le Havre est égale à 228 km, trouver à quelle heure et à quelle distance de Paris aura lieu la rencontre. 
 \item Une personne veut consacrer 3 heures à une promenade. Elle part en automobile à la vitesse de 70 km à l'heure et revient à 
 pied à 5 km à l'heure. À quelle distance du point de départ devra-t-elle descendre de l'automobile ? 
 \item Un cycliste roule pendant 1 h 40 min, puis prend pendant 20 minutes un train dont la vitesse est les $\frac53$ de la sienne. Il a parcouru au total 60 km. Quelle est la vitesse du cycliste ? 
 \item Un premier cycliste part à 9 h du matin et fait 24 km à l'heure. Un second cycliste part à sa poursuite à 9 h 48 et fait d'abord 36 km à l'heure. Mais il s'arrête 9 minutes et ne fait ensuite que 30 km à l'heure. Il rejoint le premier cycliste à 12 h 15. Pendant combien de temps le second cycliste a-t-il roulé à 36 km à l'heure ?
 \item Un train part d'une ville A à 7 h. Il arrive en B à 11 h 30. Il fait les $\frac35$ du trajet à une vitesse de 84 km à l'heure. Dans la seconde partie du trajet, sa vitesse est réduite à 70 km à l'heure. Trouver la distance entre A et B. 
 \item Un cycliste met 4 h 30 pour faire le trajet aller et retour d'une ville A à une ville B distante de 60 km. Il fait 30 km à l'heure en terrain plat, 36 km à l'heure en descente et 20 km à l'heure en montée. Trouver la longueur du terrain plat entre A et B. 
 Sachant qu'il met 12 minutes de plus à l'aller qu'au retour, trouver la longueur des montées et descentes de A vers B. 
 \item Un piéton marche pendant 3 h 40. Il monte alors dans une automobile qui a une vitesse égale aux $\frac{40}3$ de la sienne et qui 
 le dépose au bout de 10 minutes à 26,5 km de son point de départ initial. Trouver la vitesse du piéton. 
 \item Deux capitaux égaux, placés le premier à 4,5\% pendant 16 mois, le second à 4\% pendant 18 mois ont rapporté 1~800 F d'intérêt
 total. Trouver leur valeur commune initiale. 
 \item Deux sommes égales sont placées l'une à 4\%, l'autre à 5\% pendant 2 ans et 8 mois. La seconde a rapporté 800 F de plus que la première. Quelle est le montant de chacune des sommes placées ? 
 \item On place le $\frac13$ d'un capital à 3\%, les $\frac25$ à 3,5\% et le reste à 4\%. Au bout de 15 mois, on a touché 2~600 F d'intérêt. Trouver la valeur de ce capital. 
 \item Un capital est partagé en trois parts. La première est les $\frac45$ de la deuxième, et les $\frac23$ de la troisième. La première part est placée à 4 \%, la deuxième à 5\%, la troisième à 5,5\%. L'intérêt annuel est de 11~100 F. Quel est ce capital ?
 \item Une personne place les $\frac23$ de son capital à 3,5 \%, les $\frac25$ du reste à 4\% et le reste à 4,5\%. Ce dernier placement lui est remboursé au bout de huit mois et reste 4 mois improductif. Le revenu total au bout de l'année est 78 F. Trouver le capital.
 \item Un certain capital est placé à 6\% pendant 10 mois. Un autre égal aux $\frac23$ du précédent est placé à 5\% pendant huit mois. Trouver les deux capitaux, sachant que la différence des intérêts est de 2~500 F. 
 \item Une personne avait placé les $\frac56$ de son capital à 4 \% et le reste à 5 \%. Elle prélève 60~000 F sur ce capital et place le reste à 4,5 \%. Son revenu diminue de 1~200 F par an. Quel était son capital primitif ? 
 \item Une personne possède 45~00 F. Elle emploie une partie de ce capital à l'achat d'une propriété. Elle place les $\frac23$ du reste à 4\% et le tiers restant à 5\%. Ces deux placements lui assurent un revenu de 780 F. Trouver le prix de la propriété.
 \item Un propriétaire veut construire une usine. Il emploie les $\frac3{10}$ de sa fortune à l'achat du terrain qui lui revient à 7,20 F le mètre carré. Il consacre les $\frac47$ du reste à la construction des bâtiments. Il place alors les $\frac23$ de l'argent disponible à 4\% et le reste à 4,5\%, ce qui lui procure un revenu annuel de 450 F. On demande la fortune totale du propriétaire et la surface du terrain acheté. 
 \item Une personne a placé une certaine somme à 6 \% à intérêts simples. Au bout de 5 ans et 4 mois, elle retire, capital et intérêts réunis, une somme de 409~200 F. Quelle était la somme placée ? 
 \item Une personne place son argent à intérêts simples ; les $\frac25$ sont placés à 4 \%, le $\frac13$ à 5 \%, et le reste à 4,5 \%. Elle retire son argent au bout de 2 ans et 7 mois et achète une propriété de 429 mètres carré à raison de 4,30 F le mètre carré. Il lui reste alors 163 F. Trouver son avoir primitif. 
 \item On place les $\frac25$ d'un capital de 150~000 F à un certain taux et le reste à un taux supérieur de 1 \%. L'intérêt annuel étant 6~150 F, trouver les taux des deux placements. 
 \item Un capital de 924~000 F est placé à un certain taux pendant 21 mois. Un autre capital de 660~000 F est placé à un taux inférieur de 1 \% au précédent, pendant 15 mois. La différence des intérêts étant 47~850 F, trouver les taux des deux placements. 
 \item On a placé au même taux 240~000 F pendant 2 ans et 250~000 F pendant 2 ans et demi. Le second capital a rapporté 5~800 F d'intérêts de plus que le premier. Calculer le taux de ces deux placements.
 \item Une personne avait placé deux capitaux, l'un de 320 F, l'autre de 480 F au même taux de 3 \%. Elle retire capitaux et intérêts réunis 834 F. Sachant que le premier capital de 320 F est resté placé 20 mois, on demande la durée du second placement.
 \item Une personne place 210 F à 3\% et 18 mois plus tard 560 F à 4,5\%. Au bout de combien de temps les deux placements auront-ils rapportés des intérêts égaux ? 
 \item Une personne place 240 F à 4\%. 18 mois plus tard elle place une seconde somme d'argent à 6\%. Calculer le montant de cette somme placée sachant que 6 mois après les deux placements ont rapporté le même intérêt. 
 \item Les $\frac35$ d'un capital ont été placés à 4\% et le reste à 5 \%. Au bout de deux ans et demi, les intérêts se sont élevés à 77 F. Calculer ce capital. 
 \item Une personne avait placé les $\frac35$ de son capital à 3 \% et le reste à 4 \% ; elle retire les deux parts, prélève 105~000 F et replace le reste à 5\% ; son revenu annuel se trouve ainsi augmenté de 6~830 F. Quel était le capital primitivement placé ?
 \item Une personne place les $\frac37$ de son capital à 3\%, et le reste est placé à un taux différent mais rapporte annuellement le même intérêt que le premier placement. Calculer ce taux. Calculer ensuite le capital, sachant que la personne perçoit annuellement 12 F d'intérêt de plus que si elle avait placé tout son capital à 5 \%. 
 \item Calculer le montant d'un capital placé à 4\% pendant 219 jours, sachant que si l'on comptait l'année de 365 jours dans le calcul de l'intérêt, on trouverait 200 F de moins que si l'on compte l'année de 360 jours que l'on utilise habituellement.
 \item Calculer les montants de deux capitaux ayant pour différence 200 F sachant que le moins élevé, placé pendant un an à 4,5\% et l'autre placé pendant 2 ans à 3,75 \% ont rapporté ensemble 63 F. 
 \item Un cycliste effectue une promenade de 3 heures. Il parcourt la moitié du trajet à 18 km à l'heure, le tiers à 20 km à l'heure, et le reste à 15 km à l'heure. Trouver la distance parcourue par le cycliste.
 \item Un express part à 9 h 15 d'une ville A et arrive à 16 h 25 en B.  Il a parcouru la moitié de son parcours à la vitesse de 75 km à l'heure, les $\frac25$ du trajet à la vitesse de 72 km à l'heure, et le reste à la vitesse de 80 km à l'heure. La durée totale des arrêts a été de 42 minutes. Calculer la distance AB. 
 \item Un alliage d'argent et de cuivre au titre de 0,720 pèse 500 grammes. Combien faudrait-il ajouter d'argent pour élever le titre à 0,800 ?
 \item On a un lingot d'or de 1~200 grammes au titre de $\frac{11}{12}$. Quelle quantité de cuivre faudrait-il y rajouter pour avoir un alliage au titre de 0,880 ?
 \item  Un alliage de cuivre et d'argent pesant 4,75 kg est constitué par des volumes égaux de ces deux métaux. Trouver le poids de cuivre et le poids d'argent que contient cet alliage en admettant que leurs poids spécifiques sont respectivement 9 et 10. 
 \item On a fondu ensemble deux lingots d'argent de titres différents contenant respectivement 250 grammes et 175 grammes d'argent fin et l'on a obtenu un alliage ayant pour titre 0,85. Quels étaient les titres de ces lingots sachant que le second a apporté 2 fois plus de cuivre que l'autre ? 
 \item Un épicier mélange du café à 8,40 F le kg et à 9,60 F le kg; Combien doit-il en prendre de chaque sorte sachant qu'il veut obtenir 30 kg de mélange revenant à 8,80 F le kg ? 
 \item Un négociant mélange trois sortes de café qu'il veut vendre 9,36 F le kg. Ce mélange comprend 64 kg de café, acheté 8,64 F le kg, et deux autres sortes achetées 7,56 F et 6,30 F le kg. Quel poids faut-il prendre de ces deux dernières sortes, sachant que le poids du café à 7,56 F doit être les $\frac34$ du poids du café à 6,30 F et que le négociant veut gagner 20\% sur le prix de revient. 
 \item Deux cyclistes éloignés de 66 km vont à la rencontre l'un de l'autre. Le premier qui fait 12 km à l'heure part une heure plus tôt que le second qui fait 15 km à l'heure. On demande au bout de combien 
 de temps ils se rencontreront et quels trajets ils auront parcourus. 
 \item Un piéton marchant à 5km à l'heure et un cycliste roulant à 15 km à l'heure partent d'une ville A en même temps et vont dans la même direction. Arrivé dans une ville B située à 12 km de A, le cycliste y reste 20 minutes puis revient en A. On demande à quelle distance de A il croisera le piéton. 
 \item Un cycliste part de Paris à 7 h et va à Fontainebleau où il 
 reste deux heures puis revient à Paris où il arrive à 18 heures. Calculer la distance Paris-Fontainebleau sachant qu'il a fait 15 km à l'heure à l'aller et 12 km à l'heure au retour. 
 \item Un cycliste monte une côte à la vitesse de 8 km à l'heure et la descend à 15 km à l'heure. Trouver la longueur de cette côte sachant qu'il met 35 minutes de plus pour monter que pour descendre. 
 \item La route allant d'une ville A à une ville B, distante de 60 km, comprend d'abord une montée, puis une partie horizontale de 20 km et enfin une descente. Un cycliste dont la vitesse en montée est de 8 km à l'heure, et en descente de 15 km à l'heure va de A à B et revient en A. On demande combien il y a de km de montée, et combien de km de descente de A vers B, sachant que le cycliste a mis 1 h 10 de plus pour aller que pour revenir. 
 \item Deux bassins contiennent déjà l'un 210 litres d'eau, et l'autre 100 litres. Pour les remplir, on ouvre 2 robinets, qui versent l'un 7 litres par minute dans le premier bassin, l'autre 8 litres par minute dans le second. Au bout de combien de temps le contenu du second bassin sera-t-il les $\frac47$ du contenu du premier ? 
 \item Un paysan doit labourer deux champs de même superficie. Il laboure le premier à raison de 8 ares par heure, et le second à raison de 10 ares par heure. Calculer la superficie de chaque champ sachant 
 que pour labourer le second, il met 8 heures de moins que pour labourer le premier.
  
 \end{enumerate}
\chapter{Notions préliminaires} 
\begin{enumerate}
\item Vérification de la règle. 
\item Marquer deux points $A$ et $B$ sur une feuille de papier. Déterminer la droite $(AB)$ par pliage. 
\item Marquer trois points $A$, $B$, $C$ sur une feuille de papier. Déterminer les droites qui joignent deux de ces points. 
Même exercice avec quatre points $A$, $B$, $C$ et $D$. Combien de droites obtient-on dans ce cas ? Construire leurs points d'intersection ? 
\item Construire dans un plan $4$ droites se coupant deux à deux en des points distincts. Montrer que l'on obtient ainsi $6$ points d'intersection. Combien de droites nouvelles obtient-on en joignant ces points deux à deux ? 
\item On donne dans un plan $6$ points tels que $3$ quelconques d'entre eux ne soient pas alignés et on les joint deux à deux. Combien de droites obtient-on ? Montrer que le nombre de ces droites est égal à la somme des $5$ premiers nombres entiers ou au demi-produit de $6$ par $5$. Généraliser pour $n$ points. 
\item On considère deux droites $(Ox)$ et $(Oy)$ concourantes en $O$, situées dans un plan $P$. On joint par une droite un point $A$ de $(Ox)$ et un point $B$ de $(Oy)$. Montrer que tout point $M$ de la droite $AB$ est situé dans le plan $P$.\\ En déduire que si $A$ et $B$ se 
déplacent simultanément sur $(Ox)$ et $(Oy)$ la droite $(AB)$ engendre le point $P$.
\item On considère trois points $A$, $B$, $C$ d'une même droite et $3$ points $A'$, $B'$, $C'$ d'une seconde droite, distincte de la première. Les droites $(AB')$ et $(A'B)$ se coupent en $M$, les droites $(AC')$ et $(A'C)$ se coupent en $N$ et les droites $(BC')$ et $(B'C)$ se coupent en $P$. Vérifier que les trois points $M$, $N$, $P$ sont alignés.
\item Construire trois droites issues d'un même point $I$. Puis d'un point $O$ mener à ces trois droites deux sécantes. Soient $A$, $B$, $C$
les intersections des trois droites avec la première sécante ; $A'$, $B'$, $C'$ leurs intersections avec la seconde. Les droites $(AB')$ et 
$(A'B)$ se coupent en $M$, les droites $(AC')$ et $(A'C)$ en $N$ et les
droites $(BC')$ et $(B'C)$ en $P$. Vérifier que les $4$ points $M$, $N$, $O$ et $P$ sont alignés. 
\end{enumerate}
\chapter{Segments de droite}
\begin{enumerate}
\item Construire quatre points $A$, $B$, $C$ et $D$ situés dans cet ordre sur une même droite $(xy)$ sachant que $AB=CD=3\cm$ et $BC = 5\cm$. Comparer ensuite les segments $[AC]$ et $[BD]$. 
\item Quatre points $A$, $B$, $C$, $D$ sont situés dans cet ordre sur une droite $(xy)$. Construite ces quatre points sachant que $AC=BD = 9 \cm$ et $BC = 7\cm$. Comparer les segments $[AB]$ et $[CD]$. Montrer que $[AD]$ et $[BC]$ ont milieu $O$. 
\item Quatre points $A$, $B$, $C$, $D$ sont situés dans cet ordre sur une droite $(xy)$. Construire ces quatre points sachant que $AD= 14\cm$, $BC = 10\cm$ et que les segments $[AD]$ et $[BC]$ ont même milieu $O$. Démontrer les égalités $AB=CD$ et $AC = BD$. 
\item On porte bout à bout des segments avec $AB = 10\cm$, $BC=3\cm$, $CD=7\cm$. Écrire les inégalités que vérifient $AB$ et $BC$, $AB$ et $CD$, $CD$ et $BC$. Quelle est la mesure de $[AD]$ en prenant $AB$ pour unité, puis $BC$ pour unité, puis $CD$ pour unité ? 
\item Trois points $O$, $A$, $B$ sont situés dans cet ordre sur une 
droite $(xy)$ et $M$ est le milieu de $[AB]$. \begin{enumerate}
\item On donne $OA = 4\cm$ et $OB = 10\cm$. Quelle est la longueur $OM$ ? 
\item On donne $OA = a$, et $OB = b$. Démontrer les égalités : 
\[ OM = OA + AM; \phantom{meow} OM = OB - AM;\phantom{meow} OM = \frac{a+b}2.\]
\end{enumerate}
\item Trois points $A$, $O$, $B$ sont situés dans cet ordre sur une droite $(xy)$ et 
$M$ désigne le milieu de $[AB]$. \begin{enumerate}
\item On donne $OA=5\cm$ et $OB = 11\cm$. Quelle est la longueur de $[OM]$ ? 
\item On donne $OA = a$, et $OB = b\ (b>a)$. Démontrer les égalités : 
\[ OM = AM - OA ; \phantom{meow} OM = OB-AM; \phantom{meow} OM = \frac{b-a}2.\]
\end{enumerate}
\item Trois points $B$, $A$, $C$ sont situés dans cet ordre sur une droite $(xy)$. On désigne par $I$ et $J$ les milieux respectifs de $[AB]$ et $[AC]$. 
\begin{enumerate}
\item On donne $AB= 7\cm$ et $AC = 5\cm$. Quelle est la longueur de $[IJ]$ ? Comparer la longueur trouvée à celle de $[BC]$ ? 
\item Calculer les longueurs $BC$ et $IJ$ connaissant $AB = a$ et $AC = b$. 
\end{enumerate}
\item Trois points $A$, $B$, $C$ sont situés dans cet ordre sur une droite $(xy)$. On désigne par $I$ et $J$ les milieux respectifs de $[AB]$ et $[AC]$. \begin{enumerate}
\item On donne $AB=9\cm$ et $AC=13\cm$. Quelle est la longueur de $[IJ]$ ? Comparer la longueur trouvée à celle de $[BC]$. 
\item Calculer les longueurs $BC$ et $IJ$ connaissant $AB = a$ et $AC = b$. 
\end{enumerate}

\end{enumerate} 
\chapter{Angles}  
\begin{enumerate}
\item Transformer en grades les angles suivants et les construire : 
\begin{enumerate}
\item $45^o$; $30^o$; $60^o$. 
\item $120^o$; $150^o$; $135^o$. 
\item $40^o15'$; $36^o30'$.
\item $50^o17'$; $112^o17'$. 
\end{enumerate}
\item Transformer en degrés les angles suivants et les construire : 
\begin{enumerate}
\item $30$ gr; $50$ gr; $90$ gr. 
\item $120$ gr; $160$ gr; $190$ gr. 
\item $20,5$ gr; $37,7$ gr; $68,9$ gr. 
\item $42,25$ gr; $112,6$ gr; $148,4$ gr. 
\end{enumerate} 
\item Montrer sur des exemples que la différence de deux angles ne change pas si on leur ajoute (ou retranche) un même angle.
\item Découper dans une feuille de papier deux angles adjacents $\widehat{AOB}$ et $\widehat{BOC}$ mesurant respectivement $68^o$ et $42^o$. Construire par pliage leurs bissectrices $[OM)$ et $[ON)$. Mesurer l'angle $\widehat{MON}$ et le comparer à l'angle $\widehat{AOC}$. 
\item On considère dans cet ordre $4$ demi-droites $[OA)$, $[OB)$, $[OC)$ et $[OD)$. 
\begin{enumerate}
\item Sachant que $\widehat{AOB}= \widehat{COD} = 35^o$ et $\widehat{BOC}=48^o$, construire les quatre demi-droites. Calculer et comparer les angles $\widehat{AOC}$ et $\widehat{BOD}$. 
\item Soit $[OM)$ la bissectrice de l'angle $\widehat{BOC}$. Montrer que $[OM)$ est également la bissectrice de l'angle $\widehat{AOD}$. 
\end{enumerate}
\item Deux angles $\widehat{AOB}$ et $\widehat{BOC}$ sont adjacents 
et $[OM)$ est la bissectrice de l'angle $\widehat{BOC}$. \begin{enumerate}
\item Construire la figure sachant que $\widehat{AOB}=60^o$ et $\widehat{AOC}= 110^o$. Calculer la mesure de l'angle $\widehat{AOM}$. 
\item Si $\widehat{AOB}= \alpha$ et $\widehat{AOC}= \beta$, montrer que $\widehat{AOM}= \frac12(\alpha + \beta)$. 
\end{enumerate}
\item Les angles $\widehat{AOB}$ et $\widehat{AOC}$ sont adjacents et 
$[OM)$ est la bissectrice de l'angle $\widehat{BOC}$. 
\begin{enumerate}
\item Effectuer la construction de ces angles en prenant $\widehat{AOB}= 52^o$ et $\widehat{AOC}=108^o$. Mener $[OM)$ et calculer la mesure 
de l'angle $\widehat{AOM}$. 
\item On suppose $\widehat{AOB}= \alpha$ et $\widehat{AOC}= \beta$ avec $\beta>\alpha$. Montrer que $\widehat{AOM}= \frac12(\beta-\alpha)$. 
\end{enumerate}
\item Soient $[OM)$ et $[ON)$ les bissectrices des angles adjacents $\widehat{AOB}$ et $\widehat{AOC}$. 
\begin{enumerate}
\item Construire la figure pour $\widehat{AOB}= 72^o$ et $\widehat{AOC}= 48^o$. Calculer les mesures des angles $\widehat{BOC}$ et $\widehat{MON}$. Comparer ces mesures. 
\item Si $\widehat{AOB}= \alpha$ et $\widehat{AOC}= \beta$, montrer 
que $\widehat{BOC}= \alpha + \beta$ et $\widehat{MON}= \frac12(\alpha+\beta)$. 
\end{enumerate}
\item On considère deux angles adjacents $\widehat{AOB}$ et $\widehat{BOC}$. Soient $[OM)$ et $[ON)$ les bissectrices des angles $\widehat{AOB}$ et $\widehat{AOC}$.
\begin{enumerate}
\item On donne $\widehat{AOB}= 60^o$ et $\widehat{AOC}= 108^o$. Construire la figure et calculer les mesures des angles $\widehat{BOC}$ et $\widehat{MON}$. Comparer ces deux mesures. 
\item On suppose $\widehat{AOB}= \alpha$ et $\widehat{AOC}= \beta$, montrer que $\widehat{MON}= \frac12\widehat{BOC}= \frac12(\beta-\alpha)$. 
\end{enumerate}
\item Les bissectrices $[OM)$ et $[ON)$ des angles non-adjacents $\widehat{AOB}$ et $\widehat{AOC}$ font un angle de $36^o$ et l'angle $\widehat{AOB}$ mesure $64^o$. \begin{enumerate}
\item Construire la figure et calculer les angles $\widehat{AOM}$, $\widehat{AON}$ et $\widehat{AOC}$.
\item Comparer les angles $\widehat{BOC}$ et $\widehat{MON}$. En est-il toujours ainsi ? 
\end{enumerate}
\item On considère deux angles adjacents $\widehat{AOB}$ et $\widehat{AOC}$ dont les bissectrices $[OM)$ et $[ON)$ font un angle de $84^o$. \begin{enumerate}
\item Sachant que l'angle $\widehat{AOC}$ vaut $118^o$, construire la figure et calculer les mesures des angles $\widehat{AON}$, $\widehat{AOM}$ et $\widehat{AOB}$. 
\item Comparer les angles $\widehat{MON}$ et $\widehat{BOC}$. Généraliser.
\end{enumerate}
\item Autour d'un point $O$ sont construits cinq angles successivement adjacents $\widehat{AOB}$, $\widehat{BOC}$, $\widehat{COD}$, $\widehat{DOE}$, et $\widehat{EOA}$ recouvrant tout le plan. Ces angles
vérifient les relations : 
\[ \widehat{BOC}= 2\widehat{AOB}; \phantom{meo} \widehat{COD}= \widehat{AOB}+ \widehat{BOC}; \phantom{meo} \widehat{DOE}= 2\widehat{BOC}; \phantom{meo}; \widehat{EOA}= \widehat{BOC}+\widehat{COD}.\]
\begin{enumerate}
\item Calculer la mesure en degrés de chacun de ces angles. 
\item Calculer l'angle des bissectrices des angles $\widehat{AOB}$ et $\widehat{DOE}$. 
\end{enumerate}
\item\begin{enumerate}
\item Construire trois angles successivement adjacents : $\widehat{AOB}= 32^o$, $\widehat{BOC}= 72^o$, et $\widehat{COD}= 48^o$, puis les bissectrices $[OM)$, $[ON)$, $[OP)$ et $[OQ)$ des angles $\widehat{AOB}$, $\widehat{AOC}$, $\widehat{BOD}$ et $\widehat{COD}$. 
\item Calculer les angles $\widehat{MON}$ et $\widehat{POQ}$. Comparer
ces angles à l'angle $\widehat{BOC}$. 
\item Montrer que les angles $\widehat{MOQ}$ et $\widehat{NOP}$ ont la même bissectrice. 
\end{enumerate}
\item \begin{enumerate}
\item Construire un angle $\widehat{AOB}$ de $60^o$, sa bissectrice $[Ox)$, puis les angles droits $\widehat{AOC}$ et $\widehat{BOD}$ adjacents à l'angle $\widehat{AOB}$ et enfin les bissectrices $[Oy)$, $[Oz)$ et $[Ou)$ des angles $\widehat{AOC}$, $\widehat{BOD}$, et $\widehat{COD}$. 
\item Calculer la valeur des angles $\widehat{COD}$, $\widehat{xOy}$, et $\widehat{xOz}$. Montrer que $[Ox)$ est la bissectrice de l'angle $\widehat{yOz}$. 
\item Calculer les mesures des angles $\widehat{yOu}$, $\widehat{zOu}$, et $\widehat{xOu}$. Que peut-on dire des demi-droites $[Ox)$ et $[Ou]$ ? 
\end{enumerate}

\end{enumerate}
\chapter{Le cercle} 
\begin{enumerate}
\item Calculer en degrés la mesures des arcs égaux à $\frac13$, $\frac23$, $\frac15$, $\frac25$, $\frac18$, $\frac38$, $\frac1{10}$ et $\frac3{10}$ d'un cercle et construire ces arcs.
\item Effectuer en grades les calculs précédents.
\item Construire sur un cercle de $3\cm$ de rayon les arcs successifs $AB$, $BC$, $CD$ et $DE$ mesurant respectivement : $27^o$, $53^o$, $75^o$ et $108^o$. Calculer la mesure de l'arc $EA$ qui reste sur le cercle. 
\item Démontrer le théorème suivant : \emph{La bissectrice d'un angle au centre $\widehat{AOB}$ passe par le milieu $M$ de l'arc intercepté par cet angle.}\\ Énoncer et démontrer la réciproque de ce théorème. 
\item Énoncer et démontrer la réciproque de chacune des propositions : 
\begin{enumerate}
\item \emph{Si un point est intérieur à un cercle, sa distance au centre est inférieure au rayon.}
\item \emph{Si un point est extérieur à un cercle, sa distance au centre est supérieure au rayon.}
\end{enumerate}
\item On donne deux points $A$ et $B$ distants de $5\cm$. Construire un point $C$ tel que $AC= 6\cm$ et $BC = 4\cm$. Combien y a-t-il de solutions ?
\item Quelle est la figure formée par l'ensemble des centres des cercles de $5\cm$ de rayon passant par un point donné $A$ ? \\
Construire un cercle de $5\cm$ de rayon passant par deux points donné $A$ et $B$ tels que $AB=4\cm$. 
\item On donne un cercle de centre $O$, de rayon $4\cm$ et un point $A$ de ce cercle. Construire les cordes issues de $A$ ayant $2,5\cm$ de longueur. 
\item Construire les milieux $M$ et $N$ des deux arcs d'extrémités $A$ 
et $B$ d'un cercle donné de centre $O$. Que représentent $OM$ et $ON$ pour les angles saillant et rentrant $\widehat{AOB}$ ? Comment sont disposés les 3 points $M$, $O$ et $N$ ?
\item Quatre points $A$, $B$, $C$, $D$ sont disposés dans cet ordre sur un demi-cercle. \begin{enumerate}
\item Sachant que $\arc{AB}= \arc{CD}$, comparer les arcs $\arc{AC}$ et $\arc{BD}$. 
\item Dans ce cas montrer que les arcs $\arc{AD}$ et $\arc{BC}$ ont même milieu $M$. 
\end{enumerate}
\item Trois points $A$, $B$ et $C$ sont situés dans cet ordre sur un cercle. Les arcs $AB$ et $AC$ mesurent respectivement $88^o$ et $154^o$. Soit $M$ le milieu de l'arc $\arc{BC}$.\begin{enumerate}
\item Montrer que $\arc{AM}=\arc{AB}+\arc{BM}=\arc{AC}-\arc{BM} = \frac12(\arc{AB}+\arc{AC}).$
\item Évaluer l'angle $\widehat{AOM}$. Comparer sa valeur à la somme des angles 
$\widehat{AOB}$ et $\widehat{AOC}$. 
\end{enumerate}
\item On considère sur un cercle deux arcs consécutifs $\arc{BA}=78^o$ et $\arc{AC}=54^o$. 
\begin{enumerate}
\item Construire les milieux $M$ et $N$ de ces deux arcs. 
\item Calculer la mesure de l'arc $\arc{MAN}$ et la comparer à la mesure de l'arc $\arc{BAC}$. Généraliser. 
\end{enumerate}
\item Soient sur un cercle deux arcs de même sens : $\arc{AB}=42^o$ et $\arc{AC}= 108^o$. \begin{enumerate}
\item Construire les milieux $M$ et $N$ de ces deux arcs et calculer la mesure de 
l'arc $\arc{MBN}$. 
\item Comparer cette mesure à celle de l'arc $\arc{BNC}$. Généraliser. 
\end{enumerate}
\end{enumerate}
\chapter{Angles associés} 
\begin{enumerate}
\item Construire et calculer le supplément des angles suivants : 
\begin{enumerate}
\item $30^o$; $45^o$; $60^o$. 
\item $67^o35'$; $113^o43'23"$. 
\item $25$ gr; $50$ gr; $75$ gr. 
\item $56,327$ gr; $141,943$ gr. 
\end{enumerate}
\item Construire et calculer le complément des angles suivants : 
\begin{enumerate}
\item $24^o$; $36^o$; $63^o$. 
\item $23^o37'$; $67^o26'30"$.
\item $30$ gr; $48$ gr; $65$ gr. 
\item $31,09$ gr; $58,542$ gr. 
\end{enumerate}
\item Construire à l'aide du rapporteur, deux angles adjacents $\widehat{AOB}$ et $\widehat{AOC}$ de $54^o$ et $96^o$. Calculer l'angle $\widehat{MON}$ de leurs bissectrices. Le comparer à l'angle $\widehat{BOC}$.
\item Deux angles adjacents ont pour somme $102^o$. Quel est l'angle de leurs bissectrices ? Calculer et construire ces deux angles sachant que leur différence vaut $36^o$. 
\item Construire deux angles adjacents supplémentaires dont la différence vaut $54^o$. 
\item Deux angles adjacents $\widehat{AOB}$ et $\widehat{BOC}$ sont complémentaires et l'un est les $\frac23$ de l'autre. Calculer et construire ces deux angles. Construire les bissectrices $[OM)$ et $[ON)$. Calculer leur angle $\widehat{MON}$. 
\item Les bissectrices de deux angles adjacents $\widehat{AOB}$ et $\widehat{BOC}$ font un angle droit $\widehat{MON}$. Démontrer que ces deux angles sont supplémentaires. 
\item Les bissectrices $[OM)$ et $[ON)$ de deux angles égaux sont dont le prolongement l'une de l'autre. Démontrer que ces deux angles sont opposés par le sommet. 
\item Deux angles non adjacents $\widehat{AOB}$ et $\widehat{AOC}$ ont pour différence un angle droit. 
\begin{enumerate}
\item Calculer l'angle de leurs bissectrices. 
\item Construire ces deux angles sachant que $\widehat{AOB}= \frac38\widehat{AOC}$.
\end{enumerate}
\item Les bissectrices $[OM)$ et $[ON)$ de deux angles non adjacents $\widehat{AOB}$ et $\widehat{AOC}$ font un angle de $36^o$. \begin{enumerate}
\item Calculer l'angle $\widehat{BOC}$, lorsque $\widehat{AOB}= 84^o$. Comparer la valeur trouvée à celle de l'angle $\widehat{MON}$. 
\item Généraliser pour des valeurs quelconques des angles donnés. 
\end{enumerate}
\item Quatre angles consécutifs de même sommet $\widehat{AOB}$, $\widehat{BOC}$, 
$\widehat{COD}$ et $\widehat{DOA}$ sont tels que : \[ \widehat{AOB}= \widehat{COD}
\phantom{meow}\text{et}\phantom{meow}\widehat{BOC}=\widehat{DOA}.\]
Démontrer que leurs côtés sont deux à deux dans le prolongement de l'autre. 
\item Démontrer que lorsque deux angles $\widehat{AOB}$ et $\widehat{AOC}$ ont même sommet et un côté commun, l'angle $\widehat{MON}$ de leurs bissectrices et la moitié de l'angle de leurs côtés non communs, que ces angles soient adjacents ou non. 
\item On construit un angle $\widehat{AOB}$ de $48^o$ puis les angles droits $\widehat{AOA'}$ et $\widehat{BOB'}$ non adjacents au premier. 
\begin{enumerate}
\item Comparer les angles $\widehat{AOB'}$ et $\widehat{A'OB}$. Montrer que les angles $\widehat{AOB}$ et $\widehat{A'OB'}$ sont supplémentaires. 
\item Montrer que les angles $\widehat{AOB}$ et $\widehat{A'OB'}$ ont même bissectrice $[OP)$. Calculer l'angle $\widehat{MON}$ des bissectrices des angles $\widehat{AOB'}$ et $\widehat{A'OB}$. 
\end{enumerate}
\item Construire l'angle $\widehat{AOB}=108^o$ puis les angles droits $\widehat{AOA'}$ et $\widehat{BOB'}$ adjacents à celui-ci. 
\begin{enumerate}
\item Montrer que les angles $\widehat{AOB}$ et $\widehat{A'OB'}$ sont supplémentaires. Comparer les angles $\widehat{AOB'}$ et $\widehat{A'OB}$. 
\item Démontrer que les bissectrices $[OP)$ et $[OQ)$ des angles $\widehat{AOB}$ et $\widehat{A'OB'}$ sont en ligne droite et calculer l'angle $\widehat{MON}$ des bissectrices des angles $\widehat{AOB'}$ et $\widehat{A'OB}$. 
\end{enumerate}
\item Calculer la valeur et construire deux angles $\widehat{AOB}$ et $\widehat{BOC}$ sachant que leurs bissectrices font un angle de $40^o$ et que le premier est égal aux $\frac35$ de l'autre. \begin{enumerate}
\item Dans le cas où les angles $\widehat{AOB}$ et $\widehat{BOC}$ sont adjacents.
\item Dans le cas où ces angles ne sont pas adjacents. 
\end{enumerate}
\end{enumerate}
\chapter{Droites perpendiculaires} 
\begin{enumerate}
\item Vérification de l'angle droit d'une équerre.
\item Construire le complément d'un angle aigu : \begin{enumerate}
\item Par pliage ; 
\item à l'aide d'un rapporteur ; 
\item à l'aide de l'équerre.
\end{enumerate}
\item Déterminer la projection $H$ et la distance d'un point extérieur $O$ à une droite $(xy)$. \begin{enumerate}
\item Par pliage; \item à l'aide de l'équerre.
\end{enumerate}
\item Construire la médiatrice d'un segment donné $[AB]$. \begin{enumerate}
\item Par pliage; \item à l'aide d'un double décimètre et du rapporteur ou de l'équerre.
\end{enumerate}
\item Soient $I$ et $J$ deux points d'une droite $(xy)$ et un point extérieur $O$.
On trace les cercles de centres $I$ et $J$ passant par $O$. \begin{enumerate}
\item Montrer que les deux cercles passent également par le point $O'$ qui coïncide avec $O$ lorsqu'on plie la figure suivant $(xy)$.
\item En déduire une construction de la perpendiculaire menée de $O$ à $ (xy)$ et de la projection $H$ du point $O$ sur $(xy)$. 
\end{enumerate}
\item On prend trois points $A$, $B$, $C$ sur un même cercle. Mener d'un point $M$ de
ce cercle les perpendiculaires $(MP)$, $(MQ)$ et $(MR)$ aux trois droites $(BC)$, 
$(CA)$ et $(AB)$. Si la construction est précise, les trois points $P$, $Q$, $R$ sont alignés. Vérifiez-le. 
\item Soient trois points $A$, $B$, $C$ non alignés. Mener de chacun d'eux, à l'aide
de l'équerre, la perpendiculaire à la droite déterminée par les deux autres. Que constate-t-on ? On envisagera le cas où les trois angles sont aigus, et le cas où l'angle $\widehat{BAC}$ est obtus.
\item On considère un angle aigu $\widehat{AOB}$.\begin{enumerate}
\item Construire un angle aigu $\widehat{COD}$ tel que $[OC)$ soit perpendiculaire à $[OA)$ et $[OD)$ perpendiculaire à $[OB)$. 
\item Comparer les angles $\widehat{AOB}$ et $\widehat{COD}$ et énoncer le théorème
correspondant. 
\item Que peut-on dire des angles $\widehat{AOD}$ et $\widehat{BOC}$ ainsi que de 
leurs bissectrices ?
\end{enumerate}
\item Reprendre le problème précédent avec des angles $\widehat{AOB}$ et $\widehat{COD}$ obtus. 
\item On considère un angle obtus $\widehat{AOB}$ et on construit à l'intérieur de cet angle $[OC)$ perpendiculaire à $[OA)$ et $[OD)$ perpendiculaire à $[OB)$. \begin{enumerate}
\item Montrer que les angles $\widehat{AOB}$ et $\widehat{COD}$ sont supplémentaires.
\item Comment sont disposées les bissectrices des angles $\widehat{AOB}$ et $\widehat{COD}$ ? 
\item Calculer l'angle $\widehat{MON}$ des bissectrices des angles $\widehat{AOD}$ et $\widehat{BOC}$. 
\end{enumerate}
\item Soit une droite $(AB)$ et un point $O$ de cette droite. On construit d'un
même côté de cette droite deux angles $\widehat{AOC}$ et $\widehat{BOC}$ complémentaires ainsi que la perpendiculaire $[OE)$ à $(AB)$ en $O$. 
\begin{enumerate}
\item Démontrer que $(OC)$ et $(OD)$ sont perpendiculaires. 
\item Comparer les angles $\widehat{AOC}$ et $\widehat{EOD}$ puis les angles $\widehat{BOD}$ et $\widehat{EOC}$. 
\item Que peut-on dire des angles $\widehat{BOC}$ et $\widehat{DOE}$ ?
\end{enumerate}
\item Deux droites perpendiculaires $(xx')$ et $(yy')$ se coupent en $O$. Deux autres droites perpendiculaires $(uu')$ et $(vv')$ se coupent en $O$ de façon que $[Ou)$ soit dans l'angle $\widehat{xOy}$ et $[Ov)$ dans l'angle $\widehat{x'Oy}$. \begin{enumerate}
\item Trouver dans la figure les angles égaux à $\widehat{xOu}$. 
\item Déterminer les angles complémentaires, puis les angles supplémentaires de $\widehat{xOu}$. 
\end{enumerate}
\item Quatre angles consécutifs $\widehat{AOB}$, $\widehat{BOC}$, $\widehat{COD}$ et $\widehat{DOE}$ valent chacun $45^o$. 
\begin{enumerate}
\item Que peut-on dire des demi-droites $[OA)$ et $[OE)$, $[OB)$ et $[OD)$ ?
\item On construit les bissectrices des quatre angles initiaux. Montrer qu'elle sont deux à deux\footnote{Deux sont perpendiculaires à deux autres; non deux quelconques.} perpendiculaires. 
\end{enumerate}
\item On considère une feuille de papier $ABCD$ dont les angles $\widehat{A}$ et $\widehat{B}$ sont droits. Par le milieu $O$ de $[AB]$, on trace une demi-droite 
$[Ox)$.\begin{enumerate}
\item Plier la feuille de façon à amener les segments $[OA]$ et $[OB]$ sur $[Ox)$. 
Montrer que $A$ et $B$ viennent alors coïncider en $P$.
\item L'un des plis coupe $[AD]$ en $M$, l'autre coupe $[BC]$ en $N$. Montrer que $(Om)$ et $(ON)$ sont perpendiculaires.
\item Démontrer que les trois points $M$, $P$ et $N$ sont alignés sur la perpendiculaire en $P$ à $[Ox)$.
\end{enumerate}
\end{enumerate}
\chapter{Polygones. Triangles} 
\begin{enumerate}
\item Construire les médianes, les hauteurs, les bissectrices intérieures et extérieures d'un triangle. Que remarquez-vous ? 
\item Construire un triangle rectangle et isocèle dont les côtés de l'angle droit valent $32$ mm. 
\item Construire un triangle rectangle connaissant l'hypoténuse $58$ mm et un côté de 
l'angle droit $42$ mm. 
\item Construire un triangle rectangle connaissant l'hypoténuse $44$ mm et un angle
$32^o$. 
\item Construire un triangle rectangle connaissant un côté de l'angle droit $37$ mm et un angle adjacent à ce côté $53^o$. 
\item Construire un triangle isocèle dont les deux côtés égaux mesurent $54$ mm et le troisième côté $60$ mm. 
\item Construire un triangle équilatéral de $29$ mm de côté. 
\item Construire un triangle $ABC$ connaissant $\widehat{A}=52^o$, $AB=27$ mm, $AC=38$ mm. 
\item Construire un triangle $ABC$ connaissant $BC = 57$ mm, $\widehat{B}= 63^o$, $\widehat{C}= 51^o$. 
\item Construire un triangle $ABC$ connaissant la hauteur $AH=18$ mm, $AB=25$ mm, et $BC=50$ mm (on supposera $H$ entre $B$ et $C$). 
\item Construire un triangle $ABC$ connaissant la hauteur $AH= 18$ mm, $AB=25$ mm, et $AC=30$ mm.
\item Construire un triangle $ABC$ connaissant la hauteur $AH=22$ mm, $AB=29$ mm, et $\widehat{A}=37^o$. 
\item Construire un triangle $ABC$ connaissant $AH=30$ mm, $BH= 12$ mm, et $AC=40$ mm. 
\item Construire un triangle $ABC$ connaissant $AB=38$ mm, la médiane $AM=23$ mm, et $BM=23$ mm. Mesurer les angles de ce triangle. 
\item Construire un triangle $ABC$ connaissant $AB=41$ mm, $\widehat{BAC}= 54^o$
et la bissectrice intérieure $AD=45$ mm. 
\item Construire un triangle $ABC$ connaissant : $\widehat{A}= 120^o$, $AB= 30$ mm, et $BC = 57$ mm. 
\item Construire un triangle isocèle $ABC$ sachant que $AB=AC=32$ mm, et que $\widehat{B}= 65^o$. 
\item Découper dans une feuille de bristol ou de léger carton une fausse équerre triangulaire $ABC$ telle que $BC=10$ cm, $\widehat{B}=60^o$ et $\widehat{C}=72^o$. 
Vérifier que l'angle $\widehat{A}$ mesure $48^o$.
\item Utiliser la fausse équerre de l'exercice précédent pour mener par un point extérieur $A$ une oblique $[AM)$ faisant un angle de $60^o$ avec une droite donnée $(xy)$. Nombre de solutions ?
\item Construire un triangle $ABC$ rectangle en $A$ tel que $AB=40$ mm et $\widehat{C}= 72^o$. (Utiliser la fausse équerre précédente.)
\item Construire un triangle $ABC$ tel que $AB= 32$ mm, $\widehat{B}= 65^o$, et $\widehat{C}= 47^o$. (Utiliser la fausse équerre précédente.)
\item Construire un triangle isocèle $ABC$ tel que $\widehat{B}= \widehat{C}= 72^o$ avec une hauteur $AH=48$ mm. (Utiliser la fausse équerre précédente.)
\item Construire un triangle $ABC$ sachant la hauteur $AH=45$ mm et que les angles $\widehat{B}$ et $\widehat{C}$ sont égaux à $60^o$ et $72^o$. (Utiliser la fausse équerre précédente.)
\item Construire un triangle $ABC$ connaissant la hauteur $AH= 36$ mm, l'angle $\widehat{B}=120^o$ et l'angle $\widehat{C}= 48^o$. 
\item Construire un quadrilatère convexe $ABCD$ sachant que $AB=AC= BC=5\cm$, $AD=3\cm$ et $CD=4\cm$. Mesurer au rapporteur les angles $\widehat{B}$ et $\widehat{D}$ du quadrilatère obtenu.
\item Construire un quadrilatère convexe $ABCD$ connaissant $\widehat{A}=120^o$, $AB=3\cm$, $AD=2\cm$, $BC=3,5\cm$, et $CD=4\cm$. Mesurer au rapporteur les angles
$\widehat{B}$, $\widehat{C}$ et $\widehat{D}$ du quadrilatère obtenu.
\end{enumerate}
\chapter{Les deux premiers cas d'égalité des triangles} 
\begin{enumerate}
\item Démontrer que dans deux triangles égaux $ABC$ et $A'B'C'$, les médianes $AM$ et $A'M'$ sont égales. (Comparer $ABM$ et $A'B'M'$.). Vérification graphique. 
\item Même problème pour les bissectrices intérieures $AD$ et $A'D'$.
\item On porte sur les cotés d'un angle $\widehat{xOy}$ respectivement deux longueurs égales $OA$ et $OB$ et on joint $A$ et $B$ à un point quelconque $M$ de la bissectrice de l'angle $\widehat{xOy}$.\begin{enumerate}
\item Comparer les triangles $AOM$ et $BOM$. Conséquences ? 
\item Démontrer que $AM=BM$ et que $[OM)$ est bissectrice de l'angle $\widehat{AMB}$.
\end{enumerate}
\item On considère un quadrilatère convexe $ABCD$ dans lequel la droite $(AC)$ est bissectrice intérieure des angles $\widehat{A}$ et $\widehat{C}$. 
\begin{enumerate}
\item Comparer les triangles $ABC$ et $ADC$. Conséquences ? 
\item On plie la figure suivant la droite $(AC)$. Que se passe-t-il ? En déduire que $(AC)$ est médiatrice du segment $[BD]$. 
\end{enumerate}
\item Soient $[AB]$ et $[CD]$ deux diamètres pris dans deux cercles de même centre $O$.\begin{enumerate}
\item Comparer les triangles $AOC$ et $BOD$ puis les triangles $AOD$ et $BOC$. 
\item Comparer les côtés opposés et les angles opposés du quadrilatère $ACDB$.
\end{enumerate}
\item On considère dans un cercle deux angles au centre égaux $\widehat{AOB}$ et
$\widehat{DOC}$ tous deux adjacents à l'angle $\widehat{BOD}$. 
\begin{enumerate}
\item Comparer les triangles $AOB$ et $COD$. Conséquences ? 
\item Comparer les triangles $AOD$ et $COB$. Conséquences ? Montrer que les angles $\widehat{BAD}$ et $\widehat{BCD}$ sont égaux ainsi que les angles $\widehat{ABC}$ et $\widehat{ADC}$. 
\end{enumerate}
\item Deux points $A$ et $B$ situés d'un même côté de la droite $(xy)$ se projettent
en $H$ et $K$ sur cette droite et on a $AH=BK$. \begin{enumerate}
\item Comparer les triangles $AHK$ et $BHK$. Conséquences ? 
\item Comparer les triangles $ABK$ et $BAH$ puis les angles $\widehat{KAB}$ et $\widehat{HBA}$. 
\end{enumerate}
\item Une droite $(xy)$ passe entre les points $A$ et $B$ et les distances $AH$ et $BK$ à cette droite sont égales. Soit $O$ le milieu de $[HK]$. \begin{enumerate}
\item Comparer les triangles $OAH$ et $OBK$. Conséquences ? 
\item Démontrer que les points $A$, $O$ et $B$ sont alignés et que $O$ est le milieu de $[AB]$. 
\end{enumerate}
\item Dans un quadrilatère convexe la diagonale $[AC]$ fait des angles égaux avec $[AD]$ et $[BC]$ d'une part et avec $[AB]$ et $[CD]$ d'autre part. Soit $O$ le milieu de $[AC]$. \begin{enumerate}
\item Comparer les triangles $ABC$ et $CDA$. En déduire une propriété des angles opposés et des côtés opposés du quadrilatère. 
\item Comparer les triangles $OAB$ et $OCD$ et montrer que $O$ est aussi milieu de $[BD]$. 
\end{enumerate}
\item Deux segments inégaux $[AB]$ et $[CD]$ ont même milieu $O$. \begin{enumerate}
\item Comparer les triangles $OAC$ et $OBD$ puis les triangles $AOD$ et $BOC$. Conséquences pour les côtés et les angles du quadrilatère $ACBD$. 
\item Une droite $(Ox)$ est perpendiculaire en $H$ à $(AC)$. Montrer qu'elle est également perpendiculaire en $K$ à $(BD)$ et que $O$ est le milieu de $[HK]$. 
\end{enumerate}
\item On considère un quadrilatère convexe $ABCD$ tel que $AB=CD$ et $\widehat{BAC}=\widehat{CAD}$. Soit $O$ le milieu de la diagonale $[AC]$.\begin{enumerate}
\item Comparer les triangles $OAB$ et $OCD$. Conséquences ? Que représente le point 
$O$ pour le segment $[BD]$ ? 
\item Une droite $(xOy)$ coupe $(AB)$ en $M$ et $(DC)$ en $N$. Comparer les triangles
$OAM$ et $OCN$. En déduire que $AM=CN$ et que $O$ est le milieu de $[MN]$. Comparer
les angles $\widehat{OMA}=\widehat{ONC}$. 
\end{enumerate}
\item On prolonge la médiane $[AM]$ du triangle $ABC$ d'une longueur $MD$ égale à $AM$. \begin{enumerate}
\item Comparer les triangles $MAC$ et $MDB$. Conséquences pour $AC$ et $BD$ ainsi que pour les angles $\widehat{MAC}$ et $\widehat{MDB}$ ?
\item Construire la figure sachant que $AM = 3\cm$, $AB=5\cm$ et $AC=4\cm$. 
\end{enumerate}
\item Utiliser les résultats de l'exercice précédent pour construire un triangle 
$ABC$ connaissant la médiane $AM = 36$ mm, $\widehat{MAB}= 72^o$, et $\widehat{MAC}= 54^o$. 
\item On considère deux angles égaux $\widehat{BAx}$ et $\widehat{ABy}$ situés
de part et d'autre de la droite $(AB)$. Une sécante issue du milieu $O$ de $[AB]$ coupe $[Ax)$ en $M$ et $[By)$ en $N$. \begin{enumerate}
\item Comparer les triangles $AOM$ et $BON$. Conséquences ? 
\item Comparer les triangles $AON$ et $BOM$ puis les segments $[AN]$ et $[BM]$.
Montrer que le milieu de $[AN]$ et le milieu de $[BM]$ sont sur une droite issue de 
$O$. 
\end{enumerate}
\item \begin{enumerate}
\item Construire un triangle $ABC$ tel que $\widehat{A}=42^o$, $AB=48$ mm, et $AC=36$ mm, puis, extérieurement au triangle, les triangles isocèles en $A$ $ACD$ et $ABE$ tels que $\widehat{CAD}= \widehat{BAE}= 58^o$.
\item Comparer les triangles $ABD$ et $AEC$ puis les segments $[BD]$ et $[EC]$. Ces
segments se coupent en $I$ : mesurer l'angle $\widehat{CID}$. 
\end{enumerate}
\item \begin{enumerate}
\item Construire un triangle $ABC$ tel que $BC=34$ mm, $\widehat{B}=82^o$ et $\widehat{C}= 68^o$, puis extérieurement au triangle $ABC$ construire les triangles 
$ABD$ et $ACE$ tels que $\widehat{ABD}= \widehat{ACE}=90^o$, $BD=BA$ et $CE=CA$. ainsi que le triangle $ICB$ égal au triangle $ABC$ ($IC=AB$, $IB=AC$). 
\item Comparer les triangles $BDI$ et $CIE$ puis les segments $[ID]$ et $[IE]$. 
Mesurer l'angle $\widehat{DIE}$. 
\end{enumerate}
\item On donne un triangle $ABC$ et on prolonge le côté $[AB]$ d'une longueur $BD=AB$, le côté $[AC]$ d'une longueur $CF=AC$ et la médiane $[AM]$ d'une longueur $MI=AM$.
\begin{enumerate}
\item Comparer les triangles $MAB$ et $MIC$ puis $MAC$ et $MIB$. Montrer que le quadrilatère $BACI$ a ses côtés opposés égaux et ses angles opposés égaux. 
\item Comparer les triangles $BDI$ et $CIF$ puis les segments $[ID]$ et $[IF]$. 
Mesurer l'angle $\widehat{DIF}$. Que déduit-on pour les $3$ points $I$, $D$ et $F$ ?
\end{enumerate}
\item On considère un quadrilatère concave $ABCD$ dans lequel la droite $(AC)$ est bissectrice de l'angle saillant $\widehat{BAD}$ et aussi de l'angle rentrant $\widehat{BCD}$. 
\begin{enumerate}
\item Comparer les triangles $ABC$ et $ADC$. Conséquences ? 
\item Le segment $[BD]$ coupe en $H$ le prolongement de $[AC]$. Comparer les triangles $HBC$ et $HDC$. Conséquences ? 
\item Montrer que la droite $(AC)$ est la médiatrice de $[BD]$. 
\end{enumerate}
\end{enumerate}
\chapter{Symétrie par rapport à une droite}
\begin{enumerate}
\item Soient deux points $A$ et $A'$ symétriques par rapport à une droite $(xy)$ et 
deux points quelconques $I$ et $J$ de la droite $(xy)$. Montrer que les cercles de
centres $I$ et $J$ passant par $A$ se recoupent au point $A'$ En déduire une construction au compas du symétrique d'un point par rapport à la droite $(xy)$. 
\item \begin{enumerate}
\item Construire le symétrique d'un polygone $ABCDEF$ par rapport à une droite $(xy)$
en utilisant l'exercice précédent. 
\item Vérifier que $(BD)$ et son symétrique $(B'D')$ se coupent en $O$ sur $(xy)$ et
que l'angle $\widehat{BOB'}$ admet $(xy)$ pour bissectrice. 
\end{enumerate}
\item \begin{enumerate}
\item Construire le symétrique $\widehat{B'A'C'}$ d'un angle $\widehat{BAC}$ par rapport à la droite $(xy)$.
\item Dans quel cas la droite $(AB)$ est-elle confondue avec sa symétrique $(A'B')$ ?
\end{enumerate}
\item Dans un triangle $ABC$ on mène la bissectrice intérieure de l'angle $A$ coupant en $D$ le côté $[BC]$. (On suppose $AB<AC$). \begin{enumerate}
\item Construire le point $E$ symétrique de $B$ par rapport à $(AD)$. Que 
représente $(CE)$ pour les segments $[AB]$ et $[AC]$ ? 
\item Que représente la droite $(DA)$ pour l'angle $D$ du triangle $CDE$ ? 
\end{enumerate}
\item Reprendre l'exercice précédent avec $(AD)$ bissectrice extérieure de l'angle $\widehat{A}$. 
\item \begin{enumerate}
\item Construire le symétrique $A'B'C'$ du triangle $ABC$ par rapport à la droite $(xy)$. 
\item Les médianes $[BM]$ et $[CN]$ du triangle $ABC$ se coupent en $G$, les médianes
$[B'M']$ et $[C'N']$ du triangle $A'B'C'$ se coupent en $G'$. Que représente $(xy)$
pour le segment $[GG']$ ? 
\end{enumerate}
\item On considère un angle $\widehat{BAD}$ tel que $AB=AD$. On construit à l'intérieur de cet angle deux angles égaux $\widehat{ABx}$ et $\widehat{ADy}$.
\begin{enumerate}
\item Montrer que $[Bx)$ et $[Dy)$ sont symétriques par rapport à la bissectrice $[Ou)$ de l'angle $\widehat{BAD}$. Où se trouve le point d'intersection $C$ de ces
deux demi-droites ? 
\item Comparer les segments $[BC]$ et $[DC]$ et les angles $\widehat{ACB}$ et $\widehat{ACD}$.
\end{enumerate}
\item On prend les symétriques $B$ et $C$ d'un point $A$ intérieur à l'angle droit $\widehat{xOy}$ par rapport aux droites $[Ox)$ et $[Oy)$. \begin{enumerate}
\item Montrer que le point $O$ est le milieu du segment $[BC]$ et que le 
cercle de diamètre $[BC]$ passe par $A$. 
\item Comparer l'angle $\widehat{BAC}$ à la somme des angles $\widehat{B}$ et $\widehat{C}$ du triangle
$ABC$.
\end{enumerate}
\item Les deux points $A$ et $B$ sont d'un même côté de la droite $(xy)$. On construit le symétrique $A'$ de $A$ par rapport à $(xy)$. La droite $(A'B)$ coupe 
$(xy)$ en $M$. \begin{enumerate}
\item Montrer que $(MA)$ et $(MB)$ dont des angles égaux avec la droite $(xy)$.
\item Existe-t-il un autre point $N$ de la droite $(xy)$ tel que $(xy)$ soit
bissectrice extérieure de l'angle $\widehat{ANB}$ ? 
\end{enumerate}
\item On considère deux points $A$ et $B$ de part et d'autre de la droite $(xy)$
et on désigne par $A'$ le symétrique de $A$ par rapport à la droite $(xy)$.
La droite $(BA')$ coupe $(xy)$ en $M$.  \begin{enumerate}
\item Montrer que $(MA)$ et $(MB)$ dont des angles égaux avec la droite $(xy)$.
\item Existe-t-il un autre point $N$ de la droite $(xy)$ tel que $(xy)$ soit
bissectrice de l'angle $\widehat{ANB}$ ? 
\end{enumerate}
\item Soit un cercle de centre $O$ et de diamètre $[AB]$. On construit deux arcs
égaux $\arc{AC}$ et $\arc{AD}$. \begin{enumerate}
\item Montrer que $(AB)$ est la médiatrice de $[CD]$.
\item Démontrer que $AC=AD$, $BC=BD$ et que la droite $(AB)$ est bissectrice des 
angles $\widehat{COD}$, $\widehat{CAD}$ et $\widehat{CBD}$. 
\end{enumerate}
\item On construit d'un même côté d'un segment $[AB]$ deux angles égaux 
$\widehat{BAC}$ et $\widehat{ABD}$ tels que $AC=BD$. \begin{enumerate}
\item Montrer que la figure obtenue admet un axe de symétrie. En déduire l'égalité de 
$AD$ et de $BC$ et l'égalité entre les angles $\widehat{ACB}$ et $\widehat{ADB}$.
\item Retrouver ces propriétés en comparant les triangles $ABC$ et $BAD$ puis montrer
que $(AD)$ et $(BC)$ se coupent sur la médiatrice de $[AB]$. 
\end{enumerate}
\item Un quadrilatère $ABCD$ dont les diagonales se coupent en $O$ est tel que $OA=OB$ et $OC=OD$. \begin{enumerate}
\item Montrer que ce quadrilatère admet pour axe de symétrie la droite $(Ox)$ bissectrice des angles $\widehat{AOB}$ et $\widehat{COD}$. 
\item Montrer que $(AD)$ et $(BC)$ se coupent sur cet axe et démontrer $AD=BC$.
\end{enumerate}
\item On considère un angle $\widehat{xOy}$ et un point $A$ intérieur tel que $\widehat{AOx}=38^o$ et $\widehat{AOy}= 16^o$. \begin{enumerate}
\item Construire les symétriques $B$ et $C$ du point $A$ par rapport à $(Ox)$ et $(Oy)$, puis la bissectrice de l'angle $\widehat{BOC}$ sur laquelle on porte $OD=OA$. 
\item Montrer que l'angle $\widehat{BOC}$ est le double de l'angle $\widehat{xOy}$
et que les quatre points $A$, $B$, $C$ et $D$ sont sur un cercle de centre $O$. 
\item Montrer que $B$ et $C$ sont symétriques par rapport à $(OD)$ 
et que $A$ et $D$ sont symétriques par rapport à la droite $(Ou)$ bissectrice de l'angle $\widehat{xOy}$. 
\end{enumerate}
\end{enumerate}
\chapter{Triangle isocèle}
\begin{enumerate}
\item Démontrer que lorsque la hauteur $[AH]$ d'un triangle $ABC$ est en même temps médiane, le triangle est isocèle.
\item Dans un triangle $ABC$, la hauteur $[AH]$ est en même temps bissectrice intérieure de l'angle $\widehat{A}$. Démontrer que le triangle est isocèle.
\item On considère un triangle $ABC$ dans lequel la médiane $[AM]$ est 
en même temps bissectrice intérieure de l'angle $\widehat{A}$. On prolonge $[AM]$ 
d'une longueur $MD=AM$. \begin{enumerate}
\item Comparer les triangles $AMD$ et $DMC$, puis les angles $\widehat{MAB}$ et 
$\widehat{MDC}$. 
\item En déduire que le triangle $ACD$ est isocèle, et qu'il en est de même du triangle $ABC$.
\end{enumerate}
\item Démontrer que dans un triangle isocèle $ABC$ en $A$ : \begin{enumerate}
\item Les médianes $[BM]$ et $[CN]$ sont égales. 
\item Les bissectrices intérieures $[BD]$ et $[CE]$ sont égales.
\end{enumerate}
\item Dans un triangle isocèle $OAB$ de sommet $O$, on prend un point $C$ sur $[OA]$
et un point $D$ sur $[OB]$ tels que $OC=OD$.\begin{enumerate}
\item Comparer les triangles $OAD$, $OBC$ puis les triangles $ABC$ et $BAD$. Conséquences ? 
\item Montrer que le point d'intersection $I$ de $[AD]$ et $[BC]$ appartient à l'axe
de symétrie du triangle $OAB$.
\end{enumerate}
\item Reprendre l'exercice précédent avec l'hypothèse $\widehat{OAD}=\widehat{OBC}$.
\item Dans le quadrilatère $ABCD$, les angles $\widehat{B}$ et $\widehat{D}$ sont supplémentaires et $AD=BC$. On prolonge $[AB]$ d'une longueur $BE=CD$.
\begin{enumerate}
\item Comparer les triangles $ACD$ et $EBC$. Nature du triangle $ACE$ ?
\item Démontrer que les angles $\widehat{BAC}$ et $\widehat{ACD}$ sont égaux.
\end{enumerate}
\item On construit extérieurement à l'angle $\widehat{A}$ du triangle $ABC$ le 
segment $[BD]$ perpendiculaire à $(BA)$ et égal à $[AC]$, puis le segments $[CE]$
perpendiculaire à $(CA)$ et égal à $[AB]$. 
\begin{enumerate}
\item Comparer les triangles $ABD$ et $ECA$. Conséquences ? 
\item Démontrer l'égalité des angles $\widehat{ADE}$ et $\widehat{AED}$.
\end{enumerate}
\item Sur les côtés $[BC]$, $[CA]$ et $[AB]$ du triangle équilatéral $ABC$, 
on prend respectivement $A'$, $B'$ et $C'$ tels que $BA'=CB'=AC'$.\begin{enumerate}
\item Comparer les triangles $AB'C'$, $BC'A'$ et $CA'B'$. 
\item Montrer que le triangle $A'B'C'$ est équilatéral.
\end{enumerate}
\item On considère un triangle isocèle $OAB$ de sommet $O$ et un point $C$ du côté $[OA]$. On prolonge $[OB]$ dune longueur $BD=AC$. Le segment $[CD]$ coupe $[AB]$ en
$M$. On prolonge $[BA]$ d'une longueur $AP=BM$. \begin{enumerate}
\item Comparer les triangles $APC$ et $BMD$. Conséquence pour $CP$ et $MD$ et pour
les angles $\widehat{CPA}$ et $\widehat{DMB}$ ? 
\item Nature du triangle $CMP$ ? Que représente $M$ pour le segment $[CD]$ ?
\end{enumerate}
\item \begin{enumerate} \item Dans un triangle rectangle $ABC$, l'hypoténuse $[BC]$ est le double du côté de l'angle droit $[AB]$. Montrer que l'angle $\widehat{B}$ du triangle est le double 
de l'angle $\widehat{C}$.
\item Énoncer et démontrer la réciproque. On prolongera $[BA]$ d'une longueur $AD=AB$.\end{enumerate}
\item Construire un triangle isocèle $OAB$ tel que $\widehat{A} = \widehat{B}= 68^o$. 
Le cercle de centre $A$ passant par $B$ recoupe $[OB]$ en $C$. On prolonge $[OA]$ 
d'une longueur $AD=OC$.\begin{enumerate}
\item Comparer les angles $\widehat{ACO}$ et $\widehat{BAD}$ puis les triangles $ACO$ et $BAD$. 
\item Nature du triangle $OBD$. Calculer la somme des angles $\widehat{OBD}$ et 
$\widehat{BAC}$. 
\end{enumerate}
\begin{enumerate}
\item Démontrer que si un diamètre d'un cercle de centre $O$ est perpendiculaire
à la corde $[AB]$ il passe par le milieu $H$ de cette corde et par les milieux $M$ 
et $N$ des arcs qu'elle sous-tend.
\item Comparer les triangles $MAH$ et $MHB$, puis $NHA$ et $NHB$ et montrer que 
$(MN)$ est la bissectrice des angles $\widehat{AMB}$ et $\widehat{ANB}$. 
\end{enumerate}
\item On construit extérieurement au triangle isocèle $ABC$ de sommet $A$ deux triangles égaux $ABD$ et $ACE$. \begin{enumerate}
\item Comparer les triangles $ABE$ et $ACD$ puis les triangles $BCD$ et $CBE$. En 
déduire l'égalité de $BE$ et $CD$. 
\item Montrer que $[BD]$ et $[CE]$ ainsi que $[CD]$ et $[BE]$ se coupent en $I$ 
et $J$ sur la hauteur $[AH]$ du triangle $ABC$. 
\end{enumerate}
\item Soit un triangle isocèle $ABC$ tel que l'angle au sommet $\widehat{BAC}=56^o$.
On construit un segment $[AD]$ perpendiculaire à $[AB]$ et du même côté de $(AB)$ que $[AC]$, puis le segment $[AE]$ égal à $[AD]$, perpendiculaire à $(AC)$ de telle sorte que l'angle droit $\widehat{CAE}$ soit adjacent à l'angle $\widehat{BAC}$. 
\begin{enumerate}
\item Comparer les triangles $ABD$ et $ACE$, les segments $[BD]$ et $[CE]$ et les
angles $\widehat{BAC}$ et $\widehat{DAE}$.
\item On mène les hauteurs $[AH]$ et $[AK]$ des triangles $ABC$ et $ADE$. Évaluer 
l'angle $HAK$. Les droites $(BD)$ et $(CE)$ se coupent en $I$ : mesurer l'angle $\widehat{BIC}$.
\end{enumerate}
\item Construire un quadrilatère $ABCD$ tel que $AD=52$ mm, $AB=CD=24$ mm et que les angles $\widehat{A}$ et $\widehat{D}$ soient égaux à $48^o$. Les droites $(AB)$ et 
$(CD)$ se coupent en $O$. 
\begin{enumerate}
\item Montrer que les triangles $AOD$ et $BOC$ sont isocèles et que la bissectrice de l'angle $\widehat{AOD}$ est aussi médiatrice de $[AD]$ et de $[BC]$.
\item Comparer les triangles $OAC$ et $OBD$. Que peut-on dire du point d'intersection
des segments $[AC]$ et $[BD]$ ? 
\end{enumerate}
\item Dans le quadrilatère $ABCD$ l'angle $\widehat{A}$ mesure $40^o$, les côtés $[AB]$ et $[AD]$ mesurent $3$ cm, et les angles $\widehat{B}$ et $\widehat{D}$ sont égaux à $52^o$. \begin{enumerate}
\item Construire le quadrilatère. Montrer que le triangle $BAD$ est isocèle puis, qu'il en est de même du triangle $BCD$.
\item Que représente la droite $(AC)$ pour le segment $[BD]$ et les angles $\widehat{A}$ et $\widehat{C}$ du quadrilatère. 
\end{enumerate}
\item Les côtés $[AB]$ et $[AD]$ du quadrilatère convexe $ABCD$ sont égaux et les angles $\widehat{B}$ et $\widehat{D}$ sont supplémentaires. On prolonge $[CB]$ d'une
longueur $BE=CD$. \begin{enumerate}
\item Comparer les triangles $ABE$ et $ADC$. Nature du triangle $ACE$ ? 
\item Comparer les angles $\widehat{CAE}$ et $\widehat{BAD}$ puis les angles $\widehat{ACB}$ et $\widehat{ACD}$. Que représente la diagonale $(CA)$ pour l'angle 
$\widehat{BCD}$ ? 
\end{enumerate}
\end{enumerate}
\chapter{Médiatrice d'un segment. Constructions géométriques}
\begin{enumerate}
\item Partager géométriquement un segment $[AB]$ en huit parties égales. 
\item Partager géométriquement un angle donné $\widehat{xOy}$ en $4$ angles égaux. Même problème pour un arc donné. 
\item Diviser géométriquement un cercle en $16$ arcs égaux. 
\item On se donne quatre points $ABCD$ tels que $(AB)$ soit perpendiculaire à $(CD)$ et $(AC)$ à $(BD)$. Construire les quatre cercles passant par trois de ces points.
Que remarque-t-on au sujet de leurs rayons ?
\item Trois cercles égaux de $3$ cm de rayon passent par un même point $A$ et 
se recoupent deux à deux en $B$, $C$ et $D$. Construire le cercle $BCD$ et mesurer
son rayon. 
\item \begin{enumerate}
\item Construire géométriquement le cercle circonscrit à un triangle $ABC$. 
\item Tracer sur la même figure les trois médianes du triangle. Que remarquez-vous ?
\end{enumerate}
\item Construire géométriquement les trois hauteurs d'un triangle, dans le cas où les trois angles sont aigus, puis où un angle est obtus. Que remarquez-vous ? 
\item Construire les trois bissectrices intérieures et les trois bissectrices extérieures du triangle $ABC$. Y a-t-il des points communs à trois de ces droites ?
\item Trois cercles égaux de centres donnés $A$, $B$, $C$ se coupent deux à deux. 
Démontrer que les cordes communes à deux de ces cercles concourent en un point $O$. 
Que représente ce point pour le triangle $ABC$ ? 
\item 
\begin{enumerate}
\item Construire sur une droite donnée $(\Delta)$ (ou sur un cercle $C$) un point équidistant de deux points donnés $A$ et $B$. 
\item Construire un cercle passant par $A$ et $B$ sachant que son centre est sur $(\Delta)$ (ou sur $C$).
\end{enumerate}
\item On mène d'un point $O$ la perpendiculaire $OH$ et deux obliques $OA$ et $OB$ de part et d'autre de $OH$ à une droite donnée $(xy)$.
\begin{enumerate}
\item Montrer que l'une des égalités $OA=OB$ et $HA=HB$ entraîne l'autre. 
\item Montrer que chacune des égalités précédentes est aussi équivalente à $\widehat{OAH} = \widehat{OBH}$ et à $\widehat{HOA}=\widehat{HOB}$.
\end{enumerate}
\item Soit un triangle isocèle $ABC$ de base $[BC]$ et soit $(xy)$ la bissectrice extérieure de l'angle $\widehat{A}$. On porte sur $(xy)$ deux segments égaux $[AD]$
et $[AE]$ ($D$ du côté de $B$ et $E$ du côté de $C$).\begin{enumerate}
\item Que représente la hauteur $[AH]$ du triangle $ABC$ pour le segment $[DE]$ ?
Comparer $HD$ et $HE$. 
\item Comparer les triangles $ABD$ et $ACE$ puis $ABE$ et $ACD$. Conséquences
pour $BD$ et $CE$ ainsi que pour $BE$ et $CD$ ? 
\end{enumerate}
\item On considère un triangle isocèle $AOB$ de sommet $O$. On prend un point $A'$ sur le côté $[OA]$ et un point $B'$ sur le côté $[OB]$ tels que les angles 
$\widehat{ABA'}$ et $\widehat{BAB'}$ soient égaux. $[AB']$ et $[AB]$ se coupent en $I$.\begin{enumerate}
\item Montrer que le triangle $IAB$ est isocèle. Préciser la position du point $I$
par rapport à la hauteur $[OH]$ du triangle $OAB$. 
\item Réciproquement,montrer que lorsque $I$ appartient au segment $[OH]$, les 
angles $\widehat{ABA'}$ et $\widehat{BAB'}$ sont égaux. Comparer $AA'$ et $BB'$.
\end{enumerate}
\item On prolonge la base $[AB]$ du triangle isocèle $[OAB]$ de deux longueurs égales
$AC$ et $BD$.\begin{enumerate}
\item Montrer que la hauteur $[OH]$ du triangle $OAB$ est la médiatrice de $[CD]$.
\item Démontrer que le triangle $OCD$ est isocèle puis que les triangles $OAC$ 
et $OBD$ sont égaux ainsi que les triangles $OAD$ et $OBC$.
\end{enumerate}
\item Soient $D$, $E$, $F$ les symétriques d'un point $M$ intérieur à un triangle $ABC$ par rapport aux côtés $[BC]$, $[CA]$ et $[AB]$.\begin{enumerate}
\item Que représente $A$ pour le cercle $MEF$ ? Propriété analogue des points $B$
et $C$ ? En déduire une construction au compas des points $D$, $E$, $F$. 
\item Les médiatrices du triangle $DEF$ se coupent en $P$. Montrer que $[AP]$ est
la bissectrice de l'angle $\widehat{EAF}$. Sachant que $\widehat{BAM}=16^o$ et $\widehat{CAM}=45^o$ trouver la valeur de l'angle $\widehat{CAP}$ et la comparer à celle de l'angle $\widehat{BAM}$. 
\end{enumerate}
\item Dans un triangle $ABC$ les médiatrices des côtés $[AB]$ et $[AC]$ se coupent
en un point $O$ de $[BC]$. On désigne par $M$ et $N$ les milieux de $[AB]$ et $[AC]$.
\begin{enumerate}
\item Montrer que $O$ est le milieu de $[BC]$ et que le cercle de diamètre $[BC]$ passe par $A$. 
\item Quelle est la nature du triangle $MON$ ? Montrer que l'angle $\widehat{A}$ du 
triangle $ABC$ est égal à la somme des angles $\widehat{B}$ et $\widehat{C}$ de ce triangle.
\end{enumerate}
\item Les médiatrices des côtés $[AB]$ et $[AC]$ du triangle $ABC$ se coupent en $O$
à l'intérieur de l'angle $\widehat{BAC}$. Comparer la somme des angles $\widehat{B}$
et $\widehat{C}$ du triangle à la somme des angles $\widehat{ABO}$ et $\widehat{ACO}$
puis à l'angle $\widehat{A}$ du triangle suivant que le point $O$ est : \begin{enumerate}
\item intérieur au triangle ; \item sur le côté $[BC]$ ; \item extérieur au triangle.
\end{enumerate}
\item Dans un quadrilatère $ABCD$ les médiatrices des segments $[AB]$, $[AC]$ et 
$[AD]$ se coupent en un même point $O$. \begin{enumerate}
\item Montrer que les points $A$, $B$, $C$, $D$ sont sur un même cercle. 
\item Montrer que les médiatrices du triangle $BDC$ se coupent en $O$.
\end{enumerate}
\item On considère un quadrilatère convexe $ABCD$ dans lequel la médiatrice de 
$[AB]$ est également la médiatrice de $[CD]$. Soit $O$ le point où cette médiatrice coupe la médiatrice de $[BC]$. \begin{enumerate}
\item Montrer que les quatre points $A$, $B$, $C$ et $D$ sont situés sur un même cercle de centre $O$. 
\item Démontrer que les médiatrices de $[AD]$, $[AC]$ et $[BD]$ passent également par $O$. 
\end{enumerate}
\item \begin{enumerate}
\item Construire un triangle isocèle $ABC$ tel que $BC = 5$ cm et $\widehat{B}=\widehat{C}= 53^o$ ; le point $D$ de $[BC]$ tel que $DA=DB$ et enfin le point $E$ 
du segment $[AD]$ tel que $AE=CD$.
\item Comparer les triangles $ABE$ et $CAD$. Nature du triangle $BDE$.
\end{enumerate}
\end{enumerate}
\chapter{Le troisième cas d'égalité des triangles}
\begin{enumerate}
\item Démontrer que lorsque deux triangles $ABC$ et $A'B'C'$ ont deux côtés respectivement égaux $AB=A'B'$ et $BC=B'C'$ ainsi que la médiane relative à l'un d'eux $AM = A'M'$, ces deux triangles sont égaux.
\item Démontrer que lorsque deux triangles $ABC$ et $A'B'C'$ ont deux côtés 
respectivement égaux $AB=A'B'$ et $AC=A'C'$ ainsi que la médiane relative au troisième 
$AM=A'M'$ ces deux triangles sont égaux (on prolongera $[AM]$ et $[A'M']$ d'une longueur égale $MD=M'D'=AM$ et on comparera d'abord $ABD$ et $A'B'D'$). 
\item Construire le centre $O$ du cercle circonscrit au triangle équilatéral $ABC$. 
\begin{enumerate}
\item Comparer les angles $\widehat{BOC}$, $\widehat{COA}$ et $\widehat{AOB}$ puis calculer leurs valeurs.
\item Montrer que les distances de $O$ aux trois côtés du triangle $ABC$ sont égales.
\end{enumerate}
\item Deux triangles $ABC$ et $A'B'C'$ sont égaux. Que peut-on dire des triangles obtenus en joignant dans chacun d'eux les pieds des hauteurs,des bissectrices, des médianes ?
\item Deux triangles équilatéraux $ABC$ et $A'B'C'$ sont tels que $AB=A'B'$. Comparer ces triangles ainsi que les rayons de leurs cercles circonscrits. 
\item Démontrer que dans un même cercle ou dans deux cercles égaux : \begin{enumerate}
\item Deux arcs égaux sont sous-tendus par des cordes égales. 
\item Deux cordes égales sous-tendent des arcs (inférieurs à un demi-cercle) égaux.
\end{enumerate}
\item Dans un quadrilatère $ABCD$, on a $AB=AD$ et $BC=DC$. 
\begin{enumerate}
\item Montrer que $(AC)$ est médiatrice du segment $[BD]$ et bissectrice des angles $\widehat{A}$ et $\widehat{C}$.
\item Soit $M$ le milieu de $[AC]$. Comparer $MB$ et $MD$ ainsi que $\widehat{AMB}$ et $\widehat{AMD}$.
\end{enumerate}
\item Le quadrilatère convexe $ABCD$ est tel que $AB=CD$ et $AD=BC$. \begin{enumerate}
\item Comparer les triangles $ABC$ et $CDA$, puis $ABD$ et $DBC$. Montrer qu'une diagonale forme des angles égaux avec deux côtés opposés. 
\item Les diagonales $[AC]$ et $[BD]$ se coupent en $O$. Que représente ce point pour
chacune d'elles ?
\end{enumerate}
\item Dans un quadrilatère convexe $ABCD$, on a $AD=BC$ et $AC=BD$. 
\begin{enumerate}
\item Comparer les triangles $ABC$ et $BAD$ puis $BCD$ et $ADC$. Montrer que l'un des côtés $[AB]$ ou $[CD]$ forme des angles égaux avec les deux diagonales.
\item Les diagonales se coupent en $I$. Nature des triangles $IAB$ et $ICD$ ? 
comparer les triangles $IAD$ et $IBC$. 
\end{enumerate}
\item Deux cercles de centre $O$ coupent une droite donnée $(xy)$ le premier en $A$ et $B$ le second en $C$ et $D$. Soit $H$ la projection de $O$ sur $(xy)$.
\begin{enumerate}
\item Nature des triangles $OAB$ et $OCD$ ? Que représente $H$ pour chacun des segments $[AB]$ et $[CD]$ ?
\item Comparer $AC$ et $BD$ puis $AD$ et $BC$ et enfin les triangles $OAC$ et $OBD$ et les triangles $OAD$ et $OBC$.
\end{enumerate}
\item \begin{enumerate}
\item Démontrer que deux triangles qui ont deux côtés respectivement égaux et même périmètre sont égaux.
\item Démontrer que deux triangles isocèles qui ont même base et même périmètre sont égaux.
\end{enumerate}
\item On donne un triangle $ABC$ tel que $AB<AC$. On prend sur le côté $[AC]$ le point $D$ tel que $AD=AB$. Le cercle de centre $A$ et de rayon $AC$ et le cercle de centre $D$ et de rayon $BC$ se coupent en $E$ et $F$. \begin{enumerate}
\item Montrer que l'un de ces points, $E$, par exemple, est sur la droite $(AB)$. 
\item Comment sont placés $E$ et $F$ par rapport à $[AC]$ ? Que représente $(AC)$ 
pour les angles $\widehat{BAF}$, $\widehat{ECF}$ et $\widehat{EDF}$ ? 
\end{enumerate}
\item On considère trois points $A$, $B$, $C$ dans cet ordre sur un cercle de centre $O$. La médiatrice de $[BC]$ coupe $[AC]$ en $M$. \begin{enumerate}
\item Comparer les triangles $OMB$ et $OMC$. Conséquence ? 
\item Démontrer que les angles $\widehat{OAM}$ et $\widehat{OBM}$ sont égaux. 
Que représente la droite $(MO)$ pour l'angle $\widehat{M}$ du triangle $MAB$ ?
\end{enumerate}
\item Démontrer que lorsque deux triangles $AOB$ et $OA'B'$ sont superposables par 
retournement, les médiatrices de $[AA']$ et $[BB']$ sont confondues. 
\item On considère deux segments égaux $[AB]$ et $[A'B']$. Construire le point $Q$ commun aux médiatrices de $[AA']$ et $[BB']$ et comparer les triangles $AOB$ et $OA'B'$. Montrer que ces deux triangles se superposent par glissement et comparer les angles $AOA'$ et $BOB'$. 
\end{enumerate}
\chapter{Cas d'égalité des triangles rectangles}
\begin{enumerate}
\item \begin{enumerate}
\item Dans un triangle $ABC$ le pied de la hauteur $[AH]$ se trouve entre $B$ et $C$.
Montrer que les angles $\widehat{B}$ et $\widehat{C}$ sont aigus. 
\item Lorsque $H$ est sur le prolongement $[Bx)$ de $[CB]$, montrer que l'angle $\widehat{B}$ est obtus et les angles $\widehat{C}$ et $\widehat{A}$ aigus. Combien
y a-t-il au moins d'angles aigus dans un triangle ? 
\end{enumerate}
\item Deux triangles $ABC$ et $A'B'C'$ sont égaux. Comparer les hauteurs homologues 
$AH$ et $A'H'$.
\item Démontrer que dans un triangle isocèle $ABC$ de sommet $A$ les hauteurs $[BH]$ et $[CK]$ sont égales. Énoncer et démontrer la réciproque. (Utiliser les triangles 
$ABH$ et $ACK$ ou $BCH$ et $CBK$.)
\item Montrer que dans un triangle isocèle les hauteurs issues des sommets de la base se coupent sur l'axe de symétrie du triangle. 
\item Démontrer que les sommets $B$ et $C$ du triangle $ABC$ sont équidistants de la médiane $[AM]$. La propriété subsiste-t-elle pour toute droite $(xy)$ passant par $M$ ?
\item Dans un cercle donné ou dans deux cercles égaux deux cordes égales sont équidistantes du centre. Énoncer et démontrer la réciproque. 
\item On mène d'un point $O$ la perpendiculaire $(OH)$ et les obliques $(OA)$ et $(OB)$ à une droite $(xy)$. Démontrer que l'égalité $OA=OB$ entraîne $HA=HB$ et réciproquement. 
\item Les deux segments concourants $[AB]$ et $[CD]$ ont même milieu $O$. 
\begin{enumerate}
\item Comparer les triangles $OAC$ et $OBD$ ainsi que les hauteurs $OH$ et $OK$ de ces deux triangles. 
\item Montrer que le point $O$ est le milieu de $[HK]$.
\end{enumerate}
\item Dans un quadrilatère $ABCD$ on a $AD=BC$ et $\widehat{DAB}= \widehat{ABC}<90^o$. On mène les perpendiculaires $(DH)$ et $(CK)$ à $(AB)$. \begin{enumerate}
\item Comparer les triangles $ADH$ et $BCK$ puis $DH$ et $CK$. 
\item Montrer que $AB$ et $HK$ ont le même milieu $I$ et que $HC=KD$.
\end{enumerate}
\item Construire un triangle isocèle $ABC$ sachant que $AB=AC=5$ cm et que la hauteur $AH=4$ cm. Mesurer $[BC]$.
\item On connaît en position l'hypoténuse $BC = 40$ mm d'un triangle rectangle $ABC$ et la longueur $AB=24$ mm. \begin{enumerate}
\item Construire le point $D$ du prolongement de $[BA]$ tel que $AD=AB$ puis le point 
$A$. Mesurer $AC$. 
\item Généraliser.
\end{enumerate}
\item Construire un triangle $ABC$ tel que $BC=60$ mm, la hauteur $AH=32$ mm et la médiane $AM=35$ mm. 
\item Dans le triangle $ABC$ la hauteur $AH$ mesure $38$ mm et la bissectrice $AD$ $42$ mm. Effectuer la construction du triangle $AHD$ puis celle du triangle $ABC$ sachant que :\begin{enumerate}
\item $\widehat{BAC}=64^o$ ; 
\item $AB=45$ mm.
\end{enumerate}
\item On considère un triangle $ABC$ rectangle en $A$ et on construit le segment $[BD]$ perpendiculaire à $(BC)$ et égal à $BC$, puis la perpendiculaire $(Bx)$ à
$(AB)$. On mène de $D$ la perpendiculaire $(DE)$ à $(Bx)$.\begin{enumerate}
\item Comparer les angles $\widehat{ABC}$ et $\widehat{EBD}$ puis les triangles
$ABC$ et $EBD$.
\item Démontrer que $AC=DE$ et $AB=BE$.
\end{enumerate}
\item Deux points $A$ et $B$ situés de part et d'autre de $(xy)$ sont tels que leurs distances $AH$ et $BK$ soient égales. Démontrer que $(xy)$ passe par le milieu $O$ 
de $[AB]$.
\item Comparer deux triangles isocèles ayant : \begin{enumerate}
\item Des bases égales et des angles au sommet égaux.
\item Les hauteurs relatives à la base égales et des angles à la base égaux.
\end{enumerate}
\item Deux triangles $ABC$ et $A'B'C'$ ont $AB=A'B'$, $\widehat{B}=\widehat{B'}$ et $\widehat{C}=\widehat{C'}$. On mène les hauteurs $[AH]$ et $[A'H']$. \begin{enumerate}
\item Comparer les triangles $ABH$ et $A'B'H'$, puis les triangles $AHC$ et $A'H'C'$.
\item En déduire l'égalité des triangles $ABC$ et $A'B'C'$ et énoncer le cas 
d'égalité correspondant.
\end{enumerate}
\item La médiatrice du côté $[AB]$ du triangle isocèle $ABC$ coupe la base $[BC]$ en $D$. Le cercle de centre $B$ passant par $D$ recoupe $[AD]$ en $E$. \begin{enumerate}
\item Nature des triangles $DAB$ et $BDE$. Comparer les angles $\widehat{ACD}$ et 
$\widehat{BAE}$. 
\item Démontrer, en utilisant le résultat de l'exercice précédent, l'égalité
des triangles $ACD$ et $BAE$ puis que $CD=AE$.
\end{enumerate}
\item \begin{enumerate}
\item Deux triangles $ABC$ et $A'B'C'$ ont $AB=A'B'$, $\widehat{B}+\widehat{B'}=180^o$ et $\widehat{C}= \widehat{C'}$. Montrer que les hauteurs $AH$ et $A'H'$ sont égales, puis que les côtés $[AC]$ et $[A'C']$ sont égaux.
\item Montrer que réciproquement si $AB=A'B'$, $AC=A'C'$ et $\widehat{B}+\widehat{B'}=180^o$, les angles $\widehat{C}$ et $\widehat{C'}$ sont égaux. \\ (On pourra amener $[A'B']$ sur $[AB]$ et $C'$ sur le prolongement de $[CB]$.)
\item Dans un triangle isocèle $OAB$, on prend un point $C$ du côté $[OA]$ et on
prolonge $[OB]$ d'une longueur $BD=AC$. Le segment $[CD]$ coupe $[AB]$ en $M$. 
\begin{enumerate}
\item Comparer les hauteurs $CH$ et $DK$ des triangles $MAC$ et $MBD$. 
\item Montrer que $M$ est le milieu de $[HK]$ et de $[CD]$.
\end{enumerate}
\end{enumerate}
\end{enumerate}
\chapter{Bissectrice d'un angle}
\begin{enumerate}
\item Construire sur une droite donnée $(D)$ un point $M$ équidistant de deux droites
$(xx')$ et $(yy')$.
\item Construire un point $M$ équidistant de deux droites $(xx')$ et $(yy')$ connaissant en outre sa distance à un point donné $O$. 
\item Construire un point $M$ équidistant de deux points donnés $A$ et $B$ et 
équidistant de deux droites données $(xx')$ et $(yy')$.
\item Un quadrilatère convexe $ABCD$ a deux angles opposés $\widehat{B}$ et $\widehat{D}$ droits et les côtés $[AB]$ et $[AD]$ égaux. Montrer que $CD=CB$
et que $(AC)$ est bissectrice des angles $\widehat{A}$ et $\widehat{C}$. 
\item Construire un triangle $ABC$ dont les bissectrices intérieures
se coupent en $I$ sachant que $\widehat{BIC}=124^o$, $IB=3$ cm et $IC = 4$ cm.
\item Démontrer qu'il y a quatre points $I$, $J$, $K$ et $L$ équidistants des trois droites $(BC)$, $(CA)$ et $(AB)$. Que représentent $(IJ)$, $(IK)$ et $(IL)$ pour
le triangle $JKL$ ?
\item Les angles $\widehat{ABC}$ et $\widehat{ADC}$ du quadrilatère convexe $ABCD$ 
sont supplémentaires et $BC=CD$. Le point $C$ se projette en $H$ sur $[AB]$ et en 
$K$ sur $[AD]$. \begin{enumerate}
\item Comparer les triangles $BCH$ et $CDK$ puis les longueurs $CH$ et $CK$.
\item Montrer que $(AC)$ est la bissectrice intérieure de l'angle $\widehat{BAD}$.
\end{enumerate}
\item Dans le quadrilatère convexe $ABCD$ on a $DB=DC$ et $\widehat{DBA}=\widehat{DCA}$. Le point $D$ se projette en $H$ sur $[AB]$ et en $K$ sur $[AC]$.
\begin{enumerate}
\item Comparer les triangles $BDH$ et $DCK$ puis les longueurs $DH$ et $DK$. 
\item Déterminer la somme des angles $\widehat{DAB}$ et $\widehat{DAC}$.
\end{enumerate}
\item Les bissectrices intérieures des angles $\widehat{B}$ et $\widehat{C}$ d'un triangle $ABC$ se coupent en $I$. On mène les perpendiculaires en $B$ à $(IB)$ et 
en $C$ à $(IC)$. Elles se coupent en $J$. \begin{enumerate}
\item Que représentent $(BJ)$ et $(CJ)$ pour les angles $\widehat{B}$ et $\widehat{C}$ du triangle ?
\item Montrer que les points $A$, $I$ et $J$ sont alignés.
\end{enumerate}
\item Un losange est un quadrilatère dont les quatre côtés sont égaux.\begin{enumerate}
\item Démontrer que les diagonales $[AC]$ et $[BD]$ sont médiatrices l'une de l'autre et bissectrices des angles intérieurs du losange. 
\item Montrer que leur point commun $O$ est équidistant des quatre côtés. 
\end{enumerate}
\item Soient $D$, $E$, $F$ les projections, sur les côtés $[BC]$, $[CA]$ et $[AB]$,
du point de concours $I$ des bissectrices intérieures du triangle $ABC$.
\begin{enumerate}
\item Montrer que $BD=BF$, $CE=CD$ et $AF=AE$. Que représente la somme $AE+BC$ pour le périmètre du triangle ? 
\item $BC= 48$ mm, $CA=40$ mm, et $AB=28$ mm. Calculer $AF$, $BD$ et $CE$.
\end{enumerate}
\item Dans un triangle $ABC$ tel que $AB>AC$ la médiatrice de $[BC]$ coupe en $M$
la bissectrice intérieure de l'angle $\widehat{A}$. Le point $M$ se projette en 
$H$ sur $[AB]$ et en $K$ sur $[AC]$. 
\begin{enumerate}
\item Comparer $AH$ et $AK$ puis $BH$ et $CK$. 
\item Démontrer que $AH=\frac12(AB+AC)$ et $BH=\frac12(AB-AC)$.
\end{enumerate}
\item Dans le même triangle $ABC$ qu'à l'exercice précédent, la médiatrice de $[BC]$ 
coupe en $M'$ la bissectrice extérieure de l'angle $\widehat{A}$. Le point $M'$ se projette en $H'$ sur $(AB)$ et en $K'$ sur $(AC)$. \begin{enumerate}
\item Comparer $AH'$ et $AK'$ puis $BH'$ et $CK'$.
\item Démontrer que $AH'=\frac12(AB-AC)$ et $BH'=\frac12(AB+AC)$ puis que $HH'=AC$. 
\end{enumerate}
\item On prolonge la base $[BC]$ du triangle $ABC$ de deux longueurs $BM=BA$ et $CN=CA$. Les bissectrices extérieures des angles $\widehat{B}$ et $\widehat{C}$ de ce triangle se coupent en $J$. \begin{enumerate}
\item Montrer que $M$ et $N$ sont les symétriques de $A$ par rapport à $(BJ)$ et $(CJ)$. Comparer $JA$, $JN$ et $JM$. Nature du triangle $JMN$. 
\item Comparer les angles $\widehat{BAJ}$ et $\widehat{CAJ}$. Démontrer ainsi que
\emph{dans un triangle les bissectrices extérieures de deux angles et la bissectrice
intérieure du troisième sont concourantes.}
\end{enumerate}
\item Un quadrilatère convexe $BICJ$ a ses angles opposés $\widehat{B}$ et $\widehat{C}$ droits. On construit les symétriques $(x'x)$ et $(y'y)$ de la droite 
$(BC)$ par rapport à chacune des droites $(IB)$ et $(IC)$. 
\begin{enumerate}
\item Montrer que $I$ et $J$ sont équidistants des trois droites $(BC)$, $(x'x)$ et 
$(y'y)$.
\item Démontrer que les trois droites $(IJ)$, $(x'x)$ et $(y'y)$ sont en générales 
concourantes en un même point $A$.
\end{enumerate}
\item On considère un triangle $ABC$ tel que $AB<AC$. La médiatrice de $[BC]$ coupe en $D$ le côté $[AC]$ et en $I$ et $J$ respectivement la bissectrice intérieure et la
bissectrice extérieure de l'angle $\widehat{BAC}$. \begin{enumerate}
\item Que représente $(IJ)$ pour l'angle $\widehat{ADB}$ ? Montrer que les points $I$ et $J$ sont équidistants des cotés du triangle $aBD$ et que $(BI)$ et $(BJ)$ sont 
les bissectrices extérieure et intérieure de l'angle $\widehat{ABD}$. 
\item Quelle est la valeur des angles $\widehat{IBJ}$ et $\widehat{ICJ}$ ? Démontrer que les angles $\widehat{ABJ}$ et $\widehat{ACJ}$ sont égaux et que les angles $\widehat{ABI}$ et $\widehat{ACI}$ sont supplémentaires.
\end{enumerate}
\end{enumerate}
\chapter*{Problèmes de révision } 
 \begin{enumerate}
 \item On donne un angle $\widehat{xOy}$ et un point $A$. Construire un triangle isocèle dont l'angle au sommet soit l'angle $\widehat{xOy}$ et tel que le côté opposé passe par $A$. 
 \item \begin{enumerate}
 \item Soit un triangle $ABC$ et $M$ le milieu de $[BC]$. La perpendiculaire au côté $[AB]$ issue de $M$ coupe ce côté en $H$. On prolonge $[HM]$ d'un segment $MK=HM$
 et on joint $[KC]$. Comparer les triangles $MBH$ et $MCK$. En déduire la valeur de l'angle $\widehat{MKC}$. 
 \item On donne un angle $\widehat{xAy}$ et un point $M$ intérieur à cet angle. Construire une droite passant par $M$ coupant $(Ax)$ en $B$ et $(Ay)$ en $C$ de façon que $M$ soit le milieu de $[BC]$. 
 \end{enumerate}
 \item \begin{enumerate}
 \item Construire deux droites $(D_1)$ et $(D_2)$ respectivement perpendiculaires
 en $A$ et $B$ au segment $[AB]$. Une droite $(xy)$ issue du milieu $O$ de $[AB]$ coupe $(D_1)$ en $P$ et $(D_2)$ en $Q$. Comparer les triangles $OAP$ et $OBQ$.
 \item Comparer les angles $\widehat{APO}$ et $\widehat{OQB}$. Que peut-on dire des angles que forme $(PQ)$ avec les droites $(D_1)$ et $(D_2)$ ?
 \end{enumerate}
 \item On donne un triangle isocèle $ABC$ de base $[BC]$. On construit les angles droits $\widehat{BAx}$ et $\widehat{CAy}$ adjacents à l'angle $\widehat{BAC}$. 
 Puis on porte sur $(Ax)$ et $(Ay)$ les segments $[AE]$ et $[AF]$ égaux à $[AB]$.
 Soit $M$ le milieu de $[BC]$ et $N$ le milieu de $[EF]$. \begin{enumerate}
 \item Comparer les triangles $MAE$ et $MAF$.
 \item Démontrer que les points $A$, $M$ et $N$ sont alignés.
 \end{enumerate}
 \item Soit un triangle $ABC$ rectangle en $A$, et $[AD]$ la hauteur relative à l'hypoténuse. On construit les symétriques $E$ et $F$ de $D$ par rapport aux droites $(AB)$ et $(AC)$.\begin{enumerate}
 \item Montrer que $D$, $E$, $F$, appartiennent à un même cercle de centre $A$.
 \item Montrer que les points $E$, $A$, $F$ sont alignés. Que représente $A$ pour le segment $[EF]$ ?
 \end{enumerate}
 \item D'un même côté d'un segment $[AB]$, on mène deux segments égaux $[AD]$ et $[BC]$ respectivement perpendiculaires à $[AB]$ en $A$ et en $B$.
 \begin{enumerate}
 \item Comparer les segments $[AC]$ et $[BD]$. 
 \item $[AC]$ et $[BD]$ se coupent en $O$. Démontrer que la médiatrice de $[AB]$ passe par $O$ et qu'elle est aussi médiatrice de $[CD]$.
 \end{enumerate}
 \item Deux segments $[AC]$ et $[DB]$ sont perpendiculaires et leur point commun
 $O$ est le milieu de chacun d'eux. \begin{enumerate}
 \item Montrer que le quadrilatère $ABCD$ a ses quatre côtés égaux et comparer les 
 quatre triangles $OAB$, $OCB$, $OCD$ et $OAD$.
 \item On mène $OP$ perpendiculaire à $[AD]$ en $P$. La droite $(PO)$ coupe $(BC)$ en 
 $R$. Comparer les triangles $ODP$ et $OBR$. En déduire que $(OR)$ est perpendiculaire à $(BC)$ et que $OP=OR$. 
 \item Montrer de même que la droite $(OQ)$ perpendiculaire en $Q$ à $(AB)$ est 
 perpendiculaire en $S$ à $(CD)$ et que $OP=OQ=OR=OS$. 
 \end{enumerate}
 \item Construire un angle $\widehat{AOB}=30^o$ puis deux angles $\widehat{AOC}$ et
 $\widehat{BOD}$ adjacents à $\widehat{AOB}$ et valant $60^o$ chacun. \begin{enumerate}
 \item Montrer que les angles $\widehat{AOB}$ et $\widehat{COD}$ ont même bissectrice.
 \item Montrer que les bissectrices $(ON)$ et $(OP)$ des angles $\widehat{AOC}$ et $\widehat{BOD}$ dont un angle droit.
 \end{enumerate}
 \item On donne un angle $\widehat{xOy}=45^o$ et un point $A$ intérieur à cet angle.
 Construire les points $B$ et $C$ symétriques de $A$ par rapport aux droites
 $(Ox)$ et $(Oy)$.\begin{enumerate}
 \item Évaluer l'angle $\widehat{BOC}$. Nature du triangle $BOC$.
 \item Démontrer que la médiatrice du segment $[BC]$ passe par $O$. Comparer l'angle
 formé par cette médiatrice et $(Oy)$ à l'angle $\widehat{AOx}$. 
 \end{enumerate}
 \item Trois segments $OA$, $OB$, $OC$ sont égaux et tels que les angles $\widehat{AOB}$, $\widehat{BOC}$, et $\widehat{COA}$ soient égaux :  \begin{enumerate}
 \item Quelle est la valeur commune de ces trois angles ? 
 \item Démontrer que le triangle $ABC$ est équilatéral. 
 \item Montrer que $O$ est équidistant des trois côtés du triangle $ABC$ et que
 le triangle ayant pour sommets les pieds des perpendiculaires menées de $O$ 
 aux côtés du triangle $ABC$ est lui-même équilatéral.
 \end{enumerate}
 \item Construire un triangle isocèle tel que $AB=AC=5$ cm et tel que l'angle
 $\widehat{BAC}=40^o$. La médiatrice de $[AC]$ coupe la droite $(BC)$ en $D$. On joint $[DA]$ et on le prolonge de $AE=BD$.\begin{enumerate}
 \item Montrer que $DAC$ est isocèle, puis établir l'égalité des angles $\widehat{CAE}$ et $\widehat{ABD}$. 
 \item Comparer les triangles $CAE$ et $ABD$. En déduire que le triangle $CDE$ est
 isocèle.
 \end{enumerate}
 \item Soit un triangle $ABC$ et la bissectrice intérieure $(Ax)$ de l'angle
 $\widehat{A}$. sur la demi-droite $[Ax)$, on construit les points $B'$ et $C'$ tels
 que $AB'=AB$ et $AC'=AC$. Les droites $(BC')$ et $(B'C)$ se coupent en $D$. \begin{enumerate}
 \item Comparer les triangles $ABC'$ et $ACB'$ puis les segments $[BC']$ et $[B'C]$.
 \item On mène $(AE)$ perpendiculaire en $E$ à $(BC')$ et $(AF)$ perpendiculaire en $F$ à $(B'C)$. \\ Demontrer que $AE=AF$ puis que $(DA)$ est une bissectrice du triangle $BCD$.
 \end{enumerate}
 \item On donne un triangle $ABC$. Sur le côté $[AB]$ on porte $AC'=AC$ et sur le côté $[AC]$ on porte $AB'=AB$. La bissectrice intérieure de l'angle $A$ coupe $[BC]$ en $D$.\begin{enumerate}
 \item Comparer les segments $[DC']$ et $[DC]$ puis les segments $[DB']$ et $[DB]$.
 \item Démontrer que les points $C'$, $D$ et $B'$ sont alignés.
 \end{enumerate}
 \item Soit $I$ le point commun aux bissectrices intérieures du triangle $ABC$. 
 On construit les segments $[IA']$, $[IB']$ et $[IC']$ respectivement perpendiculaires en $A'$, $B'$ et $C'$ aux côtés $[BC]$, $[CA]$ et $[AB]$. 
 \begin{enumerate}
 \item Démontrer que $(IA)$, $(IB)$ et $(IC)$ sont les médiatrices du triangle $A'B'C'$ et les bissectrices des angles $\widehat{B'IC'}$, $\widehat{C'IA'}$ et
 $\widehat{A'IB'}$.
 \item Montrer que les angles $\widehat{AIB'}$ et $\widehat{BIC}$ sont supplémentaires.
 \end{enumerate}
 \item On considère deux triangles isocèles $OAB$ et $OCD$ de sommet $O$ tels que les angles $\widehat{AOB}$ et $\widehat{COD}$ aient même bissectrice $(Ox)$. \begin{enumerate}
 \item Comparer les triangles $OAC$ et $OBD$. 
 \item Soit $M$ un point quelconque de $(Ox)$. Comparer les triangles $MAC$ et $MBD$.
 \end{enumerate}
 \item\begin{enumerate}
 \item Construire un triangle $ABC$ connaissant $AB= 8$ cm; $BC= 5$ cm et sachant que l'angle $\widehat{BAC}$ vaut $30^o$. Nombre de solutions ?
 \item Déduire de cette construction que deux triangles qui ont deux côtés respectivement égaux et un angle égal opposé à l'un de ces côtés ne sont pas nécessairement égaux.
\end{enumerate}  
\item On considère deux triangles $ABC$ et $A'B'C'$ tels que les angles $\widehat{A}$ et $\widehat{A'}$ soient supplémentaires, $AB=A'B'$ et $BC=B'C'$. On prolonge $[CA]$
d'une longueur $AD$ égale à $A'C'$.\begin{enumerate}
\item Comparer les triangles $A'B'C'$ et $ABD$. Nature du triangle $BCD$ ? 
\item Démontrer que les angles $\widehat{ACB}$ et $\widehat{A'C'B'}$ sont égaux.
\end{enumerate}
\item Un angle $\widehat{xOy}$ vaut $120^o$. D'un point $A$ de sa bissectrice on mène les perpendiculaires $(AM)$ à $(Ox)$ et $(AN)$ à $(Oy)$. La droite $(AM)$ coupe $(Oy)$ en $B$, et la droite $(AN)$ coupe $(Ox)$ en $C$. \begin{enumerate}
\item Comparer les triangles $OAM$ et $OBM$ ainsi que les triangles $OAN$ et $OCN$.
\item Démontrer que $OA=OB=OC$ puis que le triangle $ABC$ est équilatéral. \\ 
Que représente le point $O$ pour le triangle $ABC$ ?
\end{enumerate}
\item Soit un cercle de diamètre $[AB]$ de centre $O$. La médiatrice de $[OA]$ coupe
le cercle en $C$ et $D$ : la médiatrice de $[OB]$ coupe le cercle en $E$ et $F$ ($C$ et $E$ sont du même côté de $[AB]$). \begin{enumerate}
\item Montrer que les triangles $OAC$, $OAD$, $OBE$ et $OBF$ sont équilatéraux et égaux.
\item Montrer que les points $C$, $O$, $F$ d'une part et $D$, $O$, $E$ sont alignés.
\item Montrer que le diamètre du cercle $O$ perpendiculaire à $[AB]$ est médiatrice
des segments $[CE]$ et $[DF]$. 
\end{enumerate}
\item Construire la hauteur $[AH]$ et la médiane $[AM]$ du triangle $ABC$, puis
prolonger $[AH]$ d'un segment $HD=AH$ et $[AM]$ d'un segment $ME=AM$. Les droites $(BD)$ et $(CE)$ se coupent en $P$. \begin{enumerate}
\item Comparer les triangles $ABH$ et $DBH$, puis les triangles $MAB$ et $MEC$. 
\item Démontrer que le triangle $PBC$ est isocèle et que $(PM)$ est médiatrice des segments $[BC]$ et $[BE]$. 
\item On suppose de plus que $MA=MB$. Montrer que dans ce cas les cinq points $ABCDE$ sont sur un même cercle.
\end{enumerate}
\item Dans un quadrilatère $ABCD$ les angles $\widehat{ABC}$ et $\widehat{ADC}$ sont égaux et la diagonale $[AC]$ est bissectrice de l'angle $\widehat{BAD}$. On mène 
les perpendiculaires $(CH)$ et $(CK)$ à $(AB)$ et à $(AD)$. \begin{enumerate}
\item Comparer les triangles $AHC$ et $AKC$, puis les triangles $BHC$ et $DKC$.
\item Démontrer que $BC=CD$, $AB=AD$ et que $(AC)$ est bissectrice de l'angle
$\widehat{BCD}$.
\end{enumerate}
\item On considère un quadrilatère convexe $ABCD$ dans lequel la diagonale $[AC]$ 
fait des angles aigus égaux avec les côtés $[AB]$ et $[CD]$. On suppose en outre que les angles $\widehat{B}$ et $\widehat{D}$ du quadrilatère sont aigus et égaux. On mène les perpendiculaires $(AH)$ à $(CD)$ et $(CK)$ à $(AB)$.\begin{enumerate}
\item Comaprer les triangles rectangles $AKC$ et $CHA$, puis les triangles $BKC$ et $DHA$. \item Démontrer que $[AC]$ fait des angles égaux avec $(AD)$ et $(BC)$, que $AB=CD$ et que $AD=BC$.
\end{enumerate}
\item Construire un quadrilatère convexe $ABCD$ dans lequel $BC=3$ cm, $\widehat{ABD}=80^o$, $\widehat{BDC}=100^o$ et $AD=BC=4$ cm. On prolonge $[AB]$ d'une longueur $BE=CD$. \begin{enumerate}
\item Comparer les triangles $DBC$ et $DBE$. Conséquences ? 
\item Nature du triangle $DAE$ ? Démontrer que les angles $\widehat{BAD}$ et $\widehat{BCD}$ sont égaux.
\end{enumerate}
\item Dans le quadrilatère convexe $ABCD$ les diagonales $[AC]$ et $[BD]$ sont égales et les angles $\widehat{ABC}$ et $\widehat{BCD}$ sont supplémentaires. On prolonge $[AB]$ d'une longueur $BE=CD$. \begin{enumerate}
\item Comparer les triangles $BCD$ et $CBE$. Nature du triangle $CAE$ ? 
\item Démontrer que les angles $\widehat{BAC}$ et $\widehat{BDC}$ sont égaux.
\end{enumerate}
\item Soit un triangle $ABC$ tel que $AB>AC$. La bissectrice intérieure de l'angle
$\widehat{A}$ coupe en $D$ la médiatrice de $[BC]$. On construit le point $E$ de $[AB]$ tel que $AE=AC$. \begin{enumerate}
\item Que représente le point $D$ pour le triangle $BCE$ ? Nature du triangle $DBE$ ?
\item Comparer les triangles $ACD$ et $AED$ et montrer que les angles $ABD$ et $ACD$
sont supplémentaires. 
\end{enumerate}
\item Dans le quadrilatère convexe $ABCD$ les côtés $[BC]$ et $[CD]$ sont égaux 
et les angles $\widehat{B}$ et $\widehat{D}$ sont supplémentaires. Soient $H$ et $K$
les projections du point $C$ sur les droites $(AB)$ et $(AD)$.
\begin{enumerate}
\item Comparer les triangles $BCH$ et $DCK$. Conséquences ?
\item Montrer que $(AC)$ est bissectrice de l'angle $\widehat{BAD}$. Comparer $AH$
et $AK$ à la demi-somme de $AB$ et $AD$.
\end{enumerate}
\item Soit $O$ le centre du cercle circonscrit au triangle $ABC$. La médiatrice 
de $[BC]$ coupe en $I$ le côté $[AB]$ et en $J$ la droite $(AC)$.\begin{enumerate}
\item Comparer les triangles $IOB$ et $IOC$ et démontrer que les angles
$\widehat{IAO}$ et $\widehat{ICO}$ sont égaux.
\item Comparer les triangles $JOB$ et $JOC$ et démontrer que les angles $\widehat{JAO}$ et $\widehat{JBO}$ sont supplémentaires.
\end{enumerate}
\item La bissectrice extérieure de l'angle $\widehat{A}$ du triangle $ABC$ coupe en $M$ la médiatrice de $[BC]$. On prolonge $[BA]$ d'une longueur $AD=AC$.\begin{enumerate}
\item Comparer $MB$ et $MC$, puis les triangles $MAC$ et $MAD$.
\item Nature du triangle $MBD$ ? Démontrer l'égalité des angles $\widehat{ABM}$ et 
$\widehat{ACM}$.
\end{enumerate}
\item Dans le quadrilatère convexe $ABCD$ on a $AB>CD$, $AD=BC$ et d'autre part la 
diagonale $[AC]$ fait des angles aigus égaux avec $[AB]$ et $[CD]$. 
\begin{enumerate}
\item Construire le point $E$ de $[AB]$ tel que $AE=CD$ et comparer les triangles $ACD$ et $CAE$. Nature du triangle $BCE$ ? 
\item Démontrer que les angles $ABC$ et $ADC$ sont supplémentaires ? 
\end{enumerate}
\item Dans le triangle $ABC$, on considère un point intérieur $D$ tel que $AD=BC$ et tel que les angles $\widehat{ABC}$ et $\widehat{ADC}$ soient supplémentaires. On
prolonge $[AB]$ d'une longueur $BE=CD$.\begin{enumerate}
\item Comparer les triangles $ADC$ et $CBE$. Nature du triangle $CAE$ ? 
\item La droite $(CD)$ coupe $(AB)$ en $M$. Démontrer que $MA=MC$ et en déduire une construction géométrique du point $D$ en connaissant le triangle $ABC$. 
\end{enumerate}
\item Les côtés $[AB]$ et $[CD]$ d'un quadrilatère convexe $ABCD$ sont égaux. Les diagonales $[AC]$ et $[BD]$ se coupent en un point $O$ de telle sorte que $OA=OC$ et 
$OB>OD$. Les droites $(AB)$ et $(CD)$ se coupent en $I$. \begin{enumerate}
\item Construire le point $E$ du segment $[OB]$ tel que $OE=OD$ et comparer les triangles $OAE$ et $OCD$. Nature du triangle $ABE$ ? 
\item Comparer les angles $\widehat{ABO}$ et $\widehat{CDO}$, puis les segments $[IB]$ et $[ID]$.
\end{enumerate}
\item On considère un quadrilatère convexe $ABCD$ tel que $AB=CD$. Les diagonales 
$[AC]$ et $[BD]$ se coupent en $O$ tel que : $OA>OD$ et $OB=OC$. \begin{enumerate}
\item Soit $E$ le point du segment $[OA]$ tel que $OE=OD$. Comparer les triangles
$OBE$ et $OCD$.
\item Nature du triangle $BAE$ ? Comparer les angles $\widehat{BAC}$ et $\widehat{BDC}$.
\item Les droites $(BE)$ et $(CD)$ se coupent en $I$. Comparer $IB$ et $IC$, ainsi que $ID$ et $IE$ et montrer que $(IO)$ est bissectrice de l'angle $\widehat{BIC}$.
\end{enumerate}
\item Soit un triangle isocèle $OAB$ de base $AB<AO$. Le cercle de centre $A$ passant par $O$ recoupe la droite $(OB)$ en $E$. On prend un point $C$ sur le segment $[BE]$
et le point $D$ du segment $[OA]$ tel que $OD=EC$. \begin{enumerate}
\item Nature du triangle $AOE$ ? Comparer les triangles $AEC$ et $BOD$. 
\item Comparer les segments $[AC]$ et $[BD]$ ainsi que les angles $\widehat{ACB}$ et
$\widehat{ADB}$. 
\end{enumerate}
\item On désigne par $D$ et $E$ les milieux des côtés $[AB]$ et $[AC]$ du triangle $ABC$ et par $G$, $H$ et $K$ les projections sur la droite $(DE)$ des points $A$, $B$ et $C$. \begin{enumerate}
\item Comparer les triangles $AGD$ et $BHD$ ainsi que les triangles $AGE$ et $CKE$. 
En déduire que $BH=CK$ et que $HK=2DE$. 
\item Soit $M$ le milieu de $[BC]$ et $I$ le milieu de $[HK]$. Comparer les triangles $IBH$ et $ICK$. Nature du triangle $BIC$ ? Montrer que $(IM)$ est médiatrice de $[BC]$ et de $[HK]$. 
\end{enumerate}
\item Un \emph{trapèze isocèle} est un quadrilatère convexe $ABCD$ dont les \emph{bases} $AB$ et $CD$ admettent la même médiatrice $(xy)$. Démontrer que : 
\begin{enumerate}
\item Les angles adjacents à une même base sont égaux. 
\item Les côtés obliques $[AD]$ et $[BC]$ sont égaux et leurs prolongements se coupent en $I$ sur la droite $(xy)$ formant deux triangles $IAB$ et $ICD$ isocèles. 
\item Les diagonales $[AC]$ et $[BD]$ sont égales et se coupent en $J$ sur $(xy)$
formant deux triangles $JAB$ et $JCD$ isocèles.
\end{enumerate}
\item Démontrer qu'un quadrilatère convexe $ABCD$ est un trapèze isocèle (cf. exercice précédent), lorsque : \begin{enumerate}
\item les angles $\widehat{A}$ et $\widehat{B}$ sont égaux ainsi que les angles 
$\widehat{C}$ et $\widehat{D}$. 
\item Lorsque les côtés $[AD]$ et $[BC]$ sont égaux ainsi que les angles $\widehat{A}$ et $\widehat{B}$.
\item Lorsque les côtés $[AD]$ et $[BC]$ sont égaux ainsi que les diagonales $[AC]$ et $[BD]$.
\end{enumerate}
\item Le quadrilatère convexe $ABCD$ admet pour axe de symétrie la droite $(xy)$ médiatrice commune de $[AB]$ et $[CD]$ (trapèze isocèle). \begin{enumerate}
\item Démontrer que les médiatrices des côtés $[BC]$ et $[AD]$ se coupent en un point $O$ de $(xy)$, centre d'un cercle circonscrit au trapèze. 
\item Comparer sur ce cercle les arcs de même sens $\arc{DA}$ et $\arc{BC}$ ainsi que les arcs $\arc{DB}$ et $\arc{AC}$. 
\end{enumerate}
\item On considère un quadrilatère convexe $ABCD$ dont les sommets sont sur un cercle de centre $O$ de telle sorte que les côtés $[AD]$ et $[BC]$ soient égaux. \begin{enumerate}
\item Compare les angles au centre $\widehat{AOD}$ et $\widehat{BOC}$.
\item Démontrer que la bissectrice $(Ox)$ de l'angle $\widehat{AOB}$ est également bissectrice de l'angle $\widehat{AOB}$ est également bissectrice de l'angle $\widehat{COD}$ et médiatrice des deux côtés $[AB]$ et $[CE]$ du quadrilatère $ABCD$. 
\end{enumerate}
\item On désigne par $M$ et $N$ les milieux des côtés obliques $[AD]$ et $[BC]$ du trapèze isocèle $ABCD$ dont les bases $[AB]$ et $[CD]$ admettent pour médiatrice commune la droite $(xy)$. Le segment $[MN]$ coupe les diagonales $[AC]$ en $P$ et $[BD]$ en $Q$. \begin{enumerate}
\item Démontrer que $AM=CN$ et que les angles $\widehat{DMN}$ et $\widehat{CNM}$ sont égaux. 
\item On prolonge $[MP]$ d'une longueur $ME$ égale à $NP$. Comparer les triangles $AME$ et $CNP$. Nature du triangle $APE$ ? En déduire que les points $P$ et $Q$ sont les milieux des diagonales $[AC]$ et $[BD]$.
\end{enumerate}
\item Construire un triangle $ABC$ tel que $AB= 6$ cm, $BC=5$ cm, $AC=4$ cm, puis le point $D$ du côté $[AB]$ et le point $E$ du prolongement de $[CA]$ tels que $AD=AE=1$ cm. La droite $(DE)$ coupe $(BC)$ en $M$. \begin{enumerate}
\item Comparer les segments $[BD]$ et $[CE]$ ainsi que les angles $\widehat{BDM}$ 
et $\widehat{CEM}$.
\item On prolonge $[DM]$ d'une longueur $MP=DE$. Comparer les triangles $BDP$ et $CEM$. Nature du triangle $BMP$ ? Démontrer que $M$ est le milieu de $[BC]$. 
\end{enumerate}
\item Soient quatre points $A$, $B$,$C$, $D$ disposés dans cet ordre sur un cercle de centre $O$ et tels que $AC=BD$. On désigne par $E$ le point d'intersection de $[AC]$
et $[BD]$, par $I$ le milieu de $[AC]$ et par $J$ le milieu de $[BD]$. La droite $(IJ)$ coupe $(AB)$ en $M$.
\begin{enumerate}
\item Comparer les triangles $OAI$ et $OBJ$, puis les triangles $OEI$ et $OEJ$. 
Nature du triangle $EIJ$ ? 
\item On prolonge $[JI]$ d'une longueur $IN=JM$. Comparer les triangles $AIN$ et $BJM$. Nature du triangle $AMN$ ? En déduire que $M$ est le milieu de $[AB]$. 
\end{enumerate}
\item On considère un angle aigu $\widehat{BAC}$. Le point $B$ se projette en $E$
sur $[AC]$ et le point $C$ se projette en $F$ sur $[AB]$. Les segments $[BE]$ et $[CF]$ se coupent en $H$. Les symétriques de la droite $(EF)$ par rapport à $(BE)$
et par rapport à $(CF)$ se coupent en $D$. 
\begin{enumerate}
\item Montrer que les quatre points $A$, $B$, $C$ et $H$ sont équidistants des trois droites $(EF)$, $(FD)$ et $(DE)$. 
\item En déduire que $(AH)$ et $(BC)$ sont les bissectrices de l'angle $\widehat{D}$ du triangle $DEF$ et qu'elles sont perpendiculaires. Démontrer ainsi que les trois 
hauteurs du triangle $ABC$ sont concourantes en $H$. 
\end{enumerate}
 \end{enumerate}
 
 
 
 
 	\end{document}
