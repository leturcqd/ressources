	\documentclass[14pt]{extreport}
\usepackage{extsizes}
	\usepackage[frenchb]{babel}
	\usepackage[utf8]{inputenc}  
	\usepackage[T1]{fontenc}
	\usepackage{amssymb}
	\usepackage[mathscr]{euscript}
	\usepackage{stmaryrd}
	\usepackage{amsmath}
	\usepackage{tikz}
\usepackage{eurosym}
	\usepackage[all,cmtip]{xy}
	\usepackage{amsthm}
	\usepackage{varioref}
	\usetikzlibrary{patterns}
	\usepackage{float}
	\usepackage[ margin=.6in]{geometry}
	\geometry{a4paper}
	\usepackage{lmodern}
	\usepackage{hyperref}
	\usepackage{array}
	\usepackage{easytable}
	 \usepackage{fancyhdr}\usepackage{longtable}

	\pagestyle{fancy}
	\theoremstyle{plain}
	\fancyfoot[C]{\empty} 
	\fancyhead[L]{Contrôle multiplication-cercles}
	\fancyhead[R]{2 avril 2024}
	
	
	\title{Contrôle chapitre 7}
	\date{}
	\begin{document}



\subsection*{Exercice 1 (3 points)}

Joachim achète $4,6$ kg de carottes à $4,90$ euros/kg et $350$ g de bœuf à $23,50$ euro/kg. Quel est le prix total à payer ?

\subsection*{Exercice 2 (4 points)}

Le livreur d'une fromagerie charge $134$ fromages pesant chacun $2,36$ kg
et $122$ autres pesant chacun $3,11$ kg dans une voiture pouvant transporter $550$ kg. Il part à $6$h $12$ et parvient à destination à $7$h $10$. Il a roulé à $70,32$ km/h en moyenne. 

Le véhicule est-il en surcharge ?
Si oui, de combien ? Si non, combien reste-t-il ?

\subsection*{Exercice 3 (4 points)}  % 4 points

On considère une table ronde formant un cercle de rayon $80$ cm. 
\begin{enumerate}
\item Calculez le périmètre de la table : donnez une valeur exacte, puis une valeur approchée au millimètre.
\item On veut cercler cette table avec un plaquage en chêne. Le prix du placage au mètre est de $19$ euros et $96$ centimes. Calculez le montant que l'on devra débourser.
\end{enumerate}




\subsection*{Exercice 4 (6 points)}
On considère les deux figures suivantes : 

\begin{figure}[H]
\center 
\begin{tikzpicture}[scale =1.3]
\draw (-2, 0) -- (2, 0) arc (0: 180 :2);
\draw (-2, 0) -- (2, 0) arc (0: 180 :2);
\draw[<->] (-2, -.2) -- (2, -.2); 
\draw (0, -.5) node {$7 \ cm$};
\draw[white](0,-1)--(3,-1);
\end{tikzpicture}
\begin{tikzpicture}[scale =1.3]
\draw[dashed](-2, 0) -- (2, 0);
\draw(2,0) arc (0: 180 :2) arc (180:360: 1) arc(180:0:1);
\draw(2,0) arc (0: 180 :2) arc (180:360: 1) arc(180:0:1);
\draw[<->] (2, -.2) -- (0, -.2); 
\draw (1, -.5) node {$4 \ cm$};
\end{tikzpicture}
\end{figure} 
\begin{enumerate}
\item Calculer une valeur exacte du périmètre de la première figure. Donnez ensuite une valeur approchée au millimètre. 

\item Calculer une valeur exacte du périmètre de la seconde figure. Donnez ensuite une valeur approchée au millimètre. 
\end{enumerate} 


\subsection*{Exercice 5 (3 points)}

Dans un ménage, le père gagne $559$ euros par mois, la mère $67$ euros par semaine, et le fils $134$ euros par trimestre. Quel est le revenu annuel de cette famille ?

\newpage



\subsection*{Exercice 1 (3 points)}

Joachim achète $3,6$ kg de carottes à $4,90$ euros/kg et $350$ g de bœuf à $23,50$ euro/kg. Quel est le prix total à payer ?

\subsection*{Exercice 2 (4 points)}

Le livreur d'une fromagerie charge $134$ fromages pesant chacun $2,36$ kg
et $122$ autres pesant chacun $3,11$ kg dans une voiture pouvant transporter $650$ kg. Il part à $6$h $12$ et parvient à destination à $7$h $10$. Il a roulé à $70,32$ km/h en moyenne. 

Le véhicule est-il en surcharge ?
Si oui, de combien ? Si non, combien reste-t-il ?

\subsection*{Exercice 3 (4 points)}  % 4 points

On considère une table ronde formant un cercle de rayon $80$ cm. 
\begin{enumerate}
\item Calculez le périmètre de la table : donnez une valeur exacte, puis une valeur approchée au millimètre.
\item On veut cercler cette table avec un plaquage en chêne. Le prix du placage au mètre est de $19$ euros et $93$ centimes. Calculez le montant que l'on devra débourser.
\end{enumerate}




\subsection*{Exercice 4 (6 points)}
On considère les deux figures suivantes : 

\begin{figure}[H]
\center 
\begin{tikzpicture}[scale =1.3]
\draw (-2, 0) -- (2, 0) arc (0: 180 :2);
\draw (-2, 0) -- (2, 0) arc (0: 180 :2);
\draw[<->] (-2, -.2) -- (2, -.2); 
\draw (0, -.5) node {$7 \ cm$};
\draw[white](0,-1)--(3,-1);
\end{tikzpicture}
\begin{tikzpicture}[scale =1.3]
\draw[dashed](-2, 0) -- (2, 0);
\draw(2,0) arc (0: 180 :2) arc (180:360: 1) arc(180:0:1);
\draw(2,0) arc (0: 180 :2) arc (180:360: 1) arc(180:0:1);
\draw[<->] (2, -.2) -- (0, -.2); 
\draw (1, -.5) node {$3 \ cm$};
\end{tikzpicture}
\end{figure} 
\begin{enumerate}
\item Calculer une valeur exacte du périmètre de la première figure. Donnez ensuite une valeur approchée au millimètre. 

\item Calculer une valeur exacte du périmètre de la seconde figure. Donnez ensuite une valeur approchée au millimètre. 
\end{enumerate} 


\subsection*{Exercice 5 (3 points)}

Dans un ménage, le père gagne $559$ euros par mois, la mère $66$ euros par semaine, et le fils $134$ euros par trimestre. Quel est le revenu annuel de cette famille ?


\end{document}