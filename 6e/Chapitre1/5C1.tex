\documentclass[12 pt]{extarticle}

	\usepackage[frenchb]{babel}
	\usepackage[utf8]{inputenc}  
	\usepackage[T1]{fontenc}
	\usepackage{amssymb}
	\usepackage[mathscr]{euscript}
	\usepackage{stmaryrd}
	\usepackage{amsmath}
	\usepackage{tikz}
	\usepackage[all,cmtip]{xy}
	\usepackage{amsthm}
	\usepackage{varioref}
	\usepackage{geometry}
	\geometry{a4paper}
	\usepackage{lmodern}
	\usepackage{hyperref}
	\usepackage{array}
	 \usepackage{fancyhdr}
\renewcommand{\theenumi}{\alph{enumi})}
	\pagestyle{fancy}
	\theoremstyle{plain}
	\fancyfoot[C]{} 
	\fancyhead[L]{Fiche d'exercices}
	\fancyhead[R]{2023-2024}\geometry{
 a4paper,
 total={170mm,257mm},
 left=20mm,
 top=20mm,
 }
	
	
	\title{Exercices Chapitre 1}
	\date{}
	\begin{document}

\begin{center}{\Large Chapitre 1 - Rappels de  numération}\\ 
 \end{center}
 
 
 \subsection*{Exercice 1 - Écriture décimale des nombres entiers}
 
 \begin{enumerate}
 \item Donner l'écriture décimale de soixante-et-onze, cent vingt-et-un, et dix millions mille deux cent treize. 
 \item Lire à haute voix $7203$, $23429$, et $1294203$. 
\end{enumerate} 
 
 
  
\subsection*{Exercice 2 - Nombre inconnu }
\begin{enumerate}
\item Trouver un nombre à trois chiffres dont le chiffre des centaines est égal à la somme des chiffres des dizaines et des unités. Combien y a-t-il de solutions possibles ? Quel est le plus grand tel nombre ? le plus petit ? 
\item Quel est le plus grand nombre dont le premier chiffre soit la somme des chiffres suivants ?
 
 \end{enumerate}
 
 
 
 
\subsection*{Exercice 3 - Nombre inconnu (difficile/long)}
 \begin{enumerate}
\item Trouver un nombre décimal dont les chiffres sont distincts et non nuls, et dont le chiffre des dizaines est égal à la somme des autres chiffres. 
\item Faire la liste des tels nombres ; combien y en a-t-il ? 
\end{enumerate}

\subsection*{Exercice 4 - Écritures d'un nombre}

1) Donnez les parties entière et décimale des nombres suivants : 
\begin{enumerate}
\item $2,43$
\item $2334,892$
\item $032,548$
\item $20345,2430$
\end{enumerate}


2)Pour chacun des nombres précédents, l'écrire sous manière simplifiée s'il ne l'est pas déjà. \\

3) Pour chacun des nombres précédents, l'écrire comme une fraction décimale. 
\\

4) Pour chacun des nombres précédents, l'écrire comme la somme d'un entier et d'une fraction décimale inférieure à $1$. 


\subsection*{Exercice 5 - Repérage}

Placer sur une demi-droite graduée de onze carreaux les nombres suivants : 

\[ 0,1 ; 0,2 ; 0,5 ; 1 ; 0,20 \]

\subsection*{Exercice 6 - Repérage}

Placer sur une demi-droite graduée de onze carreaux les nombres suivants : 

\[ \frac1{10} ; 0,83+ \frac{7}{100} ; \frac{150}{1000}; 0,300 \]

\subsection*{Exercice 7 - Repérage}

Sur une demi-droite graduée correctement choisie, faire apparaître : 
\[ \frac{12}{100} ; 0,123 ; \frac1{10} + \frac3{100} + \frac1{1000} ; 0,1 + \frac{19}{100} ; \frac{120}{1000} + 0,006\]

 	\end{document}
