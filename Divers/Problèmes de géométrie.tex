\documentclass[12 pt]{extarticle}

	\usepackage[frenchb]{babel}
	\usepackage[utf8]{inputenc}  
	\usepackage[T1]{fontenc}
	\usepackage{amssymb}
	\usepackage[mathscr]{euscript}
	\usepackage{stmaryrd}
	\usepackage{amsmath}
	\usepackage{tikz}
	\usepackage[all,cmtip]{xy}
	\usepackage{amsthm}
	\usepackage{varioref}
	\usepackage{geometry}
	\geometry{a4paper}
	\usepackage{lmodern}
	\usepackage{hyperref}
	\usepackage{array}
	 \usepackage{fancyhdr}
	 \usepackage{float}
	\pagestyle{fancy}
	\theoremstyle{plain}
	\fancyfoot[C]{\thepage} 
	\fancyhead[L]{Fiche d'exercices}
	\fancyhead[R]{2022-2023}
	
	
\renewcommand{\theenumi}{\alph{enumi})}

\newlength{\taillecellule}
\setlength{\taillecellule}{2cm}
\newcolumntype{C}{@{}>{\centering\arraybackslash}p{\taillecellule}@{}}
\usetikzlibrary{calc}
\usepackage{pstricks,multido}
\usepackage{arrayjob}
\usepackage{calc,xlop}

\newcounter{AlphNode}
\renewcommand*{\theAlphNode}{\Alph{AlphNode}}

	\title{Exercices divers}
	\date{}
	\begin{document}

\begin{center}{\Large Problèmes variés de géométrie\\}
 \end{center} 
 
\subsection*{Exercice 1 - Variations sur un même thème (I)}
 
On considère un triangle $ABC$ et on suppose donnés un point $D\in[AB]$ et un point $E\in[AC]$ tels que $(DE)$ et $(BC)$ soient parallèles. 

On suppose ici $ED=DB$. L'objectif est de déterminer comment placer précisément les points $D$ et $E$. 

\begin{enumerate}
\item Faire une figure à main levée. (Se convaincre au passage qu'il serait
non immédiat de réussir à faire la figure avec $ED=DB$.)

\item On trace aussi la parallèle à $(AB)$ passant par $E$ et on note $F$
son intersection avec le côté $[BC]$. Compléter la figure. 

\item Démontrer que $BDEF$ est un parallélogramme. 

\item Sachant que $ED=DB$, montrez que le quadrilatère $BDEF$ est un losange. 

\item En déduire que $(BE)$ est un axe de symétrie du losange $BDEF$. 

\item Que représente $(BE)$ pour l'angle $\widehat{DBF}$ ? 

\item En déduire\footnote{(Indication : puisqu'il est écrit « en déduire », on commencera par tracer la droite $(BE)$ pour trouver le point $E$.)} un protocole de construction précis de la figure complète.

\end{enumerate}
\subsection*{Exercice 2 - Variations sur un même thème (II)}
 
On considère un triangle $ABC$ et on suppose donnés un point $D\in[AB]$ et un point $E\in[AC]$ tels que $(DE)$ et $(BC)$ soient parallèles. 

On suppose ici $ED=DB+EC$. L'objectif est de déterminer comment placer précisément les points $D$ et $E$. 

\begin{enumerate}
\item Faire une figure à main levée.

\item On place le point $F$ du segment $[DE]$ pour lequel $DF= DB$. 
Compléter la figure. 

\item Sachant que $ED=DB+EC$, démontrer que $EF= EC$. 

\item Quelle est la nature des triangles $DBF$ et $FEC$ ? 

\item On note $a = \widehat{DBF}$. Montrer que $\widehat{FDB}= 180^o-a-a$. 

\item En utilisant la propriété des angles alternes-internes et un angle intérmédiaire, montrer que $\widehat{DBC}=180^o-\widehat{FDB}$. 

\item En déduire que $\widehat{DBC}= 2 \times \widehat{DBF}$. 

\item On démontrerait de même que $\widehat{ECB}= 2 \times \widehat{ECF}$. Que représentent les droites $(BF)$ et $(CF)$ pour 
les angles $\widehat{DBC}$ et $\widehat{ECB}$ ? 

\item En déduire\footnote{(Indication : puisqu'il est écrit « en déduire », on commencera par tracer les deux droites de la question précédente pour trouver le point $F$, puis le reste de la figure.)} un protocole de construction précis de la figure complète.

\end{enumerate}


\subsection*{Exercice 3 - Variations sur un même thème (III)}

On considère un triangle $ABC$ et on suppose donnés un point $D\in[AB]$ et un point $E\in[AC]$ tels que $(DE)$ et $(BC)$ soient parallèles. 

On suppose ici $AE = BD$. L'objectif est de déterminer comment placer précisément les points $D$ et $E$. 

\begin{enumerate}
\item Faire une figure à main levée.

\item On trace la parallèle à $(AB)$ passant par $E$, et on note $F$ son point d'intersection avec le côté $[BC]$. 
Compléter la figure. 

\item Quelle est la nature du quadrilatère $BDEF$ ? Justifier. 

\item En déduire que $EF=BD$. 

\item On trace maintenant la parallèle à $(AE)$ passant par $F$, et on note $G$ son point d'intersection avec $[AB]$. Compléter la figure.

\item Démontrer que $AEFG$ est un parallélogramme. 

\item Sachant que $AE=BD$, montrer que $EF = AE$, puis que $AEFG$ est un losange. 

\item Que représente donc la droite $(AF)$ pour l'angle $\widehat{CAB}$ ? 

\item En déduire\footnote{(Indication : puisqu'il est écrit « en déduire », on commencera par tracer la droite de la question précédente pour trouver le point $F$, puis le reste de la figure.)} un protocole de construction précis de la figure complète.

\end{enumerate}
 
 	\end{document}
